\documentclass[11pt, class=article, crop=false]{standalone}
\usepackage{amsmath,amssymb,amsfonts,amsthm}
\usepackage{enumitem}
\usepackage{fancyhdr}
\usepackage{tikz-cd}
\usepackage{mathabx}
\usepackage{geometry}
\usepackage{natbib}
\usepackage{braket}
\usepackage{graphicx}
\usepackage{simpler-wick}
\usepackage{hyperref}
\usepackage{ytableau}
\usepackage{cancel}
\usepackage{listings}
\usepackage{relsize}
\usepackage{xcolor}
\usepackage{stmaryrd}
\usepackage{slashed}
\usepackage{tikz-feynman}
\usepackage{kiritsis}
\geometry{margin = 0.5in}


\begin{document}
\section*{Chapter 14: The Bulk/Boundary (Holographic) Correspondence} % (fold)
\label{sec:chapter_14_the_bulk_boundary_holographic_correspondence}

\begin{enumerate}
	\item The two diagrams are as follows:
	
	\begin{center}
		\textbf{Insert drawing}
	\end{center}
	
	\item Let us focus on vacuum diagrams. We will rescale the fundamental field (call it $\phi$) so that the action has a leading factor going as $N$. Schematically this is:
	\[
		N\left(\frac{1}{\lambda} \Tr F^2 + \tr (D \phi)^2  \right)
	\]
	 By the same argument as for fermions in QED, any fermion worldline will give a closed curve, which we interpret at giving a boundary of the Riemann surface. Moreover, each fundamental field loop will contain exactly as many propagators as vertices (except for the trivial disconnected loop). 
	
	Each fundamental vertex will contribute $N$ while each propagator will contribute $\frac{1}{N}$. The connected fundamental loops could be viewed as \emph{not} contributing $N$ because each fundamental line will join with gauge boson lines which are already traced over. The fact that the fundamental loops do not contribute a face is consistent with their interpretation as enclosing boundaries of a Riemann surface. Thus, the introduction of each fundamental loop is equivalent to including a cycle with no interior, so we will \emph{not} count the face, and the vertices and edges don't contribute either. The  counting is then unmodified
	\[
		\left(\frac{\lambda}{N} \right)^E \left(\frac{N}{\lambda} \right)^V N^F  = N^{\chi}\, \lambda^{E - V}
	\]
	
	\begin{center}
		\textbf{Insert drawing}
	\end{center}
	
	\item I think that the normalization of this operator (which is the same as in MAGOO) is just wrong. I think it should be $\Phi_a = \Tr[\prod_i X_i^{n_i}]$
	\textbf{McGreevy confirms this}.
	
	Our single trace operator is a product over the distinct $X_i$ fields
	\[
		\Phi_a = \Tr \prod_{i} X_i^{n_i}
	\]
	Each such insertion has $\sum_i n_i$ external legs.
		
	For $\Phi_a$ distinct, we consider:
	\[
		S = N \frac{1}{\lambda} \Tr[d X^i d X^j + c_{ijk} X^i X^j X^k + \dots] + \sum_{a=1}^m N g_a(x)  \Phi_a (x)
	\]
	Here the $g_a$ are taken to be functions of $x$. % Note that the factor of $N$ out front of the last term comes from our choice of normalization of the $\Phi_a$.
	
	We then compute:
	\[
		\braket{ \prod_{a=1}^m \Phi_a(x_a)} = \frac{1}{N^m} \frac{\delta}{\delta g_1(x_1)} \dots \frac{\delta}{\delta g_m(x_m)}   \log \mathcal Z[\{g_a\}]
	\]
	We can compute vacuum bubbles for this modified action as before. We now get a new vertex type involving the single-trace operator $\Phi_a$. Because it still appears with a coefficient $N$ in the action, we can still apply the same counting logic to the calculation of $\log \mathcal Z$. The spherical contribution dominates.
	
	Thus, again the leading contributions to the free energy goes as $N^{2}$, and we get that the correlator expression behaves as $N^{2-m}$. In particular, the three-point function vanishes as $1/N$, as said in the text. 
	
	% This also justifies our normalization convention - this way the two-point functions are normalized to be $N$-independent in the large $N$ limit.
	
	\item 
	Again, let's take the operator without the $\frac{1}{N^2}$ out front. To add in a double-trace operator requires two factors of $N$ out front. 
	
	We thus add to the action the term:
	\[
		S_{new} = \sum_{a=1}^N N^2 g_a(x) \Psi_a(x)
	\]
	We get $m$-point correlators by differentiating $m$ times by $N^2 g_a$. With our choice of $S_{new}$, $\log Z$ still satisfies the same Euler characteristic rules as before, and has a dominant contribution going as $N^2$. We see that the two-point function then goes as $N^{-2}$. Normalizing the two-point function two $1$ requires that we look at the fields $\tilde \Psi_a = N \Psi_a$. The three point function then goes as $N^{-1}$.
	
	\textbf{Understand the Silverstein paper about the connecting $S^2$s contributing for double twist.}
	
	\item We add to the action the following (schematic) terms:
	\[
		N \Tr[(D q)^2 + (D \bar q)^2 + q \bar q + q \bar q \Phi + \dots ]
	\]
	Again the propagators will give $1/N$ and the vertices will give $N$, so we can apply the same analysis to get that planar diagrams dominate. For the two point correlator of $q \bar q$, we must introduce a quark loop into our worldsheet. The lowest-genus such surface is the disk, with two $q \bar q$s inserted on the boundary. This is genus $1$. Differentiating with respect to $N$ twice I get a two-point function going as $N^{-1}$. To get this to be unity I must rescale my mesons operator to be $\sqrt{N} q \bar q$. Now the $m$-point correlation function goes as $N^{m/2} N^{1-m}$.
	We thus get scaling behavior behavior $N^{1-m/2}$ for mesons. In particular the 3-point function goes as $1/\sqrt{N}$. 
	
	% I can write the mesons as $\phi = \tr q \bar q$ where the trace is in the fundamental representation. Adding $\int N g(x) \phi(x)$ to the action and differentiating
	% \[
	% 	\frac{1}{N^m} \frac{\delta}{\delta g(x_1)} \dots \frac{\delta}{\delta g(x_m)} \log \mathcal Z
	% \]
	% we get contributions from $m$ insertions of the meson traces. This corresponds to a Riemann surface with $m$ boundaries. This gives correlation functions going as $N^{2 - m}$
	
	\item It is important that in this case, for both $\SO(N)$ and $\Sp(2N)$, the fundamental representation $\mathbf{F}$ is real. Consequently, the adjoint can be written (up to $1/N$ corrections) as the antisymmetrized (resp symmetrized) part of $\mathbf{F} \otimes \mathbf{F}$. In double-line notation we can understand the gluons as being labeled by two ``fundamental lines''. Because there is no difference between $\mathbf{F}$ and $\overline{\mathbf{F}}$, there is no inherent orientation to the strips, and we can twist to form unoriented surfaces. Thus, the string theory that these would correspond to must necessarily be non-oriented. 
	
	The difference between the orthogonal and symplectic projection will be in the relative sign of a propogator with intermediate twist between $\O(2N)$ and $\Sp(2N)$. For $\O(2N)$ we have the same sign contribution between the propagator and the propagator-with-crosscap. For $\Sp(2N)$, we have the opposite sign.
	
	\textbf{Draw this}
	
	We can talk about a large $N$ expansion of diagrams identically. The only additional ingredient is incorporating points where edges swap. These play roles identical to cross-caps. The diagrams we can draw will have $V$ vertices with $N/\lambda$ coefficient, $E$ edges with $\lambda/N$ coefficient, $F$ faces, with $N$ coefficient, and $C$ ``cross-caps'' with $N^{-1}$ coefficient. Altogether these gives 
	\[
		N^{V-E+F-C} \lambda^{E-V} = N^{\chi} \lambda^{E-V}
	\]
	generalizing the prior discussion.
	
	\item Take $L = 1$. The relationships \textbf{14.4.7} become:
	\[
		\ell_s = \lambda^{-1/4} = (4 \pi g_s N)^{-1/4}
	\]
	and
	\[
		G_N = \frac{(2 \pi)^7 \ell_s^8}{16 \pi} g_s^2 = \frac{(2 \pi)^7 \ell_s^8}{16 \pi} \frac{1}{(4 \pi \ell_s^4 N)^2} = \frac{\pi^4}{4 N^2}.
	\]
	
	\item 
	 Starting with IIB the gravitational constant is $16 \pi G_N = (2\pi)^7 \ell_s^8 g_s^2$. The volume of $S^5$ is $\pi^3 L^3$. We get:
	 \[
	 	G_5 = \frac{8 \pi^3 \ell_s^8}{L^5} g_s^2
	 \]
	 Recall in AdS/CFT, the coupling constant $\lambda$ is the 4th power of $L$ in string units: 
	 	\[
	 		\frac{L^4}{\ell_s^4} = 4 \pi g_s N
	 	\]
		Substituting this gives:
		\[
			G_5 = \frac{8 \pi^3 L^3}{(4 \pi N)^2} = \frac{\pi L^3}{2 N^2}
		\]
		This is a nice relationship, independent of the string length, and only dependent on the size of AdS and the number of D-branes. 
	
	\item \textbf{Massless and massive vector fields in AdS}
	\begin{enumerate}
		\item \textbf{Massless Case} Take the gauge $A_u = 0$. Let $A_\mu (u, \vec x) = u^{\Delta - 1} A_\mu (\vec x)$. Note that this way $A_\mu dx^\mu$ has scaling dimension $u^\Delta$. The equations of motion for the Maxwell theory give:
		\[
			0 = \partial_M (\sqrt{-g} F^{MN}) = \partial_M (\sqrt{-g} g^{M A} g^{N B} \partial_{\small[ A} A_{B\small]}) = \partial_{u} \left( \frac{1}{u^{p+2}} u^{4} \partial_{u} (u^{\Delta-1} A_\mu(\vec x)) \right) + \dots
		\]
		 where $\dots$ are terms with higher powers of $z$. Altogether this is:
		 \[
		 	(\Delta - p)(\Delta - 1) (A_\mu (\vec x) = 0.
		 \]
		 This implies that either $\Delta = 1$ or $\Delta = p$.
		 
		% We can get the scaling dimension by looking at how the invariant field $A = A_\mu dx ^\mu$ falls off at infinity. The reason for the discrepancy is that $dx^\mu$ is not a unit vector, but rather has length $R/z$. Incorporating this effect, we see that:
		% \[
		% 	A_M (z, \vec x) = z A(\vec x) + \dots
		% \]
		% Thus the leading dimension is $1$.
		% The scaling of $A$ is $|A| = \sqrt{g^{\mu \nu} A_\mu A_\nu} \to \sqrt{\frac{z^2}{R^2}} z^{d-2} = z^{d-1}$.
		
		Consequently this gives a scaling dimension of $d-1$ to $J^\mu$, exactly what we want for a conserved current. 
		
		\item \textbf{Massive case} Now again we have:
		\[
		\begin{aligned}
			0 &= \partial_M (\sqrt{-g} F^{MN}) - \sqrt{-g} \frac{m^2}{L^2} A^N\\
			& = \partial_M (\sqrt{-g} g^{MA} g^{NB} \partial_{\small[ A} A_{B\small]}) - \sqrt{-g} \frac{m^2}{L^2} A_M\\
			& = L^{p-2} \partial_{z} \left( \frac{1}{u^{p+2}} u^{4} \partial_u (u^{\Delta-1} A_M(\vec x)) \right) - L^{p} m^2 z^{\Delta-p-1} A_M + \dots
		\end{aligned}
		\]
		Then this gives a new quadratic equation for $\Delta$:
		\[
			0 = (\Delta - 1) (\Delta - p) - L^2 m^2 \Rightarrow \Delta = \frac{p+1}{2} \pm \sqrt{\frac{(p-1)^2}{4} + m^2 L^2}
		\]
		% The
		%  % As before, looking at $|A| \propto \sqrt{\frac{z^2}{R^2}} z^\Delta \to z^{\Delta + 1}$ gives a conformal dimension of one higher:
		% \[
		% 	\tilde \Delta = \frac{p+1}{2} \pm \sqrt{\frac{(p-1)^2}{4} + m^2 L^2}
		% \]
		We see that the massive field picks up an $A_z$ component which cannot be gauged away, as the mass term is not gauge invariant. Because gauge invariance is lost, we see that the corresponding operator in the CFT does not have dimension $d-1$ anymore and no longer gives rise to a conserved current.
		
	\end{enumerate}
	
	\item Taking $A_\mu \to A_\mu + \d_\mu \epsilon$ gives:
	\[
		W[A_\mu] \to W[A_\mu + \d_\mu \epsilon] = \braket{e^{\int d^d x\, J^\mu A_\mu - \int d^d x\, \epsilon \d_\mu J^\mu }} = W[A_\mu]
	\]
	here $A_\mu$ is the source for the boundary $J^\mu$ current in the CFT. So the partition function is invariant under any local gauge transformation of $A_\mu$. This gives us that a global $U(1)$ in the CFT corresponds to a gauged $U(1)$ on the boundary.
	
	\item We couple our CFT stress tensor $T^{\mu \nu}$ to an external field $h_{\mu \nu}$. Note that under any shift $h_{\mu \nu} + \d_\mu \epsilon_\nu + \d_\nu \epsilon_\mu$ we get:
	\[
		W[h_{\mu \nu}] \to W[h_{\mu \nu} + \d_\mu \epsilon_\nu + \d_\nu \epsilon_\mu] = \braket{e^{\int d^d x\, T^{\mu \nu} h_{\mu \nu}- 2 \int d^d x\, \epsilon_\nu \d_\mu T^{\mu \nu}}} = W[h_{\mu \nu}]
	\]
	
	Thus, the translation invariance given by $\d_\mu T^{\mu \nu} = 0$ of the CFT$_d$ has given diffeomorphism invariance in the bulk. This is gravity. 
	
	\item In global coordinates the metric is
	\[
		ds^2 = \frac{L^2}{\cos^2 \theta} \left(-d\tau^2 + d \theta^2 + \sin^2 \theta d\Omega_3^2 \right)
	\]
	Let's find $u$ by integrating:
	\[
		\int_0^{\theta'} \frac{d\theta}{\cos \theta} = \int_0^{u'} \frac{2 du}{(1-u^2)} \Rightarrow u = \tan \tfrac\theta2
	\]
	Consequently,
	\[
		\frac{4 u^2}{(1-u^2)^2} = \tan^2 \theta = \frac{\sin^2 \theta}{\cos^2 \theta}
	\]
	so we get agreement for the $du, d\Omega_3$ terms. Finally
	\[
		\frac{(1+u^2)^2}{(1-u^2)^2} = \frac{1}{\cos^2 \theta}
	\]
	and we get agreement for the $d\tau$ term, without having to rescale $\tau$. Since $0 < \theta < \pi/2$ we have $0 < u < 1$ as required. 
	
	\item Take $d\tau = 0$. First, take the points to be on opposite sides of the AdS cylinder. We get a distance equal to
	\[
		\int_{1-\epsilon}^{1+\epsilon} \frac{2 du}{1-u^2} = 2 \log\left(\frac{2-\epsilon}{\epsilon}\right) \approx 2 \log\left(\frac{2}{\epsilon}\right)
	\]
	Given that AdS is a homogenous space, we can use symmetry arguments from group theory to get the general formula, replacing $2$ with $|x_1 - x_2|$. For finding the geodesic distance through direct methods, a better coordinate system would be global coordinates. In AdS$_3$ this looks like:
	\[
	\begin{aligned}
		X_{-1} &= L \cosh \rho \cos T\\
		X_{0} &= L \cosh \rho \sin T\\
		X_{1} &= L \sinh \rho \cos \theta\\
		X_{2} &= L \sinh \rho \sin \theta\\
	\end{aligned}
	\] 
	this generalizes directly to AdS$_{p+2}$. The argument will be the same there. Note 
	\[
		ds^2 = L^2 (-\cosh^2 \rho dT^2 + d\rho^2 + \sinh^2 \rho d\theta^2).
	\]
	Now take $dT= 0$. The Lagrangian takes the form:
	\[
		\mathcal L = L \sqrt{\dot \rho^2 + \sinh^2 \rho \, \dot \theta^2}
	\]
	Take $\tau = \rho$ so that the EOM for $\theta$ quickly gives:
	\[
		\theta' = \frac{c}{\sinh \rho \sqrt{\sinh^2 \rho - c^2}}
	\]
	Take $c = \sinh \rho_0$. In AdS this corresponds to the minimum distance $\rho$ that the geodesic will approach, since there we have $\theta' = \infty \Rightarrow d\rho/d\theta = 0$.
	
	\textbf{Insert figure}
	
	To get $\Delta \theta$, we integrate this, giving
	\[
		\Delta \theta = 2 \arctan \frac{\cosh \rho\, \sqrt{2} \sinh \rho_0}{\sqrt{\cosh 2 \rho - \cosh 2 \rho_0}} + c_0
	\]
	The factor of $2$ comes from the fact that we need to do the $\rho$ integration twice to get the full geodesic curve. Now we must take $\rho \to \infty$ to approach the boundary of AdS. In this case, the equation for $\Delta \theta$ simplifies to:
	\[
		\tan \frac{\Delta \theta - c}{2}  = \sinh \rho_0
	\]
	Taking $\theta \to 0$ (corresponding to $\rho_0 \to \infty$) shows that $c = -\pi /2$. This gives
	\[
		\tan \frac{\Delta \theta }{2}  = \frac{1}{\sinh \rho_0}
	\]
	The total length of the trajectory is:
	\[
		L \int d\rho \sqrt{1 + \sinh^2 \rho
		\, (\theta'(\rho))^2} = L \int d\rho \frac{\sinh \rho}{\sqrt{\sinh^2 \rho - \sinh^2 \rho_0}} = 2 L  \log\left(\frac{\cosh \rho_f}{\cosh \rho_0}  +  \sqrt{\frac{\sinh^2 \rho_f - \sinh^2 \rho_0}{\cosh^2 \rho_0}}\right)
	\]
	Take $\rho_f \to \infty$. The leading behavior of this goes as
	\[
		2 L \log \frac{2 e^{\rho_f}}{\cosh \rho_0} = 2 L \log \left(2 e^{\rho_f} \sin \frac{\theta}{2} \right)
	\]
	Now note that for $x, y$ coordinates in $\RR^{d+2}$ lying on the unit $S^{d+1}$, we have 
	\[
		|x-y|^2 = (\cos^2 \theta - 1)^2 + \sin^2 \theta = 4 \sin^2 \frac{\theta}{2} \Rightarrow |x-y| = 2 \sin \frac{\theta}{2}.
	\]
	Finally, $\sinh \rho_f = \tan \theta_f$. This gives $\theta_f = \pi/2 - \epsilon$, with $\epsilon = e^{-\rho_f}$. Then $\tilde u = \tan \frac{\theta_f}{2} \approx 1-\epsilon$. So indeed we get the final entropy formula:
	\[
		A = 2 L \log (|x_1-x_2|/\epsilon)
	\]
	
	\item The volume element goes as
	\[
		16 L^4 \int_0^{1-\epsilon} \frac{u^3}{(1-u^2)^4} du d\Omega_3 \approx \frac{16 L^4}{6 \epsilon^3} 2 \pi^2 = \frac{16 \pi^2 L^4}{3 \epsilon^3}
	\]
	The $\epsilon^3$ scaling valid in the small $\epsilon$ limit is exactly area scaling.
	
	\item Because of a horizon, particles approaching this horizon will be arbitrarily redshifted. This implies that the frequencies reaching the boundary can be shifted to arbitrarily low values, giving a continuum of states above the vacuum state with no mass gap.
	
	\item Yes (Maldacena already does this in his seminal paper). Take the branes to be separated by a distance $r$. The supergravity solution will look like
	\[
		ds^2 = \frac{1}{\sqrt{H}} dx_{\parallel}^2 + \sqrt{H} dx_\perp^2, \quad H = 1 + \frac{4 \pi g N \ell_s^4}{r^4}
	\]
	and take $\ell_s, r \to 0$ while holding $U = r/\ell_s^2$ fixed (we've done this type of near-horizon analysis in chapter 11). This keeps the masses of the stretched strings fixed even as we bring the branes together. If all the branes are coincident, the coordinate $r$ in the supergravity solution gives an equivalent coordinate $U = r/\ell_s^2$, giving the metric
	\[
		ds^2 = \ell_s^2 \left[ \frac{U^2}{\sqrt{4 \pi g N}} dx_\parallel^2 + \sqrt{4 \pi g N} \frac{dU^2}{U^2} + \sqrt{4 \pi g N} d\Omega_5^2 \right]
	\]
	Now, pulling $M$ of the $N$ D3 branes off by a distance $W$ gives a supergravity solution:
	\[
			ds^2 = \ell_s^2 \left[ \frac{U^2}{\sqrt{4 \pi g} \sqrt{N - M + \frac{M U^4}{|U - W|^4}}} dx_\parallel^2 + \sqrt{4 \pi g} \frac{dU^2}{U^2} \sqrt{N - M + \frac{M U^4}{|U - W|^4}} + \sqrt{4 \pi g N} d\Omega_5^2 \right]
	\]
	As long as $U \gg W$ we still effectively see $AdS_5 \times S_5$. For smaller values of $U$, this splits into two separated AdS backgrounds, with two different near-horizon limits. We can trust these limits when both $M$ and $N$ are large, but splitting off single branes gives geometries with singular curvatures that we cannot trust. 
	
	But since finite $U$ already means that we have taken the near-horizon limit, the entire moduli space $\RR^{6N}/S_N$ is visible at that level.
	
	\item The eigenvalues of the Laplacian on a 5-sphere of radius $L$ are in correspondence with the quadratic casimir of $\SO(6)$ for the $(0, k, 0)$ representations, giving:
	\[
		\frac{k(k+5)}{L^2}
	\]
	For a proof of this, consider a homogenous harmonic function of degree $k$. Any such homogenous harmonic function takes the form $f = |x|^{k} \sum_{m_i} Y^k_{m_i}(\gamma)$. Look then at the Laplacian
	\[
		f = -\nabla^2 (|x|^{-k} f) = k (k + 6 - 2) |x|^{-(k+2)} f + x^{-s} \cancel{\nabla^2 f}
	\]
	Thus the spherical harmonics on a unit $S^5$ have eigenvalue $k(k+4)$. Rescaling the sphere will contravariantly rescale these eigenvalues by $L^{-2}$ as required.
	
	
	\item The wave equation for a massive scalar field is quickly seen to be:
	\[
		\nabla^2 \phi = \frac{u^2}{L^2} \left[\d_u^2 - \frac{p}{u} \d_u - \d_t^2 + \d \cdot \d \right] \phi = m^2 \phi 
	\]
	Upon Fourier transforming
	\[
		\phi(u, x) = \int \frac{d^{p+1} q}{(2 \pi)^{p+1}} \phi(u, q) e^{i q \cdot x}, \quad q \cdot x = - q^0 t + \vec q \cdot \vec x
	\]
	we get
	\[
		 \left[\d_u^2 - \frac{p}{u} \d_u - q^2 + \d \cdot \d - \frac{m^2 L^2}{u^2} \right] \phi(u, q) = 0
	\]
	solving this in Mathematica directly yields two solutions
	\[
		\phi_\pm(u, q) = A u^{\frac{1+p}{2}} J_\nu ( i q u ) + B u^{\frac{1+p}{2}} Y_\nu ( i q u ), \quad \nu = \frac12 \sqrt{(p+1)^2 + 4 m^2 L^2}
	\]
	We can rewrite this in terms of modified Bessel functions as:
	\[
		A u^{\frac{1+p}{2}} I_\nu (  q u ) + B u^{\frac{1+p}{2}} K_\nu ( i q u )
	\]
	These two solutions have the desired scaling dimensions of $\Delta_\pm$ respectively ($K$ will have to be defined differently from how it is defined in mathematica and wikipedia).
	
	Now WLOG take $x' = 0$. Rotating to Euclidean space, it is a quick check to see that the bulk-to-boundary propagator as written in \textbf{L.51} does indeed satisfy the massive Laplace equation precisely when $\Delta (\Delta - p - 1) = m^2 L^2$.
	\begin{center}
		\includegraphics[scale=0.5]{"Figures/Bulk-to-Boundary"}
	\end{center}
	
	Next, we see that
	\[
		\int d^{p+1} x \, f(u, x; 0) = \int d^{p+1} x\, \frac{u^\Delta}{(u^2 + x^2)^\Delta} = u^{p+1-\Delta} \int d^{p+1} y \frac{1}{(1+\zeta^2)^\Delta} = u^{p+1-\Delta} \Omega_{p} \int_0^\infty \frac{\zeta^p d\zeta}{(1 + \zeta^2)^\Delta}
	\]
	This last integral can easily be evaluated using $\Gamma$-functions. The final result is then:
	\[
		u^{p+1-\Delta} \Omega_{p} \frac{\Gamma(\frac{1+p}{2}) \Gamma(\Delta - \frac{1+p}{2})}{2\Gamma(\Delta)} = u^{p+1-\Delta} \pi^{\frac{p+1}{2}} \frac{\Gamma(\Delta - \frac{1+p}{2})}{\Gamma(\Delta)}
	\]
	Thus, the normalized bulk-to-boundary propagator is:
	\[
		\frac{\Gamma(\Delta)}{\pi^{\frac{p+1}{2}}  \Gamma(\Delta -  \frac{1+p}{2})} \frac{u^\Delta}{(u^2 + |x-x'|^2)^\Delta}.
	\]
	In the above, we pick $\Delta = \Delta_+$. By convolving a boundary configurations $\phi_0(x)$ with this propagator, we obtain a field $\phi$ in AdS satisfying the massive Laplace equation. 
	\[
		\phi(u, x) = \frac{\Gamma(\Delta)}{\pi^{\frac{p+1}{2}}  \Gamma(\Delta -  \frac{1+p}{2})} \int d^{p+1} x' \frac{u^\Delta}{(u^2 + |x-x'|^2)^\Delta} \phi_0(x')
	\]
	This is clear. It is also quick to see that as $u \to 0$ the leading behavior comes from the bulk-to-boundary propagator going as $u^{p+1-\Delta}  \phi_0(x) = u^{\Delta_-} \phi_0(x)$. This is exactly proportional to the leading solution, which asymptotes with the lower power $\Delta_-$.
	
	It is worth noting that there is another very clean way to obtain this propagator, as originally written in Witten's paper. Take the delta function source to be at $u = \infty$ and let's look for a solution of the massive wave equation. Because the source is at $\infty$ the solution has full symmetry under translations in $x$, and so can only depend on $u$. The equations of motion are:
	\[
		\d_u u^{-p - 2} u^{2} \d_u \phi(u) - m^2 L^2 \phi = 0 \Rightarrow \phi(u) = u^\Delta, \quad \Delta (\Delta - p - 1) = m^2 L^2
	\]
	this gives the correct $\Delta_{\pm}$ as required. To relate this to our solution for a $\delta$ function at $0$ we must do an inversion, which consists of taking $u, x \to \frac{u, x}{u^2 + |x|^2}$ yielding our desired propagator. 
	
	\item The extrinsic curvature $K$ is defined as
	\[
		K_{\mu \nu} = \frac12 (\nabla_\mu n_\nu + \nabla_\nu n_\mu), \quad K = h^{\mu \nu} K_{\mu \nu}
	\]
	where $n_\mu$ is the normal vector and $h$ is the pullback of the metric $g$ to $\d M$. When the coordinate system is hypersurface-orthogonal, then this simplifies nicely to 
	\[
		K_{\mu \nu} = \frac12 n^\rho \d_\rho G_{\mu \nu}
	\]
	For the poincare patch, we have the normal vector $-\frac{1}{\sqrt g_{uu} }\d_u = -\frac{u}{L} \d_u$. Note the minus sign, because $u \to 0$ gives the boundary, which is the opposite from the outward normal orientation. The contraction gives:
	\[
		K_{\mu \nu} = \frac12 ((-)^2 \frac{2}{L} h_{\mu \nu}) \Rightarrow K = \frac{p+1}{L}
	\]
	as required.
	
	\item This problem is done for $p=3$, but I will solve it for general $p$. The subleading order equation of motion for $\phi$ of the form $u^{\Delta} (\phi_0 + u^2 \phi_2)$ is
	\[
		u^\Delta \Delta (\Delta -p-1) \phi_0 -  u^{\Delta} m^2 L^2 \phi_0 + u^{\Delta + 2} \Box \phi_0  + u^{\Delta + 2} (\Delta+2) (\Delta - p + 1)  \phi_2 + u^{\Delta - 2} m^2 L^2 \phi_2 = 0
	\]
	At leading $u^{\Delta}$ order we get the quadratic constraint on $\Delta$, giving $\Delta_\pm$ as solutions. Solving for $\phi_2$ at subleading order gives:
	\[
		\phi_2 = - \frac{\Box \phi_0}{(\Delta+2)(\Delta-p+1) + m^2 L^2}
	\]
	Plugging in for $\Delta = \Delta_-$ yields:
	\[
		\phi_2 = - \frac{\Box \phi_0}{4 \Delta_- + 2 - 2p}
	\]
	This is consistent with what is written when $p=3$. Taking now $\Delta \to p+1 - \Delta$ gives
	\[
		A_2 = \frac{\Box \phi_0}{4 \Delta_- - 6 - 2 p}
	\]
	Again, taking $p = 3$ gives the correct $4 (\Delta_- - 3)$ denominator.
	%
	% It is worth remarking that this is easily extendable:
	% \[
	% 	\phi_{2n}(x) = - \frac{\Box \phi_{2n-2}}{2n (2 \Delta_- - p-1 + 2n)}
	% \]
	% This is valid
	
	\item Let's repeat the argument for clarity. We stick to $p+1=d=4$. We have a bulk-to-boundary propagator given by:
	\[
		K_{\Delta}(u, x; x') = \frac{1}{C_3} \frac{u^\Delta}{(u^2 + |x-x'|^2)^\Delta}, \quad C_3 = \pi^2 \frac{\Gamma(\Delta-2)}{\Gamma(\Delta)}
	\]
	At this stage we take $\Delta$ to be arbitrarily. We do not identify it with $\Delta_\pm$. In the small $u$-limit the propagator looks like:
	\[
		u^{4-\Delta} (\delta(x- x') + O(u^2)) + u^{\Delta} \left(\frac{C_3^{-1}}{|x_1 - x_2'|^2} + O(u^2)\right)
	\]
	A general field $\phi_0(x)$ on the boundary sources a the bulk field $\phi(u, x)$ to take the form:
	\begin{equation}\label{eq:small-u-phi}
	\begin{aligned}
		\phi(u, x) &= \int dx' K(u, x; x') \phi_0 (x') \\ 
		&= u^{4-\Delta} (\phi_0 (x) + u^2 \phi_2 (x) + \dots) +  u^{\Delta} \left(C_3^{-1} \int d^4x' \frac{\phi_0(x')}{|x-x'|^{2\Delta}} + u^2 A_2(x) + \dots \right)
	\end{aligned}
	\end{equation}
	Here we have written the additional terms that we obtained in the prior exercise. This gives for the on-shell action at leading order:
	\[
	\begin{aligned}
		S_{\text{on-shell}} &= - \frac{M_{pl}^3}{2} \int d^4 x \sqrt{g} g^{uu} \phi \d_u \phi\Big|_{u = \epsilon}\\
		&= - \frac{(M_{pl} L)^3}{2} \int d^4 x_1 d^4 x_2 \phi(x_1) \phi(x_2) \int d^4 x \frac{K(u, x_1; x) \d_u K(u, x_2; x)}{u^3}\Big|_{u = \epsilon}\\
	\end{aligned}
	\]
	This last integral is given by:
	\[
		(4-\Delta) u^{4-2\Delta} \delta^4 (x_1 - x_2) + \frac{4 C_3^{-1}}{|x_1 - x_2|} + \frac{\Delta}{C_3^2} u^{2 \Delta - 4} \int d^4 x \frac{1}{|x-x_1|^{2\Delta} |x-x_2|^{2\Delta}} + \dots
	\]
	When $1 < \Delta < 3$ the remaining terms vanish. \textbf{The third term doesn't though, at least not for $1<\Delta<2$. Different kind of counterterm needed?}
	
	Now let's take $3 < \Delta < 4$. the third term and all of its higher-order contributions will vanish. on the other hand, not only will the first term require a counter-term going as $\phi^2$, but so will the $u^{6-2 \Delta}$ term. From Equation~\eqref{eq:small-u-phi} we see that this must go as 
	\[
		u^{4-\Delta}(\delta(x-x') - \frac{u^2}{4 (3- \Delta)} \Box_x \delta(x- x') )
	\]
	Then we get divergent terms from:
	\[
	\begin{aligned}
		&\int d^4 x \, u^{-3} \, u^{4-\Delta}\left(\delta(x-x_1) - \frac{u^2}{4 (3- \Delta)} \Box_x \delta(x- x_1) \right) \, \d_u \left[
		u^{4-\Delta} \left(\delta(x-x_2) - \frac{u^2}{4 (3- \Delta)} \Box_x \delta(x- x_2) \right) \right]\\
		&= (4-\Delta) u^{4-2\Delta} \delta^4 (x_1 - x_2) + (4-\Delta) \frac{u^{6-2\Delta}}{4 (\Delta-3)} \Box_{x_1} \delta^4 (x_1 - x_2) + (6-\Delta) \frac{u^{6-2\Delta}}{4 (\Delta-3)} \Box_{x_1} \delta^4 (x_1 - x_2)\\
		&= (4-\Delta) u^{4-2\Delta} \delta^4 (x_1 - x_2) +  u^{6-2\Delta} \frac{10-2\Delta}{4 (\Delta-3)} \Box_{x_1} \delta^4 (x_1 - x_2)
	\end{aligned}
	\]
	This altogether contributes:
	\[
		- \frac{(M_{pl} L)^3}{2} \epsilon^{6-2\Delta} \frac{10-2\Delta}{4 (\Delta-3)} \int dx \phi_0 \Box \phi_0
	\]
	
	In the original action, expanding $\phi$ to quadratic order in $\epsilon$ gives
	\[
		\phi = \epsilon^{4-\Delta} \phi_0 + \frac{\epsilon^{2 + 4 - \Delta}}{4 (\Delta - 3)} \Box \phi_0 \Rightarrow \frac{(ML)^ \Delta_{-}}{2} \phi^2 =   \frac{(ML)^3 (4- \Delta)}{2} \epsilon^{2 (4- \Delta)} \phi_0^2  + \epsilon^{2 (4 - \Delta) + 2} (ML)^3 (4-\Delta) \frac{\phi_0 \Box \phi_0}{4 (\Delta - 3)} + \dots
	\]
	This shows that the $\phi^2$ term contributes a $\phi_0 \Box \phi_0$ term as well. 
	
	We thus need a counterterm action given by:
	\[
		S_{ct} = \int \sqrt{h} \left( \frac{M^3 \Delta_-}{2L} \phi^2 + \frac{M_{pl}^3 L}{2} \frac{1}{4 (\Delta - 3)} \phi \Box \phi \right)% \frac{10-2\Delta}{4 (\Delta-3)} \int dx \sqrt{h^\epsilon} \phi(\epsilon, x) \Box \phi(\epsilon, x).
	\]
	Here, now, $\Box$ is taken with respect to the boundary metric $h^\epsilon$, meaning it absorbs two factors of $\epsilon/L$. \textbf{This is correct :)}
	
	\item Our correlator is:
	\[
		\braket{\phi(k) \phi(q)} % = \int d^4 x e^{i p x - i q y} \braket{\phi(x) \phi(y)} =
		= \frac{\d}{\d \lambda_k \d \lambda_q} S_{\text{on-shell}}(\phi_0(x)  = \lambda_k e^{i k x} + \lambda_q e^{i q x} )\Big|_{\lambda_k, \lambda_q = 0}
	\]
	We take the on-shell action and get
	\[
	\begin{aligned}
	S_{\text{on-shell}}\left(\phi \right) 
	&= - \frac{M^3}{2} \int d^4 x \sqrt g g^{uu} \phi \d_u \phi 
	\\&=  - \frac{M^3}{2} \int d^4 k d^4 q \lambda_k \lambda_q \sdelta^4(k+q) u^{-3} \phi(u, p) \d_u \phi(u, q) \Big|_{u = \epsilon}
	\end{aligned}
	\]
	Using the form \textbf{L.50} of the bulk-to-boundary propagator we can solve this:
	\begin{center}
		\includegraphics[scale=0.5]{"Figures/AdS 2-point"}
	\end{center}
    \textbf{Factor of 2 off from Gubser, Klebanov, Polyakov, but I don't think this problem asks for the coefficients, rather just the form of the correlator.}
	
	There is always a term going as $\log (p\epsilon/2)$. All other terms are positive polynomials in $p$ and so will contribute contact terms that we will need to supply counterterms to renormalize. There is an easy way to see this. A term going as $p^0$ simply contributes a $\delta(x)$ in position space. Even polynomials in $p$ therefore contribute terms of the form $\Box^n \delta(x)$. All of these need to be regulated and subtracted.
	
	We thus recognize the pattern and get
	\[
		\braket{\phi(k) \phi(q)} = - M^2 \sdelta^4 (k + q) \log \left( \frac{p^2 \epsilon^2}{2} \right) \frac{\sin(\pi \nu)}{\Gamma(\nu)^2} \left(\frac p2\right)^{2 \nu} = -\frac{M^2}{2} (2 \Delta - 4) \frac{\Gamma(3- \Delta)}{\Gamma(\Delta - 1)} \sdelta^4(p + q) \left(\frac p2\right)^{2 \Delta - 4}
	\]
	Up to a factor of $2$ this is consistent with 3.40 of MAGOO. My argument at the moment works only for integral $\nu$, but based off of remarks that I have read, this is what should be expected. 
	
	Integrating this is not hard if you know a trick:
	\[
		\frac{2 \pi^2}{(2\pi)^4} \int_0^\infty dp p^{2 \nu + 3} \frac{e^{i p |x|}}{p|x|} \log p \to \frac{1}{8\pi^2} \int_0^\infty dp p^{2 \nu + \epsilon} e^{i p x} 
	\]
	And look at the $O(\epsilon)$ part of this expansion. 
	
	Altogether this gives:
	\[
		\braket{\O(x) \O(y)} \approx \frac{1}{|x-y|^{2 \Delta}}
	\]
	Where I am not sure about the constant, but am sure about the $x$-scaling. % The $x$-scaling can be seen right away from dimensional analysis.
	
	\item In this problem I will freely exchange $p+1$ and $d$ whenever suitable. The dual current is constrained to have scaling dimension $p$. Upon choosing a gauge:
	\[
		A_u(u, x) = 0, \qquad \nabla^\mu A_\mu = 0
	\]
	We therefore have that $A$ satisfies the differential equation:
	\[
		0 = \nabla^M F_{MN} = \d_u \left(\frac{1}{u^{p+2}} u^4 u^{-1} \d_u A_\nu \right) + \frac{1}{u^{p+2}} u^4 u^{-1} \d_\mu \d_{[\mu} A_{\nu]}
	\]
	You will notice that there is an extra factor of $u$ accompanying $A$ in this PDE. This should be viewed as turning $\mu$ into a Vielbein index, so that $dx^\mu$ has trivial scaling properties. 
	
	The full solution $A_\mu (u, x)$ is then given by:
	\[
		u^{p/2} \int \frac{d^{p+1} p}{(2\pi)^{p+1}} a_\mu (p) e^{i p x} K_{p/2}(p u), \quad p^\mu a_\mu(p) = 0.
	\]
	
	
	It will be nicer to write the bulk-to-boundary propagator in position space. For this, I'll follow Witten's argument by putting the $\delta$ function source at $u = \infty$. This simplifies things since the solution must be independent of the $x$ variables. Take the solution $A= f(u) dx^i$. This gives the equations of motion for a free field:
	\[
		\dd \star \dd A = \d_u \frac{1}{u^{p+2}} u^4 f'(u) = 0 \Rightarrow f(u) = u^\Delta, \quad \Delta = d-2
	\]
	Doing the inversion again we have:
	\[
		u^{d-2} \dd x^i \to \frac{u^{d-2}}{(u^2 + |x|^2)^{d-2}} \dd\left(\frac{x^i}{u^2 + |x|^2} \right)=  \frac{u^{d-2}}{(u^2 + |x|^2)^{d}} (u^2 - x_i^2) dx^i - \frac{2 x_i u^{d-1}}{(u^2 + |x|^2)^{d-2}} du
	\]
	adding the exact term:
	\[
		- \frac{1}{d-1} \dd \left[\frac{x_i}{u} \left(\frac{u}{u^2 + |x|^2} \right)^{d-1} \right]
	\]
	and rescaling by an overall factor of $\frac{d-1}{d-2}$ yields:
	\begin{equation}\label{eq:vectorprop}
		G_{\mu i}(u, x; 0) = \frac{u^{d-2}}{(u^2 + |x|^2)^{d-1}} \left(dx^i - x^i \frac{du}{u} \right)
	\end{equation}
	This does not satisfy Lorenz gauge, but as expected can be brought to satisfy it by adding appropriate pure gauge terms
	\[
		G_{\mu i}(u, x; x') \to G_{\mu i}(u, x; x') + \d_\mu \Lambda(u, x; x')
	\]
	Freedman et al have a slightly different propagator, which can be obtained from this one by gauge transform. It takes the form:
	\[
		G_{\mu i} = \frac{ \Gamma(d)}{2\pi^{\frac d2} \Gamma(\frac d2)} \frac{u^{d-2}}{(u^2 + |x|^2)^{d-1}} I_{\mu i}(x) 
		= \frac{ \Gamma(d)}{2\pi^{\frac d2} \Gamma(\frac d2)} \frac{u^{d-2}}{(u^2 + |x|^2)^{d-1}} \left(\delta_{\mu i} - 2 \frac{x_\mu x_i}{x^2} \right)
	\]
	This propagator can be seen to naturally come from the embedding space formalism. 
	
	Upon integration by parts, the action leads to only a boundary term:
	\[
		\frac{1}{4 g^2} \int F_{\mu \nu} F^{\mu \nu} \to \frac{1}{2 g^2} \int A \wedge \star F
	\]
	The nonvanishing components will come from the parts of $\star F$ that do not involve a $du$. Consequently, we only need to calculate the $du$ parts of $F$. 
	Said equivalently, the on-shell action is:
		\begin{equation}\label{eq:vectaction}
			S_{on-shell} = \frac{1}{2g^2} \int dx_1 dx_2 J_i(x_1) J_j(x_2) \int \frac{du d^d z}{u^{d+1}} G_{\nu i} u^{4} \d_{[0} G_{\nu] j} 
		\end{equation}
	Computing this is straightforward, using \eqref{eq:vectorprop}:
	\begin{equation}\label{eq:fieldstrengthprop}
		\begin{aligned}
			\d_{[\mu} G_{\nu] i} &= (d-2) \frac{u^{d-3}}{(u^2 + |x|^2)^{d-1}} du \wedge dx^i - \frac{u^{d-3}}{(u^2 + |x|^2)^{d-1}} dx^i \wedge du \\
			& \quad - 2 (d-1) \frac{u^{d-1}}{(u^2 + |x|^2)^d} du \wedge dx^i + 2 (d-1) \frac{u^{d-3} x^i x^j}{(u^2 + |x|^2)^{d}} dx^j \wedge du\\
			&= (d - 1) \frac{u^{d-3}}{(u^2 + |x|^2)^{d-1}} du \wedge dx^i - 2 (d-1) \frac{u^{d-1}}{(u^2 + |x|^2)^d} du \wedge dx^i - 2 (d-1) \frac{u^{d-3} x^i x^j}{(u^2 + |x|^2)^{d}} du \wedge dx^j
		\end{aligned}
	\end{equation}
	% Taking the hodge star of this gives an extra factor of $u^{-d-1}$ wedged with $A$.
	For the $u\to 0$ limit, recall the scalar propagator for a field with dimension $d-1$ would take the form:
	\[
		\frac{\Gamma(d-1)}{\pi^{\frac d2} \Gamma(\frac d2-1)} \frac{u^{d-1}}{(u^2 + |x-x'|^2)^{d-1}} = \frac{ \Gamma(d)}{2\pi^{\frac d2} \Gamma(\frac d2)} \frac{u^{d-1}}{(u^2 + |x-x'|^2)^{d-1}} \to u \delta(x - x') dx^i + u^{d-1} \frac{ \Gamma(d)}{2\pi^{\frac d2} \Gamma(\frac d2)} \frac{1}{|x-x'|^{2d -2}} dx^i
	\]
	So that $G_{\mu i}(u, x; x')$ will (up to a constant) approach $\delta(x- x') dx^i$ in the $u \to 0$ limit. On the other hand, \eqref{eq:fieldstrengthprop} will approach (up to a constant):
	\[
		\d_{[0} G_{i] j} = u^{d-3} \frac{1}{|x_{12}|^{2d-2}} \left(\delta_{ij} - 2 \frac{x_{12}^i x^{j}_{12}}{|x_{12}|^2} \right)
	\]
	The leading $u^{d-3}$ cancels exactly against the metric-dependent terms in the integral \eqref{eq:vectaction}. The final result is proportional to:
	\[
		 % \frac{\Gamma(d)}{2 \pi^{\frac d2} \Gamma(\frac d2)}
		 \frac{1}{g^2}\int dx_1 dx_2 \, J_i(x_1) J_j(x_2)  \left(\frac{\delta^{ij}}{|x_{12}|^{2d-2}} - 2 \frac{x^i_{12} x^j_{12}}{|x_{12}|^{2d}} \right)
	\]
	Giving a two-point correlator (after rescaling):
	\[
		\braket{J_i(x_1) J_j(x_2)} = \frac{1}{|x_{12}|^{2d-2}} I_{ij}(x_{12}), \quad I_{ij}(x) = \delta_{ij} - 2 \hat x_i \hat x_j
	\] 

	
	\item I think this is pretty direct. If a field $\phi_0$ diverges as $\epsilon^{\Delta}$ in the IR, it must couple with an operator $\mathcal O$ of scaling dimension $d-\Delta$ in order for the interaction term $\int \phi_0 \mathcal O$ to be conformally invariant. This situation is generic
	
	\textbf{What more does this question ask for?}
	
	\item In what follows, there are many variables and the story becomes rapidly confusing if one does not understand what everything stands for. I will review this. My conventions will mostly by those of Kiritsis, although occasionally I will adopt notation from de Haro, Skenderis, and Solodukin arXiv:000223
	\begin{center}
		\begin{tabular}{c | c}
			Symbol & Definition\\
			\hline
			$g(u, x)$ & Full AdS5 Metric\\
			$R_{\mu \nu}$ & Ricci Curvature of $g$\\
			$g^{(2n)}$ & \parbox{4in}{\centering The coefficient of $u^{2n}$ in the expansion of $g$ about $u = 0$. \newline
			Note that $g^{(4)}$ is undetermined by $g_0$. Its trace and divergence is however determined.} \\
			$h^{(4)}$ & The coefficient of $u^4 \log u^2$\\
			$\rho = u^2$ & Alternative coordinate, usually easier to work with.\\ 
			$R_{ij}, R$ & Ricci curvatures of $g^{(0)}_{ij}$ \emph{only}\\
			$t_{ij}$ & Undetermined integration constant in  $g^{(4)}$\\
			$\gamma(x) = \frac{L^2}{\epsilon^2} g_{ij}(\epsilon, x)$ & Induced metric on the renormalization hypersurface $u = \epsilon$\\
			$\braket{T_{ij}}$ & Stress tensor of varying renormalized action w.r.t. $g^{(0)}$\\
			$T_{ij}[\gamma]$ & Stress tensor w.r.t. metric on renormalized hypersurface\\
			$T^{\mathcal A}_{ij}$ & Stress tensor of varying anomaly $\mathcal A$ w.r.t. $g^{(0)}$.
		\end{tabular}
	\end{center}
	
	 The theorem of Fefferman and Graham states that a general asymptotically-AdS metric can be written as
	\[
		\frac{ds^2}{L^2} = \frac{du^2}{u^2} + \frac{g_{ij}}{u^2} dx^i dx^j, \quad g_{ij}(u^2, x) = g^{(0)}(x) + u^2 g^{(2)})(x) + \dots
	\]
	Einstein's equations in this setting are:
	\[
		R_{\mu \nu} - \frac12 g_{\mu \nu} R = \frac{6}{L^2} g_{\mu \nu}
	\]
	It will be rather annoying to derive how this looks like in terms of $g$ and its derivatives. Thankfully, Henningston and Skenderis already have this formula in equation 6 of their paper, and I am happy to quote it directly. Take $\rho = u^2$. Denote differentiation with respect to $\rho$ by $g'$. Then we get 
	\begin{equation}\label{eq:fefferman-graham}
		\rho [2 g'' - 2 g' g^{-1} g' + \Tr(g') g']_{ij} + R_{ij} - (d-2) g'_{ij} - \Tr(g') g_{ij} = 0
	\end{equation}
	Here all traces mean $\Tr(X) = g^{ab} X_{ab}$. 
	At leading order in $\rho$, the first term is set to zero and we get for $d=4$ at $u = 0$:
	\[
		 g^{(2)}_{ij} + \frac12 g_{ij} g^{ab} g^{(2)}_{ab}  = \frac12 R_{ij}
	\]
	Taking the ansatz:
	\[
		g^{(2)}_{ij}  = \alpha R_{ij} + \beta R g_{ij} \Rightarrow g^{ab} g'_{ab} = (\alpha + 4 \beta) R \Rightarrow \beta +\frac12 (\alpha  + 4 \beta) = 0 \Rightarrow \beta = - \frac16 \alpha
	\]
	We see that to get the Ricci tensor to match we need $\alpha = 1/2$. Thus we get the solution
	\[
		g^{(2)}_{ij}  = \frac12 R_{ij} - \frac{1}{12} R g_{ij}
	\]
	as required. To next order, we have:
	\begin{equation}\label{eq:hard}
		\begin{aligned}
			\cancel{4 g^{(4)}_{ij}} + 6 h^{(4)}_{ij} & - 2 (g^{(2)} g^{-1} g^{(2)})_{ij} + \cancel{\Tr(g^{(2)}) g^{(2)}_{ij}} \\
			& + R_{ij}^{(2)} - \cancel{4 g^{(4)}_{ij}} - 2 h_{ij}^{(4)} - 2 \Tr[g^{(4)}] g_{ij} - \Tr[h^{(4)}] g_{ij}  - \cancel{\Tr[g^{(2)}] g_{ij}^{(2)}} + \Tr[g^{(2)} g^{(2)}] g_{ij} = 0\\
			&\Rightarrow -4 (h^{(4)}_{ij} + g_{ij} \frac14 \Tr h^{(4)}) = - 2 g^{(2)} g^{-1} g^{(2)}_{ij} - 2\Tr[g^{(4)}] g_{ij}^{(0)} + R_{ij}^{(2)}  + \Tr[g^{(2)} g^{(2)}] g_{ij}
		\end{aligned}
	\end{equation}
	Note that $g^{(4)}$ has canceled. This is generic. In $d$ dimensions $g^{(d)}$ will cancel. This is a reflection of the fact that there are generally two solutions to the Einstein equations. We cannot determine $g^{(4)}$ uniquely without an additional constraint. We can still trace over both sides and get a relation between traces. Alternatively, this comes from the $R_{rr}$ part of the Einstein equations:
	\[
		\Tr[g^{(4)}] =  \frac14 \Tr[g^{(2)} g^{(2)}].  % + R^{(2)}
	\]
	
	The Einstein equations for $R_{i\rho}$ give the further constraint that
	\[
		0 = \nabla_i (g^{ab} g'_{ab}) - g^{ab} \nabla_b g_{ia}' \Rightarrow \nabla_i \Tr g' = \nabla^j g'_{ij}
	\]
	
	\textbf{I'm missing how to actually get $g^{(4)}$ in order to get agreement with Kiritsis and 0002230. Thankfully, explicit calculation of $g^{(4)}$ is not necessary for anomaly analysis.}
	
	This gives the final desired result:
	\[
		g^{(4)}_{ij} = \frac18 g^{(0)}_{ij} \left[(\Tr g^{(2)})^2 - \Tr[(g^{(2)})^2]\right] + \frac12 (g^{(2)})^2_{ij} - \frac14 g_{ij}^{(2)} \Tr(g^{(2)}) + t_{ij}
	\]
	Consistency of divergence and trace requires:
	\[
		\nabla^i t_{ij} = 0, \quad t^i_i = -\frac14 \left[(\Tr g^{(2)})^2 - \Tr[(g^{(2)})^2]\right]
	\]
	Now, revisiting \eqref{eq:hard}, the trace conditions $\Tr g^{(4)} = \Tr[ (g^{(2)})^2]$ and $\Tr h^{(4)} =0$ simplify it to:
	\[
	\begin{aligned}
		h_{ij}^{(4)} &= \frac12 (g^{(2)})^2 + \frac18 \Tr[(g^{(2)})^2] g^{(0)}_{ij} - \frac14 R_{ij}^{(2)}\\
		& = \frac12 (g^{(2)})^2 + \frac18 \Tr[(g^{(2)})^2] g^{(0)}_{ij} + \frac18 (\nabla^k \nabla_{i} g_{j k}^{(2)} + \nabla^k \nabla_{j} g_{i k}^{(2)}  - \nabla^2 g^{(2)} -\nabla_i \nabla_j \Tr g^{(2)})
	\end{aligned}
	\]
	where we have used identities for the variation of the Ricci tensor from appendix \textbf{C.2}. This can be written as:
	\[
	\begin{aligned}
h^{(4)}&= \frac18 R_{ijkl} R^{kl} + \frac{1}{48} \nabla_i \nabla_j R - \frac{1}{12} \nabla^2 R_{ij} - \frac{1}{24} R R_{ij} + \frac{1}{32} \left(\frac13 \nabla^2 R + \frac13 R^2 - R_{kl} R^{kl} \right) g^{(0)}_{ij}\\
&= - \frac{1}{\sqrt{g^{0}}} \frac{\delta}{\delta (g^{(0)})^{ij}} \left(- \frac18 \left[R_{kl} R^{kl} - \frac13 R^2 \right]\right)
	\end{aligned}
	\]

	
	As a bonus, for $d=2$, Equation~\eqref{eq:fefferman-graham} gives:
	\[
		R_{ij} = \Tr(g^{-1} g') g_{ij} \Rightarrow \frac12 R = \Tr(g^{-1} g') % \Rightarrow g^{(2)}_{ij} = \frac{1}{4} R g^{(0)}_{ij}
	\]
	Together with $\nabla^i g_{ij} = \nabla^i g_{ij} \Tr g'$ we get 
	\[
		g^{(2)}_{ij} = \frac12 (R g^{(0)}_{ij} + t_{ij})
	\]
	with $t_{ij}$ divergence-free and $t^i_i = - R$
	We then get:
	\[
		\left< T_{ij} \right> = \frac{L}{16 \pi G_{N}} = - \frac{c}{24 \pi} R
	\]
	This is the 2D Weyl anomaly. 
	
	\item Take:
	\begin{equation}\label{eq:gravonshell}
				\frac{L^3}{2 \pi G_5} \int d^4 x\left( \int_{\epsilon} \frac{du}{u^5} \sqrt{\det g(u, x)} - \frac{1}{u^4} (1 - u \d_u) \sqrt{g(u, x)}|_{u = \epsilon} \right)
	\end{equation}
	It will be useful to recall:
	\[
	\begin{aligned}
				\sqrt{-\det(g+h)}& = \exp \frac12 \log (-\det(g+h))\\& = \sqrt{\deg g} \exp\left[ \frac12 \tr \log(1 + g^{-1} h) \right] \\&= \sqrt{\deg g} \exp\left[ \frac12 \tr [g^{-1} h - \frac12 (g^{-1} h)^2] \right]\\& = \sqrt{\deg g}\left(1 + \frac12 \tr (g^{-1} h) - \frac14 \tr[(g^{-1} h)^2] + \frac18 \tr[g^{-1} h]^2 \right)
	\end{aligned}
	\]
	This implies
	\[
		\sqrt{g(u, x)} = \sqrt{g^{(0)}} \left(1 + \frac12 u^2 \Tr[g^{(2)}] + \frac12 u^4 \Tr[g^{(4)}] + \cancel{ \frac12 u^4 \log u^2 \Tr[h^{(4)}]} + \frac18 u^4 \Tr[g^{(2)}]^2  - \frac14 u^4 \Tr[(g^{(2)})^2 ] \right)
	\]
	Again, indices are raised and lowered with $g^{(0)}$.
	
	At zeroth order we get:
	\[
		\frac{L^3}{8 \pi G_5} \frac{1}{\epsilon^4} - \frac{L^3}{2 \pi G_5 \epsilon^4} = -6 \frac{L^3}{16 \pi G_5} \frac{1}{\epsilon^4} \Rightarrow A_0 = -6
	\]
	At second order we get:
	\[
		\frac{L^3}{4 \pi G_5} \frac12  \frac{1}{\epsilon^2} \Tr[ g^{(2)}] - \frac{L^3}{2 \pi G_5} (1 - \frac12 ) \frac12 u^2 \Tr[g^{(2)}] = 0 \Rightarrow A_2 = 0
	\]
	At fourth order only the first term of \eqref{eq:gravonshell} contributes and we get:
	\[
	\begin{aligned}
		-\frac{L^3}{4 \pi G_5} \log \epsilon^2 \Big( \frac12 \Tr[g^{(4)}] + & \frac18 \Tr[g^{2}]^2 - \frac14 \Tr[(g^{(2)})^2] \Big) = - \frac{L^3}{16 \pi G_5} \log \epsilon^2 \frac12 \left(\Tr[g^{2}]^2  - \Tr[(g^{2})^2]\right)\\
		 \Rightarrow A_4 &= \frac12 \left(\Tr[g^{2}]^2  - \Tr[(g^{2})^2]\right) = \mathcal A
	\end{aligned}
	\]
	\item 
	In order to get the correct counterterms, we need to solve for $\Tr g^{(2)}, \Tr [(g^{(2)})^2]$ in terms of the induced metric \emph{on the renormalization surface}, that is at $u = \epsilon$. I will call this $\gamma$  (Kiritsis calls it $h$) as as not to be confused with $h^{(4)}!$. We have $\gamma = \frac{L^2}{\epsilon^2} g_{ij}$. This gives:
	\[
		\sqrt{g^{(0)}} = \frac{\epsilon^4}{L^4} \sqrt{\gamma} \left(1 - \frac12 \frac{\epsilon^2}{L^2} \Tr g^{(2)} + \frac18 \frac{\epsilon^4}{L^4} \left(-2 \Tr [(g^{(2)})^2]  + \Tr[ g^{(2)}]^2 \right) \right)
	\]
	Next, recall:
	\[
		g^{(2)}_{ij} = \frac12 R_{ij} - \frac{1}{12} R g_{ij}^{(0)}
	\]
	where $R_{ij}$ is taken w.r.t. $g^{(0)}$
	This gives:
	\[
		\Tr g^{(2)} = \frac{1}{6} R[g^{(0)}] = \frac{1}{6} \left[R[g_{ij}] - g^{(2)}_{ij} \frac{\delta R[g^{(0)}]}{\delta g^{(0)}_{ij}}  \right] = \frac16 \frac{L^2}{\epsilon^2} \left(R[\gamma] + \frac{1}{2}\left[R^{ij}[\gamma] R_{ij}[\gamma] - \frac{1}{6} R[\gamma]^2 \right] \right)
	\]
	Finally, to leading order:
	\[
		\Tr[ (g^{(2)})^2] = \frac{L^4}{\epsilon^4} \frac12 (R_{ij}[\gamma] - \frac{1}{6} R[\gamma]\, \gamma_{ij}) \frac12 (R^{ij}[\gamma] - \frac{1}{6} R[\gamma]\, \gamma^{ij}) = \frac{L^4}{\epsilon^4} \frac14 ( R_{ij}[\gamma] R^{ij}[\gamma] - \frac{2}{9} R[\gamma]^2)
	\]
	As a check, we see (up to quadratic order in the curvature):
	\[
		A_4 = \frac12 [\Tr[g^{(2)}]^2 - \Tr[(g^{(2)})^2]] = \frac{L^4}{\epsilon^4} \frac12 \left( \frac{1}{36} R[\gamma]^2 - \frac14 ( R_{ij}[\gamma] R^{ij}[\gamma] - \frac{2}{9} R[\gamma]^2) \right) = -\frac{L^4}{\epsilon^4 }\frac{1}{8}\left(R_{ij}[\gamma] R^{ij}[\gamma] - \frac13 R[\gamma]^2 \right)
	\]
	
	All together we get
	\[
	\begin{aligned}
		&-\frac{L^3}{16 \pi G_5} \int d^4 x \sqrt{g^{(0)}} \left[-\frac{6}{\epsilon^4} - \log \epsilon^2 \, A_4\right]\\
		& = \frac{L^3}{16 \pi G_5} \int d^4 x \sqrt{\gamma} \left[6 + \frac{6}{2} \frac{\epsilon^2}{L^2} \Tr g^{(2)}  + \Tr[ g^{(2)}]^2  + \log \epsilon^2 \, A_4 + \dots \right]\\
		&=  \frac{1}{16 \pi G_5} \int d^4 x \sqrt{\gamma} \left[\frac{6}{L} - \frac{L}{2} R[\gamma] - \frac{L^3}{8} \log \epsilon^2 \left(R_{ij}[\gamma] R^{ij}[\gamma] - \frac13 R[\gamma]^2 \right) \right]
	\end{aligned}
	\]
	Where the $\dots$ denotes ``up to finite terms''. \textbf{I think Kiritsis has a mistake and it should be $-\frac{L}{2} R$ not $+$. 0002230 confirms this.}

	\item Let's review first. The stress tensor comes from varying the renormalized action by the initial source field $g^{(0)}$. This is given by:
	\[
		\braket{T_{ij}} = \lim_{\epsilon \to 0} \frac{2}{\sqrt {g_{ij}(\epsilon, x)}} \frac{\delta}{\delta g^{ij}(\epsilon, x)} S_{ren}[\gamma] = \frac{L^2}{\epsilon^2} T_{ij}[\gamma]
	\]
	Thus, we can look at variations with respect to the induced metric $\gamma$ at $u = \epsilon$ and take $\epsilon \to 0$ at the very end. 
	
	There are two variations to consider. The first is the variation of the on-shell effective action. Generalizing the argument giving \textbf{14.8.35} to a $u$-dependent metric, we get:
	\[
		T^{sugra}_{ij} = -\frac{1}{8 \pi G_5} (K_{ij} - K \gamma_{ij}) = -\frac{L^3}{8 \pi G_5} \left(-\d_\epsilon g_{ij}(\epsilon, x) + g_{ij}(\epsilon, x) \Tr[g^{-1}(\epsilon, x) \d_\epsilon g(\epsilon, x)] - \frac{3}{\epsilon^2} g_{ij}(\epsilon, x) \right)
	\]
	Varying the counterterm action with respect to $\gamma$ gives directly:
	\[
		T^{c}_{ij} = - \frac{1}{8 \pi G_5} \left[\frac{3}{L} \gamma_{ij} - \frac{L}{2} \left(R_{ij} - \frac12 R \gamma_{ij} \right) - \frac{L^3}{2} \log{\epsilon^2} T^{\mathcal A}_{ij} \right]
	\]
	The next step is to write these in terms of the $g^{(2n)}, h^{(4)}$.
	
	It is useful to note:
	\[
	\begin{aligned}
		R_{ij}[\gamma] &= R_{ij} [g_0] + \frac14 \frac{\epsilon^2}{L^2} \left(R_{ik} R^{k}_{\, j} - 2 R_{ikjl}  R^{kl} - \frac13 \nabla_i \nabla_j R + \nabla^2 R_{ij} - \frac16 \nabla^2 R g_{ij}^{(0)}\right).\\
		\Rightarrow R[\gamma] &= R[g_0] - \frac14 \frac{\epsilon^2}{L^2} R_{ij} R^{ij}
	\end{aligned}
	\]
	where the curvatures on the RHS are those of the metric $g^{(0)}$. Now we get:
	\[
	\begin{aligned}
		\braket{T_{ij}} &= \frac{L^2}{\epsilon^2} [T^{sugra}_{ij} + T^c_{ij}] \\
		&= -\frac{L^3}{8 \pi G_5} \Bigg[\frac{L^2}{\epsilon^2} (-g^{(2)}_{ij} + g^{(0)} \Tr g^{(2)}_{ij} + \frac12 R_{ij} - \frac14 R g_{ij}^{(0)}) - \log \epsilon^2 \left(2 h^{(4)} +  T^{\mathcal A}_{ij} \right)\\
		&\qquad \qquad -2 g^{(4)} - h^{(4)} - g_{ij}^{(2)} \Tr g^{(2)} - \frac12 g^{(0)}_{ij} \Tr g^2
		\\
		& \quad - \frac{L}{8} \cancel{\frac{\epsilon^2}{L^2}} \left(2 R_{ik} R^{k}_{\, j} - 2 R_{ikjl}  R^{kl} - \frac13 \nabla_i \nabla_j R + \nabla^2 R_{ij} - \frac16 \nabla^2 R g_{ij}^{(0)}\right) - \frac{L}{4} R g^{(2)}_{ij} + \frac{L}{8} g^{(0)}_{ij}  R_{ij} R^{ij} - \frac16 \nabla^2 R g^{(0)}_{ij}
		 \Bigg]
	\end{aligned}
	\]
	We see that the $\frac{L^2}{\epsilon^2}$ and $\log \epsilon^2$ terms vanish by \textbf{14.8.39, 14.8.43}. The curvature terms combine to give yet another two copies of $-h^{(4)})_{ij}$, and the remaining terms cancel everything but $t_{ij}$ from the $g^{(4)}_{ij}$ to give:
	\[
		\braket{T_{ij}} = \frac{L^3}{8 \pi G_5}[2 t_{ij} + 3 h^{(4)}_{ij}]
	\]
	
	
	\item In embedding space, it is easy, using L.10 to calculate:
	\[
		\begin{aligned}
			2\sigma := \eta_{MN} (\xi - \xi')^M (\xi - \xi')^N &= -(\Delta X^0)^2 + (\Delta X^{p+1})^2 + (\Delta X^i)^2 \\
			&= -\left(\tfrac{u-u'}{2} + \tfrac{L^2 + x^2}{2u} - \tfrac{L^2 + {x'}^2}{2u'}\right)^2 
			+ \left(\tfrac{u-u'}{2} -  \tfrac{L^2 + x^2}{2u} + \tfrac{L^2 + {x'}^2}{2u'}\right)^2 + \left(\tfrac{L x}{2u} - \tfrac{L x'}{2u'} \right)^2\\
			&= - (u-u')\left(\frac{L^2}{2u} - \frac{L^2}{2u'} \right) - \left(\frac{L^2}{u} - \frac{L^2}{u'} \right) \left(\frac{x^2}{u} - \frac{{x'}^2}{u'} \right) + L^2 \left(\frac{x}{u} - \frac{x'}{u'} \right)^2\\
			&= L^2 \frac{(u-u')^2 + (x-x')^2}{u u'}
		\end{aligned}
	\]
	Then $\eta^2 = 1+\sigma$. Now let us calculate:
	\[
		-iG_F = \braket{0| T \phi(x) \phi(x')| 0} =  \Theta(x-x')\sum_{n, \ell} \phi_{\Delta_+ n \ell}(x) \phi_{\Delta_+ n \ell}(x') + (x \leftrightarrow x').
	\]
	The modes of a scalar field are given by:
	\[
	\begin{aligned}
		\phi_{\Delta, \ell, n} &=  N_{\Delta, n, \ell} \, \sin^\ell \theta\, \cos^{\Delta_\pm} \theta\, {_2F_1}(a, b, c; \sin^2 \theta) \, Y_{\ell}(\Omega_p) (\Omega_p),\\ \quad a = \frac12 (\ell + \Delta_{\pm}& - \omega L), \quad b = \frac12 (\ell + \Delta_{\pm} + \omega L),\quad c = \ell+\frac{p+1}{2}, \quad \omega L = \Delta_\pm + \ell + 2n
	\end{aligned}
	\] 
	Because of the homogeneity of AdS, pick the AdS coordinate origin at $x'$. Then $\phi_{\Delta_+, \ell, n}(x') = 0$ for $\ell \neq 0$. This reduces the sum to:
	\[
		\frac{\Gamma(\frac{p+1}{2})}{2 \pi^{\frac{p+1}{2}}} e^{-i \Delta_+}\sum_{n=0}^\infty N_{\Delta, n, \ell}^2 \phi_{\Delta, 0, n}(x) \phi_{\Delta, 0, n}(0) = \frac{e^{-i \Delta t} \cos^{\Delta} \theta}{2 \pi^{\frac{p+1}{2}} L^p} \sum_{n=0}^\infty \frac{(\frac{p+1}{2})_n \Gamma(\Delta + n)}{n! \Gamma(\Delta + n + \frac{p-1}{2})} {_2F_1}\left(\begin{matrix}
			-n \quad  \Delta + n\\ \frac12(p+1)
		\end{matrix}; \sin^2 \theta \right) e^{-2 i n |t|}
	\]
	I have started using (ascending) Pochammer symbols. 
	Writing this in terms of a Jacobi poylnomial:
	\[
		\sum_{n=0}^\infty \frac{\Gamma(\Delta + n)}{\Gamma(\Delta + n + \frac{p-1}{2})} P^{\frac{p-1}{2}, \Delta - \frac{p+1}{2}}_n (\cos 2 \theta) e^{-2 i n |t|}
	\]
	Using a Jacobi polynomial identity from 45.1.4 ``A table of series and products'' by Eldon R. Hansen we get the full greens function to be 
	\[
		\frac{\Gamma(\Delta) e^{-i \Delta t} \cos^\Delta \theta}{2 \pi^{\frac{p+1}{2}} \Gamma(\Delta + \frac{1-p}{2}) L^p} (1 + e^{-2 i |t|})^{-\Delta} {_2F_1}\left(\begin{matrix}
			\frac{\Delta}{2} \quad \frac{\Delta + 1}{2}\\ \Delta + \frac{1-p}{2}
		\end{matrix}; \frac{\cos^2 \theta}{\cos^2 t}\right) = \frac{\Gamma(\Delta)}{2^{\Delta + 1} \pi^{\frac{p+1}{2}} \Gamma(\Delta + \frac{1-p}{2}) L^p } \eta^{-2 \Delta} {_2F_1}\left(\begin{matrix}
			\frac{\Delta}{2} \quad \frac{\Delta + 1}{2}\\ \Delta + \frac{1-p}{2}
		\end{matrix}; \frac{1}{\eta^4}\right) 
	\]
	And $\eta^2 = \frac{\cos t}{\cos \theta}$ is the geodesic distance. Up to a minus sign this is correct. \textbf{I think Kiritsis means $2^{\Delta+1}$ in the denominator}. Either $\Delta = \Delta_+$ or $\Delta=\Delta_-$ works. 
	% In the above expression, as we take $x \to x'$ we get
% 	\[
% 		\eta^2 = 1 + \frac{(u-u')^2}{2 u u'}
% 	\]
% \textbf{Important to appreciate the subtleties of contrasting this to the euclidean continuation of the sphere propagator. C.f. Burgess + Lutken.}
	
	
	\item Upon taking $u \to 0$, $\eta^{-4}$ goes to zero and the ${_2F_1} \to 1$. What remains is (I'm including Kiritsis' minus sign):
	\[
		-\frac{\Gamma(\Delta)}{2^{\Delta+1} \pi^{\frac{p+1}{2}} \Gamma(\Delta + \frac{1-p}{2})} (2u)^{\Delta} \left(\frac{u'}{{u'}^2 + x^2} \right)^\Delta = -\frac{1}{2 (\Delta - \frac{p + 1}{2})} \frac{u^\Delta \Gamma(\Delta)}{ \pi^{\frac{p+1}{2}} \Gamma(\Delta - \frac{1+p}{2})} \left(\frac{u'}{{u'}^2 + x^2} \right)^\Delta 
	\]
	This is the negative of what is in Klebanov and Witten. 
	\[
		G_{\Delta}(u, x; u', x') t\to \frac{u^\Delta}{p+1 - 2 \Delta} K_\Delta(x; u', x')
	\]
	Take a source field $\phi(u,x)$ in the bulk. As $u \to 0$ away from the source we have $\phi(u, x) \to u^{\Delta} A(x)$.
	
	The expectation value $\braket{\O(x)}$ is given by contracting the bulk source with a bulk-to-boundary propagator going to $x$. On the other hand, contracting with a bulk-to-bulk propagator and taking $u$ close to $0$ gives $u^{\Delta} A(x)$. Thus, we see:
	\[
		u^{\Delta} A(x) = \frac{u^{\Delta}}{p+1-2\Delta} \braket{\O(x)} \Rightarrow A(x) = \frac{1}{p+1-2 \Delta} \braket{\O(x)}
	\]
	
	I think the bulk-to-bulk greens function should have a minus sign from how Kiritsis defined it.
	
	\item 
	
	\item This is a difficult problem in general. Hong Liu has a method of doing this using conformal partial waves. 
	
\end{enumerate}

% section chapter_14_the_bulk_boundary_holographic_correspondence (end)

\end{document}
	