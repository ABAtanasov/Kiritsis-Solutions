\documentclass[11pt, class=article, crop=false]{standalone}
\usepackage{amsmath,amssymb,amsfonts,amsthm}
\usepackage{enumitem}
\usepackage{fancyhdr}
\usepackage{tikz-cd}
\usepackage{mathabx}
\usepackage{geometry}
\usepackage{natbib}
\usepackage{braket}
\usepackage{graphicx}
\usepackage{simpler-wick}
\usepackage{hyperref}
\usepackage{cancel}
\usepackage{listings}
\usepackage{relsize}
\usepackage{xcolor}
\usepackage{stmaryrd}
\usepackage{tikz-feynman}
\usepackage{kiritsis}
\geometry{margin = 0.5in}


\begin{document}
\section*{Chapter 7: Superstrings and Supersymmetry} % (fold)
\label{sec:chapter_7_superstrings_and_supersymmetry}
\begin{enumerate}
	\item We already know that $T T$ will have the desired OPE, since the bosons and fermions are uncoupled and we already have shown their own respective stress tensor OPEs. Next
	\[
	\begin{aligned}
		G(z) G(w) &= - \frac{2}{\ell_s^4} \psi_\mu(z) \d X^\mu(z) \psi_\nu(w) \d X^\nu (w)\\ 
		&= - \frac{2}{\ell_s^4} \left(\ell_s^2 \frac{\eta_{\mu \nu}}{z-w} + (z-w) :\d \psi_\mu \psi_\nu(w): \right) \left(-\frac{\ell_s^2}{2} \frac{\eta_{\mu \nu}}{(z-w)^2} + :\d X_\mu \d X_\nu (w): \right)\\
		&= \frac{D}{(z-w)^3} + \frac{-\frac{2}{\ell_s^2} \d X_\mu \d X^\mu (w) - \frac{1}{\ell_s^2} \psi^\mu \d \psi_\mu (w)}{z-w} \\
		&= \frac{\hat c}{(z-w)^3} + \frac{2 T(w)}{z-w}
	\end{aligned}
	\]
	Finally
	\[
		\begin{aligned}
			T(z) G(w) &= - \frac{1}{\ell_s^2} \, \left(:\d X_\mu \d X^\mu (z): + \frac12 \psi^\mu \d \psi_\mu (z)\right) i \frac{\sqrt 2}{\ell_s^2} \psi_\nu \d X^\nu(w)\\
			&= -i \frac{\sqrt 2}{\ell_s^4} \left(-\frac{\ell_s^2}{2} \frac{ \psi_\mu \d X^\mu (w) + \psi_\mu \d^2 X^\mu(w) (z-w)}{(z-w)^2} - \frac{\ell_s^2}{2} \frac{ \psi_\mu \d X^\mu (w)}{(z-w)^2} + (-) \frac{\ell_s^2}{2} \frac{\d_\mu \psi \d X^\mu(w)}{(z-w)} \right)\\
			&= \frac32 \frac{G(w)}{(z-w)^2} + \frac{\d G(w)}{z-w}
		\end{aligned}
	\]
	
	\item We will take the OPE of $j_B(z) j_B(w)$, but just look at the $(z-w)^{-1}$ term as a function of $w$, as this, when integrated around the origin in $w$ will give $Q_B^2$. This is an extension of exercise \textbf{4.45}, and there is nothing conceptually further, except for some $\beta \gamma$ manipulation. There are altogether $16$ terms to consider, and we will get $c=15$. The algebra is heavy, so I will skip this. An alternative is to do this as in \textbf{Polchinski 4.3}.
	
	To do it this way, note the following OPEs:
	\[
		\begin{aligned}
			j_B(z) b(w) &\sim \frac{T_{matter}(z)}{z-w} -\frac{1}{(z-w)^2} \left(b c (z) + \frac34 \beta \gamma(z)\right) + \frac{1}{z-w} \left(-b \d c(z) + \frac14 \d \beta \gamma(z) - \frac34 \beta \d \gamma(z) \right)\\
			&= \dots + \frac{1}{z-w} \left[T_{matter}(z) - \d b\, c(w) - 2 b \d c(w) - \frac12 \d \beta \gamma(w) - \frac32 \beta \d \gamma(w) \right]\\
			&= \dots + \frac{T_{matter}(w) + T_{gh}(w)}{z-w} \Rightarrow \{Q_B, b_n\} = L_n
		\end{aligned}
	\]
	Similarly
	\[
		j_B(z) \beta(w) = \dots + \frac{G_{matter}(w) + G_{gh}(w)}{z-w} \Rightarrow [Q_B, \beta_n] = G_n
	\]
	Now note that the Jacobi identity on $Q_B$ reads:
	\[
		\{[Q_B, L_m], b_n \} - \{\overbrace{[L_m, b_n]}^{(m-n)b_{m+n}}, Q_B\} - [\overbrace{\{b_n, Q_B \}}^{L_n}, L_m] = 0 \Rightarrow \{[Q_B, L_m], b_n \} = (m-n) L_{m+n} - [L_m, L_n]
	\]
	So if the total central charge is zero we'll get $\{[Q_B, L_m], b_n \} = 0$, implying that $[Q_b, L_m]$ is independent of the $c$ ghost. But on the other hand this operator has ghost number $1$, so it must therefore vanish. Further, the Jacobi identity also yields
	\[
		[\{Q_B, Q_B\}, b_n] = - 2 [\{b_n , Q_B\}, Q_B] = 2 [Q_B, L_n]
	\]
	since we just showed that this last term vanishes, we must have $Q_B, Q_B$ is also independent of $c$, but again since $Q_B^2$ has positive ghost number, we get that it is in fact zero. We can do the same argument with $\beta$ and $G$ and get that the superstring BRST operator is zero, as long as the total central charge vanishes. This was much cleaner than the OPE way. 

	
	\item
	First a lemma: An abelian $p$-form field $A$ has ${D - 2 \choose p}$ on shell DOF. To prove this, note that we have a gauge symmetry of $A \to A + \d \Lambda$ which has ${D \choose p-1}$ parameters.
	Next, the Euler-Lagrange equations give us that the components $A^{0 i_1 \dots i_{p-1}}$ are non-propagating. We thus get ${D-1 \choose p}$ massless propagating off-shell d.o.f. which have ${D-2 \choose p-1}$ gauge symmetries left over. These can be used to enforce Coulomb gauge conditions which allow for there to be no polarizations along one of the spatial directions. 
	We thus get ${D-1 \choose p} - {D-2 \choose p-1} = {D - 2 \choose p}$ massless on-shell degrees of freedom. For $A_\mu$ this is $D-2$ and for $B_{\mu \nu}$ this is $(D-2)(D-3)/2$.
	
	The metric has $\frac12 D (D-3)$ on-shell degrees of freedom. There are two ways to see this, first, that the dynamically allowed variation $\delta g$ may on-shell be described by a symmetric traceless tensor in dimension $D-2$ which gives
	\[
		\frac{(D-1)(D-2)}{2} - 1 = \frac12 D (D-3)
	\]
	or by noting that since we are gauging translation symmetry locally, each translation makes $2$ polarizations unphysical and so we get:
	\[
		\frac{D(D+1)}{2} - 2 D = \frac12 D (D-3)
	\]
	as required.
	
	We now consider the R-R, R-NS, NS-R, NS-NS sectors together. For NS-NS we have the scalar $=1$ both on-shell and off-shell, the antisymmetric two-form, which has only transverse degrees of freedom $=8 * 7/2 = 28$ and the gravity, $= 10 * 7/2 = 35$ altogether we get $64$ on-shell degrees of freedom.

	In both the R-NS and NS-R sector, we have a Weyl representation of dimension $2^{5-1} = 16$. There are however only $8$ on-shell degrees of freedom. Similarly, we only consider the on-shell $\psi^\mu_{-1/2}$ acting on the NS part of the vacuum which gives another factor of $8$. This gives $64$ fermionic variables in each sector for a grand total of $128$.

	In R-R for IIA we have a 0, 2, and \emph{self-dual} 4-form. This gives:
	\[
	1 + {8 \choose 2} + \frac12 {8 \choose 4} = 64
	\]
	For IIB we have a 1 and 3-form. This gives
	\[
	{8 \choose 1} + {8 \choose 3} = 64
	\]
	so in either case we have 64 on-shell degrees of freedom here. This is consistent with each $\ket{S}$ state having $8$ on-shell degrees of freedom giving $8 \times 8 = 64$. All together, we have the same number of on-shell fermionic and bosonic degrees of freedom. 

	Now for the massive case. In the NS sector you might expect the next excitations come from the bosons $\alpha_{-1}$, but this gets projected out by GSO, so in fact the next states come from $\psi_{-3/2}^i$, $C_{ijk} \psi_{-1/2}^i \psi_{-1/2}^j \psi_{-1/2}^k$ and $C_{ij} \psi_{-1/2}^i \alpha_{-1}^j$. These have dimensions $8+56+64 = 128$, which decomposes as the traceless symmetric \textbf{44} and three-index antisymmetric \textbf{84} representation of $\SO(9)$. In the R sector, we must look at $\alpha_{-1}^i \ket{S_\alpha}$ and $\psi_{-1}^i \ket{C_\alpha}$ for $S_\alpha, C_\alpha$ suitably chosen so that the state satisfies $G_0 = 0$. This constraint gives a factor of two reduction for the dimension of the space of candidate $S_\alpha$. Consequently, we get $\mathbf{8}_v \otimes \mathbf{8}_s \oplus \mathbf{8}_v \otimes \mathbf{8}_{s'}$ which has dimension $128$. This indeed turns out to be a spinor representation of $\SO(9)$, and it comes from looking at the tensor product of \emph{the} fundamental spinor representation with the vector representation $\mathbf{16}_s \otimes \mathbf{9}_v$. This turns must decompose as a sum of two spinor representations $\mathbf{16}_s \oplus \mathbf{128}_s$. One is again the fundamental, while the other is the required \textbf{128}. 
	
	For the massive states in the type IIA and type IIB, we must tensor we wish to look at the lowest-level masses. Note we must match massive states with massive states. In this case, we match $2/\alpha$ on both sides to get massive states of mass $4/\alpha$. Since the particles already organize into representations of $\SO(9)$ on each side, the closed string massive spectrum will again clearly organize intro representations of $\SO(9)$. Also since fermionic and bosonic degrees of freedom already were equal on each side, they will be equal in the closed string as well. We will have $2 \times 128^2 = 32768$ bosonic and fermionic degrees of freedom. 
	 % Consequently, the NS-NS sector will have a spin 6, spin 4, and two spin 5 massive particles. 	From R-NS and NS-R I can tensor the R vacuum $\ket{S_\alpha}$ or $\ket{C_\alpha}$ with the NS states and get two copies of $\mathbf{8}^4$ and $\mathbf{8}^3$. These will appropriately combine to give representations of $\mathrm SO(9)$. \textbf{SHOW THIS PART}

	
	\item In terms of theta functions:
	\[
	\begin{aligned}
		\chi_O &= \frac12 \left(\prod_{i=1}^4 \frac{\theta_3(\nu_i)}{\eta} - \prod_{i=1}^4 \frac{\theta_4(\nu_i)}{\eta} \right)\\
		\chi_V &= \frac12 \left(\prod_{i=1}^4 \frac{\theta_3(\nu_i)}{\eta} + \prod_{i=1}^4 \frac{\theta_4(\nu_i)}{\eta} \right)\\
		\chi_S &= \frac12 \left(\prod_{i=1}^4 \frac{\theta_2(\nu_i)}{\eta} - \prod_{i=1}^4 \frac{\theta_1(\nu_i)}{\eta} \right)\\
		\chi_C &= \frac12 \left(\prod_{i=1}^4 \frac{\theta_2(\nu_i)}{\eta} + \prod_{i=1}^4 \frac{\theta_1(\nu_i)}{\eta} \right)\\
	\end{aligned}
	\]
	We'll take $\nu_i = 0$ here \textbf{(I assume this is what I'm supposed to do)} and so $\theta_1 = 0 \Rightarrow \chi_S= \chi_C$.
	
	For IIB we look at 
	\[
		\frac{|\chi_V - \chi_C|^2}{(\sqrt \tau_2 \eta \bar \eta)^8} = \frac{1}{(\sqrt \tau_2 \eta \bar \eta)^8} \frac12 \sum_{a,b=0}^1 (-1)^{a+b} \frac{\theta^4\twist{a}{b} }{\eta^4}
		\times \frac12 \sum_{\bar a, \bar b=0}^1 (-1)^{\bar a + \bar b} \frac{\bar \theta^4 \twist{\bar a}{\bar b}}{\bar \eta^4}
	\]
	Under modular transformations $\tau \to \tau+1$ $\theta^4\twist01 \leftrightarrow \theta^4\twist00$, $\theta^4\twist10\to-\theta^4\twist10$ while $\eta^{12} \to -\eta^{12}$. In the holomorphic and anti-holomorphic parts separately, each term in the sum picks up a minus sign that is cancelled by the minus sign in the $\eta^4$. 
	
	Under $\tau \to -1/\tau$, the $\frac1{(\sqrt{\tau_2} \eta \bar \eta)^8}$ out front is invariant. On the other hand, the $\theta$ functions transform as $\theta^4\twist00 \to (-i\tau)^2 \theta^4\twist00$, $\theta^4\twist01 \to (-i\tau)^2 \theta^4\twist10$, $\theta^4\twist10 \to (-i\tau)^2 \theta^4\twist01$. These are exactly compensated by the $\eta$ transformations in the denominator, and no overall sign is picked up
	
	For IIA we have similarly
	\[
		\frac{(\chi_V - \chi_C)(\bar \chi_V - \bar \chi_S)}{(\sqrt \tau_2 \eta \bar \eta)^8} = \frac{1}{(\sqrt \tau_2 \eta \bar \eta)^8} \frac12 \sum_{a,b=0}^1 (-1)^{a+b} \frac{\theta^4\twist{a}{b} }{\eta^4}
		\times \frac12 \sum_{\bar a, \bar b=0}^1 (-1)^{\bar a + \bar b + \bar a \bar b} \frac{\bar \theta^4 \twist{\bar a}{\bar b}}{\bar \eta^4}
	\]
	Again, the holomorphic part transforms as before and as we have set the $\nu_i$ to zero, we have the same partition function. Using \textbf{D.18}, we see that each of the four above sums are zero since they are equal to a product of $\theta_1 = 0$.
	
 	\item Again, these are identical if I set the $\nu_i = 0$ (am I not supposed to be doing this? What do the $\nu_i$ represent physically?). They are equal to
	\[
		\frac{1}{(\sqrt {\tau_2} \eta \bar \eta)^8 4 \eta^4 \bar \eta^4} (\cancel{|\theta_1^4|^2} + |\theta_2^4|^2 + |\theta_3^4|^3 + |\theta_4^4|^2)
	\]
	We have $\theta_3$ and $\theta_4$ swapping under $\tau \to \tau+1$, generating no signs in this case, while the denominator looks like $|\eta|^{24}$ and also doesn't generate a sign. Then, under $\tau \to -1/\tau$ we have $\theta_2$ and $\theta_4$ swapping generating a $|\tau|^4$, identical to what is generated by the $(\eta \bar \eta)^4$. 
	
	\item The partition function is
	\[
		Z_{\mathrm{SO}(16) \times \mathrm{SO}(16)}^{\mathrm{het}} 
		= \frac12 \sum_{h,g} \frac{\bar Z_{E_8}\twist hg ^2}{(\sqrt{\tau_2} \eta \bar \eta)^8} \frac12 \sum_{a,b} (-1)^{a+b+ab+ ag+bh+gh} \frac{\theta^4 \twist ab}{\eta^4}, 
		\quad \bar Z_{E_8}\twist hg = \frac12 \sum_{\gamma, \delta} (-1)^{\gamma g + \delta h} \frac{\bar \theta^8 \twist \gamma \delta}{\bar \eta^8}
	\]
	First look at $\bar Z_{E_8}$. Under modular transformations $\tau \to -1/\tau$ we get $\bar Z_{E_8}\twist hg \to \bar Z_{E_8} \twist gh$. Under $\tau \to \tau + 1$, we get $\bar Z_{E_8}\twist hg \to (-1)^{h-2/3} \bar Z_{E_8}\twist{h}{g+h}$. With this, we can look at $Z_{\mathrm{SO}(16) \times \mathrm{SO}(16)}^{\mathrm{het}}$ under $\tau \to -1/\tau$
	\[
		\frac12 \sum_{h,g} \frac{\bar Z_{E_8}\twist gh ^2}{(\sqrt{\tau_2} \eta \bar \eta)^8} \frac12 \sum_{a,b} (-1)^{a+b+ab+ ag+bh+gh} \frac{\theta^4 \twist ba}{\eta^4}
	\]
	Under relabeling of $a\leftrightarrow b, g\leftrightarrow h$, this is the same. Next, under $\tau \to \tau+1$:
	\[
	\begin{aligned}
		&\frac12 \sum_{h,g} \frac{(-1)^{-4/3} \bar Z_{E_8}\twist{h}{g+h}^2}{(-1)^{4/3} (\sqrt{\tau_2} \eta \bar \eta)^8} \frac12 \sum_{a,b} (-1)^{a+b+ab+ ag+bh+gh} \frac{(-1)^a \theta^4 \twist{a}{a+b-1}}{(-1)^{1/3} \eta^4}\\ 
		&= \frac12 \sum_{h,g} - \frac{ \bar Z_{E_8}\twist{h}{g+h}^2}{(\sqrt{\tau_2} \eta \bar \eta)^8} \frac12 \sum_{a,b} (-1)^{b+ab+ ag+bh+gh} \frac{\theta^4 \twist{a}{a+b-1}}{\eta^4}\\
		&=	\frac12 \sum_{h,g'} \frac{ \bar Z_{E_8}\twist{h}{g'}^2}{(\sqrt{\tau_2} \eta \bar \eta)^8} \frac12 \sum_{a,b} (-1)^{1+b+ab+ ag' + (a+b) h +g'h + h} \frac{\theta^4 \twist{a}{a+b-1}}{\eta^4}\\
		&= \frac12 \sum_{h,g'} \frac{ \bar Z_{E_8}\twist{h}{g'}^2}{(\sqrt{\tau_2} \eta \bar \eta)^8} \frac12 \sum_{a,b} (-1)^{\cancel 1 + (b' + a + \cancel 1) + (ab' + \cancel a - \cancel a) + ag' + (b' h + \cancel{h}) +g'h + \cancel h } \frac{\theta^4 \twist{a}{b'}}{\eta^4}\\
		&= \frac12 \sum_{h,g'} \frac{ \bar Z_{E_8}\twist{h}{g'}^2}{(\sqrt{\tau_2} \eta \bar \eta)^8} \frac12 \sum_{a,b} (-1)^{a + b' + a b' + a g' + b' h + g' h} \frac{\theta^4 \twist{a}{b'}}{\eta^4}\\
	\end{aligned}
	\]
	Keep in mind that $x^2 = x\, \mathrm{mod}\, 2$. 
	
	Before we do the next part, let's elaborate on why $Z_{E_8} = \frac12 \sum_{a, b} \theta^8 \twist ab$ is the partition function of the $E_8$ lattice. From the sixteen fermion picture, this is just the $(-1)^F = 1$ in the NS sector (corresponding to the $\chi_O = \frac12 (\theta^8 \twist00 + \theta^8 \twist01)$ character) together with the R sector $\chi_S = \frac12 \theta^8 \twist10$ giving the spinor representation.
	 
	Indeed, the roots of $E_8$ consist of the roots of $\mathrm O(16)$ as well as the spinor weights of $\mathrm O(16)$. Note that the spinor representation comes from the half-integral points, corresponding to $\theta \twist10$ in the sum, while the adjoint representation comes from $\theta \twist 01$ and $\theta \twist 00$. Consequently the action of $\mathcal S_i$ that fixes the adjoint vectors but flips the sign of the spinor acts on our partition function as $\mathcal S_i Z_{E_8} = \frac12 \sum_{a, b} (-1)^{a}\, \theta^8 \twist ab$. It of course also gives rise to a twisted sector, so altogether we get the four twisted blocks $\bar Z_{E_8} \twist hg$ as required.
	
	Since we have projected out the spinor representation, the current algebra only contains the NS currents $\bar J^{ij}$ corresponding to the adjoint of $\mathrm{SO}(16)$, and we have two copies of this for each group of 16 fermions. 
	
	From the factor of $(\sqrt \tau_2 \eta \bar \eta)^{-8}$ we see that we have $8$ on-shell noncompact massless bosonic excitations as well as all of their descendants (on both left and right moving sides). We also see on the left-moving side we get a theta-function corresponding to $N=8$ fermions transforming under a spacetime $\mathrm{SO}(8)$, forming the superpartners of the bosons. On the right side instead of the superpartner fermions, we have the $16$ internal fermions that transform in the adjoint representations. 
	
	Let's see what massless states we can build. In the NS sector of the left-movers, we have $L_0 = 1/2, \bar L_0 = 1$ and so we get $\psi^i_{-1/2} \alpha^j_{-1} \ket p$ which gives us our usual graviton, two-form field, and dilaton. We also have $\psi^i_{-1/2} \bar J^a_{-1} \ket{p}$ for the $\mathrm{O}(16) \times \mathrm{O}(16)$ currents. This gives us vectors corresponding to gauge bosons valued in the adjoint of $\mathrm{O}(16) \times \mathrm{O}(16)$ as required. 
	
	In the R sector we have $G_0 = 0, \bar L_0 = 1$ we'll get a gravitino, fermion, and gaugino as before, but again this time valued in $\mathrm{O}(16) \times \mathrm{O}(16)$.
	
	\item Because we have seen that T-duality flips the antichiral $U(1)$ $\bar \d X \to -\bar \d X$, and we want to preserve the (1,1) supersymmetry $G$ in the type II string (and so must keep it as a periodic variable \textbf{Why is this absolutely necessary. Can we not work with double covers in some clever way when defining supercurrents?} ), we must consequently flip $\bar \psi$. This corresponds to inserting $(-1)^{F_R}$. For the right-moving R sector, this changes the chirality of the R spinor, taking $S_\alpha \to \Gamma^9 \Gamma^{11} S_\alpha$ (there can be no phase, by reality conditions of $\Gamma$). We thus flip IIA to IIB and vice versa.
	
	From this we get that 
	\[
		F_{\alpha \beta} = S_{\alpha} (\Gamma^{0})_{\beta \gamma} \tilde S_\gamma \to S_{\alpha} (\Gamma^{0} \Gamma^{9} \Gamma^{11})_{\beta \gamma} \tilde S_\gamma = - \xi \, S_{\alpha} (\Gamma^{9} \Gamma^{0})_{\beta \gamma} \tilde S_\gamma = - \xi F \Gamma^{9}
	\]
	Expanding in terms of the $F_{\mu_1 \dots \mu_k}$ gives the action:
	\[
		F_{\alpha \beta} \to - \xi \sum_{k=0}^{10} \frac{(-1)^k}{k!} F_{\mu_1 \dots \mu_k} \Gamma^{\mu_1 \dots \mu_k} \Gamma^{9}
	\]
	This gives that
	\[
		\tilde F_{\mu_1 \dots \mu_k, 9} = - \xi F_{\mu_1 \dots \mu_k}, \quad \tilde F_{\mu_1 \dots \mu_k} = F_{\mu_1 \dots \mu_k, 9}
	\]
	Then
	\[
		\d_{\mu_1} \tilde C_{\mu_2 \dots \mu_k 9} = -\xi \d_{\mu_1} C_{\mu_2 \dots \mu_k}, \quad \d_{\mu_1} \tilde C_{\mu_2 \dots \mu_k} = \d_{\mu_1} \tilde C_{\mu_2 \dots \mu_k 9}
	\]
	so that (up to a closed term)
	\[
		\tilde C_{\mu_1 \dots \mu_{p-1} 9}^{(p)} = -\xi C_{\mu_1 \dots \mu_{p-1}}^{p-1}, \quad \tilde C^{(p)}_{\mu_1 \dots \mu_p} = C^{(p+1)}_{\mu_1 \dots \mu_p 9}
	\]
	\textbf{Get rid of the $\xi$ factor}
	
	\item We have that $\Omega \ket{S_\alpha \tilde S_\beta} = \varepsilon_R \ket{S_\beta \tilde S_\alpha}$. Further, it acts trivially on $\Gamma^0$ \textbf{(you sure?)}. Now, in the operator language we will have $\Omega S_\alpha \Omega^{-1} = \epsilon_1 \tilde S_\alpha$ and $\Omega \tilde S_\beta \Omega^{-1} = \epsilon_2 S_\beta$. In any case, we must have for the bi-spinor that $\Omega S_\alpha \tilde S_\beta \Omega^{-1} = \epsilon_R S_\beta \tilde S_\alpha$, which gives that $\epsilon_1 \epsilon_2 = -\epsilon_R$ Thus, we have:
	\[
		\Omega F_{\alpha \beta} \Omega^{-1} = \Omega S_\alpha \Gamma^0_{\beta \gamma} \tilde S_\gamma \Omega^{-1} = - \epsilon_R \Gamma^0_{\beta \gamma} S_{\gamma} \tilde S_\alpha = - \epsilon_R \Gamma^0_{\beta \gamma} F_{\gamma \delta} \Gamma^0_{\delta \alpha} = -\epsilon_R (\Gamma^0 F \Gamma^0)_{\beta \alpha} =  -\epsilon_R (\Gamma^0 F^T \Gamma^0)_{\beta \alpha }
	\]
	\textbf{I think 7.3.3 of Kiritsis has the derivation wrong. Ask Nathan/Xi.}
	\item When we take $\epsilon_R = -1$m the scalar and four-index self-dual tensor survive. In this case, we will \emph{not} have consistent interactions. Since the graviton survives, there must be an equal number of massless bosonic and fermionic excitations. The fermions come just from the NS-R sector (there is no R-NS now), giving 64 on-shell fermionic excitations. From the NS-NS sector, the dilaton and gravity will give $1+35 = 36$ on-shell bosonic degrees of freedom. We are missing 28 bosonic degrees of freedom. 
	
	The scalar and four-index self dual tensor contribute $1 + \frac12 \frac{8 \times 7 \times 6 \times 5}{4!} = 36$ on-shell bosonic degrees of freedom. This is too much. The two-form, on the other hand, contributes the requisite $8 \times 7 / 2= 28$. Consistency of interaction thus \emph{demands} we keep only the 2-form and drop the 0 and self-dual 4-form. This necessitates $\epsilon_R = 1$.
	
	\item We are just looking at the \emph{open} superstrings here. Any open string that consistently couples to type I or type II string theory must have a GSO projection as well. We have already seen how the oriented open strings look like in exercise \textbf{7.3}. In the NS sector we have at $-p^2 = m^2 = 2/\ell_s^2$
	\begin{equation}\label{eq:NStype1}
		\begin{aligned}
			&\psi_{-3/2}^i \lambda_{ab} \ket{p; ab}_{NS}\\
			&C_{ijk} \psi_{-1/2}^i \psi_{-1/2}^j \psi_{-1/2}^k \lambda_{ab} \ket{p; ab}_{NS}\\
			&C_{ij} \psi_{-1/2}^i \alpha_{-1}^j \lambda_{ab} \ket{p; ab}_{NS}
		\end{aligned}
	\end{equation}
	 In the $R$ sector we have (for $S_\alpha$ suitably chosen so that the state satisfies $G_0 = 0$):
	 \begin{equation}\label{eq:Rtype1}
	 	\begin{aligned}
	 		& \alpha_{-1}^i \lambda_{ab} \ket{S_\alpha; ab}_{R}\\
			& \psi_{-1}^i \lambda_{ab} \ket{C_\alpha; ab}_{R}
	 	\end{aligned}
	 \end{equation}
	I will assume NN boundary conditions. In this case
	\[
	\begin{aligned}
		\Omega \alpha_{-1} \Omega^{-1} &= - \alpha_{-1}\\
		\Omega \psi_{-1} \Omega^{-1} &= - \psi_{-1}\\
		\Omega \psi_{-\frac12} \Omega^{-1} &= -i \psi_{-\frac12}\\
		\Omega \psi_{-\frac32} \Omega^{-1} &= i \psi_{-\frac32}
	\end{aligned}
	\]
	So all of the terms in \eqref{eq:NStype1} are terms of the form $\mathcal A \lambda_{ab} \ket{p; ab}_{NS}$ with the operator $\mathcal A$ transforming as $\mathcal A \to i \mathcal A$ under parity. Doing parity twice therefore will generate a $- \epsilon_{NS}^2 \mathcal A (\gamma {\gamma^T}^{-1})_{ii'} \ket{p; a'b'} (\gamma^T {\gamma}^{-1})_{j'j}$. This is exactly the same as in \textbf{7.3.10}.  % Imposing that this is identical will give that $\gamma = \zeta \gamma^T$, $\zeta^2 \epsilon_{NS}^2 = -1$.
	Demanding that $\Omega$ act on the state with eigenvalue $+1$ will make it so that $\lambda = i \epsilon_{NS} \gamma \lambda^{T} \gamma^{-1}$. We already have $\epsilon_{NS} = -i$ so $\lambda = \gamma \lambda^T \gamma^{-1}$ here. % Using the same arguments as in the section, we will conclude that $\epsilon_{NS}=-i$ (this is always true for all levels) and $\zeta^2 = 1$.
	Imposing the tadpole cancelation condition $\zeta = 1$ and we get gauge group $\SO(32)$. So we get that states at this level will transform in the  \emph{the traceless symmetric tensor + singlet representation} of $SO(32)$.
	
	All of the terms in \eqref{eq:Rtype1} will transform under parity twice as as $\epsilon_R^2 \mathcal A (\gamma {\gamma^T}^{-1})_{ii'} \ket{S_\alpha; a'b'} (\gamma^T {\gamma}^{-1})_{j'j}$. We will have the same $\gamma$ matrix as in the NS sector, as required for consistency of interactions. Here, though, we will get $\epsilon_R = -1 \Rightarrow \epsilon_R^2 = 1$ and we will get $\lambda = -\gamma \lambda^T \gamma^{-1}$ (this is what we got from the massless sector with an extra minus sign since $\psi_{-1}, \alpha_{-1}$ now transform with minus signs). Again we will have that these states will transform in the symmetric representation of $\SO(32)$. 

	Again we get $128$ bosonic states that will transform as the $\mathbf{44} \oplus \mathbf{84}$ representation of $\SO(9)$. We will also get fermions transforming in the $\mathbf{128}$ spinor representation as in exercise \textbf{3}. 
	All of these states will transform in the traceless symmetric representation of $\SO(32)$. \textbf{Confirm}
	
	
	\item Certainly in the untwisted sector, the theory we get corresponds to tracing over the projection operator $\frac12 (1 + g)$ where $g$ is orientation-reversal. Now in the twisted sector, we still have $X^\mu$ satisfies the Laplace equation $\d_+ \d_- X = 0$ so we can write
	\[
		X(\sigma, \tau) = x^\mu + \tau \ell_s^2 \frac{p^\mu + \bar p^\mu}{2} + \sigma \ell_s^2 \frac{p^\mu - \bar p^\mu}{2} + \frac{i \ell_s}{\sqrt 2} \sum_{n} \left( \frac{\alpha_n}{n} e^{-i n (\tau  + \sigma)} + \frac{\tilde \alpha_n}{n} e^{-i n (\tau  + \sigma)} \right)
	\]
	The condition that $X(\sigma + 2\pi) = X(2\pi-\sigma)$ give that $p^\mu = \bar p^\mu$ and the $\sigma$ term vanishes. We must have $n$ is a half integer. For integer modding we have $e^{-i n (\tau \pm \sigma)} \to e^{-i n (\tau \mp \sigma)}$. For half-integer modding we have $e^{-i n (\tau \pm \sigma)} = (-1)^n e^{-i n (\tau \mp \sigma)}$. We should thus have $\alpha_n = \tilde \alpha_n$ for $n$ integral and $\alpha_{n} = -\tilde \alpha_{n}$ We thus get
	\[
			X(\sigma, \tau) = x^\mu + 2 \ell_s^2 p^\mu \tau + \sigma  i \sqrt 2\ell_s \sum_{n \in \ZZ \setminus \{0\}} \frac{\alpha_n}{n} \cos(n \sigma) e^{- i n \tau}  - \sqrt{2} \ell_s \sum_{n \in \ZZ + \frac12} \frac{\alpha_n}{n} \sin(n \sigma) e^{-i n \tau}
	\]
	This is the twisted sector. The last sum picks up a minus sign under orientation reversal, and so will be projected out. We are left with the equations of motion for the open string.
	
	\item In NS we have (up to a factor of $i^{-1/2}$)
	\[
		\psi(\sigma, \tau) = \sum_{n\in \ZZ} \psi_{n+1/2} e^{(n+1/2) (\tau + i \sigma)}, \quad \bar \psi(\sigma, \tau) = \sum_{n\in \ZZ} \bar \psi_{n+1/2} e^{(n+1/2) (\tau - i \sigma)}
	\]
	In the closed string case have that $\Omega \psi_{n+1/2} \Omega^{-1} = \bar \psi_{n+1/2}$. 
	Given that $\Omega \psi(\sigma, \tau) \Omega^{-1} = \bar \psi(\pi - \sigma, \tau)$, we directly get $\Omega \psi_{n+1/2} \Omega^{-1} = i (-1)^n \bar \psi_{n+1/2}$. For DD boundary conditions we get an extra minus sign to this, since there $\Omega \psi(\sigma, \tau) \Omega^{-1} = - \bar \psi(\pi - \sigma, \tau)$. 
	
	In the R sector we have
	\[
		\psi(\sigma, \tau) = \sum_{n\in \ZZ} b_n e^{n (\tau + i \sigma)}, \quad \bar \psi(\sigma, \tau) = \sum_{n\in \ZZ} \bar b_n e^{n (\tau - i \sigma)}
	\]
	Following the same logic we get that $\Omega \psi_{n} \Omega^{-1} = (-1)^n \psi_n$ for NN and $\Omega \psi_{n} \Omega^{-1} = -(-1)^n \psi_n$ for DD.
	
	All of these cases can be summarized by
	\[
		\begin{aligned}
			\text{NN:}\quad & \Omega \psi_r \Omega^{-1} = (-1)^r \psi_r \\ 
			\text{DD:}\quad & \Omega \psi_r \Omega^{-1} = - (-1)^r \psi_r.
		\end{aligned}
	\]
	\item Let's first not twist the left sector with the right one. We can either take all the 32 fermions together and orbifold by $(-1)^F$. This gives the partition function for the $\SO(32)$ heterotic string $\frac12 \sum_{ab} \theta^16 \twist ab$. We can also split them up into groups of 16 + 16 and orbifold by $(-1)^{F_1}$ and $(-1)^{F_2}$ on each of the two groups ($\cong \ZZ_2^2$). This gives the partition function for the $E_8 \oplus E_8$ heterotic string $\left(\frac12 \sum_{ab} \theta^8 \twist ab\right)^2$. We cannot split things any further, since splitting the fermions into groups of $8$ will require us to have minus signs between the different sectors in order to keep modular invariance. This would give negative weight to the R sector (which would give the wrong spin-statistics, since the $R$ sector ground state are spacetime scalars) correspond do doing the projection $(-1)^F = -1$ in the NS sector (which would not close under OPE). 
	
	We also have no other internal symmetry to twist the right-movers by. The remaining partition functions that we can write down necessarily must twist together the fermions on the left-moving and right-moving sides.
	
	The simplest thing we can twist by is $(-1)^{F_L} (-1)^{F_R}$ \emph{diagonally now} rather than separately. Our fermion partition function will be:
	\[
		\frac12 \sum_{ab} \frac{(-1)^{a + b + ab} \theta^4 \twist ab \bar \theta^{16} \twist ab}{\eta^4 \bar \eta^{16}}
	\]
	Here all the fermions are either R or NS and have been projected out by $(-1)^{F_L + F_R} = 1$
	This theory is consistent, but at level $0$ we have the tachyon associated with the identity. The normalization is of the identity is indeed 1.  
	
	Now lets try to twist this further. Note that twisting this theory by $(-1)^F_R$ or $(-1)^F_L$ would just give group elements $\frac{1 + (-1)^{F_L + F_R}}{2} \frac{1 + (-1)^{F_R}}{2} = \frac{1 + (-1)^{F_L}}{2} \frac{1 + (-1)^{F_R}}{2}$ so we would recover the $\SO(32)$ superstring. 
	
	Consider instead twisting by $(-1)^{F_{R, 1}}$ which flips the sign of the first 16 righr-moving fermions. 
		%
	% The characters for $O(32)$ are $\frac12 \frac{\theta^{16} \twist ab}{\eta^16}$ and the characters for $O(8)$ are $\frac12 \frac{\theta^4 \twist ab}{\eta^4}$. Each of these has $4$ sectors, so there are $16$ sectors that we can choose from.
	
	
	
	We thus have 9 theories satisfying modular invariance, spin-statistics. Six of them have tachyons, so we get \textbf{three} theories left over. Of these three, only two have spacetime supersymmetry--exactly heterotic O and heterotic E. 
	
	\item I think this problem is backwards. For 32 fermions \emph{all} with the same boundary conditions, its immediate to see that they will reproduce the partition function for the $\mathrm{Spin}(32)/\ZZ_2$ string:
	\[
		\frac12 \sum_{a,b} \theta^{16}\twist ab
	\]
	Just by considering the $O(N)$ fermion at $N=32$. On the other hand, if we split the fermions into $16+16$, and consider separately boundary conditions for each of \emph{those}, then our partition function is the square of the $16$-fermion system. We then get $E_8 \times E_8$ as required
	\[
		\left[ \frac12 \sum_{a,b} \theta^{8}\twist ab \right]^2
	\]
	\item Note this was a Lorentzian lattice of signature $(n,n)$. The norm was thus $P_L^2 - P_R^2 = 2 m n \in 2 \ZZ$. It is also self dual, since it is already integral, and there is no integral sublattice. 
	
	\item We have
	\[
		\gamma G_{ghost} = -c \gamma \d \beta - \frac32 \d c \gamma \beta - 2 \gamma^2 b, \quad c T_{ghost} = 2 b c \d c - \frac12 c \gamma \d \beta - \frac32 c \d \gamma \beta
	\]
	\textbf{Here Kitisis' conventions are different than Polchinski.}= Recall upon bosonization $\beta(z) = e^{-\phi(z)} \d \xi(z), \gamma = e^{\phi(z)} \eta(z)$. 
	Although we can solve this problem very quickly since we already know what the stress tensor looks like in the bosonized variables, I think it's way more instructive to explicitly compute OPEs to $O(z-w)$. First let's look at the $\eta, \xi$ theory, which is a fermoinic $bc$ theory of weights $1, 0$. We get
	\[
		\xi(z) \eta(w) = \frac{1}{z-w} + :\xi \eta:(w) + O(z-w)
	\]
	We can bosonize this theory in terms of hermitian $\chi$ field so that $\eta = e^{-\chi}, \xi = e^{-\chi}$. Using these coordinates
	\[
	\begin{aligned}
		\xi(z) \eta(w) &= e^{\chi(z)} e^{-\chi(w)} = \frac{1}{z-w} \left[1 + (z-w) \d \chi + \frac12 (z-w)^2 (\d^2 \chi + (\d \chi)^2)  + \dots \right]\\
		\Rightarrow \d \xi(z) \eta(w) &= -\frac{1}{(z-w)^2} + \frac12 (\d^2 \chi + (\d \chi)^2)
	\end{aligned}
	\]
	Using this we can write
	\[
	\begin{aligned}
		\beta(z) \gamma(w) &= e^{-\phi(z)} \d \xi(z) \, e^{\phi(w)} \eta(w) \\
		&= (z-w) \left[1 - (z-w) \d \phi(w) + \frac12 (z-w)^2 ((\d\phi)^2 - \d^2 \phi) \right] \left[-\frac{1}{(z-w)^2} + \frac12 (\d^2 \chi + (\d \chi)^2) \right]
	\end{aligned}
	\]
	The constant term gives $:\!\!\beta \gamma\!\!: = \d \phi \Rightarrow :\!\d(\beta \gamma)\!: = \d^2 \phi$. The $(z-w)$ term gives exactly the stress tensor of the $\beta \gamma$ theory at $\lambda = 0$, which makes sense since this is exactly $\d \beta \gamma$
	\[
	\begin{aligned}
		:\!\d \beta \gamma\!: &= -\frac12 (\d \phi)^2 + \frac12 \d^2 \phi + \frac12 (\d\chi)^2 + \frac12 \d^2 \chi\\
		\Rightarrow 	T_{\beta \gamma } &= \d \beta \gamma - \lambda \d(\beta \gamma) = -\frac12 (\d \phi)^2 + \left(\frac12 - \lambda\right) \d^2 \phi + \frac12 (\d\chi)^2 + \frac12 \d^2 \chi.
	\end{aligned}
	\]
	In our case we have $\lambda = 3/2$. % but we will leave $\lambda$ in, unevalauted. In terms of these nice bosonic variables we have 
	\[
	\begin{aligned}
		\gamma G_{ghost} &= -c \left(-\frac12 (\d \phi)^2 + \frac12 \d^2 \phi + \frac12 (\d\chi)^2 + \frac12 \d^2 \chi\right) - \frac32 \d \phi\, \d c - 2 \gamma^2 b\\
		c T_{ghost} &= 2 b c \d c + c \left(-\frac12 (\d \phi)^2 - \d^2 \phi + \frac12 (\d\chi)^2 + \frac12 \d^2 \chi \right)
	\end{aligned}
	\]
	Altogether this gives a BRST current:
	\[
	\begin{aligned}
		j_B &= c T_X + \gamma G_X +  \frac12 \left(c T_{gh} + \gamma G_{gh} \right)\\
		&=  c T_X + \gamma G_X + b c \d c - \frac34 \d \phi \d c - \frac34 c \d^2 \phi - \gamma^2 b
	\end{aligned}
	\]
	\item We are looking at $[Q_B, \xi e^{-\phi/2} S_\alpha e^{i p X}]$. It suffices to look at the $1/(z-w)$ pole in the OPE of $j_B$ with $\xi e^{-\phi/2} S_\alpha e^{i p X}$. The terms that contribute to this pole must involve pairing $\xi$ with its conjugate $\eta$. $\eta$ appears in $j_B$ wherever $\gamma = e^{\phi} \eta$ appears. From the previous exercise, we see that we need only look at the terms $\gamma G_X$ and $-\gamma^2 b$.
	
	These two terms contribute poles:
	\[
		- \left[\frac{ :e^{\phi/2} G_X S_\alpha e^{i p X}: }{z-w} - \frac{:e^{3\phi/2} \eta b S_\alpha e^{i p X}: }{z-w} \right]
	\]
	The minus sign comes from commuting across an odd number of fermions for the Wick-contraction. We will need to recall two things:
	\[
		\psi^\mu(z) \cdot S_\alpha(w) \sim \frac{\ell_s^2}{\sqrt 2 {\sqrt{z-w}}} \left( \Gamma^\mu_{\alpha \beta} S^{\beta}(w) +  \Gamma^\nu_{\alpha \beta} S^\beta \psi_\nu \psi^\mu (z-w) \right),
		\quad e^{\phi(z)} e^{-\phi(w)/2} \sim i \sqrt{z-w} e^{\phi(w)/2}
	\]
	\textbf{Justify the $i$ in that last OPE}. That means that  first term is:
	\[
	\begin{aligned}
		e^{\phi (z)} i \frac{\sqrt{2}}{\ell_s^2} &\psi^\mu(z) \d X_\mu(z) \cdot e^{-\phi(w)/2} S_\alpha(w) e^{i p X(z)} \\
		&\sim  i \frac{\sqrt 2}{\ell_s^2} \,\sqrt{z-w} \, e^{\phi(w)/2} \left(i \frac{\ell_s^2}{\sqrt 2} \frac{\Gamma_{\alpha \beta}^{\mu} S^\beta \d X_\mu e^{i p \cdot X}}{\sqrt{z-w}} + i  \frac{-i p_\mu e^{i p \cdot X}}{z-w} \Gamma^\nu_{\alpha \beta} S^\beta \psi_\nu \psi^\mu \sqrt{z-w} \right)\\
		& = -e^{\phi/2} \left( \Gamma^\mu_{\alpha \beta} S^\beta \d X_\mu 
		- i \Gamma^\nu_{\alpha \beta} S^\beta \psi_\nu p \cdot \psi  \right) e^{i p \cdot X}
	\end{aligned}
	\]
	Note this OPE has no singularity, so we exactly got the normal ordered term we required: $:e^{\phi/2} G_X S_\alpha e^{i p \cdot X}:$. 
	Altogether this gives us:
	\[
		V^{(1/2)}_{\text{fermion}} (u, p) = \left[e^{\phi/2} \Gamma_{\alpha \beta}^\mu S^\beta \d X^\mu -i e^{\phi/2} \Gamma_{\alpha \beta}^\nu S^\beta \psi_\nu p \cdot \psi  + e^{3 \phi/2} \eta b S_\alpha
		\right] e^{i p X} u^\alpha(p).
	\]
	\item It is useful to note that we can write the PCO
	\[
		X(z) := Q_B \xi (z) = T_F \delta(\beta(z)) - \d b \delta'(\beta(z))
	\]
	In the language of Polchinski, $\delta(\gamma) = e^{-\phi}, \delta(\beta) = e^{\phi}$.  \textbf{Finish}
	\item It is enough to look at the $1/(z-w)$ term in the OPE
	\[
		e^{-\phi(z)/2} S_\alpha(z) V^{(-1/2)}_{\text{fermion}}(w) = e^{-\phi(z)/2} S_\alpha(z) u^\beta(p) e^{-\phi(w)/2} S_\beta(w) e^{i p \cdot X(w)}
	\]
	We will use the fact of \textbf{4.12.42}:
	\[
		S_\alpha(z) S_\beta(w) = \frac{C_{\alpha \beta}}{(z-w)^{N/8}} + \frac{\Gamma^\mu_{\alpha \beta} \psi_\mu(w)}{\sqrt 2 (z-w)^{N/8-1/2}}
	\]
	where $C_{\alpha \beta}$ is the charge conjugation matrix and here $N=8$. The only solution is if $e^{\phi(z)/2} e^{-\phi(w)/2} \sim \frac{\sqrt{2} e^{-\phi(w)}}{\sqrt{z-w}}$ \textbf{Why would this possibly happen?}. This leaves the $(z-w)^{-1}$ term to be the requisite 
	\[
		e^{-\phi} u^\beta(p) \Gamma^\mu_{\alpha \beta} \psi_\mu e^{i p \cdot X} = V^{(-1)}_{\text{boson}}	
	\] 
	
	For the second one, we will look at the $(z-w)^{-1}$ term in the OPE
	\[
		e^{-\phi(z)/2} S_\alpha(z) \epsilon_\mu \left(\d X^\mu- \frac i2 p_\mu \psi^\mu\, \psi^\nu \right) e^{i p \cdot X}.
	\]
	The first term in parentheses will not contribute to the singular term. Also the $e^{-\phi/2}$ and $e^{i p \cdot X}$ contract with nothing. Here, we use \textbf{4.12.41} to evaluate
	\[
		 S_\alpha(z) \cdot \underbrace{\psi^\mu \psi^\nu}_{J^{\mu \nu}} (w) \sim - \frac{(\Gamma_{\mu \nu})^\beta_\alpha S_\beta (w)}{2 (z-w)}
	\]
	The $-$ sign comes from the fact that the fermion current is coming from the \emph{right} this time so $z$ and $w$ are swapped.  
	This gives a variation
	\[
		e^{-\phi/2} i p^\mu \epsilon^\nu (\Gamma_{\mu \nu})^{\beta}_{\alpha} S_\beta e^{i p \cdot X} = V^{(-1/2)}_{\text{boson}}
	\]
	\textbf{I'm off by \emph{two} factors of $2$ }
	\item 
	
	\item 
	
	\item 
	
	\item 
	
	\item The bosonic action in 11D is:
	\[
		\frac{1}{2\kappa^2} \int d^{11} x \sqrt{-\det \hat G} \left[R - \frac{1}{2 \cdot 4!} G_4^2 + \frac{1}{(144)^2} \epsilon^{M_1 \dots M_{11}} G_{M_1 \dots M_4} G_{M_4 \dots M_8} \hat C_{M_9 M_{10} M_{11}}\right]
	\]
	where $G_4$ is the field strength of the 3-form $\hat C$. From Appendix F, we have that the dilaton $\Phi = 0$ in 11D. So the field $\sigma$ will just be $-2\phi = \frac12 \log G_{00}$, and $A$ here is as it is in appendix \textbf{F}. Directly using the bosonic equation \textbf{F.3} gives the terms
	\[
		\frac{1}{2\kappa^2 }\int d^8 x \sqrt g e^{\sigma} \left[R - \frac14 e^{2\sigma} F_2^2  \right]
	\]
	Now let's look at the $3$-form potential contribution. Because $F$ is antisymmetric in all four indices, and we only are compactifying along one dimension, only the first two terms of \textbf{F.28} can contribute. They give
	\[
		\frac{1}{2\kappa^2} \int d^{10} x \sqrt{g} e^\sigma \left[-\frac{1}{2 \cdot 4!} F_4^2 - \frac{1}{2 \cdot 3!} H_3^2 \right]
	\]
	and $H_3 = \d B_2 = \d C_{\mu \nu, 11}$ consistent with \textbf{F.30}

	Finally, let's look at that last $\epsilon^{M_1 \dots M{11}}$ term. At first it looks quite scary. Note we can write this last term as $\dd \hat C \wedge \dd \hat C \wedge \hat C$. \textbf{(Finish this part)}
	
	This translates to our requisite term
	\[
		\frac{1}{4\kappa^2} \int d^{10} x B_2 \wedge \dd C_3 \wedge \dd C_3
	\]
	\item Under $A \to A + \d \epsilon$ and $C \to C + \epsilon H_3$
	
	Note $F_2$ will stay the same, as will $H_3' = \dd B_2' = \dd C_{\mu \nu 11}' = \dd C_{\mu \nu 11} + \epsilon \dd H_3 = \dd C_{\mu \nu 11} = H_3$ since $H_3$ is exact. $\dd C_3$ will also stay the same. 
	
	Finally, $F_4' = F_4 - \dd \epsilon \wedge H_3$
	

\end{enumerate}
% section chapter_7_superstrings_and_supersymmetry (end)
	
\end{document}
	