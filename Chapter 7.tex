\documentclass[11pt, class=article, crop=false]{standalone}
\usepackage{amsmath,amssymb,amsfonts,amsthm}
\usepackage{enumitem}
\usepackage{fancyhdr}
\usepackage{tikz-cd}
\usepackage{mathabx}
\usepackage{geometry}
\usepackage{natbib}
\usepackage{braket}
\usepackage{graphicx}
\usepackage{simpler-wick}
\usepackage{hyperref}
\usepackage{ytableau}
\usepackage{cancel}
\usepackage{listings}
\usepackage{relsize}
\usepackage{xcolor}
\usepackage{stmaryrd}
\usepackage{tikz-feynman}
\usepackage{kiritsis}
\geometry{margin = 0.5in}


\begin{document}
\section*{Chapter 7: Superstrings and Supersymmetry} % (fold)
\label{sec:chapter_7_superstrings_and_supersymmetry}
\begin{enumerate}
	\item We already know that $T T$ will have the desired OPE, since the bosons and fermions are uncoupled and we already have shown their own respective stress tensor OPEs. Next
	\[
	\begin{aligned}
		G(z) G(w) &= - \frac{2}{\ell_s^4} \psi_\mu(z) \d X^\mu(z) \psi_\nu(w) \d X^\nu (w)\\ 
		&= - \frac{2}{\ell_s^4} \left(\ell_s^2 \frac{\eta_{\mu \nu}}{z-w} + (z-w) :\d \psi_\mu \psi_\nu(w): \right) \left(-\frac{\ell_s^2}{2} \frac{\eta_{\mu \nu}}{(z-w)^2} + :\d X_\mu \d X_\nu (w): \right)\\
		&= \frac{D}{(z-w)^3} + \frac{-\frac{2}{\ell_s^2} \d X_\mu \d X^\mu (w) - \frac{1}{\ell_s^2} \psi^\mu \d \psi_\mu (w)}{z-w} \\
		&= \frac{\hat c}{(z-w)^3} + \frac{2 T(w)}{z-w}
	\end{aligned}
	\]
	Finally
	\[
		\begin{aligned}
			T(z) G(w) &= - \frac{1}{\ell_s^2} \, \left(:\d X_\mu \d X^\mu (z): + \frac12 \psi^\mu \d \psi_\mu (z)\right) i \frac{\sqrt 2}{\ell_s^2} \psi_\nu \d X^\nu(w)\\
			&= -i \frac{\sqrt 2}{\ell_s^4} \left(-\frac{\ell_s^2}{2} \frac{ \psi_\mu \d X^\mu (w) + \psi_\mu \d^2 X^\mu(w) (z-w)}{(z-w)^2} - \frac{\ell_s^2}{2} \frac{ \psi_\mu \d X^\mu (w)}{(z-w)^2} + (-) \frac{\ell_s^2}{2} \frac{\d_\mu \psi \d X^\mu(w)}{(z-w)} \right)\\
			&= \frac32 \frac{G(w)}{(z-w)^2} + \frac{\d G(w)}{z-w}
		\end{aligned}
	\]
	
	\item We will take the OPE of $j_B(z) j_B(w)$, but just look at the $(z-w)^{-1}$ term as a function of $w$, as this, when integrated around the origin in $w$ will give $Q_B^2$. This is an extension of exercise \textbf{4.45}, and there is nothing conceptually further, except for some $\beta \gamma$ manipulation. There are altogether $16$ terms to consider, and we will get $c=15$. The algebra is heavy, so I will skip this. An alternative is to do this as in \textbf{Polchinski 4.3}.
	
	To do it this way, note the following OPEs:
	\[
		\begin{aligned}
			j_B(z) b(w) &\sim \frac{T_{matter}(z)}{z-w} -\frac{1}{(z-w)^2} \left(b c (z) + \frac34 \beta \gamma(z)\right) + \frac{1}{z-w} \left(-b \d c(z) + \frac14 \d \beta \gamma(z) - \frac34 \beta \d \gamma(z) \right)\\
			&= \dots + \frac{1}{z-w} \left[T_{matter}(z) - \d b\, c(w) - 2 b \d c(w) - \frac12 \d \beta \gamma(w) - \frac32 \beta \d \gamma(w) \right]\\
			&= \dots + \frac{T_{matter}(w) + T_{gh}(w)}{z-w} \Rightarrow \{Q_B, b_n\} = L_n
		\end{aligned}
	\]
	Similarly
	\[
		j_B(z) \beta(w) = \dots + \frac{G_{matter}(w) + G_{gh}(w)}{z-w} \Rightarrow [Q_B, \beta_n] = G_n
	\]
	Now note that the Jacobi identity on $Q_B$ reads:
	\[
		\{[Q_B, L_m], b_n \} - \{\overbrace{[L_m, b_n]}^{(m-n)b_{m+n}}, Q_B\} - [\overbrace{\{b_n, Q_B \}}^{L_n}, L_m] = 0 \Rightarrow \{[Q_B, L_m], b_n \} = (m-n) L_{m+n} - [L_m, L_n]
	\]
	So if the total central charge is zero we'll get $\{[Q_B, L_m], b_n \} = 0$, implying that $[Q_b, L_m]$ is independent of the $c$ ghost. But on the other hand this operator has ghost number $1$, so it must therefore vanish. Further, the Jacobi identity also yields
	\[
		[\{Q_B, Q_B\}, b_n] = - 2 [\{b_n , Q_B\}, Q_B] = 2 [Q_B, L_n]
	\]
	since we just showed that this last term vanishes, we must have $Q_B, Q_B$ is also independent of $c$, but again since $Q_B^2$ has positive ghost number, we get that it is in fact zero. We can do the same argument with $\beta$ and $G$ and get that the superstring BRST operator is zero, as long as the total central charge vanishes. This was much cleaner than the OPE way. 

	
	\item
	First a lemma: An abelian $p$-form field $A$ has ${D - 2 \choose p}$ on shell DOF. To prove this, note that we have a gauge symmetry of $A \to A + \d \Lambda$ which has ${D \choose p-1}$ parameters.
	Next, the Euler-Lagrange equations give us that the components $A^{0 i_1 \dots i_{p-1}}$ are non-propagating. We thus get ${D-1 \choose p}$ massless propagating off-shell d.o.f. which have ${D-2 \choose p-1}$ gauge symmetries left over. These can be used to enforce Coulomb gauge conditions which allow for there to be no polarizations along one of the spatial directions. 
	We thus get ${D-1 \choose p} - {D-2 \choose p-1} = {D - 2 \choose p}$ massless on-shell degrees of freedom. For $A_\mu$ this is $D-2$ and for $B_{\mu \nu}$ this is $(D-2)(D-3)/2$.
	
	The metric has $\frac12 D (D-3)$ on-shell degrees of freedom. There are two ways to see this, first, that the dynamically allowed variation $\delta g$ may on-shell be described by a symmetric traceless tensor in dimension $D-2$ which gives
	\[
		\frac{(D-1)(D-2)}{2} - 1 = \frac12 D (D-3)
	\]
	or by noting that since we are gauging translation symmetry locally, each translation makes $2$ polarizations unphysical and so we get:
	\[
		\frac{D(D+1)}{2} - 2 D = \frac12 D (D-3)
	\]
	as required.
	
	We now consider the R-R, R-NS, NS-R, NS-NS sectors together. For NS-NS we have the scalar $=1$ both on-shell and off-shell, the antisymmetric two-form, which has only transverse degrees of freedom $=8 * 7/2 = 28$ and the gravity, $= 10 * 7/2 = 35$ altogether we get $64$ on-shell degrees of freedom.

	In both the R-NS and NS-R sector, we have a Weyl representation of dimension $2^{5-1} = 16$. There are however only $8$ on-shell degrees of freedom. Similarly, we only consider the on-shell $\psi^\mu_{-1/2}$ acting on the NS part of the vacuum which gives another factor of $8$. This gives $64$ fermionic variables in each sector for a grand total of $128$.

	In R-R for IIA we have a 0, 2, and \emph{self-dual} 4-form. This gives:
	\[
	1 + {8 \choose 2} + \frac12 {8 \choose 4} = 64
	\]
	For IIB we have a 1 and 3-form. This gives
	\[
	{8 \choose 1} + {8 \choose 3} = 64
	\]
	so in either case we have 64 on-shell degrees of freedom here. This is consistent with each $\ket{S}$ state having $8$ on-shell degrees of freedom giving $8 \times 8 = 64$. All together, we have the same number of on-shell fermionic and bosonic degrees of freedom. 

	Now for the massive case. In the NS sector you might expect the next excitations come from the bosons $\alpha_{-1}$, but this gets projected out by GSO, so in fact the next states come from $\psi_{-3/2}^i$, $C_{ijk} \psi_{-1/2}^i \psi_{-1/2}^j \psi_{-1/2}^k$ and $C_{ij} \psi_{-1/2}^i \alpha_{-1}^j$. These have dimensions $8+56+64 = 128$, which decomposes as the traceless symmetric \textbf{44} and three-index antisymmetric \textbf{84} representation of $\SO(9)$. In the R sector, we must look at $\alpha_{-1}^i \ket{S_\alpha}$ and $\psi_{-1}^i \ket{C_\alpha}$ for $S_\alpha, C_\alpha$ suitably chosen so that the state satisfies $G_0 = 0$. This constraint gives a factor of two reduction for the dimension of the space of candidate $S_\alpha$. Consequently, we get $\mathbf{8}_v \otimes \mathbf{8}_s \oplus \mathbf{8}_v \otimes \mathbf{8}_{s'}$ which has dimension $128$. This indeed turns out to be a spinor representation of $\SO(9)$, and it comes from looking at the tensor product of \emph{the} fundamental spinor representation with the vector representation $\mathbf{16}_s \otimes \mathbf{9}_v$. This turns must decompose as a sum of two spinor representations $\mathbf{16}_s \oplus \mathbf{128}_s$. One is again the fundamental, while the other is the required \textbf{128}. 
	
	For the massive states in the type IIA and type IIB, we must tensor we wish to look at the lowest-level masses. Note we must match massive states with massive states. In this case, we match $2/\alpha$ on both sides to get massive states of mass $4/\alpha$. Since the particles already organize into representations of $\SO(9)$ on each side, the closed string massive spectrum will again clearly organize intro representations of $\SO(9)$. Also since fermionic and bosonic degrees of freedom already were equal on each side, they will be equal in the closed string as well. We will have $2 \times 128^2 = 32768$ bosonic and fermionic degrees of freedom. 
	 % Consequently, the NS-NS sector will have a spin 6, spin 4, and two spin 5 massive particles. 	From R-NS and NS-R I can tensor the R vacuum $\ket{S_\alpha}$ or $\ket{C_\alpha}$ with the NS states and get two copies of $\mathbf{8}^4$ and $\mathbf{8}^3$. These will appropriately combine to give representations of $\mathrm SO(9)$. \textbf{SHOW THIS PART}

	
	\item In terms of theta functions:
	\[
	\begin{aligned}
		\chi_O &= \frac12 \left(\prod_{i=1}^4 \frac{\theta_3(\nu_i)}{\eta} - \prod_{i=1}^4 \frac{\theta_4(\nu_i)}{\eta} \right)\\
		\chi_V &= \frac12 \left(\prod_{i=1}^4 \frac{\theta_3(\nu_i)}{\eta} + \prod_{i=1}^4 \frac{\theta_4(\nu_i)}{\eta} \right)\\
		\chi_S &= \frac12 \left(\prod_{i=1}^4 \frac{\theta_2(\nu_i)}{\eta} - \prod_{i=1}^4 \frac{\theta_1(\nu_i)}{\eta} \right)\\
		\chi_C &= \frac12 \left(\prod_{i=1}^4 \frac{\theta_2(\nu_i)}{\eta} + \prod_{i=1}^4 \frac{\theta_1(\nu_i)}{\eta} \right)\\
	\end{aligned}
	\]
	We'll take $\nu_i = 0$ here \textbf{(I assume this is what I'm supposed to do)} and so $\theta_1 = 0 \Rightarrow \chi_S= \chi_C$.
	
	For IIB we look at 
	\[
		\frac{|\chi_V - \chi_C|^2}{(\sqrt \tau_2 \eta \bar \eta)^8} = \frac{1}{(\sqrt \tau_2 \eta \bar \eta)^8} \frac12 \sum_{a,b=0}^1 (-1)^{a+b} \frac{\theta^4\twist{a}{b} }{\eta^4}
		\times \frac12 \sum_{\bar a, \bar b=0}^1 (-1)^{\bar a + \bar b} \frac{\bar \theta^4 \twist{\bar a}{\bar b}}{\bar \eta^4}
	\]
	Under modular transformations $\tau \to \tau+1$ $\theta^4\twist01 \leftrightarrow \theta^4\twist00$, $\theta^4\twist10\to-\theta^4\twist10$ while $\eta^{12} \to -\eta^{12}$. In the holomorphic and anti-holomorphic parts separately, each term in the sum picks up a minus sign that is cancelled by the minus sign in the $\eta^4$. 
	
	Under $\tau \to -1/\tau$, the $\frac1{(\sqrt{\tau_2} \eta \bar \eta)^8}$ out front is invariant. On the other hand, the $\theta$ functions transform as $\theta^4\twist00 \to (-i\tau)^2 \theta^4\twist00$, $\theta^4\twist01 \to (-i\tau)^2 \theta^4\twist10$, $\theta^4\twist10 \to (-i\tau)^2 \theta^4\twist01$. These are exactly compensated by the $\eta$ transformations in the denominator, and no overall sign is picked up
	
	For IIA we have similarly
	\[
		\frac{(\chi_V - \chi_C)(\bar \chi_V - \bar \chi_S)}{(\sqrt \tau_2 \eta \bar \eta)^8} = \frac{1}{(\sqrt \tau_2 \eta \bar \eta)^8} \frac12 \sum_{a,b=0}^1 (-1)^{a+b} \frac{\theta^4\twist{a}{b} }{\eta^4}
		\times \frac12 \sum_{\bar a, \bar b=0}^1 (-1)^{\bar a + \bar b + \bar a \bar b} \frac{\bar \theta^4 \twist{\bar a}{\bar b}}{\bar \eta^4}
	\]
	Again, the holomorphic part transforms as before and as we have set the $\nu_i$ to zero, we have the same partition function. Using \textbf{D.18}, we see that each of the four above sums are zero since they are equal to a product of $\theta_1 = 0$.
	
 	\item Again, these are identical if I set the $\nu_i = 0$ (am I not supposed to be doing this? What do the $\nu_i$ represent physically?). They are equal to
	\[
		\frac{1}{(\sqrt {\tau_2} \eta \bar \eta)^8 4 \eta^4 \bar \eta^4} (\cancel{|\theta_1^4|^2} + |\theta_2^4|^2 + |\theta_3^4|^3 + |\theta_4^4|^2)
	\]
	We have $\theta_3$ and $\theta_4$ swapping under $\tau \to \tau+1$, generating no signs in this case, while the denominator looks like $|\eta|^{24}$ and also doesn't generate a sign. Then, under $\tau \to -1/\tau$ we have $\theta_2$ and $\theta_4$ swapping generating a $|\tau|^4$, identical to what is generated by the $(\eta \bar \eta)^4$. 
	
	\item The partition function is
	\[
		Z_{\mathrm{SO}(16) \times \mathrm{SO}(16)}^{\mathrm{het}} 
		= \frac12 \sum_{h,g} \frac{\bar Z_{E_8}\twist hg ^2}{(\sqrt{\tau_2} \eta \bar \eta)^8} \frac12 \sum_{a,b} (-1)^{a+b+ab+ ag+bh+gh} \frac{\theta^4 \twist ab}{\eta^4}, 
		\quad \bar Z_{E_8}\twist hg = \frac12 \sum_{\gamma, \delta} (-1)^{\gamma g + \delta h} \frac{\bar \theta^8 \twist \gamma \delta}{\bar \eta^8}
	\]
	First look at $\bar Z_{E_8}$. Under modular transformations $\tau \to -1/\tau$ we get $\bar Z_{E_8}\twist hg \to \bar Z_{E_8} \twist gh$. Under $\tau \to \tau + 1$, we get $\bar Z_{E_8}\twist hg \to (-1)^{h-2/3} \bar Z_{E_8}\twist{h}{g+h}$. With this, we can look at $Z_{\mathrm{SO}(16) \times \mathrm{SO}(16)}^{\mathrm{het}}$ under $\tau \to -1/\tau$
	\[
		\frac12 \sum_{h,g} \frac{\bar Z_{E_8}\twist gh ^2}{(\sqrt{\tau_2} \eta \bar \eta)^8} \frac12 \sum_{a,b} (-1)^{a+b+ab+ ag+bh+gh} \frac{\theta^4 \twist ba}{\eta^4}
	\]
	Under relabeling of $a\leftrightarrow b, g\leftrightarrow h$, this is the same. Next, under $\tau \to \tau+1$:
	\[
	\begin{aligned}
		&\frac12 \sum_{h,g} \frac{(-1)^{-4/3} \bar Z_{E_8}\twist{h}{g+h}^2}{(-1)^{4/3} (\sqrt{\tau_2} \eta \bar \eta)^8} \frac12 \sum_{a,b} (-1)^{a+b+ab+ ag+bh+gh} \frac{(-1)^a \theta^4 \twist{a}{a+b-1}}{(-1)^{1/3} \eta^4}\\ 
		&= \frac12 \sum_{h,g} - \frac{ \bar Z_{E_8}\twist{h}{g+h}^2}{(\sqrt{\tau_2} \eta \bar \eta)^8} \frac12 \sum_{a,b} (-1)^{b+ab+ ag+bh+gh} \frac{\theta^4 \twist{a}{a+b-1}}{\eta^4}\\
		&=	\frac12 \sum_{h,g'} \frac{ \bar Z_{E_8}\twist{h}{g'}^2}{(\sqrt{\tau_2} \eta \bar \eta)^8} \frac12 \sum_{a,b} (-1)^{1+b+ab+ ag' + (a+b) h +g'h + h} \frac{\theta^4 \twist{a}{a+b-1}}{\eta^4}\\
		&= \frac12 \sum_{h,g'} \frac{ \bar Z_{E_8}\twist{h}{g'}^2}{(\sqrt{\tau_2} \eta \bar \eta)^8} \frac12 \sum_{a,b} (-1)^{\cancel 1 + (b' + a + \cancel 1) + (ab' + \cancel a - \cancel a) + ag' + (b' h + \cancel{h}) +g'h + \cancel h } \frac{\theta^4 \twist{a}{b'}}{\eta^4}\\
		&= \frac12 \sum_{h,g'} \frac{ \bar Z_{E_8}\twist{h}{g'}^2}{(\sqrt{\tau_2} \eta \bar \eta)^8} \frac12 \sum_{a,b} (-1)^{a + b' + a b' + a g' + b' h + g' h} \frac{\theta^4 \twist{a}{b'}}{\eta^4}\\
	\end{aligned}
	\]
	Keep in mind that $x^2 = x\, \mathrm{mod}\, 2$. 
	
	Before we do the next part, let's elaborate on why $Z_{E_8} = \frac12 \sum_{a, b} \theta^8 \twist ab$ is the partition function of the $E_8$ lattice. From the sixteen fermion picture, this is just the $(-1)^F = 1$ in the NS sector (corresponding to the $\chi_O = \frac12 (\theta^8 \twist00 + \theta^8 \twist01)$ character) together with the R sector $\chi_S = \frac12 \theta^8 \twist10$ giving the spinor representation.
	 
	Indeed, the roots of $E_8$ consist of the roots of $\mathrm O(16)$ as well as the spinor weights of $\mathrm O(16)$. Note that the spinor representation comes from the half-integral points, corresponding to $\theta \twist10$ in the sum, while the adjoint representation comes from $\theta \twist 01$ and $\theta \twist 00$. Consequently the action of $\mathcal S_i$ that fixes the adjoint vectors but flips the sign of the spinor acts on our partition function as $\mathcal S_i Z_{E_8} = \frac12 \sum_{a, b} (-1)^{a}\, \theta^8 \twist ab$. It of course also gives rise to a twisted sector, so altogether we get the four twisted blocks $\bar Z_{E_8} \twist hg$ as required.
	
	Since we have projected out the spinor representation, the current algebra only contains the NS currents $\bar J^{ij}$ corresponding to the adjoint of $\mathrm{SO}(16)$, and we have two copies of this for each group of 16 fermions. 
	
	From the factor of $(\sqrt \tau_2 \eta \bar \eta)^{-8}$ we see that we have $8$ on-shell noncompact massless bosonic excitations as well as all of their descendants (on both left and right moving sides). We also see on the left-moving side we get a theta-function corresponding to $N=8$ fermions transforming under a spacetime $\mathrm{SO}(8)$, forming the superpartners of the bosons. On the right side instead of the superpartner fermions, we have the $16$ internal fermions that transform in the adjoint representations. 
	
	Let's see what massless states we can build. In the NS sector of the left-movers, we have $L_0 = 1/2, \bar L_0 = 1$ and so we get $\psi^i_{-1/2} \alpha^j_{-1} \ket p$ which gives us our usual graviton, two-form field, and dilaton. We also have $\psi^i_{-1/2} \bar J^a_{-1} \ket{p}$ for the $\mathrm{O}(16) \times \mathrm{O}(16)$ currents. This gives us vectors corresponding to gauge bosons valued in the adjoint of $\mathrm{O}(16) \times \mathrm{O}(16)$ as required. 
	
	In the R sector we have $G_0 = 0, \bar L_0 = 1$ we'll get a gravitino, fermion, and gaugino as before, but again this time valued in $\mathrm{O}(16) \times \mathrm{O}(16)$.
	
	\item Because we have seen that T-duality flips the antichiral $U(1)$ $\bar \d X \to -\bar \d X$, and we want to preserve the (1,1) supersymmetry $G$ in the type II string (and so must keep it as a periodic variable \textbf{Why is this absolutely necessary. Can we not work with double covers in some clever way when defining supercurrents?} ), we must consequently flip $\bar \psi$. This corresponds to inserting $(-1)^{F_R}$. For the right-moving R sector, this changes the chirality of the R spinor, taking $S_\alpha \to \Gamma^9 \Gamma^{11} S_\alpha$ (there can be no phase, by reality conditions of $\Gamma$). We thus flip IIA to IIB and vice versa.
	
	From this we get that 
	\[
		F_{\alpha \beta} = S_{\alpha} (\Gamma^{0})_{\beta \gamma} \tilde S_\gamma \to S_{\alpha} (\Gamma^{0} \Gamma^{9} \Gamma^{11})_{\beta \gamma} \tilde S_\gamma = - \xi \, S_{\alpha} (\Gamma^{9} \Gamma^{0})_{\beta \gamma} \tilde S_\gamma = - \xi F \Gamma^{9}
	\]
	Expanding in terms of the $F_{\mu_1 \dots \mu_k}$ gives the action:
	\[
		F_{\alpha \beta} \to - \xi \sum_{k=0}^{10} \frac{(-1)^k}{k!} F_{\mu_1 \dots \mu_k} \Gamma^{\mu_1 \dots \mu_k} \Gamma^{9}
	\]
	This gives that
	\[
		\tilde F_{\mu_1 \dots \mu_k, 9} = - \xi F_{\mu_1 \dots \mu_k}, \quad \tilde F_{\mu_1 \dots \mu_k} = F_{\mu_1 \dots \mu_k, 9}
	\]
	Then
	\[
		\d_{\mu_1} \tilde C_{\mu_2 \dots \mu_k 9} = -\xi \d_{\mu_1} C_{\mu_2 \dots \mu_k}, \quad \d_{\mu_1} \tilde C_{\mu_2 \dots \mu_k} = \d_{\mu_1} \tilde C_{\mu_2 \dots \mu_k 9}
	\]
	so that (up to a closed term)
	\[
		\tilde C_{\mu_1 \dots \mu_{p-1} 9}^{(p)} = -\xi C_{\mu_1 \dots \mu_{p-1}}^{p-1}, \quad \tilde C^{(p)}_{\mu_1 \dots \mu_p} = C^{(p+1)}_{\mu_1 \dots \mu_p 9}
	\]
	\textbf{Get rid of the $\xi$ factor}
	
	\item We have that $\Omega \ket{S_\alpha \tilde S_\beta} = \varepsilon_R \ket{S_\beta \tilde S_\alpha}$. Further, it acts trivially on $\Gamma^0$ \textbf{(you sure?)}. Now, in the operator language we will have $\Omega S_\alpha \Omega^{-1} = \epsilon_1 \tilde S_\alpha$ and $\Omega \tilde S_\beta \Omega^{-1} = \epsilon_2 S_\beta$. In any case, we must have for the bi-spinor that $\Omega S_\alpha \tilde S_\beta \Omega^{-1} = \epsilon_R S_\beta \tilde S_\alpha$, which gives that $\epsilon_1 \epsilon_2 = -\epsilon_R$ Thus, we have:
	\[
		\Omega F_{\alpha \beta} \Omega^{-1} = \Omega S_\alpha \Gamma^0_{\beta \gamma} \tilde S_\gamma \Omega^{-1} = - \epsilon_R \Gamma^0_{\beta \gamma} S_{\gamma} \tilde S_\alpha = - \epsilon_R \Gamma^0_{\beta \gamma} F_{\gamma \delta} \Gamma^0_{\delta \alpha} = -\epsilon_R (\Gamma^0 F \Gamma^0)_{\beta \alpha} =  -\epsilon_R (\Gamma^0 F^T \Gamma^0)_{\beta \alpha }
	\]
	\textbf{I think 7.3.3 of Kiritsis has the derivation wrong. Ask Nathan/Xi.}
	\item When we take $\epsilon_R = -1$ the scalar and four-index self-dual tensor survive. In this case, we will \emph{not} have consistent interactions. Since the graviton survives, there must be an equal number of massless bosonic and fermionic excitations. The fermions come just from the NS-R sector (there is no R-NS now), giving 64 on-shell fermionic excitations. From the NS-NS sector, the dilaton and gravity will give $1+35 = 36$ on-shell bosonic degrees of freedom. We are missing 28 bosonic degrees of freedom. 
	
	The scalar and four-index self dual tensor contribute $1 + \frac12 \frac{8 \times 7 \times 6 \times 5}{4!} = 36$ on-shell bosonic degrees of freedom. This is too much. The two-form, on the other hand, contributes the requisite $8 \times 7 / 2= 28$. Consistency of interaction thus \emph{demands} we keep only the 2-form and drop the 0 and self-dual 4-form. This necessitates $\epsilon_R = 1$.
	
	\item We are just looking at the \emph{open} superstrings here. Any open string that consistently couples to type I or type II string theory must have a GSO projection as well. We have already seen how the oriented open strings look like in exercise \textbf{7.3}. In the NS sector we have at $-p^2 = m^2 = 2/\ell_s^2$
	\begin{equation}\label{eq:NStype1}
		\begin{aligned}
			&\psi_{-3/2}^i \lambda_{ab} \ket{p; ab}_{NS}\\
			&C_{ijk} \psi_{-1/2}^i \psi_{-1/2}^j \psi_{-1/2}^k \lambda_{ab} \ket{p; ab}_{NS}\\
			&C_{ij} \psi_{-1/2}^i \alpha_{-1}^j \lambda_{ab} \ket{p; ab}_{NS}
		\end{aligned}
	\end{equation}
	 In the $R$ sector we have (for $S_\alpha$ suitably chosen so that the state satisfies $G_0 = 0$):
	 \begin{equation}\label{eq:Rtype1}
	 	\begin{aligned}
	 		& \alpha_{-1}^i \lambda_{ab} \ket{S_\alpha; ab}_{R}\\
			& \psi_{-1}^i \lambda_{ab} \ket{C_\alpha; ab}_{R}
	 	\end{aligned}
	 \end{equation}
	I will assume NN boundary conditions. In this case
	\[
	\begin{aligned}
		\Omega \alpha_{-1} \Omega^{-1} &= - \alpha_{-1}\\
		\Omega \psi_{-1} \Omega^{-1} &= - \psi_{-1}\\
		\Omega \psi_{-\frac12} \Omega^{-1} &= -i \psi_{-\frac12}\\
		\Omega \psi_{-\frac32} \Omega^{-1} &= i \psi_{-\frac32}
	\end{aligned}
	\]
	So all of the terms in \eqref{eq:NStype1} are terms of the form $\mathcal A \lambda_{ab} \ket{p; ab}_{NS}$ with the operator $\mathcal A$ transforming as $\mathcal A \to i \mathcal A$ under parity. Doing parity twice therefore will generate a $- \epsilon_{NS}^2 \mathcal A (\gamma {\gamma^T}^{-1})_{ii'} \ket{p; a'b'} (\gamma^T {\gamma}^{-1})_{j'j}$. This is exactly the same as in \textbf{7.3.10}.  % Imposing that this is identical will give that $\gamma = \zeta \gamma^T$, $\zeta^2 \epsilon_{NS}^2 = -1$.
	Demanding that $\Omega$ act on the state with eigenvalue $+1$ will make it so that $\lambda = i \epsilon_{NS} \gamma \lambda^{T} \gamma^{-1}$. We already have $\epsilon_{NS} = -i$ so $\lambda = \gamma \lambda^T \gamma^{-1}$ here. % Using the same arguments as in the section, we will conclude that $\epsilon_{NS}=-i$ (this is always true for all levels) and $\zeta^2 = 1$.
	Imposing the tadpole cancelation condition $\zeta = 1$ and we get gauge group $\SO(32)$. So we get that states at this level will transform in the  \emph{the traceless symmetric tensor + singlet representation} of $SO(32)$.
	
	All of the terms in \eqref{eq:Rtype1} will transform under parity twice as as $\epsilon_R^2 \mathcal A (\gamma {\gamma^T}^{-1})_{ii'} \ket{S_\alpha; a'b'} (\gamma^T {\gamma}^{-1})_{j'j}$. We will have the same $\gamma$ matrix as in the NS sector, as required for consistency of interactions. Here, though, we will get $\epsilon_R = -1 \Rightarrow \epsilon_R^2 = 1$ and we will get $\lambda = -\gamma \lambda^T \gamma^{-1}$ (this is what we got from the massless sector with an extra minus sign since $\psi_{-1}, \alpha_{-1}$ now transform with minus signs). Again we will have that these states will transform in the symmetric representation of $\SO(32)$. 

	Again we get $128$ bosonic states that will transform as the $\mathbf{44} \oplus \mathbf{84}$ representation of $\SO(9)$. We will also get fermions transforming in the $\mathbf{128}$ spinor representation as in exercise \textbf{3}. 
	All of these states will transform in the traceless symmetric representation of $\SO(32)$. \textbf{Confirm}
	
	
	\item Certainly in the untwisted sector, the theory we get corresponds to tracing over the projection operator $\frac12 (1 + g)$ where $g$ is orientation-reversal. Now in the twisted sector, we still have $X^\mu$ satisfies the Laplace equation $\d_+ \d_- X = 0$ so we can write
	\[
		X(\sigma, \tau) = x^\mu + \tau \ell_s^2 \frac{p^\mu + \bar p^\mu}{2} + \sigma \ell_s^2 \frac{p^\mu - \bar p^\mu}{2} + \frac{i \ell_s}{\sqrt 2} \sum_{n} \left( \frac{\alpha_n}{n} e^{-i n (\tau  + \sigma)} + \frac{\tilde \alpha_n}{n} e^{-i n (\tau  + \sigma)} \right)
	\]
	The condition that $X(\sigma + 2\pi) = X(2\pi-\sigma)$ give that $p^\mu = \bar p^\mu$ and the $\sigma$ term vanishes. We must have $n$ is a half integer. For integer modding we have $e^{-i n (\tau \pm \sigma)} \to e^{-i n (\tau \mp \sigma)}$. For half-integer modding we have $e^{-i n (\tau \pm \sigma)} = (-1)^n e^{-i n (\tau \mp \sigma)}$. We should thus have $\alpha_n = \tilde \alpha_n$ for $n$ integral and $\alpha_{n} = -\tilde \alpha_{n}$ We thus get
	\[
			X(\sigma, \tau) = x^\mu + 2 \ell_s^2 p^\mu \tau + \sigma  i \sqrt 2\ell_s \sum_{n \in \ZZ \setminus \{0\}} \frac{\alpha_n}{n} \cos(n \sigma) e^{- i n \tau}  - \sqrt{2} \ell_s \sum_{n \in \ZZ + \frac12} \frac{\alpha_n}{n} \sin(n \sigma) e^{-i n \tau}
	\]
	This is the twisted sector. The last sum picks up a minus sign under orientation reversal, and so will be projected out. We are left with the equations of motion for the open string.
	
	\item In NS we have (up to an overall irrelevant factor of $i^{-1/2}$)
	\[
		\psi(\sigma, \tau) = \sum_{n\in \ZZ} \psi_{n+1/2} e^{(n+1/2) (\tau + i \sigma)}, \quad \bar \psi(\sigma, \tau) = \sum_{n\in \ZZ} \bar \psi_{n+1/2} e^{(n+1/2) (\tau - i \sigma)}
	\]
	In the closed string case have that $\Omega \psi_{n+1/2} \Omega^{-1} = \bar \psi_{n+1/2}$. 
	Given that $\Omega \psi(\sigma, \tau) \Omega^{-1} = \bar \psi(\pi - \sigma, \tau)$, we directly get $\Omega \psi_{n+1/2} \Omega^{-1} = i (-1)^n \bar \psi_{n+1/2}$. For DD boundary conditions we get an extra minus sign to this, since there $\Omega \psi(\sigma, \tau) \Omega^{-1} = - \bar \psi(\pi - \sigma, \tau)$. 
	
	In the R sector we have
	\[
		\psi(\sigma, \tau) = \sum_{n\in \ZZ} b_n e^{n (\tau + i \sigma)}, \quad \bar \psi(\sigma, \tau) = \sum_{n\in \ZZ} \bar b_n e^{n (\tau - i \sigma)}
	\]
	Following the same logic we get that $\Omega \psi_{n} \Omega^{-1} = (-1)^n \psi_n$ for NN and $\Omega \psi_{n} \Omega^{-1} = -(-1)^n \psi_n$ for DD.
	
	All of these cases can be summarized by
	\[
		\begin{aligned}
			\text{NN:}\quad & \Omega \psi_r \Omega^{-1} = (-1)^r \psi_r \\ 
			\text{DD:}\quad & \Omega \psi_r \Omega^{-1} = - (-1)^r \psi_r.
		\end{aligned}
	\]
	\item Let's clarify a bit of terminology before we begin. We are looking at just the fermions of the left moving and right moving sides of the heterotic string theory. On the left-hand (supersymmetric) side, in the light-cone gauge these form an $\widehat{O(8)}$ current algebra at \emph{level 1}. On the right-hand side the form a $\widehat{O(32)}$ current algebra at level 1 again (\textbf{why must we always have level 1? Ask Xi.}). 
	
	The characters of $\widehat{O(N)}_1$ for $N$ even correspond to the integrable representations labeled by $O, V, S, C$ corresponding to the trivial, vector, spinor, and conjugate spinor. For our purposes (ie the heterotic string), we do not need to distinguish between $S$ and $C$, which will have the same character. The characters can be written in terms of $\theta$ functions as
	\[
		\chi_O = \frac12 \left[\left(\frac{\theta_3}{\eta}\right)^{N/2} + \left(\frac{\theta_4}{\eta}\right)^{N/2} \right], \quad
		\chi_V = \frac12 \left[\left(\frac{\theta_3}{\eta}\right)^{N/2} - \left(\frac{\theta_4}{\eta}\right)^{N/2} \right], \quad
		\chi_S = \frac12 \left(\frac{\theta_2}{\eta}\right)^{N/2}
	\]
	\begin{enumerate}
		\item Now let us first look at $O(32)$. 
	The $\widehat{O(8)}_1$ characters transform under $\tau \to \tau+1$ as 
	\[
		\chi_O^8 \to (-1)^{-1/6} \chi_O, \quad \chi_V^8 \to - (-1)^{-1/6} \chi_V, \quad \chi_S^8 \to - (-1)^{-1/6} \chi_S
	\]
	And under $\tau \to -1/\tau$ they transform as
	\[
			\chi_O^8 \to \frac12 (\chi_O^8 + \chi_V^8) + \chi_S^8, 
			\quad \chi_V^8 \to \frac12 (\chi_O^8 + \chi_V^8) - \chi_S^8,
			\quad \chi_S^8 \to \frac12 (\chi_O^8 - \chi_V^8)
	\]
	The $\widehat{O(32)}_1$ characters \emph{depending on $\bar q$} transform the same way under $\tau \to -1/\tau$ but under $\tau \to \tau + 1$ transform as
	\[
		\chi_O^{32} \to (-1)^{2/3} \chi_O^{32}, \quad \chi_V^{32} \to - (-1)^{2/3} \chi_V^{32}, \quad \chi_S^{32} \to (-1)^{2/3} \chi_S^{32}
	\]
	Our partition functions in question can be constructed from a linear combination of products of exactly one $\widehat{O(8)}_1$ and one $\widehat{O(32)}_1$ character. This gives 9 possible terms $\chi_i^8 {\chi^{32}_j}^*$. I label these in the table below. I cancel all terms that are not invariant under $\tau \to \tau + 1$. 
	\[
		\begin{tabular}{c c c}
			\cancel{O O} & O V & \cancel{O S}\\
			V O & \cancel{V V} & V S\\
			S O & \cancel{S V} & S S
		\end{tabular}
	\]
	But we are not done. It is easy to see that while that $\chi_V^8, \chi_S^8$ blocks have Taylor series $O(q^{1/3})$, the $\chi_O$ block contains a singular term going as $q^{-1/6}$. Similarly, $\chi_O^{32}$ contains a singular term going as $O(q^{-2/3})$ while $\chi_V^{32} = O(\bar q^{-1/6})$ and $\chi_{S}^{32} = O(\bar q^{4/3})$. The tachyon can come exactly (and only!) from combining $\chi_O^{8} {\chi_V^{32}}^*$ to get $1/|q|^{1/6}$ that will be singular and satisfy level-matching. Thus we must drop $O V$ above as well. We are left with four possible terms that can work. 
	
	Modular invariance under $\tau \to 1/\tau$ further constrains this to take a form proportional to
	\[
		(\chi^8_V - \chi^8_S)(\chi^{32}_1 + \chi^{32}_S)
	\]
	The normalization of the identity to 1 fixes this entirely. Note that we get spin statistics for \emph{free}, as the only character combinations appearing with a minus sign are precisely those containing $\chi^8_S$, associated with the spacetime fermions. 
	\item Having $\widehat{O(32)}_1$ out of the way, let's move on to $O(16) \times E_8$. $\widehat{E_8}_1$ has only one integrable representation and thus one corresponding character, $\chi^{E_8}$. As pointed out in the text, it is related to the characters of $\widehat{O(16)}_1$ by $\chi^{E_8} = \chi_O^{16} + \chi_S^{16}$. Thus we have trilinear combinations $\chi_i^{8}(q) \chi_j^{16}(\bar q) \chi^{E_8} (\bar q)$. Upon noting that the characters of $\widehat{O(16)}_1$ multiplied by $\chi_{E_8}$ transform the \emph{same way} under modular transformations as $\widehat{O(32)}_1$ and the \emph{same} combination $\chi^8_1 \chi^{16}_V \chi^{E_8}$ uniquely gives the tachyon, we see that \emph{again} the argument goes as before and the only viable character we can have is
	\[
		(\chi^8_V - \chi^8_S)(\chi^{32}_1 + \chi^{32}_S) \chi^{E_8} = (\chi^8_V - \chi^8_S) \chi^{E_8 \times E_8}.
	\]
	This is exactly the heterotic E string theory. 
	\item Finally we get to the hard one: $O(16) \times O(16)$. Here we have 27 trilinear terms that can contribute. I will write them out, and again cross out the ones that are not invariant under $\tau \to \tau+1$ as well as double crossing out the tachyons. Here, though, the notation $OO, OV, SV$ etc will represent \emph{just} the right-moving characters $\chi_O^{16}(\bar q) \chi_O^{16} (\bar q), \chi_O^{16}(\bar q) \chi_V^{16} (\bar q), \chi_S^{16}(\bar q) \chi_V^{16} (\bar q)$ respectively. 
	\begin{center}
	\[
		\chi^{8}_O \times 		\begin{tabular}{c c c}
			\cancel{O O} & \xcancel{O V} & \cancel{O S}\\
			\xcancel{V O} & \cancel{V V} & V S\\
			\cancel{S O} & SV & \cancel{S S}
		\end{tabular},
		\qquad \chi^{8}_V \times 		\begin{tabular}{c c c}
			O O & \cancel{O V} & O S\\
			\cancel{V O} & V V & \cancel{V S}\\
			S O & \cancel{S V} & S S
		\end{tabular},
		\quad \chi^{8}_S \times 		\begin{tabular}{c c c}
			O O & \cancel{O V} & O S\\
			\cancel{V O} & V V & \cancel{V S}\\
			S O & \cancel{S V} & S S
		\end{tabular}
	\]
	\end{center}
	It may look that 12 independent terms remain. The fact that the characters are symmetric under exchange of the last two labels mean that there are in fact only 12. 
	
	Let us look at two cases. First, assume $\chi^8_O {\chi_V^{16}}^* {\chi_S^{16}}^*$ does \emph{not} contribute (ie its coefficient vanishes). Then the first 9 terms are all zero. The remaining constraint of modular invariance under $\tau \to \tau+1$ constrains the partition function to take the form
	\[
		(\chi_V - \chi_S) \left[\chi^{16}_O \chi^{16}_O + 2 \alpha \chi^{16}_O \chi^{16}_S + (1-\alpha) \chi^{16}_V \chi^{16}_V + (2-\alpha) \chi^{16}_S \chi^{16}_S\right]
	\]
	for any value of $\alpha$. Spin statistics requires all these contributions to come in with positive coefficient, so $0 \leq \alpha \leq 1$. Moreover, if $\alpha$ is non-integral we will have coefficients that are not integers in the character expansion, which would lack a Hilbert space interpretation \textbf{Think more about the integrality condition}. Thus we can have only $\alpha = 0$ and $\alpha =1$ corresponding exactly to the $O(32)$ and $E_8 \times E_8$ superstrings. 
	
	So our remaining possibility is that $\chi^8_O {\chi_V^{16}}^* {\chi_S^{16}}^*$ does \emph{not} have vanishing coefficient. WLOG set this coefficient to 1. Invariance under $\tau \to -1/\tau$ constrains us to:
	\[
	\begin{aligned}
		2 \chi_O^8 \chi_V^{16} \chi_S^{16} &+ \chi_V^8 \left[\alpha \chi^{16}_O \chi^{16}_O + 2 \beta \chi^{16}_O \chi^{16}_S + (-1+\alpha-\beta) \chi^{16}_V \chi^{16}_V + (-1+2\alpha - \beta) \chi^{16}_S \chi^{16}_S\right]\\
		& + \chi_S^8 \left[(1-\alpha) \chi^{16}_O \chi^{16}_O - 2 (1+\beta) \alpha \chi^{16}_O \chi^{16}_S + (-\alpha + \beta) \chi^{16}_V \chi^{16}_V + (1-2\alpha +\beta) \chi^{16}_S \chi^{16}_S\right]
	\end{aligned}
	\]
	Again, spin-statistics requires the coefficient of all the characters involving $\chi_V$ to have positive sign and all the characters involving $\chi_S$ to have negative sign. This makes $1 \leq \alpha, 0 \leq \beta \leq \alpha -1$. Integrality then forces $\alpha = 1, \beta = 0$. \textbf{More general solution? We need to impose that $\chi_i^{8} \chi_O^{16} \chi_O^{16}$ has coefficient 1 or 0}
	\end{enumerate}
	Of all these theories, the first two theories have vanishing partition function - an indicator of spacetime supersymmetry, but not necessarily an identifier. Of course, we can identify them as the heterotic string theories, which indeed have space time SUSY. The last theory has nonvanishing partition function and thus cannot have spacetime SUSY as the fermions and bosons do not cancel at one loop.
	
	% Let's first not twist the left sector with the right one. We can either take all the 32 fermions together and orbifold by $(-1)^F$. This gives the partition function for the $\SO(32)$ heterotic string $\frac12 \sum_{ab} \theta^16 \twist ab$. We can also split them up into groups of 16 + 16 and orbifold by $(-1)^{F_1}$ and $(-1)^{F_2}$ on each of the two groups ($\cong \ZZ_2^2$). This gives the partition function for the $E_8 \oplus E_8$ heterotic string $\left(\frac12 \sum_{ab} \theta^8 \twist ab\right)^2$. We cannot split things any further, since splitting the fermions into groups of $8$ will require us to have minus signs between the different sectors in order to keep modular invariance. This would give negative weight to the R sector (which would give the wrong spin-statistics, since the $R$ sector ground state are spacetime scalars) correspond do doing the projection $(-1)^F = -1$ in the NS sector (which would not close under OPE).
	
	% We also have no other internal symmetry to twist the right-movers by. The remaining partition functions that we can write down necessarily must twist together the fermions on the left-moving and right-moving sides.
%
% 	The simplest thing we can twist by is $(-1)^{F_L} (-1)^{F_R}$ \emph{diagonally now} rather than separately. Our fermion partition function will be:
% 	\[
% 		\frac12 \sum_{ab} \frac{(-1)^{a + b + ab} \theta^4 \twist ab \bar \theta^{16} \twist ab}{\eta^4 \bar \eta^{16}}
% 	\]
% 	Here all the fermions are either R or NS and have been projected out by $(-1)^{F_L + F_R} = 1$
% 	This theory is consistent, but at level $0$ we have the tachyon associated with the identity. The normalization is of the identity is indeed 1.
%
% 	Now lets try to twist this further. Note that twisting this theory by $(-1)^F_R$ or $(-1)^F_L$ would just give group elements $\frac{1 + (-1)^{F_L + F_R}}{2} \frac{1 + (-1)^{F_R}}{2} = \frac{1 + (-1)^{F_L}}{2} \frac{1 + (-1)^{F_R}}{2}$ so we would recover the $\SO(32)$ superstring.
%
% 	Consider instead twisting by $(-1)^{F_{R, 1}}$ which flips the sign of the first 16 right-moving fermions.
		%
	% The characters for $O(32)$ are $\frac12 \frac{\theta^{16} \twist ab}{\eta^16}$ and the characters for $O(8)$ are $\frac12 \frac{\theta^4 \twist ab}{\eta^4}$. Each of these has $4$ sectors, so there are $16$ sectors that we can choose from.
		%
	% In order for spacetime supersymmetry to exist, we must have $e^{-\phi/2} S_\alpha$ have well-defined OPE with any vertex operator. \textbf{FINISH HERE}. This corresponds exactly to the condition that the GSO projection must act as $(-1)^{F_L} = 1$. This recovers exactly heterotic E and heterotic O.
	%
	% We thus have 9 theories satisfying modular invariance, spin-statistics. Six of them have tachyons, so we get \textbf{three} theories left over. Of these three, only two have spacetime supersymmetry--exactly heterotic O and heterotic E.
	
	\item I think this problem is backwards. For 32 fermions \emph{all} with the same boundary conditions, its immediate to see that they will reproduce the partition function for the $\mathrm{Spin}(32)/\ZZ_2$ string:
	\[
		\frac12 \sum_{a,b} \theta^{16}\twist ab
	\]
	Just by considering the $O(N)$ fermion at $N=32$. On the other hand, if we split the fermions into $16+16$, and consider separately boundary conditions for each of \emph{those}, then our partition function is the square of the $16$-fermion system. We then get the $E_8 \times E_8$ lattice theta-function, as required
	\[
		\left[ \frac12 \sum_{a,b} \theta^{8}\twist ab \right]^2
	\]
	\item Note this was a Lorentzian lattice of signature $(n,n)$. The norm was thus $P_L^2 - P_R^2 = 2 m n \in 2 \ZZ$. It is also self dual, since it is already integral, and there is no integral sublattice. 
	
	\item We have
	\[
		\gamma G_{ghost} = -c \gamma \d \beta - \frac32 \d c \gamma \beta - 2 \gamma^2 b, \quad c T_{ghost} = 2 b c \d c - \frac12 c \gamma \d \beta - \frac32 c \d \gamma \beta
	\]
	\textbf{Here Kitisis' conventions are different than Polchinski.} Recall upon bosonization $\beta(z) = e^{-\phi(z)} \d \xi(z), \gamma = e^{\phi(z)} \eta(z)$. 
	Although we can solve this problem very quickly since we already know what the stress tensor looks like in the bosonized variables, I think it's way more instructive to explicitly compute OPEs to $O(z-w)$. First let's look at the $\eta, \xi$ theory, which is a fermoinic $bc$ theory of weights $1, 0$. We get
	\[
		\xi(z) \eta(w) = \frac{1}{z-w} + :\xi \eta:(w) + O(z-w)
	\]
	We can bosonize this theory in terms of hermitian $\chi$ field so that $\eta = e^{-\chi}, \xi = e^{-\chi}$. Using these coordinates
	\[
	\begin{aligned}
		\xi(z) \eta(w) &= e^{\chi(z)} e^{-\chi(w)} = \frac{1}{z-w} \left[1 + (z-w) \d \chi + \frac12 (z-w)^2 (\d^2 \chi + (\d \chi)^2)  + \dots \right]\\
		\Rightarrow \d \xi(z) \eta(w) &= -\frac{1}{(z-w)^2} + \frac12 (\d^2 \chi + (\d \chi)^2)
	\end{aligned}
	\]
	Using this we can write
	\[
	\begin{aligned}
		\beta(z) \gamma(w) &= e^{-\phi(z)} \d \xi(z) \, e^{\phi(w)} \eta(w) \\
		&= (z-w) \left[1 - (z-w) \d \phi(w) + \frac12 (z-w)^2 ((\d\phi)^2 - \d^2 \phi) \right] \left[-\frac{1}{(z-w)^2} + \frac12 (\d^2 \chi + (\d \chi)^2) \right]
	\end{aligned}
	\]
	The constant term gives $:\!\!\beta \gamma\!\!: = \d \phi \Rightarrow :\!\d(\beta \gamma)\!: = \d^2 \phi$. The $(z-w)$ term gives exactly the stress tensor of the $\beta \gamma$ theory at $\lambda = 0$, which makes sense since this is exactly $\d \beta \gamma$
	\[
	\begin{aligned}
		:\!\d \beta \gamma\!: &= -\frac12 (\d \phi)^2 + \frac12 \d^2 \phi + \frac12 (\d\chi)^2 + \frac12 \d^2 \chi\\
		\Rightarrow 	T_{\beta \gamma } &= \d \beta \gamma - \lambda \d(\beta \gamma) = -\frac12 (\d \phi)^2 + \left(\frac12 - \lambda\right) \d^2 \phi + \frac12 (\d\chi)^2 + \frac12 \d^2 \chi.
	\end{aligned}
	\]
	In our case we have $\lambda = 3/2$. % but we will leave $\lambda$ in, unevalauted. In terms of these nice bosonic variables we have 
	\[
	\begin{aligned}
		\gamma G_{ghost} &= -c \left(-\frac12 (\d \phi)^2 + \frac12 \d^2 \phi + \frac12 (\d\chi)^2 + \frac12 \d^2 \chi\right) - \frac32 \d \phi\, \d c - 2 \gamma^2 b\\
		c T_{ghost} &= 2 b c \d c + c \left(-\frac12 (\d \phi)^2 - \d^2 \phi + \frac12 (\d\chi)^2 + \frac12 \d^2 \chi \right)
	\end{aligned}
	\]
	Altogether this gives a BRST current:
	\[
	\begin{aligned}
		j_B &= c T_X + \gamma G_X +  \frac12 \left(c T_{gh} + \gamma G_{gh} \right)\\
		&=  c T_X + \gamma G_X + b c \d c - \frac34 \d \phi \d c - \frac34 c \d^2 \phi - \gamma^2 b
	\end{aligned}
	\]
	\item We are looking at $[Q_B, \xi e^{-\phi/2} S_\alpha e^{i p X}]$. Therefore we should look at the $1/(z-w)$ pole in the OPE of $j_B$ with $\xi e^{-\phi/2} S_\alpha e^{i p X}$. The terms that contribute to this pole must involve pairing $\xi$ with its conjugate $\eta$. $\eta$ appears in $j_B$ wherever $\gamma = e^{\phi} \eta$ appears. From the previous exercise, we see that we need only look at the terms $\gamma G_X$ and $-\gamma^2 b$.
	
	These two terms contribute poles:
	\[
		- \left[\frac{ :\!e^{\phi} G_X\!: :\!e^{-\phi/2} S_\alpha e^{i p X}\!: }{z-w} - \frac{:e^{3\phi/2} \eta b S_\alpha e^{i p X}: }{z-w} \right]
	\]
	The overall minus sign comes from commuting across an odd number of fermions for the Wick-contraction. We will need to recall two things:
	\[
		\psi^\mu(z) \cdot S_\alpha(w) \sim \frac{\ell_s}{\sqrt 2 {\sqrt{z-w}}} \left( \Gamma^\mu_{\alpha \beta} S^{\beta}(w) + \frac{1}{\ell_s^2 (\frac D2 - 1)} \Gamma^\nu_{\alpha \beta} S_\beta \psi_\nu \psi^\mu \, (z-w)\right),
		\quad e^{\phi(z)} e^{-\phi(w)/2} \sim \sqrt{z-w} e^{\phi(w)/2}
	\]
	The subleading term in the first expansion is taken from \emph{Blumenhagen 13.81}.
	That means that first term is:
	\[
	\begin{aligned}
		e^{\phi (z)} i \frac{\sqrt{2}}{\ell_s^2} &\psi^\mu(z) \d X_\mu(z) \cdot e^{-\phi(w)/2} S_\alpha(w) e^{i p X(z)} \\
		&\sim  i \frac{\sqrt 2}{\ell_s^2} \,\sqrt{z-w} \, e^{\phi(w)/2} \frac{\ell_s}{\sqrt 2} \left(\frac{\Gamma_{\alpha \beta}^{\mu} S^\beta \d X_\mu e^{i p \cdot X}}{\sqrt{z-w}} +  \frac{-i \ell_s^2 p_\mu e^{i p \cdot X} }{2(z-w)} \frac{1}{4 \ell_s^2} \Gamma^\nu_{\alpha \beta} S^\beta \psi_\nu \psi^\mu \sqrt{z-w} \right)\\
		& = -\frac{e^{\phi/2}}{\ell_s} \left( \Gamma^\mu_{\alpha \beta} S^\beta \d X_\mu 
		- i \Gamma^\nu_{\alpha \beta} S^\beta \psi_\nu p \cdot \psi  \right) e^{i p \cdot X}
	\end{aligned}
	\]
	Note this OPE has no singularity, so we exactly got the normal ordered term we required: $:e^{\phi/2} G_X S_\alpha e^{i p \cdot X}:$. 
	Altogether this gives us:
	\[
		V^{(1/2)}_{\text{fermion}} (u, p) =  u^\alpha(p) \left[\frac{e^{\phi/2}}{\ell_s} \Gamma_{\alpha \beta}^\mu S^\beta \d X^\mu - \frac i8 \frac{e^{\phi/2}}{\ell_s} \Gamma_{\alpha \beta}^\nu S^\beta \psi_\nu p \cdot \psi  + e^{3 \phi/2} \eta b S_\alpha
		\right] e^{i p X}.
	\]
	I believe this is right, and moreover that the inclusion of the $\ell_s^{-1}$ factor is necessary for the dimensional analysis to make sense. 
	\item Here, I followed the discussion of Polchinski \textbf{12.5}. The picture changing operator is:
	\[
		\mathrm X(z) := Q_B \cdot \xi (z) % = T_F \delta(\beta(z)) - \d b \delta'(\beta(z))
	\]
	Over the sphere, the $\beta \gamma$ path integral is equivalent to the $\phi, \eta, \xi$ path integral \emph{plus} an additional insertion of $\xi$ to make up for the fact that it pics up a zero mode due to to the vacuum degeneracy it produces. Because the expectation value is \emph{just} proportional to the zero-mode of $\xi$, which depends on global information rather than the specific local insertion point, $\braket{\chi(z)}$ is independent of position and we can normalize $\xi$ so that this is $1$. 
	
	Say we have a null state. This means it is BRST exact. This means that we can rewrite its pointlike insertion as a local operator surrounded by a BRST contour (direct, from the definition of exact).
	For that null state to decouple, we need to be able to contract the BRST contour off the sphere (i.e. by pulling it off to the north pole). The fact that $\xi$ is inserted will seem to obstruct this. What happens now as we pull the BRST charge to infinity is that it will circle $\xi$, creating the PCO $\mathrm X(z)$. However, when the $\xi$ insertion is replaced by $\mathrm X$, the path integral will \emph{vanish} since there is now no $\xi$ insertion to avoid the zero-mode.
	
	Now consider a path integral with a PCO insertion as well as additional BRST-invariant operators (meaning. the contour integral around them of $j_B$ is zero). Then we can write $\mathrm X(z_1) \xi(z_2) = Q_B \xi(z_1) \xi(z_2) = (-)^2 \xi(z_1) Q^B \xi(z_2) = \xi(z_1) \mathrm X(z_2) $ where I have pulled the $Q_B$ contour around the sphere (there two minus signs, one from commuting $Q_B$ across a fermionic variable and one from reversing the orientation of the contour.) 
	
	This is interesting: although $\mathrm X$ is null, it does \emph{not} vanish in the path integral, since pulling $Q_B$ off of it will make $Q_B$ encircle $\xi(z_2)$ but leave behind $\mathrm X(z_1)$'s $\xi(z_1)$, so the $\xi$ zero-mode will remain saturated and we won't get zero. 
	
	The $\mathrm X$ can be brought near any of the local BRST closed operators to change their picture (the OPE is nonsingular). Ie note that the main term we look at is $\gamma G_X = e^{\phi} \eta G_X$ in $j_B$ so that $\mathrm X \O^{(-1)}(z) = z G_X(z) \O(0) \to G_{-1/2} \O(0)$. We can move $\mathrm X$ to any other point on the sphere - since the exact position of $\mathrm X$ does not matter any more than the position of $\xi$.
	
	\item It is enough to look at the $1/(z-w)$ term in the OPE
	\[
		:e^{-\phi(z)/2} S_\alpha(z): V^{(-1/2)}_{\text{fermion}}(w) = e^{-\phi(z)/2} S_\alpha(z) u^\beta(p) e^{-\phi(w)/2} S_\beta(w) e^{i p \cdot X(w)}
	\]
	We will use the fact of \textbf{4.12.42}:
	\[
		S_\alpha(z) S_\beta(w) = \frac{C_{\alpha \beta}}{(z-w)^{N/8}} + \frac{\Gamma^\mu_{\alpha \beta} \psi_\mu(w)}{\sqrt 2 \ell_s (z-w)^{N/8-1/2}}
	\]
	where $C_{\alpha \beta}$ is the charge conjugation matrix and here $N=10$. We also have $e^{-\phi/2} e^{-\phi/2} = (z-w)^{-1/4} e^{-\phi}$. This leaves the $(z-w)^{-1}$ term to be the requisite  
	\[
		e^{-\phi} u^\beta(p) \frac{\Gamma^\mu_{\alpha \beta}}{\sqrt 2 \ell_s} \psi_\mu e^{i p \cdot X} = V^{(-1)}_{\text{boson}}(\epsilon = \frac{\Gamma^\mu_{\alpha \beta} u^\beta}{\sqrt 2 \ell_s}, p, z)
	\] 
	For the second example, we will look at the $(z-w)^{-1}$ term in the OPE
	\[
		e^{-\phi(z)/2} S_\alpha(z) \epsilon_\mu \left(\d X^\mu- \frac i2 p_\mu \psi^\mu\, \psi^\nu \right) e^{i p \cdot X}.
	\]
	The first term in parentheses will not contribute to the singular term. Also the $e^{-\phi/2}$ and $e^{i p \cdot X}$ contract with nothing. Here, we use \textbf{4.12.41} to evaluate
	\[
		 S_\alpha(z) \cdot \underbrace{\psi^\mu \psi^\nu}_{-i J^{\mu \nu}} (w) \sim - \frac{\ell_s^2 (\Gamma_{\mu \nu})^\beta_\alpha S_\beta (w)}{2 (z-w)}
	\]
	The $-$ sign comes from the fact that the fermion current is coming from the \emph{right} this time so $z$ and $w$ are swapped.  
	This gives a variation
	\[
		e^{-\phi/2} i p^\mu \epsilon^\nu \frac{\ell_s^2}{4} \epsilon^\nu (\Gamma_{\mu \nu})^{\beta}_{\alpha} S_\beta e^{i p \cdot X} =  V^{(-1/2)}_{\text{fermion}} (u^\beta = \frac{i p^\mu \epsilon^\nu \ell_s^2  (\Gamma_{\mu \nu})^\beta_\alpha}{4}, p, z)
	\]
	
	\item We are in type I. We have 
	\[
		\frac{1}{\ell_s^4 g_o^2} \braket{:c V^{(-1)}(w_1):\, :c V^{-1}(w_2):\, :c V^{0}(w_3):} + 1 \leftrightarrow 2, \quad x_1 > x_2 > x_3
	\]
	\textbf{That constant out front is not obvious from Kiritsis, c.f. the discussion in Polchinski 12.4 and allow for another factor of $\ell_s^2$ since the fermions are dimensionful}
	\begin{center}
		\includegraphics[scale=0.1]{"Drawings/720"}
	\end{center}
	
	The relevant expectation values are
	\[
	\begin{aligned}
		\braket{c(w_1) c(w_2) c(w_3)} &= |w_{12} w_{13} w_{23}|, \quad \braket{e^{-\phi(w_1)} e^{-\phi(w_2)}} = w_{12}^{-1},\\
		\quad \braket{\psi^\mu(w_1) \psi^\nu (w_2)} &= \ell_s^2 \eta^{\mu \nu} w_{12}^{-1}, 
		\qquad \braket{\dot X^\mu(w_1) e^{i k_1 X} (w_2)} = -2 i \ell_s^2 k_1^\mu e^{i k_1 X} w_{12}^{-1} 
	\end{aligned}
	\]
	In the matter CFT we get (here $k_i \cdot k_j = 0$ so the pure $e^{i k X}$ terms contract to $1$):
	\[
	\begin{aligned}
		&\braket{\psi^\mu (w_1) e^{i k_1 \cdot X(w_1)}\, \psi^\nu (w_2) e^{i k_2 \cdot X(w_2)} \, (i \dot X^\rho + 2 k_3 \cdot \psi \psi^\rho) e^{i k_3 \cdot X(w_3)} } \\
		&= 2 \ell_s^4 \sdelta^{10}(\Sigma k) \left(\frac{\eta^{\mu \nu}k_1^\rho}{w_{12} w_{13}} + \frac{\eta^{\mu \nu} k_2^\rho}{w_{12} w_{23}} + \frac{\eta^{\mu \rho} k_3^\nu - \eta^{\nu \rho} k_3^\mu}{w_{13} w_{23}} \right)
	\end{aligned}
	\]
	So altogether we get an amplitude of (taking $x_1 \to 0, x_2 \to 1, x_3 \to \infty$)
	\[
	\begin{aligned}
		& \frac{i}{g_{open}^2 \ell_s^4} \times \frac{2 i \ell_s^4 g_{open}^3}{\sqrt 2 \ell_s} \sdelta^{10}(\Sigma k) \left(\frac{\eta^{\mu \nu}k_1^\rho x_{23}}{x_{12}} + \frac{\eta^{\mu \nu} k_2^\rho x_{13}}{x_{12}} + \eta^{\mu \rho} k_3^\nu - \eta^{\nu \rho} k_3^\mu \right) ([123] - [132])
		\\ &= 
		\frac{i g_{open}}{\sqrt 2 \ell_s} \sdelta^{10}(\Sigma k)  \left(\eta^{\mu \nu} k_{12}^\rho + \eta^{\mu \rho} k_{31}^\nu + \eta^{\nu \rho} k_{23}^\mu \right) ([123] - [132])
	\end{aligned}
	\]
	Note unlike the Bosonic string this is \emph{exactly the same} as the ordinary Yang-Mills amplitude, there is no $k^3$ correction term (what would correspond to at $\Tr F^3$ term in the Lagrangian).
	
	
	\item This is also in type I. We should put the gaugini in the $-1/2$ picture and the boson in the $-1$  picture.
	\begin{center}
		\includegraphics[scale=0.1]{"Drawings/721"}
	\end{center}
	 We have 
	\[
		\frac{1}{\ell_s^4 g_o^2} \braket{:c V^{(-1)}(w_1):\, :c V^{-1/2}(w_2):\, :c V^{-1/2}(w_3):} + 1 \leftrightarrow 2, \quad x_1 > x_2 > x_3
	\]
	The relevant expectation values are
	\[
	\begin{aligned}
		\braket{c(w_1) c(w_2) c(w_3)} &= |w_{12} w_{13} w_{23}|, \quad \braket{e^{-\phi(w_1)/2} e^{-\phi(w_2)/2} e^{-\phi(w_3)}} = w_{12}^{-1/4} w_{13}^{-1/2} w_{23}^{-1/2},\\
		\quad \braket{S_\alpha(w_1) S_\beta(w_2)} &= C_{\alpha \beta} w_{12}^{-5/4} 
		\qquad \Rightarrow \braket{S_\alpha(w_1) S_\beta(w_2) \psi^\mu(w_3)} = \frac{\ell_s^2}{\sqrt 2} (C \Gamma)^\mu_{\alpha \beta} w_{12}^{-3/4} w_{13}^{-1/2} w_{23}^{-1/2}
	\end{aligned}
	\]
	So altogether this gives 
	\[
		\frac{i}{\ell_s^4 g_{open}^2} \times (g_{open} \sqrt{\ell_s})^2 g_{open} \frac{\ell_s^2}{\sqrt 2}\, \sdelta^{10}(\Sigma k) C \Gamma^\mu_{\alpha \beta} \times ([123] - [132]) = \frac{i g_{open}}{\sqrt{2} \ell_s} \sdelta^{10}(\Sigma k) C \Gamma^\mu_{\alpha \beta} ([123] - [132])
	\]
	This is $k$-independent so is an even \emph{simpler} amplitude that the last in some sense. 
	
	\item We are now in type II. Gravitons are NS-NS states. We take two of them in the $(-1, -1)$ picture and the remaining one in the $(0, 0)$ picture. Again now, the constant demanded from unitarity now gets modified to $\frac{8 pi i}{g_c^2 \ell_s^6}$ We look at
	\[
	\begin{aligned}
		&\frac{8 \pi i}{g_{c}^2 \ell_s^6} \frac{2 g_{c}^3}{\ell_s^2} \Big< \big[c \tilde c e^{-\phi - \bar \phi} \psi^\mu \tilde \psi^\sigma e^{i k_1 X} \big] (z_1) \big[c \tilde c e^{-\phi - \bar \phi} \psi^\nu \tilde \psi^\omega e^{i k_2 X}\big] (z_2) \big[ c \tilde c(\d X^\rho - \frac i2 k \cdot \psi \psi^\rho) (\d X^\lambda - \frac i2 k \cdot \psi \psi^\lambda) e^{i k_3 X} \big] (z_3)\Big> 
	\end{aligned}
	\]
	Let's just look at the holomorphic part of the matter CFT, and the calculation goes almost exactly as in the last problem
	\[
	\begin{aligned}
		&\braket{\psi^\mu (z_1) e^{i k_1 \cdot X(z_1)}\, \psi^\nu (z_2) e^{i k_2 \cdot X(z_2)} \, (i \dot X^\rho + \frac12 k_3 \cdot \psi \psi^\rho) e^{i k_3 \cdot X(z_3)} } \\
		&= \frac12 \ell_s^4 \left(\frac{\eta^{\mu \nu}k_1^\rho}{z_{12} z_{13}} + \frac{\eta^{\mu \nu} k_2^\rho}{z_{12} z_{23}} + \frac{\eta^{\mu \rho} k_3^\nu - \eta^{\nu \rho} k_3^\mu}{z_{13} z_{23}} \right)\\
		& \rightarrow  \frac{\ell_s^4}{4} \underbrace{\left(\eta^{\mu \nu} k_{12}^\rho + \eta^{\mu \rho} k_{31}^\nu + \eta^{\nu \rho} k_{23}^\mu \right)}_{ =: V^{\mu \nu \rho}}
	\end{aligned}
	\]
	So the total amplitude becomes
	\[
		\pi i g_c \sdelta^{10}(\Sigma k) V^{\mu \nu \rho} V^{\sigma \omega \lambda}
	\]
	consistent with Polchinski.
	
	\item We can put all our gaugini in the $-1/2$ picture thankfully.
	\begin{center}
		\includegraphics[scale=0.1]{"Drawings/723"}
	\end{center}
	 Our vertex operators are $g_{open} \sqrt{\ell_s} \lambda^\alpha e^{-\phi/2} S_\alpha e^{ikX}$. The relevant two-point correlator is
	\[
		S_\alpha(z) S_\beta(w) \sim \frac{\ell_s (C \Gamma^\mu)_{\alpha \beta} \psi_\mu}{\sqrt 2 (z-w)}% , \quad \braket{e^{-\phi(w_1)/2}e^{-\phi(w_2)/2} e^{-\phi(w_3)/2} e^{-\phi(w_4)/2} } = \prod_{i < j} w_{ij}^{-1/4}
	\]
	 From considerations of the singularity structure, we get that the four-point correlator is:
	\[
		\frac{\ell_s^2 (C \Gamma^\mu)_{\alpha \beta} (C \Gamma_\mu)_{\gamma \delta}}{2 z_{12} z_{34} z_{23} z_{34}}
		 + \frac{\ell_s^2 (C \Gamma^\mu)_{\alpha \gamma} (C \Gamma_\mu)_{\alpha \delta}}{2 z_{13} z_{24} z_{32} z_{42}} 
		 + \frac{\ell_s^2 (C \Gamma^\mu)_{\alpha \delta} (C \Gamma_\mu)_{\beta \gamma}}{2 z_{14} z_{23} z_{42} z_{43}}
	\]
	Take $z_1 = 0, z_2 = w, z_3 = 1, z_4 = \infty$. In order for the term going as $1/z$ to cancel so that the integral over the line is well-defined, we need the (physical on-shell condition):
	\[
		\Gamma^\mu_{\alpha \beta} \Gamma^\mu_{\gamma \delta} +
		\Gamma^\mu_{\alpha \gamma} \Gamma^\mu_{\beta \delta} +
		\Gamma^\mu_{\alpha \delta} \Gamma^\mu_{\beta \gamma} = 0
	\]
	 and defining $- (k_1 + k_2)^2 = s$ etc. gives
	\[
	\begin{aligned}
		&\frac{i}{\ell_s^4 g_{open}^2} \times (g_{open} \sqrt{\ell_s})^4 \sdelta^{10}(\Sigma k) \times \frac{\ell_s^2}{2} \int_0^1 x^{-\ell_s^2 s - 1} (1-x)^{-\ell_s^2 u - 1} (\Gamma_{\alpha \beta}^\mu \Gamma_{\gamma \delta}^\mu + x \Gamma_{\alpha \gamma}^\mu \Gamma_{\beta \delta}^\mu) [1234]\\
		&= - \frac{i g_{open}^2 \ell_s^2 }{2} \sdelta^{10}(\Sigma k) \, 2 \times \left( \frac{\Gamma(-\ell_s^2 s) \Gamma(-\ell_s^2 u)}{\Gamma(1 - \ell_s^2 s - \ell_s^2 u)} (u \Gamma_{\alpha \beta}^\mu \Gamma_{\gamma \delta}^\mu - s \Gamma_{\alpha \delta}^\mu \Gamma_{\beta \gamma}^\mu) [1234] + 2 \perms \right)
	\end{aligned}
	\]
	The minus sign comes from pulling an $s$ or $u$ out of the $\Gamma$ functions. The factor of $2$ comes from summing over both orientations. Altogether we can write this as
	\[
	\begin{aligned}
		&-8 i g_{open}^2 \ell_s^2 \sdelta^{10}(\Sigma k) K(u_1, u_2, u_3, u_4) \left( \frac{\Gamma(-\ell_s^2 s) \Gamma(-\ell_s^2 u)}{\Gamma(1 - \ell_s^2 s - \ell_s^2 u)} [1234] + 2 \perms \right)\\
		& \qquad \quad K(u_1, u_2, u_3, u_4) = \frac18 (u \, \bar u_1\Gamma^\mu u_2 \bar u_3 \Gamma_\mu u_4 - s\, \bar u_1 \Gamma^\mu u_4 \bar u_3 \Gamma_\mu u_2)
	\end{aligned}
	\]
	
	\item The bosonic action in 11D is:
	\[
		\frac{1}{2\kappa^2} \int d^{11} x \sqrt{-\det \hat G} \left[R - \frac{1}{2 \cdot 4!} G_4^2 + \frac{1}{(144)^2} \epsilon^{M_1 \dots M_{11}} G_{M_1 \dots M_4} G_{M_4 \dots M_8} \hat C_{M_9 M_{10} M_{11}}\right]
	\]
	where $G_4$ is the field strength of the 3-form $\hat C$. From Appendix F, we have that the dilaton $\Phi = 0$ in 11D. So the field $\sigma$ will just be $\sigma = -2\phi = \frac12 \log G_{11\, 11}$, and $A$ here is as it is in appendix \textbf{F}. Directly using the bosonic equation \textbf{F.3} gives the terms
	\[
		\frac{1}{2\kappa^2 }\int d^8 x \sqrt{-g} e^{\sigma} \left[R - \frac14 e^{2\sigma} F_2^2  \right]
	\]
	(Here the $\frac14 \d_\mu G_{11\, 11} \d^\mu (G_{11\, 11})^{-1}$) will exactly cancel the $4 \d_\mu \phi d^\mu \phi$. 
	
	Now let's look at the $3$-form potential contribution. Because $F$ is antisymmetric in all four indices, and we only are compactifying along one dimension, only the first two terms of \textbf{F.28} can contribute. They give
	\[
		% \frac{1}{2\kappa^2} \int d^{10} x \sqrt{-g} e^\sigma \left[-\frac{1}{2 \cdot 4!} F_4^2 - \frac{1}{2 \cdot 3!} G^{11\, 11} C_{\mu \nu \rho\, 11} C_{\mu \nu \rho\, 11} \right]
		=\frac{1}{2\kappa^2} \int d^{10} x \sqrt{-g} e^\sigma \left[-\frac{1}{2 \cdot 4!} F_4^2 - \frac{1}{2 \cdot 3!} e^{-2\sigma} H_3^2 \right]
	\]
	and $(H_3)_{\mu \nu \rho} = \d_{[\mu} (B_2)_{\nu \rho]} = \d C_{\mu \nu\, 11}$ so that $H_3^2 = G^{\mu \sigma} G^{\nu \lambda} G^{\rho \kappa} H_{\mu \nu \rho} H_{\sigma \lambda \kappa}$ and $C_{\mu \nu \rho} = \hat C_{\mu \nu \rho} - (\hat C_{\nu \rho\, 11} A_\mu + 2 \perms)$  consistent with \textbf{F.30}.

	Finally, let's look at that last $\epsilon^{M_1 \dots M{11}}$ term. At first it looks quite scary. Note we can write this last term as $\frac16 \dd \hat C_3 \wedge \dd \hat C_3 \wedge \hat C_3$. I have $11$ indices to pick to be index $11$. If I pick any of the indices of the last $\hat C$ I get the term
	\[
		\frac{1}{12 \kappa^2} \dd C_3 \wedge \dd C_3 \wedge B_2
	\]
	If I pick either of the $\dd C$ terms, then after an integration by parts I get the same term. So the same term contributes three times \textbf{Revisit this logic}. We thus get the requisite action contribution:
	\[
		\frac{1}{4\kappa^2} \int d^{10} x B_2 \wedge \dd C_3 \wedge \dd C_3
	\]
	\item Under $A_1 \to A_1 + \dd \epsilon$ and $C_3 \to C_3 + \epsilon H_3$ we see that obviously $R$, $F_2$, $B_2$, and $H_3$ will stay the same. Now $dC_3 \to dC_3 + d\epsilon \wedge H_3$ while $A \wedge H_3 \to A \wedge H_3 - d \epsilon \wedge H_3$. Thus, $F_4$ will stay the same. 
	
	It remains to look at the variation of $B_2 \wedge \dd C_3 \wedge \dd C_3$. This is
	\[
	B_2 \wedge \dd \epsilon \wedge H_3 \wedge \dd C_3 + B_2 \wedge \dd C_3 \wedge \dd \epsilon \wedge H_3.	
	\]
	 These two terms cancel by antisymmetry of the indices.
	
	\item Defining $C'_3 = C_3 + A \wedge B_2$ give the above transformation as $A_1 \to A_1 + \dd \epsilon, C'_3 \to C'_3 + \epsilon H_3 + \dd \epsilon \wedge B_2$ so that $\dd C'_3 \to \dd C'_3 + \dd \epsilon \wedge H_3 - \dd \epsilon \wedge H_3 = \dd C'_3$. 
	
	Now $G_4 = \dd C_3' - \dd A \wedge B_2$ (Kiritsis wrote a small $A$, which I believe is a typo). Under the same transformation we get that $G_4$ is invariant as required. 
	
	Further, the transformation of $C_3 \to C_3 + \dd \Lambda_2$ implies that $C_3' \to C' + \dd \Lambda_2 \Rightarrow \dd C_3' \to \dd C_3' \Rightarrow G_4 \to G_4$ as required. So $C_3'$ now transforms trivially under the $A$ transformation.
	
	\item Take $S = S_0 + i e^{-\phi}$. The $\SL_2(\RR)$ transformation acts on IIB supergravity as:
	\[
		S \to \frac{a S + b}{c S + d}, \quad \begin{pmatrix}
			B_2\\
			C_2
		\end{pmatrix}
		\to \begin{pmatrix}
			d & -c\\
			-b & a
		\end{pmatrix} \begin{pmatrix}
			B_2 \\ 
			C_2
		\end{pmatrix}
	\]
	The latter is also how $H_3, F_3$ will transform together. Now think of $S$ as the modular parameter $\tau$, $e^{-\phi}= S_2 = \tau_2$ is the imaginary part. Think of $H_3, F_3$ as periods $\omega_1, \omega_2$ The IIB action in the Einstein frame can be written as
	\[
		S_{IIB} = \frac{1}{2\kappa^2} \int d^{10} x \sqrt{-g} \left[R - \frac12 \frac{\d S\, \d \bar S}{S_2^2} - \frac1{2\cdot 3!} \frac{|G_3|^2}{S_2} -\frac{1}{4 \cdot 5!} F_5^2 \right] + \frac{1}{8 i \kappa^2} \int C_4 \wedge \frac{G_3 \wedge \bar G_3}{S_2}
	\]
	Now $R, C_4,$ and $F_5$ do not change. The term $\frac{\d S\, \d \bar S}{S^2}$ transforms under $\SL(2, \RR)$ exactly like the invariant measure $\frac{d\tau d \bar \tau}{\tau^2_2}$. 
	
	Finally, any term consisting of a pair $G_3, \bar G_3$ in the numerator (either wedged or wedged with a hodge star) divided by $S_2$ will also remain modular invariant, as a quick Mathematica check confirms for us:
	\begin{center}
		\includegraphics[scale=0.5]{"Figures/IIB Calculation"}
	\end{center}
	\item Again for some reason Kiritsis writes a small $a$, which again I think is a typo. We need to find which gauge transformations need to be modified for $C_0, B_2, C_2, C_4$. There is a Chern-Simons term only in the definition of $F_5 = d C_4 - C_2 \wedge H_3$ so we see that the $C_0 \to C_0 + c$ (for $c$ a constant, the only closed 0-form) keeps the action invariant.
	
	 Taking $B_2 \to B_2 + \dd \Lambda_1$ will keep $H_3$ and therefore $F_5$ invariant, so this transform is legitimate. 
	 
	 Also, taking $C_4 \to C_4 + \dd \Lambda_3$ will keep $F_5$ invariant as well and will modify the Chern-Simons term in the full action $C_4 \wedge H_4 \wedge F_3$ by closed form, which will give no contribution. 
	
	Finally, taking $C_2 \to C_2 + \dd \Lambda_1$ will change $F_5 \to F_5 - \dd \Lambda_1 \wedge H_3$. This must be compensated by changing $C_4 \to C_4 + \frac12 \Lambda_1 \wedge H_3 + \frac12 \dd \Lambda_1 \wedge B_2$. Then $F_5$ will be invariant. Moreover, the Chern Simons term in the action $C_4 \wedge H_3 \wedge F_3$ have a variation
	\[
		\frac12 \Lambda_1 \wedge H_3 \wedge H_3 \wedge F_3 + \frac12  \dd \Lambda_1 \wedge B_2 \wedge H_3 \wedge F_3
	\]
	After integration by parts this variation will contribute nothing, as required. 
	
	\item Clearly $\dim O(32) = 32 \times 31/2 = 496$, which is necessary. For $N=32$ we also get from \textbf{7.9.29} that $\mathrm{Tr}(F^6) = 15\, \mathrm{tr}(F^4) \mathrm{tr}(F^2)$ where $\mathrm{tr}$ is the trace of the curvature form in (an associated bundle for) the fundamental representation Also using \textbf{7.9.30} we get $\mathrm{Tr}(F^4) = 24 \mathrm{tr}(F^4) + 3 (\mathrm{tr}(F^2))^2$ and $\mathrm{Tr}(F^2) = 30 \mathrm{tr}(F^2)$. Then, both sides of equation \textbf{7.9.26} become
	\[
		15 \mathrm{tr}(F^4) \mathrm{tr}(F^2) = -\frac{15}{8} \mathrm{tr}(F^2)^3 + \frac{15}{8} \mathrm{tr}(F^2)^3 + 15 \mathrm{tr}(F^4) \mathrm{tr}(F^2)
	\]
	and we have agreement!
	
	\item For an individual $E_8$, we have exactly
	\[
		\mathrm{Tr}(F^6) = \frac{1}{7200} \mathrm{Tr}{F^2} = \frac{1}{48} \mathrm{Tr}{F^2} \cdot \frac{1}{100} (\mathrm{Tr}{F^2})^2 - \frac{1}{14400} (\mathrm{Tr}{F^2})^3
	\]
	Now for $E_8 \times E_8$ we have the property that $\mathrm{Tr}(F^n) = \mathrm{Tr}_1(F^n) + \mathrm{Tr}_2(F^n)$ where $\mathrm{Tr}_i$ means tracing over the direct summands in the associated bundle that are acted on by the $i$th $E_8$. Then we will have the relations
	\[
		\mathrm{Tr}(F^6) = \frac{1}{7200} (\mathrm{Tr}_1(F^2) + \mathrm{Tr}_2(F^2))^3, \quad \mathrm{Tr}(F^4) = \frac{1}{100} (\mathrm{Tr}_1(F^2) + \mathrm{Tr}_2(F^2))^2
	\]
	So replacing $\mathrm{Tr}(F^2)$ with $\mathrm{Tr}_1(F^2) + \mathrm{Tr}_2(F^2)$ in the prior derivation gives us again exact matching and thus anomaly cancelation. 
	
	On the other hand $U(1)$ \emph{has} no Casimirs so $\mathrm{Tr}(F^m) = 0$ for all $m$. In particular this allows us to take $E_8 \times U(1)^{248}$ or $U(1)^{496}$ as a gauge group and remain anomaly-free. \textbf{Check with Nick. Reconcile this with} 
	
	\item Now let us turn to the $SO(16) \times SO(16)$ theory. To check anomalies, we look at the chiral terms. In this case we have massless content consisting of spin $1/2$ Majorana-Weyl fermions transforming in the $(16,16)$ (positive chirality), and $(1, 128) \oplus (128, 1)$ (negative chirality) representations. We also have \emph{massive} fermion fields in the $(128,128)$ representation that do \emph{not} contribute to the anomaly. Note that we do \emph{not} have a gravitino, as there is no spacetime SUSY in this theory.
	
	The positive chirality $(16,16)$ MW fermions have field strength $F_+ = F_1 \otimes 1 + 1 \otimes F_2$ valued in the vector representation. 
	
	The negative chirality $(1, 128) \oplus (128, 1)$ MW fermions have field strength $\hat F_- = \hat F_1 \oplus \hat F_2$ valued in the spinor representation. 

	Our anomaly polynomial is thus
	\[
		\frac12 (I_{1/2}(R, F_+) - I_{1/2}(R, F_-))
	\]
	Both representations have dimension 256, so the $\frac{n}{64}(\dots)$ term in \textbf{7.9.22} cancels (note we did \emph{not} need to use that the dimension of the gauge group was $496$ here!). We are left with (using $\tr$ for the trace in the fundamental representation and $\tr_s$ for the spinor rep'n)
	\begin{equation}\label{eq:anomaly}
		\begin{aligned}
			&-\frac{\tr[F^6_+]}{720}+ \frac{\tr[F^4_+] \Tr[R^2]}{24 \cdot 48} - \frac{\tr[F^2_+]}{256} \left(\frac{\Tr[R^4]}{45} + \frac{(\Tr[R^2])^2}{36} \right)\\
			&-\left(-\frac{\tr_S[\hat F^6_-]}{720}+ \frac{\tr_S[\hat F^4_-] \Tr[R^2]}{24 \cdot 48} - \frac{\tr_S[\hat F^2_-]}{256} \left(\frac{\Tr[R^4]}{45} + \frac{(\Tr[R^2])^2}{36} \right) \right)
		\end{aligned}
	\end{equation}
	From explicitly expanding out $(F_1 \otimes 1 + 1\otimes F_2)^{2,4,6}$ we get:
	\[
	\begin{aligned}
		\tr F_+^2 &= 16 (\tr F_1^2 + \tr F_2^2) \\
		\tr F_+^4 &= 16 (\tr F_1^4 + \tr F_2^4) + 6 \tr F_1^2 \tr F_2^2 \\
		\tr F_+^6 &= 16 (\tr F_1^6 + \tr F_2^6) + 15 \tr F_1^2 \tr F_2^4 + 15 \tr F_1^4 \tr F_2^2
	\end{aligned}
	\]
	Together with the results \textbf{7.4E, 7.5E} relating $\tr_S$ to $\tr$ we get 
	\[
	\begin{aligned}
		\tr_S F_-^2 &= 16 (\tr F_1^2 + \tr F_2^2) \\
		\tr_S F_-^4 &= -8 (\tr F_1^4 + \tr F_2^4) + 6 (\tr F_1^2 + \tr F_2^2)^2 \\
		\tr_S F_-^6 &= 16 (\tr F_1^6 + \tr F_2^6) - 15 (\tr F_1^2 \tr F_1^4 + \tr F_2^2 \tr F_2^4) + \frac{15}{4} \left((\tr F_1^2)^3 + (\tr F_2^2)^3\right)
	\end{aligned}
	\]
	Altogether we get
	\begin{center}
		\includegraphics[scale=0.5]{"Figures/Anomaly Cancelation"}
	\end{center}
	

	We thus get a Green-Schwarz term
	\[
		\propto \int d^{10} x B \left[\Tr_1 (F^4) + \Tr_2 (F^4) - \frac14 (\Tr_1(F^2)^2 + \Tr_2(F^2)^2 - \Tr(F_1^2) \Tr(F_2^2) \right]
	\]
	
	We have exhausted the set of supersymmetric chiral anomaly-free theories, so the question remains whether there are any \emph{non}-supersymmetric theories that are chiral and anomaly-free in 10D. We will have only MW fermions and perhaps self-dual 5-form fields contributing. It does not seem possible to cancel the $I_A(R)$ with \emph{just} the $I_{1/2}(R, F)$, so I expect any such 10 D non-SUSY theory will in fact contain \emph{only} MW fermions. They must come in pairs of opposite parities with equal particle number to cancel the gravitational anomaly. \textbf{Exhaustively showing this seems really difficult. Xi didn't know the full answer}
	
	I know for a fact there is at least \emph{one} other anomaly free theory in 10D, namely the $USp(32)$ open string Sugimoto theory (c.f. question \textbf{7.36}).

	\item We are orbifolding by a $\ZZ_2$. In the sector of the left-moving worldsheet fermions, only $(-1)^\mathbf{F}$ acts nontrivially. The twisted blocks are
	\[
		Z_{fermions} \twist hg = \frac12 \sum_{a, b = 0}^1 (-1)^{a + b + ab + a g + b h + gh} \frac{\theta^4 \twist ab}{\eta^4}
	\]
	On the $E_8 \times E_8$s the untwisted block is just $(\frac12 \sum_{ab} \bar \theta^8 \twist ab / \bar \eta^8)^2$. Performing the projection $g$ requires that $a, b$ match for both factors, giving:
	\[
		Z_{E_8^2} \twist 01 = \frac14 \sum_{a, b=0}^1 \frac{(-1)^b \bar \theta^{8} \twist{a}{b} (2 \tau)}{\bar \eta^{8}(2 \tau)} = \frac14 \frac{\bar \theta^8\twist00 + \bar \theta^8\twist01}{\bar \eta^4 (\tau) \bar \theta^4 \twist10 (\tau)}
	\]
	Taking $\tau \to -1/\tau$ gives
	\[
		Z_{E_8^2} \twist 10 = \frac14 \frac{\bar \theta^8\twist00 + \bar \theta^8\twist10}{\bar \eta^4 (\tau) \bar \theta^4 \twist01 (\tau)}
	\]
	Finally taking $\tau \to \tau+1$ gives
	\[
		Z_{E_8^2} \twist 11 = \frac14 \frac{\bar \theta^8\twist01 + \bar \theta^8\twist10}{\bar \eta^4 (\tau) \bar \theta^4 \twist00 (\tau)}
	\]
	The full partition function is thus
	\[
		Z = \frac12 \sum_{h, g = 0}^1 Z_{E_8^2} \twist hg Z_{fermions} \twist hg
	\]
	We see that this is modular invariant, as individually both $Z_{fermions}$ and $Z_{E_8^2}$ are invariant under $\tau \to -1/\tau$. Their anomalous changes under $\tau \to \tau+1$ from the $\eta^4$ powers in the denominator are cancelled in pairs. 
	
	The gauge group corresponds to the invariant (diagonal) $E_8$ sublattice of $E_8 \times E_8$ (\textbf{Confirm}). At the massless level, we still have the gravity supermultiplet ($G, B, \Phi$), as well as gauge bosons with gauge group $E_8$ from the untwisted sector. \textbf{What about the twisted sector?}
	
	The gravitino has been projected out, so this theory no longer has spacetime supersymmetry. 
	The theory is still chiral, and since the partition function is modular invariant, we are also guaranteed that it is anomaly free. However, it has a tachyon. 

	\textbf{Unfinished}
	
	\item Let's assume we do not have self-dual 2-form gauge fields that give self-dual $3$-form field strengths and we do not consider an $I_A$ contribution. Recall we can write
	\[
		I_{1/2} = \prod_{i=1}^{D/2} \frac{x_i/2}{\sinh(x_i/2)}
	\]
	where $x_i$ are the off-diagonal entries in the $2\times2$ block decomposition of $R_0 = \dd \omega$. All of this is easy to do in Mathematica. 
	\begin{center}
		\includegraphics[scale=0.45]{"Figures/6D Anomaly"}
	\end{center}
	For $n$ spin 1/2 fermions and a gravitino we thus get the forms
	\[
	\begin{aligned}
		I_{1/2}(R, F) &= \frac{n}{576} \left(\frac{\Tr (R^4)}{10} + \frac{(\Tr R^2)^2}{8} \right) - \frac{\Tr F^2}{96} \Tr R^2 + \frac{\Tr F^4}{24}\\
		I_{3/2}(R) &= \frac{49}{576 \times 2} \Tr (R^4) - \frac{43}{576 \times 8} \Tr(R^2)^2
	\end{aligned}
	\]
	The \emph{Anomalies.nb} also has the 10D cancelation if anyone is interested.
	
	\item In the absence of a linear dilaton background, the RR fields simply satisfy the equations of motion $\dd \star F_{p+2} = 0$, as well as the Bianchi identities $\dd F_{p+2} = 0$. We want to show that, the tree level effective action of type II SUGRA in the string frame will have no coupling at tree level between the RR field strengths and the dilaton.
	
	In a linear dilaton background $\Phi = \frac{Q}{\sqrt{2} \ell_s} X^9$, the supercurrent $G$ will be modified to
	\[
		G= i \frac{\sqrt 2}{\ell_s^2}\,\psi \cdot \d X  - i \sqrt 2\;  \Phi_{,\mu} \d \psi^\mu \Rightarrow G_0 \, \propto \, \frac{1}{\sqrt 2} \psi_0 (p_\mu + i \Phi_{, \mu})
	\]
	I'm not sure about a possible constant factor multiplying the second term in the definition of $G$, but it is as in Polchsinki 12.1.18. Acting on the RR ground states, $\psi_0$ gives an additional $\Gamma$ matrix, but the $\Phi_{\, \mu}$ term will modify the Bianchi and free massless equations as:
	\[
		(\d_\mu -\Phi_{, \mu}) \wedge F = (\d_\mu - \Phi_{, \mu}) \wedge \star F = 0 \Rightarrow e^{\Phi} \dd e^{-\Phi} F = e^{\Phi} \dd \star (e^{-\Phi} F) = 0.
	\]
	This implies that we should view $\hat F = e^{-\Phi} F$ as the field strength, and so the RR states correspond to $e^{\Phi} F$ (ie they already incorporate a factor of $e^{\Phi}$). The RR charges are surface integrals of $\hat F = \dd C$. Thus the dilaton coupling to the RR field strength $\hat F^{2m}$ is $e^{2m \Phi} e^{2 (k -1) \Phi} F^{2m}$. In particular, at tree level the $F^2$ term does not couple to the dilaton.
	
	\item The (minimal) supergravity multiplet contains  the left-handed $3/2$ gravitino as well as a right-handed self-dual 3-form field. The tensor multiplet contains a left-handed anti-self-dual 3-form field and the right-handed $1/2$ dilatino. Combining \emph{one} of the $N_T$ tensor multiplets with the gravity multiplet gives an anomaly contribution of:
	\[
		I_{3/2} - I_{1/2}
	\]
	The vector multiplet contains the left-handed gaugino. The hypermultiplet (which BTW Kiritsis has not yet defined this) apparently contains a \emph{right-handed} \emph{hyperino} (wow fancy). 
	
	So far this gives
	\[
		I_{3/2} + (N_V - N_H - 1) I_{1/2} (R)
	\]
	But we have $N_T - 1$ addition tensor multiplets which will then contribute
	\[
		I_{3/2} + (N_V - N_H - N_T) I_{1/2} (R) + (N_T - 1) I_A(R) 
	\]
	A quick calculation for an anti-self-dual tensor gives
	\[
		I_{ASD} = - \left( \frac{7 \Tr R^4}{1440} - \frac{(\Tr R^2)^2}{144 \times 4} \right)
	\]
	the minus sign out front is from being \emph{anti}-self dual. 
	
	As before, in order to have factorization of the anomaly polynomial for the GS mechanism to work, we need the $\Tr R^4$ terms to cancel. This gives our desired constraint
	\[
		\frac{49}{144 \times 8} + \frac{(N_V - N_H - N_T)}{144 \times 40} - \frac{7}{144 \times 10} (N_T- 1) = 0 \Rightarrow N_H - N_V + 29 N_T = 273
	\]
	I think Kiritsis has a typo in this equation and it should be $\mathbf{+29} N_T$ rather than $-29$. This is consistent with \textbf{BBS exercise 5.9}.
	
	\item Another 10D nonsupersymmetric string theory without tachyon! This one is open+closed. The $O(16) \times O(16)$ is the only closed non-SUSY string theory in 10D without tachyon. \textbf{Is this the only open one?} The relevant reference is \href{https://arxiv.org/abs/hep-th/9905159}{arXiv:hep-th/9905159}
	
	This is a theory of strings stretching $D9-D\bar 9$ branes.
	
	We have $\lambda, \tilde \lambda$ are positive chirality spinors belonging to the adjoint of $\mathrm{Sp}(n)$ (equivalent to the symmetric representation {\tiny \ydiagram{2}} and traceless antisymmetric representation {\tiny \ydiagram{1,1}} of $\mathrm{Sp}(m)$ respectively, while $\psi, \bar \psi$ are negative chirality spinors belonging to the bi-fundamental representation of $\SO(n) \times \SO(m)$. We take $n=0, m = 32$.
	
	As in the $\SO(16) \times \SO(16)$ example, the lack of spacetime SUSY means there is no gravitino contribution, and we look only at the massless fermion content:
	\[
		I_{\lambda} + I_{\bar \lambda} - 2 I_{\psi}
	\]
	The gravitational anomaly cancels for free since we have the same number of left and right chirality fermions. $\bar \lambda$ does not contribute to the gauge or mixed anomaly, since it transforms trivially under $\mathrm{USp}(32)$. 
	
	\textbf{Finish}
	
	
\end{enumerate}
% section chapter_7_superstrings_and_supersymmetry (end)
	
\end{document}
	