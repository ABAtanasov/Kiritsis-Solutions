\documentclass[11pt, class=article, crop=false]{standalone}
\usepackage{amsmath,amssymb,amsfonts,amsthm}
\usepackage{enumitem}
\usepackage{fancyhdr}
\usepackage{tikz-cd}
\usepackage{mathabx}
\usepackage{geometry}
\usepackage{natbib}
\usepackage{braket}
\usepackage{graphicx}
\usepackage{simpler-wick}
\usepackage{hyperref}
\usepackage{ytableau}
\usepackage{cancel}
\usepackage{listings}
\usepackage{relsize}
\usepackage{xcolor}
\usepackage{stmaryrd}
\usepackage{slashed}
\usepackage{tikz-feynman}
\usepackage{kiritsis}
\geometry{margin = 0.5in}


\begin{document}
\section*{Chapter 16: String Theory and Matrix Models} % (fold)
\label{sec:chapter_16_string_theory_and_matrix_models}

\begin{enumerate}
	\item The Nambu-Goto action is
	\[
		-T_2 \int d^3 \xi [\sqrt{\det \hat g} + \hat C_{\alpha \beta \gamma} \epsilon^\alpha \beta \gamma], \quad \hat g_{\alpha \beta} = G_{\mu \nu} \d_\alpha X^\mu \d_\beta X^\nu, \quad C_{\alpha \beta \gamma} = C_{\mu \nu \rho} \d_\alpha X^\mu \d_\beta X^nu \d_\gamma X^\rho
	\]
	 Let's set $C_{\alpha \beta \gamma} = 0$. The EOM for the scalar field is quickly seen to be $\Box X = 0$, where $\Box$ is the Laplacian from the induced metric. 
	 
	 In the Polyakov action, the equations of motion for $\gamma$ are the vanishing the energy-momentum tensor, giving:
	 \[
	 	\d_\alpha X^\mu \d_\beta X_\mu - \frac12 \gamma_{\alpha \beta} ( \gamma^{\gamma \delta} \d_\gamma X^\mu \d_\delta X_\mu - 1)
	 \]
	 This is harder to solve than the $p=1$ case, as we can't just take the square root of the determinant of both sides. Taking the ansatz that $\gamma_{\alpha \beta} = \lambda \d_\alpha X^\mu \d_\beta X_\mu$ we get:
	 \[
	 	\lambda \gamma_{\alpha \beta} - \frac12 \gamma_{\alpha \beta}(3 \lambda - 1 )
	 \]
	 And we get a solution with $\lambda = 1$. Note there is no Weyl rescaling here. Similarly, the $X$ field must satisify
	 \[
	 	\frac{1}{\sqrt{-h}} \d_\alpha (\sqrt{-h} h^{\alpha \beta} \d_\beta X^\mu) = 0
	 \]
	 which agrees with $\Box X = 0$ upon the identification of the induced and auxiliary metrics. 
	
	
	Note importantly that the \emph{p-brane action for $p\neq 1$ requires a cosmological constant term}.
	
	\item Take the gauge $\gamma_{00} = -\det \hat g_{ij}$ with $\gamma_{0i} = 0$ so that $\sqrt{-\gamma} = \det g_{ij}$. The action then becomes:
	\[
		-\frac{T_2}{2}  \sqrt{\gamma} \left(\gamma^{00} \dot X \cdot \dot X - 1 \right) = \frac{T_2}{2} ( \dot X \cdot \dot X  + \det \hat g_{ij})
	\]
	\textbf{Here there is a small typo in Kiritsis}. Rewriting 
	\[
		\det \hat g_{ij} = \d_1 X^\mu \d_1 X^\nu \d_2 X_\mu \d_2 X_\nu - \d_1 X^\mu \d_2 X^\nu \d_1 X_\mu \d_2 X_\nu = - \frac12 \{X^\mu, X^\nu\} \{X_{\mu}, X_{\nu}\}
	\]
	We thus get total action:
	\[
		\frac{T_2}{2} \int d^3 \xi \left(\dot X^\mu \dot X_\mu - \frac12 \{X^\mu, X^\nu\} \{X_{\mu}, X_{\nu}\} \right)
	\]
	Giving equations of motion 
	\[
		\ddot X^\mu = \{ \{X^\mu, X^\nu \}, X_\nu \}
	\]
	Taking now lightcone gauge $X^+(\tau, \sigma_1, \sigma_2) = \tau$
	
	The transverse momenta are:
	\[
		p^i = \frac{\delta L}{\delta(\d_\tau X^i)}= T_2 \dot X^i = \frac{p^+}{V} \dot X^i
	\]
	The Hamiltonian is thus
	\[
		p_- \dot X^- - \mathcal L + \int d^2 \xi \, p_i \dot X^i
	\]
	
	\item This rescaling is very straightforward once one has the hamiltonian 16.1.14. One rescales $X^i \to \left(\frac{N}{V_2}\right)^{1/4} X^i$ and $t \to \left(\frac{N}{4 V_2}\right)^{-1/4}$ yielding:
	\[
		\frac{T_2}{2} \int d^2 \sigma \dot X^i \dot X^i \to \frac{T_2}{4} \frac{N}{V_2} \int d^2 \sigma \frac{\dot X^i \dot X^i}{2} = \frac{T_2}{4} \Tr[\frac{\dot X^i \dot X^i}{2} ]
	\]
	and 
	\[
		\frac{T_2}{4} \int d^2 \sigma  \{X^i, X^j\} \{X_{i}, X_{j}\} \to -\frac{T_2}{4} \frac{N}{V_2} \int d^2 \sigma \frac14 [X^i, X^j] [X_{i}, X_{j}] = \frac{T_2}{4} \Tr\left(-\frac14 [X^i, X^j] [X_{i}, X_{j}] \right)
	\]
	
	\item For a string, imagine a rectangular spike of cross-section $\epsilon$ and length $L$. Its total energy is $2 L + \epsilon$, where $\epsilon$ does not multiply $L$ now. Therefore, taking $L$ large will give a large energy deviation, regardless of how small we take $\epsilon$. Thus, the string is stable against decaying into these small spikes. 
	
	\item Its immediate that the $C_{\mu \nu \rho}$ term multiplies a Nambu bracket, by antisymmetry. Now by permutation invariance we can write:
	\[
		\frac16 \{X^\mu, X^\nu, X^\rho\}  \{X_\mu, X_\nu, X_\rho\} = \d_1 X^\mu \d_2 X^\nu \d_3 X^\rho \epsilon_{\alpha \beta \gamma} (\d_\alpha X_\mu \d_\beta X_\nu \d_\gamma X_\rho)
	\]
	Its not hard to see that this reproduces the formula for a 3x3 determinant, as we have an antisymmetric object involving one element from every row and column multiplied together, all with unital coefficients. 
	
	The bracket is not associative \textbf{show}
	
	\item 
	
\end{enumerate}

% section chapter_16_string_theory_and_matrix_models (end)
\end{document}
	