\documentclass[11pt, class=article, crop=false]{standalone}
\usepackage{amsmath,amssymb,amsfonts,amsthm}
\usepackage{enumitem}
\usepackage{fancyhdr}
\usepackage{tikz-cd}
\usepackage{mathabx}
\usepackage{geometry}
\usepackage{natbib}
\usepackage{braket}
\usepackage{graphicx}
\usepackage{simpler-wick}
\usepackage{hyperref}
\usepackage{ytableau}
\usepackage{cancel}
\usepackage{listings}
\usepackage{relsize}
\usepackage{xcolor}
\usepackage{stmaryrd}
\usepackage{slashed}
\usepackage{tikz-feynman}
\usepackage{kiritsis}
\geometry{margin = 0.5in}


\begin{document}
\section*{Chapter 11: Duality Connections and Nonperturbative Effects} % (fold)
\label{sec:chapter_11_duality_connections_and_nonperturbative_effects}
\begin{enumerate}
	\item Taking the expression for a toroidal heterotic compactification from exercise \textbf{9.1} 
	\[
		\left[\frac{R}{\sqrt{\tau_2} \eta \bar \eta^{17}} \sum_{m,n} e^{-\frac{\pi R^2}{\tau_2} |m + n \tau|^2} e^{-i \pi \sum_I n Y^I (m + n \bar \tau) Y^I Y^I}\frac12 \sum_{a,b=0}^1  \prod_{i=1}^{16} \bar \theta \twist ab (Y^I (m + \bar \tau n) | \bar \tau) \right] \times \frac{1}{\tau_2^{7/2} \eta^7 \bar \eta^7} \frac12 \sum_{a,b = 0}^1 \frac{\theta^4 \twist a b}{\eta^4}
	\]
	Using $\theta$ function identites as in the second equation in appendex \textbf{E}, we get
	\[
		\Gamma_{1,17}(R, Y) = \frac{R}{\sqrt{\tau_2}} \sum_{m,n} e^{-\frac{\pi R^2}{\tau_2} |m + n \tau|^2} \frac12 \sum_{a,b=0}^1 e^{i \pi m Y^I Y^I n - i \pi b n Y^I} \bar \theta \twist{a - 2n Y^I}{b - 2m Y^I}
	\]
	Now take $Y^I = 0$ for $I = 1 \dots 8$ and $Y^I = 1/2$ for $I = 1 \dots 16$. Then
	\[
		\prod_I e^{i \pi m Y^I Y^I n - i \pi b n Y^I}  = e^{i \pi m \sum_I (Y^I)^2 - i \pi b \sum_I Y^I} = 1
	\]
	and we can ignore this term. Similarly because we are taking a product over 16 $\bar \theta$, no phases will interfere with us replacing $\theta\twist uv$ with $\theta \twist{-u}{-v}$ for integer $u, v$. This gives us the desired first step
	\[
		\Gamma_{1,17}(R, Y) = R \sum_{m,n} e^{-\frac{\pi R^2}{\tau_2} |m + n \tau|^2} \frac12 \sum_{a,b=0}^1 \bar \theta \twist ab^8 \bar \theta \twist{a+n}{b+m}^8
	\]
	Now again because we have enough $\theta \twist{a+n}{b+m}$ that phases do not interfere, we see that we only care about $n,m$ modulo $2$ in the fermion term. We know how to divide the partition function of the compact boson into parity odd and even blocks by doing the $\mathbb Z^2$ stratification corresponding to the $\pi R$ translation orbifold of the circle. This gives our desired answer:
	\[
		\frac12 \sum_{h,g} \Gamma_{1,1}(2R) \twist hg \frac12 \sum_{a,b} \bar \theta \twist ab^8 \bar \theta \twist{a+h}{b+g}^8
	\]
	with 
	\[
		\Gamma_{1,1} (2R) = 2R \sum_{m,n} \exp\left[\frac{-\pi R^2}{\tau_2} |2m + g + (2n + h)\tau|^2 \right]
	\]
	
	\item As before, take the ansatz
	\[
		ds^2 = e^{2A(r)} \eta_{\mu \nu} dx^\mu dx^\nu + e^{2B(r)} dx^i \cdot dx^i, \qquad A_{012} = \pm e^{C(r)} \Rightarrow G_{r 012} = \pm C'(r)\, e^{C(r)}
	\]
	
	The BPS states in 11D require only the gravitino variation to vanish:
	\[
		\delta \psi_M = \d_M \epsilon + \frac14 \omega_{M}^{PQ} \Gamma_{PQ} \epsilon + \frac{1}{2 \cdot 3! \cdot 4!} G_{PQRS} \Gamma^{PQRS}  \Gamma_M \epsilon - \frac{8}{2 \cdot 3! \cdot 4!} G_{MQRS} \Gamma^{QRS} \epsilon
	\]
	We have worked out $\omega$ in \textbf{8.43}.
	\[
		\omega_{\hat \mu \hat \nu} = 0,\quad \omega_{\hat \mu \hat i} = (-)^{\mu = 0}\, \d_i A \, e^{A-B} dx^\mu, \quad \omega_{\hat i \hat j} = \d_j B dx^i - \d_i B dx^j
	\]
	Let's look first at $M = \mu$ parallel. Since $\epsilon$ is Killing we expect no longitudinal variation and we get
	\[
	\begin{aligned}
		0 &= \cancel{\d_\mu \epsilon} + \frac12 A' \, e^{A - B} \Gamma^{\hat \mu \hat r} \epsilon \pm \frac{1}{2 \cdot 3!} C'(r) e^{C} \cancel{\Gamma^{r 0 12} \Gamma_\mu} \epsilon \mp \frac{1}{3!} C'(r) e^{C} \Gamma_\mu \Gamma^{r 0 1 2} \epsilon\\
		&= \frac12 A' \, e^{A - B} \Gamma^{\hat \mu \hat r} \epsilon \mp \frac{1}{3!} C' e^{C-B-2A} \Gamma^{\hat \mu \hat r \hat 0 \hat 1 \hat 2 } \epsilon \\
		& \Rightarrow 0 = A' \epsilon \mp \frac{1}{3} C' e^{C-3A} \Gamma^{\hat 0 \hat 1 \hat 2 } \epsilon\\
	\end{aligned}
	\] 
	If we would like these two terms to be proportional, then we should take $C= 3 A$, and we get the following condition for $\epsilon$
	\[
		(1 \mp \Gamma^{\hat 0 \hat 1 \hat 2}) \epsilon = 0
	\]
	So half the dimension of the space of spinors satisfies this at any given point. We thus get 
	
	For $M = i$ transverse, we recall $\Gamma_{ij}$ generates rotations, so assuming rotational invariance in the transverse space, we'll cancel this. We get
	\[
	\begin{aligned}
		&\d_r \epsilon + \cancel{\frac14 \omega_{r}^{jk} \Gamma_{jk} \epsilon} +\cancel{\frac{1}{2 \cdot 3!} G_{r012} \Gamma^{r 012} \Gamma_r \epsilon} \mp \frac{1}{3!} G_{r012} \Gamma^{012} \epsilon = 0\\
		\Rightarrow& \d_r \epsilon \mp \frac{1}{3!} G_{r012} \Gamma^{012} \epsilon = 0\\
		\Rightarrow& \d_r \epsilon \mp \frac{e^{-3A}}{3!} C' e^C \Gamma^{\hat 0 \hat 1 \hat 2} \epsilon
	\end{aligned}
	\]
	Solving this gives us that 
	\[
		\epsilon(r) = e^{C(r)/6} \epsilon_0
	\]
	for $\epsilon_0$ some constant spinor. We still do not have a relationship between $C$ and $B$. This can be obtained by not assuming rotational invariance but rather imposing cancelation of the second and third terms above as follows:
	\[
	\begin{aligned}
		&\frac12 \d_j B \, \Gamma^{\hat i \hat j} \epsilon  \pm \frac{1}{2 \cdot 3!} \d_j C \, e^{C} \, \Gamma^{j012} \Gamma_i \epsilon \\
		&= \frac12 \d_j B \, \Gamma^{\hat i \hat j} \epsilon  \pm \frac{1}{2 \cdot 3!} \d_j C \, e^{C-3A} \, \Gamma^{\hat i \hat j \hat 0 \hat 1 \hat 2} \epsilon\\
		& \Rightarrow  \d_j B + \frac{1}{3!} \d_j C = 0
	\end{aligned}
	\]
	where we have used the condition on $\epsilon$ already obtained. Thus $C = 3A = -6B$. Finally 
	Let's look at $G$'s equation of motion: 
	\[
		\dd G = 0, \qquad \frac{1}{3!} \dd \star G + \frac{3}{(144)^2} \epsilon^{MNOPQRST} G_{MNOP} G_{QRST} = 0
	\]
	By assumption, the term quadratic in $G$ vanishes. What remains gives us:
	\[
		0 = \d_r (e^{3A+8B} e^{-6A - 2B}  C'(r) e^C) = \d_r (e^{-3A + 6B + C} C') = \d_r (C' e^{-C}) \Rightarrow \d_r^2 e^{-C} = 0
	\]
	So we have that $e^{-C} = H(r)$ as required, where
	\[
		H(r) = 1 + \frac{L^6}{r^6}
	\]
	I'm happy with this. I could use Mathematica to show that the other EOM:
	\[
		R_{MN} - \frac12 g_{MN} R = \kappa^2 T_{MN}, \quad \kappa^2 T_{MN} = \frac{1}{2 \cdot 4!} \left(4 G_{M P Q R} G_{N}^{PQR} - \frac12 g_{MN} G^2\right)
	\]
	is satisfied - but this is barely different from what I've done several times before for the D-branes and fundamental string solutions in chapter 8. 
	
	As before, this generalizes straightforwardly to multi-membrane configurations. 
	
	The charge of the M2 brane with $H = 1 + \frac{32 \pi^2 N \ell_s^6}{r^6}$ is given by integrating $\frac{\star G}{2\kappa_{11}^2}$ on a seven-sphere at infinity. Here $2 \kappa_{11}^2 = (2\pi)^8 \ell_{11}^9$ Asymptotically we will get the field strength going as
	\[
		\frac{32 \times 6 \pi^2 N \ell_{11}^6}{r^6}
	\]
	Altogether, using $\Omega_7 = \frac{\pi^4}{3}$ this gives a total charge of
	\[
		\frac{\pi^4}{3} \frac{32 \times 6 \pi^2 N \ell_{11}^6}{(2\pi)^8 \ell_{11}^9} = \frac{N}{(2\pi)^2 \ell_{11}^2}
	\]
	This is exactly consistent with \textbf{11.4.10-13}, with $\mu = N = 1$ corresponding to a single M2 brane. 
	
	Calculating the Ricci scalar curvature in fact gives a \emph{constant} as $r \to 0$ so we do \emph{not} encounter a divergence. This signifies that this is just a coordinate singularity and we can extend past. 
	\begin{center}
		\includegraphics[scale=0.5]{"Figures/M2 Ricci"}
	\end{center}
	Finally, we can take the near-horizon limit and get
	\[
	\begin{aligned}
		ds^2 &= \frac{r^4}{L^4} \eta_{\mu \nu} dx^\mu dx^\nu + \frac{L^2}{r^2} dx^i \cdot dx^i\\
		 &= \frac{r^4}{L^4} \eta_{\mu \nu} dx^\mu dx^\nu + \frac{L^2}{r^2} dr^2  + L^2 d \Omega_7^2
	\end{aligned}
	\]
	Take now $r = L/\sqrt{z}$ to get the first term to look like $1/z^2$ while not affecting the second term much:
	\[
		\frac{1}{z^2} (\eta_{\mu \nu} dx^\mu dx^\nu + 4 L^2 dz^2) + L^2 d \Omega_7^2
	\]
	We can rescale $z, x^\mu$ and see that  this geometry is AdS$_4 \times S^7$ 
	
	\item The M5 brane is now magnetically charged under $C_3$. Now the equations of motion $\dd \star \dd C = 0$ are trivially satisfied but the Bianchi identity is nontrivial, giving
	\[
		\d^2_r H = 0 \Rightarrow H = 1 + \frac{L^3}{r^3}
	\]
	The metric form can be fixed by analyzing the gravitino variation similar to before. Longitudinally:
	\[
	\begin{aligned}
		0 &= \frac12 A' e^{A-B} \Gamma^{\hat \mu \hat r} + \frac{1}{2 \cdot 3!} C' e^{C+A-4B} \Gamma^{\hat \theta_1 \hat \theta_2 \hat \theta_3 \hat \theta_4 \hat \mu}\\
		&\Rightarrow A' \epsilon + \frac{1}{3!} C' e^{C-3B} \Gamma^{\hat r \hat \theta_1 \hat \theta_2 \hat \theta_3 \hat \theta_4 } \epsilon
	\end{aligned}
	\]
	We see that we must take $C = 3B$ and $A=-C/6$, and we get the half-BPS condition:
	\[
		(1 - \Gamma^{\hat 7 \hat 8 \hat 9 \hat{10} \hat{11}}) \epsilon = 0
	\]
	 The transverse components will give the profile for $\epsilon$.
	\[
		\d_r \epsilon + \frac{1}{2 \cdot 3!} C' e^{C-3B} \Gamma^{\hat \theta_1 \hat \theta_2 \hat \theta_3 \hat \theta_4 \hat r}  \epsilon
	\] 
	and this gives a profile 
	\[
		\epsilon = e^{-C/12} \epsilon_0
	\]
	The membrane charge is given by integrating $G$ on a $4$-sphere whose area is given by $8\pi^2/3$, so we get
	\[
		\frac{8\pi^2}{3} \frac{3 \pi N \ell_{11}^3}{(2\pi^8) \ell_{11}^9} = \frac{N}{(2\pi \ell_{11})^5 \ell_{11}}
	\]
	Again we get that the Ricci scalar tends to a constant as $r \to 0$, giving regularity at the horizon. Again, this signifies that this is just a coordinate singularity and we can extend past. 
	\begin{center}
		\includegraphics[scale=0.5]{"Figures/M5 Ricci"}
	\end{center}
	
	Taking the near-horizon limit we arrive at
	\[
		ds^2 = \frac{r}{L}  \eta_{\mu \nu} dx^\mu dx^\nu + \frac{L^2}{r^2} dx^i \cdot dx^i = \frac{r}{L} \eta_{\mu \nu} dx^\mu dx^\nu + \frac{L^2}{r^2} dr^2 + L^2 d\Omega_4^2
	\]
	Now take $r = L/z^2$ yielding 
	\[
		\frac{1}{z^2}  (\eta_{\mu \nu} dx^\mu dx^\nu  + 4 L^2 dr^2) + L^2 d\Omega_4^2
	\]
	so again after rescaling the same was as before we get AdS$_7 \times S^4$.
	
	As before, a solution can consist of an arbitrary number of $M5$ branes at different places, in which case we get
	\[
		H(r) = 1 + \sum_{i} \frac{L_i}{|r-r_i|^3}
	\]
	This remains half-BPS.

	\item First look at the field strengths. The general M5 brane solution 
	 For a uniform distribution of M5 charges, we know that in the transverse (3D) space the potential must now decay as
	\[
		H = 1 + \int dx^{11} \frac{L}{|\vec r - x^{11} \hat e_{11}|^2} = 1 + \frac{2L}{r_{10D}^2}
	\]
	where $L$ depends on the density of the distribution. Then the $3$-form field strength in 10D will just be
	\[
		(dB)_{abc} = \epsilon_{abce} \d_e H
	\]
	
	 Given this source in 10D, we have already worked out Einstein's equations in \textbf{Chapter 8}. Another way to see this is that we remain half-BPS after adding even an infinite number of parallel branes. 
	
	We have that $e^{4\Phi/3} = G_{11,11}$ so that $e^\Phi = H^{1/2}$ consistent with the NS5 solution. 
	
	Using the perscription of dimensional reduction in appendix \textbf{I.2}, we take $e^{\sigma} = e^{2\Phi/3} = H^{1/3}$. Using $g_{\mu \nu} = e^{-\sigma} g^S_{\mu\nu}$, we see that multiplying by $H^{1/3}$ takes us to the \emph{string frame} NS5 metric solution. 
	\[
		ds^2 = \eta_{\mu \nu} dx^\mu dx^\nu + H(r) dx^i \cdot dx^i
	\]
	This is exactly the NS5 metric in string frame.
	
	We can further take $g^S_{\mu \nu} = e^{\Phi/2} g^E_{\mu \nu}$ and multiply the string frame by $e^{-\Phi/2} = H^{-1/4}$ to get us to the Einstein frame. 
	
	\item Recall the BPS D6 brane in 10D is described by 
	\[
		H^{-1/2} \eta_{\mu \nu} dx^\mu dx^\nu + H^{1/2} d\vec x \cdot d \vec x , \quad H = 1 + \frac{L}{|x|}, L = g_s \ell_s N /2, \qquad F = L \, \dd \Omega_2, \quad e^{\Phi} = g_s^2 H^{-3/4}
	\]
	This means that $e^{-2\Phi/3} = H^{1/2}$. Multiplying $ds^2_{string}$ by this factor, we the 10D part of 11D metric
	\[
		\eta_{ab} d\gamma^a d \gamma^b + V d\vec x \cdot d \vec x
	\]
	Here we've picked notation consistent with the problem so that $\gamma^{0 \dots 6} = x^{0 \dots 6}$, $H(r) = V(r)$, and $x^i$ is the same.
	
	Note also that
	\[
		\frac{1}{2 \kappa_{10}^2} \int_{S^2} F = \frac{L 4 \pi}{(2\pi)^7 \ell_s^8 g_s^2} = n T_p \Rightarrow 2 L = \ell_s n g_s
	\]
	This should be supplemented by $e^{4 \Phi/3} (d\gamma + A_\mu \cdot d \vec x)^2 = V^{-1} (d\gamma + A_\mu \cdot d \vec x)^2$ where $A_\mu$ is the 10D gauge field generated by the monopole solution. 
	
	Now $A$ cannot be globally defined because of the monopole. Given $L = 2N$, it takes the same form as $A_\mu$ does in 3D about a monopole of charge $n = N/\ell_s$.
	
	We could have taken a more ``active'' approach, demonstrating that this metric ansatz does indeed solve Einstein's equations, and shown that for the field strength to satisfy the Bianchi identity in this geometry it needed to indeed be a harmonic function of the transverse coordinates taken with flat metric. 
	
	% To describe the lift of the D6 brane take $\gamma^a, a = 0, \dots 6$ parallel, $x^i, i = 7 \dots 9$ the 3D transverse directions and $\gamma$ the eleventh dimension. Consider a KK monopole-type metric:
	% \[
	% 	ds_{11}^2 = \sqrt{}
	% \]
	
	\item The DBI action for a two-brane \emph{in flat space with vanishing $B$-field and constant dilaton} is given in euclidean signature as
	\[
		- T_2 \int d^3 x \sqrt{\det (\delta_{ab} + \d_a X^\mu \d_b X^\nu + 2 \pi \ell_s^2 F_{ab})} + i \int C^{(3)} \wedge \Tr[e^{\mathcal F}] \wedge \mathcal G,
	\]
	where the second integral consists of Chern-Simons terms that we will ignore in this argument. We can work with the field variable $F$ rather than $A$ by imposing the Bianchi identity ``by hand'', namely writing the (non-CS) part of the action as
	\[
		-T_2 \int d^3 x \left[ \sqrt{\det (\delta_{ab} + \d_a X^\mu \d_b X^\nu + 2 \pi \ell_s^2 F_{ab})} + \frac{i}{2} \lambda \epsilon^{abc} \d_a F_{b c} \right]
	\]
	This last term can just as well be integrated by parts to give $\epsilon^{abc} \d_a \lambda F_{bc}$.
	
	We now introduce an auxiliary $V$ variable to rewrite the action as
	\[
	\begin{aligned}
		&-T_2 \int d^3 x \left[\frac12 V \det(\delta_{ab} + \d_a X^\mu \d_b X^\nu + 2 \pi \ell_s^2 F_{ab})  + \frac12 \frac1V + \frac{i}{2}\epsilon^{abc} \d_a \lambda F_{bc} \right]\\
		& = -T_2 \int d^3 x \left[\frac12 V (1 + \frac12 (2\pi \ell_s^2)^2 F_{ab}^2  + \dots ) + \frac12 \frac{1}{V} + \frac{i}{2}\epsilon^{abc} \d_a \lambda F_{bc} \right]
	\end{aligned}
	\]
	here $\dots$ involves terms depending on the $\d_a X^\mu$.
	The equations of motion for $F$ then give 
	\[
		F_{ab} = - i \frac{\epsilon^{abc} \d_a \lambda}{(2 \pi \ell_s^2)^2 V}
	\]
	Substituting this back in gives
	\[
		 -T_2 \int d^3 x  \left[ \frac12 V (1 + (- \frac12 + 1) (2 \pi \ell_s^2)^{-2} (\d \lambda)^2 + \dots ) + \frac12 \frac{1}{V} \right]
	\]
	Integrating out $V$ gives us the square root action again, but now with $F$ replaced by $\d \lambda$, a new coordinate 
	\[
		- T_2 \int d^3 x \sqrt{\det (\delta_{ab} +\d_a X^\mu \d_b X^\nu +  (2 \pi \ell_s^2)^{-2}  \d_a \lambda \d_b \lambda )}
	\]
	Taking $X = \lambda / 2\pi \ell_s^2$ gives our desired result
	
	\textbf{I have only shown classical equivalence. How to I prove this is quantum-mechanically true as well?}
	
	% Now, we can introduce an auxiliary metric and rewrite this action in Polyakov form to get rid of the square root. The  $F$ terms in the Polyakov action would look like
% 	\[
% 		- \frac{T}{2} \int \sqrt{-h} h^{a b} F_{a b} - \frac{i}{2} \d_a \lambda \epsilon^{abc} F_{bc}.
% 	\]
% 	Then, we can integrate out $F$ directly, performing the gaussian integral and getting
% 	\[
% 		- \frac{1}{2 T} h^{a b} \d_a \lambda \d_b \lambda
% 	\]
	
	\item We are looking at the transformation $\tau \to -1/\tau$. We see taht 
	\[
		C_0 + i e^{- \Phi} \to \frac{-1}{C_0 + i e^{- \Phi}} = \frac{-C_0 + i e^{- \Phi}}{C_0^2 + e^{-2\Phi}}
	\]
	So we see $C_0 \to - \frac{C_0}{C_0^2 + e^{-2\phi}}$ and $e^{-\Phi} \to \frac{e^{-\Phi}}{C_0^2 + e^{-2\Phi}}$. On the other hand, $C_0$ will not affect the $C_2, B_2$ transformations. Nor will it affect $C_4$, which remains invariant
	
	In the Einstein frame the metric is invariant. That means that $e^{\Phi/2} g_{string}$ is invariant, which means $g_{string}$ transforms as $e^{-\Phi/2}$ times the Einstein frame metric. Consequently, in the string frame $g_{string}' = e^{-\Phi} g_{string}$ (I think Kiritsis is wrong here, and Polchinski agrees with this)

	\textbf{Am I missing anything with that last one?}
	
	\item There's effectively nothing to derive. Translating the the Einstein frame means multiplying all lengths by $e^{-\Phi/4}$. At fixed dilaton this is $g_s^{-1/4}$. Given $\ell_s^2$ in the denominator will then contribute a factor $\sqrt{g_s}$ overall, that's exactly what was done here.  
	
	\item We have that $C_4$ is invariant. That means that objects charged under $C_4$ remain charged under $C_4$, with the same charge. These are precisely the D3/anti-D3 branes. Now recall the DBI action has coupling constant
	\[
		g_{YM}^2 = \frac{1}{(2\pi \ell_s^2)^2 T_3} = 2 \pi g_s
	\]
	note that this is dimensionless, as it should be for a gauge theory in 4D. At low energies, the closed strings decouple we can reliably trust the DBI action, considering the D-brane gauge theory on its own. In the absence of axion, the $\SL(2, \ZZ)$ of IIB takes $g_s \to 1/g_s$. This corresponds to
	\[
		g^2_{YM} \to \frac{4\pi^2}{g_{YM}^2}
	\]
	So this is the Weak-Strong Montonen-Olive duality of $\mathcal N=4$ SYM.
	
	The only subtlety is that one must take care to include the Chern-Simons terms in the DBI action in order to get the full duality, specifically 
	\[
		\int C_0 \Tr [F \wedge F].
	\]
	At fixed $C_0 = \theta/2\pi$ this produces the instanton number. The duality $C_0 \to C_0 + 1$ is a bona-fide duality of the $\mathcal N=4$ theory, a consequence of the fact that instanton charge is quantized. 
	
	\textbf{Is there anything else that I can say that constitutes any form of ``showing'' that this fact is true?}
	
	\textbf{The only thing is I think I'm assuming that the D3 brane is the only object charged under $C_3$ at leading order in $\ell_s$. Can I safely assume this?}
	
	\item We should go to the Einstein frame, ie multiply the F1 solution by $H^{1/4}$. The F1 solution is then:
	\[
		ds^2_E = H^{-3/4}(-dt^2 + (dx^1)^2)+ H^{1/4} d \vec x \cdot d \vec x, \qquad H = 1 + \frac{L^6}{r^6}
	\]
	Here $L^6 = \frac{2 \kappa_{10}^2 T_p}{6 \Omega_7} = 32 \ell_s^6 \pi^2$ % where $N \in \ZZ$ is the $B_2$ charge, interpretable as the number of fundamental strings.
	Note this is the same metric as the D1 solution, and indeed the metric will stay the same for all $(p,q)$ strings.
	
	The $C_0$ field has been set to zero. For F1 the dilaton and $B$-field have the profile
	\[
		e^{\Phi} = g_s H^{-1/2}, \quad B_{01} = H^{-1}(r)
	\]
	and indeed the dilaton has the inverse of this for the D1 while $B$ and $C$ exchange. Indeed, consider the $\SL_2(\ZZ)$ action 
	\[
		\Lambda = \begin{pmatrix}
			a & b \\  c & d
		\end{pmatrix}.
	\]
	Here, we have $ad-bc = 1$, implying $c, d$ are relatively prime. Further $\mathcal S = C_0 + i e^{-\phi}$ and $C_2, B_2$ transform as
	\[
		\mathcal S \to \frac{a \mathcal S + b}{c \mathcal S + d}, \qquad \begin{pmatrix}
			B_2\\C_2
		\end{pmatrix} \to (\Lambda^T)^{-1} \cdot \begin{pmatrix}
			B_2 \\ C_2
		\end{pmatrix} = \begin{pmatrix}
			d & -c\\ -b & a
		\end{pmatrix}   \begin{pmatrix}
			B_2 \\ C_2
		\end{pmatrix}
	\]
	There is only one subtlety in the problem, which gives us the unique $\Lambda$ we should take, which I got from reading arXiv:hep-th/9508143. We need to fix the dilaton's asymptotic value as $r \to \infty$ so as to define the vacuum of our string theory. First, consider $\phi, C_0 = 0$ asymptotically, ie $\mathcal S \to i$. We then stay within the $\SO(2) \subset  \SL_2(\RR)$ that fixes $\mathcal S = i$. We want to take $(1,0)$ to the string $p, q$. This is now uniquely determined:
	\[
		\Lambda = \frac{1}{\sqrt{q_1^2 + q_2^2}} \begin{pmatrix}
			p & -q\\
			q & p
		\end{pmatrix}
	\]
	Applying this to $B_2, C_2$ gives
	\[
		\begin{pmatrix}
					B_2\\ C_2
				\end{pmatrix} = \frac{H^{-1}}{\sqrt{p^2 + q^2}} \begin{pmatrix}
					p\\q
				\end{pmatrix}
	\]
	Upon doing this, the $B_2, C_2$ fluxes will have coefficients that get modified from just $p, q$ by a factor of $\frac{1}{\sqrt{p^2 + q^2}}$, so will no longer be integers satisfying the quantization condition. We can fix this by modifying $T \to T_{p,q} = \sqrt{p^2 + q^2} T$. Since this only serves to modify $L$, which was an arbitrary parameter of the classical solution, we still remain in a solution.
	
	This means: $H_{p,q} = 1 + \frac{L^6_{p,q}}{r^6}$, $L_{p,q}^6 = \frac{2 \kappa^2 T_{p,q}}{6 \Omega_7} = \sqrt{q_1^2 + q_2^2}\,  \frac{2 \kappa^2 T_{1,0}}{6 \Omega_7} = \sqrt{q_1^2 + q_2^2} L^6$.
	
	Our solution is now:
	\[
		ds^2_E = H_{p,q}^{-3/4} (-dt^2 + (dx^1)^2) + H_{p,q}^{1/4} d\vec x \cdot d \vec x \qquad \begin{pmatrix}
			B_2\\ C_2
		\end{pmatrix} = \frac{H_{p,q}^{-1}}{\sqrt{p^2 + q^2}} \begin{pmatrix}
			p\\q
		\end{pmatrix}\qquad \mathcal S = \frac{i p H_{p,q}^{-1/2} - q}{i q H_{p,q}^{-1/2} + p}
	\]
	Now, let us generalize this for different asymptotic values of the dilaton and axion. After applying $\Lambda$, compose with:
	\[
		\Lambda' = \begin{pmatrix}
			e^{-\phi_0 / 2} & \chi_0 e^{\phi_0/2}\\
			0 & e^{\phi_0/2}
		\end{pmatrix}
	\]
	$\mathcal S$ originally asymptotes to $i$, and now it asymptotes to
	\[
		\frac{e^{-\phi_0/2} i + \chi_0 e^{\phi_0/2}}{0 + e^{\phi_0/2}} = \chi + i e^{-\phi_0}
	\]
	exactly as we want. To get this to play well with the field strengths, we should make our first $\Lambda$ take the form
	\[
		(\Lambda^T)^{-1} = \begin{pmatrix}
			\cos \theta & -\sin \theta\\
			\sin \theta & \cos \theta
		\end{pmatrix}
	\]
	Again, applying this will break our quantization condition. The asymptotic value of the charges of $B_2, C_2$ is given by $(p, q)/\Delta^{1/2}_{p,q}$ where $\Delta_{p,q}$ is the invariant
	\[
		\Delta_{p,q} = \begin{pmatrix}
			p & q
		\end{pmatrix} \;
		\mathcal S_2^{-2}
		\begin{pmatrix}
			|\mathcal S|^2 & \mathcal S_1\\
			\mathcal S_1 & 1
		\end{pmatrix}\begin{pmatrix}
			p \\ q
		\end{pmatrix}  = e^{\phi_0} (p - q \chi_0)^2 + e^{-\phi_0} q^2
	\]
	\textbf{Worth working out a bit more explicitly?}

	This gives in full generality:
	\[
		T_{p,q} = \sqrt{e^{\phi_0} (p - q \chi_0)^2 + e^{-\phi_0} q^2} \; T_{F1}
	\]
	
	Because (aside from redefining $L$) the metric is unchanged, the singularity structure of $(p,q)$ strings is no different from $(1,0)$ or $(0,1)$ strings. Neither of these has a regular horizon.
	
	
	\item
	
	The 11-D SUGRA Lagrangian is
	\[
		L_{D=11} = \frac{1}{2\kappa_{11}^2}\left[R - \frac12 |G_4|^2 + G_4 \wedge G_4 \wedge \hat C_3 \right]
	\]
	Let's take M-theory to $9$ dimensions. The $R$ term becomes:
	% \overbrace{(2 \pi R_{11})^2 \tau_2}^A
	\[
		\frac{1}{2 \kappa_{11}^2} e^{-2 \phi} \Big[R + 4 \d^\mu \phi \d_\mu \phi + \frac14 \d_\mu G_{\alpha \beta} \d^\mu G^{\alpha \beta} - \frac14 G_{\alpha \beta} {F_{\mu \nu}^A}^\alpha {F^{A\, \mu \nu}}^\beta \Big]
	\]
	with $\phi = -\frac14 \det G_{\alpha \beta}, F_{\mu \nu}^A = \d_\mu A_\nu^\alpha - \d_\nu A^\mu \alpha$. The kinetic 3-form potential yields
	\[
		\frac{1}{2 \kappa_{11}^2} e^{-2 \phi} \Big[-\frac12 |F_4|^2 - 4 \times \frac12 |F_3|^2 - 6 \times \frac12 |F_2|^2 \Big]
	\]
	
	The IIB SUGRA Lagrangian is
	\[
		e^{-2\Phi} \left[R + 4 (\nabla \Phi)^2 - \frac12 |H_3|^2 \right] - \frac12 |F_1|^2 - \frac12 |F_3|^2 - \frac14 |F_5|^2
	\]
	supplemented by $\star F_5 = F_5$. Taking this to 9 dimensions, the $R + 4 (\nabla \Phi)^2$ term becomes
	\[
		\frac{1}{2 \kappa_{10}^2} e^{-2 (\Phi) + \sigma} \Big[R + 4 (\nabla \Phi)^2 + (\d_\mu \sigma)^2 - \frac14 e^{-2 \sigma} {F_{\mu \nu}^A} {F^{A\, \mu \nu}}  - \frac{1}{2} |H_3|^2 - \frac12 e^{-4 \rho} |H_{2}|^2 \Big]
	\]
	with $G_{10,10} = e^{2 \sigma}$. The RR forms give
	\[
		e^{\sigma} \Big[ -\frac12 (\d_\mu C_0)^2 - \frac12 |F_3|^2 - \frac12 e^{-2 \sigma} |F_2|^2 - \underbrace{\frac14 |F_5|^2}_{\text{dualize} } - \frac14 e^{-2\sigma} |F_4|^2 \Big]
	\]
	Here $F_2$ comes from $F_3$ and we can dualize the 9D $F_5$ to an $F_4$.
	
	To get to the 9D Einstein frame 
	\item 
	
	\item Again, we know the $(1,0)$ and $(0,1)$ 5-brane, namely the D5 and NS5. 
	
	\item 
	
	
\end{enumerate}

% section chapter_11_duality_connections_and_nonperturbative_effects (end)
\end{document}
	