\documentclass[11pt, class=article, crop=false]{standalone}
\usepackage{amsmath,amssymb,amsfonts,amsthm}
\usepackage{enumitem}
\usepackage{fancyhdr}
\usepackage{tikz-cd}
\usepackage{mathabx}
\usepackage{geometry}
\usepackage{natbib}
\usepackage{braket}
\usepackage{graphicx}
\usepackage{simpler-wick}
\usepackage{hyperref}
\usepackage{ytableau}
\usepackage{cancel}
\usepackage{listings}
\usepackage{relsize}
\usepackage{xcolor}
\usepackage{stmaryrd}
\usepackage{slashed}
\usepackage{tikz-feynman}
\usepackage{kiritsis}
\geometry{margin = 0.5in}


\begin{document}
\section*{Chapter 11: Duality Connections and Nonperturbative Effects} % (fold)
\label{sec:chapter_11_duality_connections_and_nonperturbative_effects}
\begin{enumerate}
	\item Taking the expression for a toroidal heterotic compactification from exercise \textbf{9.1} 
	\[
		\left[\frac{R}{\sqrt{\tau_2} \eta \bar \eta^{17}} \sum_{m,n} e^{-\frac{\pi R^2}{\tau_2} |m + n \tau|^2} e^{-i \pi \sum_I n Y^I (m + n \bar \tau) Y^I Y^I}\frac12 \sum_{a,b=0}^1  \prod_{i=1}^{16} \bar \theta \twist ab (Y^I (m + \bar \tau n) | \bar \tau) \right] \times \frac{1}{\tau_2^{7/2} \eta^7 \bar \eta^7} \frac12 \sum_{a,b = 0}^1 \frac{\theta^4 \twist a b}{\eta^4}
	\]
	Using $\theta$ function identitiess as in the second equation in appendix \textbf{E}, we get
	\[
		\Gamma_{1,17}(R, Y) = \frac{R}{\sqrt{\tau_2}} \sum_{m,n} e^{-\frac{\pi R^2}{\tau_2} |m + n \tau|^2} \frac12 \sum_{a,b=0}^1 e^{i \pi m Y^I Y^I n - i \pi b n Y^I} \bar \theta \twist{a - 2n Y^I}{b - 2m Y^I}
	\]
	Now take $Y^I = 0$ for $I = 1 \dots 8$ and $Y^I = 1/2$ for $I = 1 \dots 16$. Then
	\[
		\prod_I e^{i \pi m Y^I Y^I n - i \pi b n Y^I}  = e^{i \pi m \sum_I (Y^I)^2 - i \pi b \sum_I Y^I} = 1
	\]
	and we can ignore this term. Similarly because we are taking a product over 16 $\bar \theta$, no phases will interfere with us replacing $\theta\twist uv$ with $\theta \twist{-u}{-v}$ for integer $u, v$. This gives us the desired first step
	\[
		\Gamma_{1,17}(R, Y) = R \sum_{m,n} e^{-\frac{\pi R^2}{\tau_2} |m + n \tau|^2} \frac12 \sum_{a,b=0}^1 \bar \theta \twist ab^8 \bar \theta \twist{a+n}{b+m}^8
	\]
	Now again because we have enough $\theta \twist{a+n}{b+m}$ that phases do not interfere, we see that we only care about $n,m$ modulo $2$ in the fermion term. We know how to divide the partition function of the compact boson into parity odd and even blocks by doing the $\mathbb Z^2$ stratification corresponding to the $\pi R$ translation orbifold of the circle. This gives our desired answer:
	\[
		\frac12 \sum_{h,g} \Gamma_{1,1}(2R) \twist hg \frac12 \sum_{a,b} \bar \theta \twist ab^8 \bar \theta \twist{a+h}{b+g}^8
	\]
	with 
	\[
		\Gamma_{1,1} (2R) = 2R \sum_{m,n} \exp\left[\frac{-\pi R^2}{\tau_2} |2m + g + (2n + h)\tau|^2 \right]
	\]
	
	\item As before, take the ansatz
	\[
		ds^2 = e^{2A(r)} \eta_{\mu \nu} dx^\mu dx^\nu + e^{2B(r)} dx^i \cdot dx^i, \qquad A_{012} = \pm e^{C(r)} \Rightarrow G_{r 012} = \pm C'(r)\, e^{C(r)}
	\]
	
	The BPS states in 11D require only the gravitino variation to vanish:
	\[
		\delta \psi_M = \d_M \epsilon + \frac14 \omega_{M}^{PQ} \Gamma_{PQ} \epsilon + \frac{1}{2 \cdot 3! \cdot 4!} G_{PQRS} \Gamma^{PQRS}  \Gamma_M \epsilon - \frac{8}{2 \cdot 3! \cdot 4!} G_{MQRS} \Gamma^{QRS} \epsilon
	\]
	We have worked out $\omega$ in \textbf{8.43}.
	\[
		\omega_{\hat \mu \hat \nu} = 0,\quad \omega_{\hat \mu \hat i} = (-)^{\mu = 0}\, \d_i A \, e^{A-B} dx^\mu, \quad \omega_{\hat i \hat j} = \d_j B dx^i - \d_i B dx^j
	\]
	Let's look first at $M = \mu$ parallel. Since $\epsilon$ is Killing we expect no longitudinal variation and we get
	\[
	\begin{aligned}
		0 &= \cancel{\d_\mu \epsilon} + \frac12 A' \, e^{A - B} \Gamma^{\hat \mu \hat r} \epsilon \pm \frac{1}{2 \cdot 3!} C'(r) e^{C} \cancel{\Gamma^{r 0 12} \Gamma_\mu} \epsilon \mp \frac{1}{3!} C'(r) e^{C} \Gamma_\mu \Gamma^{r 0 1 2} \epsilon\\
		&= \frac12 A' \, e^{A - B} \Gamma^{\hat \mu \hat r} \epsilon \mp \frac{1}{3!} C' e^{C-B-2A} \Gamma^{\hat \mu \hat r \hat 0 \hat 1 \hat 2 } \epsilon \\
		& \Rightarrow 0 = A' \epsilon \mp \frac{1}{3} C' e^{C-3A} \Gamma^{\hat 0 \hat 1 \hat 2 } \epsilon\\
	\end{aligned}
	\] 
	If we would like these two terms to be proportional, then we should take $C= 3 A$, and we get the following condition for $\epsilon$
	\[
		(1 \mp \Gamma^{\hat 0 \hat 1 \hat 2}) \epsilon = 0
	\]
	So half the dimension of the space of spinors satisfies this at any given point. We thus get 
	
	For $M = i$ transverse, we recall $\Gamma_{ij}$ generates rotations, so assuming rotational invariance in the transverse space, we'll cancel this. We get
	\[
	\begin{aligned}
		&\d_r \epsilon + \cancel{\frac14 \omega_{r}^{jk} \Gamma_{jk} \epsilon} +\cancel{\frac{1}{2 \cdot 3!} G_{r012} \Gamma^{r 012} \Gamma_r \epsilon} \mp \frac{1}{3!} G_{r012} \Gamma^{012} \epsilon = 0\\
		\Rightarrow& \d_r \epsilon \mp \frac{1}{3!} G_{r012} \Gamma^{012} \epsilon = 0\\
		\Rightarrow& \d_r \epsilon \mp \frac{e^{-3A}}{3!} C' e^C \Gamma^{\hat 0 \hat 1 \hat 2} \epsilon
	\end{aligned}
	\]
	Solving this gives us that 
	\[
		\epsilon(r) = e^{C(r)/6} \epsilon_0
	\]
	for $\epsilon_0$ some constant spinor. We still do not have a relationship between $C$ and $B$. This can be obtained by not assuming rotational invariance but rather imposing cancelation of the second and third terms above as follows:
	\[
	\begin{aligned}
		&\frac12 \d_j B \, \Gamma^{\hat i \hat j} \epsilon  \pm \frac{1}{2 \cdot 3!} \d_j C \, e^{C} \, \Gamma^{j012} \Gamma_i \epsilon \\
		&= \frac12 \d_j B \, \Gamma^{\hat i \hat j} \epsilon  \pm \frac{1}{2 \cdot 3!} \d_j C \, e^{C-3A} \, \Gamma^{\hat i \hat j \hat 0 \hat 1 \hat 2} \epsilon\\
		& \Rightarrow  \d_j B + \frac{1}{3!} \d_j C = 0
	\end{aligned}
	\]
	where we have used the condition on $\epsilon$ already obtained. Thus $C = 3A = -6B$. Finally 
	Let's look at $G$'s equation of motion: 
	\[
		\dd G = 0, \qquad \frac{1}{3!} \dd \star G + \frac{3}{(144)^2} \epsilon^{MNOPQRST} G_{MNOP} G_{QRST} = 0
	\]
	By assumption, the term quadratic in $G$ vanishes. What remains gives us:
	\[
		0 = \d_r (e^{3A+8B} e^{-6A - 2B}  C'(r) e^C) = \d_r (e^{-3A + 6B + C} C') = \d_r (C' e^{-C}) \Rightarrow \d_r^2 e^{-C} = 0
	\]
	So we have that $e^{-C} = H(r)$ as required, where
	\[
		H(r) = 1 + \frac{L^6}{r^6}
	\]
	I'm happy with this. I could use Mathematica to show that the other EOM:
	\[
		R_{MN} - \frac12 g_{MN} R = \kappa^2 T_{MN}, \quad \kappa^2 T_{MN} = \frac{1}{2 \cdot 4!} \left(4 G_{M P Q R} G_{N}^{PQR} - \frac12 g_{MN} G^2\right)
	\]
	is satisfied - but this is barely different from what I've done several times before for the D-branes and fundamental string solutions in chapter 8. 
	
	As before, this generalizes straightforwardly to multi-membrane configurations. 
	
	The charge of the M2 brane with $H = 1 + \frac{32 \pi^2 N \ell_s^6}{r^6}$ is given by integrating $\frac{\star G}{2\kappa_{11}^2}$ on a seven-sphere at infinity. Here $2 \kappa_{11}^2 = (2\pi)^8 \ell_{11}^9$ Asymptotically we will get the field strength going as
	\[
		\frac{32 \times 6 \pi^2 N \ell_{11}^6}{r^6}
	\]
	Altogether, using $\Omega_7 = \frac{\pi^4}{3}$ this gives a total charge of
	\[
		\frac{\pi^4}{3} \frac{32 \times 6 \pi^2 N \ell_{11}^6}{(2\pi)^8 \ell_{11}^9} = \frac{N}{(2\pi)^2 \ell_{11}^2}
	\]
	This is exactly consistent with \textbf{11.4.10-13}, with $\mu = N = 1$ corresponding to a single M2 brane. 
	
	Calculating the Ricci scalar curvature in fact gives a \emph{constant} as $r \to 0$ so we do \emph{not} encounter a divergence. This signifies that this is just a coordinate singularity and we can extend past. 
	\begin{center}
		\includegraphics[scale=0.5]{"Figures/M2 Ricci"}
	\end{center}
	Finally, we can take the near-horizon limit and get
	\[
	\begin{aligned}
		ds^2 &= \frac{r^4}{L^4} \eta_{\mu \nu} dx^\mu dx^\nu + \frac{L^2}{r^2} dx^i \cdot dx^i\\
		 &= \frac{r^4}{L^4} \eta_{\mu \nu} dx^\mu dx^\nu + \frac{L^2}{r^2} dr^2  + L^2 d \Omega_7^2
	\end{aligned}
	\]
	Take now $r = L/\sqrt{z}$ to get the first term to look like $1/z^2$ while not affecting the second term much:
	\[
		\frac{1}{z^2} (\eta_{\mu \nu} dx^\mu dx^\nu + 4 L^2 dz^2) + L^2 d \Omega_7^2
	\]
	We can rescale $z, x^\mu$ and see that  this geometry is AdS$_4 \times S^7$ 
	
	\item The M5 brane is now magnetically charged under $C_3$. Now the equations of motion $\dd \star \dd C = 0$ are trivially satisfied but the Bianchi identity is nontrivial, giving
	\[
		\d^2_r H = 0 \Rightarrow H = 1 + \frac{L^3}{r^3}
	\]
	The metric form can be fixed by analyzing the gravitino variation similar to before. Longitudinally:
	\[
	\begin{aligned}
		0 &= \frac12 A' e^{A-B} \Gamma^{\hat \mu \hat r} + \frac{1}{2 \cdot 3!} C' e^{C+A-4B} \Gamma^{\hat \theta_1 \hat \theta_2 \hat \theta_3 \hat \theta_4 \hat \mu}\\
		&\Rightarrow A' \epsilon + \frac{1}{3!} C' e^{C-3B} \Gamma^{\hat r \hat \theta_1 \hat \theta_2 \hat \theta_3 \hat \theta_4 } \epsilon
	\end{aligned}
	\]
	We see that we must take $C = 3B$ and $A=-C/6$, and we get the half-BPS condition:
	\[
		(1 - \Gamma^{\hat 7 \hat 8 \hat 9 \hat{10} \hat{11}}) \epsilon = 0
	\]
	 The transverse components will give the profile for $\epsilon$.
	\[
		\d_r \epsilon + \frac{1}{2 \cdot 3!} C' e^{C-3B} \Gamma^{\hat \theta_1 \hat \theta_2 \hat \theta_3 \hat \theta_4 \hat r}  \epsilon
	\] 
	and this gives a profile 
	\[
		\epsilon = e^{-C/12} \epsilon_0
	\]
	The membrane charge is given by integrating $G$ on a $4$-sphere whose area is given by $8\pi^2/3$, so we get
	\[
		\frac{8\pi^2}{3} \frac{3 \pi N \ell_{11}^3}{(2\pi^8) \ell_{11}^9} = \frac{N}{(2\pi \ell_{11})^5 \ell_{11}}
	\]
	Again we get that the Ricci scalar tends to a constant as $r \to 0$, giving regularity at the horizon. Again, this signifies that this is just a coordinate singularity and we can extend past. 
	\begin{center}
		\includegraphics[scale=0.5]{"Figures/M5 Ricci"}
	\end{center}
	
	Taking the near-horizon limit we arrive at
	\[
		ds^2 = \frac{r}{L}  \eta_{\mu \nu} dx^\mu dx^\nu + \frac{L^2}{r^2} dx^i \cdot dx^i = \frac{r}{L} \eta_{\mu \nu} dx^\mu dx^\nu + \frac{L^2}{r^2} dr^2 + L^2 d\Omega_4^2
	\]
	Now take $r = L/z^2$ yielding 
	\[
		\frac{1}{z^2}  (\eta_{\mu \nu} dx^\mu dx^\nu  + 4 L^2 dr^2) + L^2 d\Omega_4^2
	\]
	so again after rescaling the same was as before we get AdS$_7 \times S^4$.
	
	As before, a solution can consist of an arbitrary number of $M5$ branes at different places, in which case we get
	\[
		H(r) = 1 + \sum_{i} \frac{L_i}{|r-r_i|^3}
	\]
	This remains half-BPS.

	\item First look at the field strengths. The general M5 brane solution 
	 For a uniform distribution of M5 charges, we know that in the transverse (3D) space the potential must now decay as
	\[
		H = 1 + \int dx^{11} \frac{L}{|\vec r - x^{11} \hat e_{11}|^2} = 1 + \frac{2L}{r_{10D}^2}
	\]
	where $L$ depends on the density of the distribution. Then the $3$-form field strength in 10D will just be
	\[
		(dB)_{abc} = \epsilon_{abce} \d_e H
	\]
	\begin{center}
		\includegraphics[scale=0.2]{"Drawings/M5"}
	\end{center}
	
	
	 Given this source in 10D, we have already worked out Einstein's equations in \textbf{Chapter 8}. Another way to see this is that we remain half-BPS after adding even an infinite number of parallel branes. 
	
	We have that $e^{4\Phi/3} = G_{11,11}$ so that $e^\Phi = H^{1/2}$ consistent with the NS5 solution. 
	
	Using the perscription of dimensional reduction in appendix \textbf{I.2}, we take $e^{\sigma} = e^{2\Phi/3} = H^{1/3}$. Using $g_{\mu \nu} = e^{-\sigma} g^S_{\mu\nu}$, we see that multiplying by $H^{1/3}$ takes us to the \emph{string frame} NS5 metric solution. 
	\[
		ds^2 = \eta_{\mu \nu} dx^\mu dx^\nu + H(r) dx^i \cdot dx^i
	\]
	This is exactly the NS5 metric in string frame.
	
	We can further take $g^S_{\mu \nu} = e^{\Phi/2} g^E_{\mu \nu}$ and multiply the string frame by $e^{-\Phi/2} = H^{-1/4}$ to get us to the Einstein frame. 
	
	\item Recall the BPS D6 brane in 10D is described by 
	\[
		H^{-1/2} \eta_{\mu \nu} dx^\mu dx^\nu + H^{1/2} d\vec x \cdot d \vec x , \quad H = 1 + \frac{L}{|x|}, L = g_s \ell_s N /2, \qquad F = L \, \dd \Omega_2, \quad e^{\Phi} = g_s^2 H^{-3/4}
	\]
	This means that $e^{-2\Phi/3} = H^{1/2}$. Multiplying $ds^2_{string}$ by this factor, we the 10D part of 11D metric
	\[
		\eta_{ab} d\gamma^a d \gamma^b + V d\vec x \cdot d \vec x
	\]
	Here we've picked notation consistent with the problem so that $\gamma^{0 \dots 6} = x^{0 \dots 6}$, $H(r) = V(r)$, and $x^i$ is the same.
	
	Note also that
	\[
		\frac{1}{2 \kappa_{10}^2} \int_{S^2} F = \frac{L 4 \pi}{(2\pi)^7 \ell_s^8 g_s^2} = n T_p \Rightarrow 2 L = \ell_s n g_s
	\]
	This should be supplemented by the metric component in the internal 11th dimension, given by $e^{4 \Phi/3} (d\gamma + A_\mu \cdot d \vec x)^2 = V^{-1} (d\gamma + A_\mu \cdot d \vec x)^2$ where $A_\mu$ is the 10D gauge field generated by the monopole solution. 
	
	Now $A$ cannot be globally defined because of the monopole. Given $L = 2N$, it takes the same form as $A_\mu$ does in 3D about a monopole of charge $n = N/\ell_s$.
	
	We could have taken a more ``active'' approach, demonstrating that this metric ansatz does indeed solve Einstein's equations, and shown that for the field strength to satisfy the Bianchi identity in this geometry it needed to indeed be a harmonic function of the transverse coordinates taken with flat metric. 
	
	\begin{center}
		\includegraphics[scale=0.2]{"Drawings/Cigar"}
	\end{center}
	
	\item The DBI action for a two-brane \emph{in flat space with vanishing $B$-field and constant dilaton} is given in euclidean signature as
	\[
		- T_2 \int d^3 x \sqrt{\det (\delta_{ab} + \d_a X^\mu \d_b X^\nu + 2 \pi \ell_s^2 F_{ab})} + i \int C^{(3)} \wedge \Tr[e^{\mathcal F}] \wedge \mathcal G,
	\]
	where the second integral consists of Chern-Simons terms that we will ignore in this argument. We can work with the field variable $F$ rather than $A$ by imposing the Bianchi identity ``by hand'', namely writing the (non-CS) part of the action as
	\[
		-T_2 \int d^3 x \left[ \sqrt{\det (\delta_{ab} + \d_a X^\mu \d_b X^\nu + 2 \pi \ell_s^2 F_{ab})} + \frac{i}{2} \lambda \epsilon^{abc} \d_a F_{b c} \right]
	\]
	This last term can just as well be integrated by parts to give $\epsilon^{abc} \d_a \lambda F_{bc}$.
	
	We now introduce an auxiliary $V$ variable to rewrite the action as
	\[
	\begin{aligned}
		&-T_2 \int d^3 x \left[\frac12 V \det(\delta_{ab} + \d_a X^\mu \d_b X^\nu + 2 \pi \ell_s^2 F_{ab})  + \frac12 \frac1V + \frac{i}{2}\epsilon^{abc} \d_a \lambda F_{bc} \right]\\
		& = -T_2 \int d^3 x \left[\frac12 V (1 + \frac12 (2\pi \ell_s^2)^2 F_{ab}^2  + \dots ) + \frac12 \frac{1}{V} + \frac{i}{2}\epsilon^{abc} \d_a \lambda F_{bc} \right]
	\end{aligned}
	\]
	here $\dots$ involves terms depending on the $\d_a X^\mu$.
	The equations of motion for $F$ then give 
	\[
		F_{ab} = - i \frac{\epsilon^{abc} \d_a \lambda}{(2 \pi \ell_s^2)^2 V}
	\]
	Substituting this back in gives
	\[
		 -T_2 \int d^3 x  \left[ \frac12 V (1 + (- \frac12 + 1) (2 \pi \ell_s^2)^{-2} (\d \lambda)^2 + \dots ) + \frac12 \frac{1}{V} \right]
	\]
	Integrating out $V$ gives us the square root action again, but now with $F$ replaced by $\d \lambda$, a new coordinate 
	\[
		- T_2 \int d^3 x \sqrt{\det (\delta_{ab} +\d_a X^\mu \d_b X^\nu +  (2 \pi \ell_s^2)^{-2}  \d_a \lambda \d_b \lambda )}
	\]
	Taking $X = \lambda / 2\pi \ell_s^2$ gives our desired result
	
	\textbf{I have only shown classical equivalence. How to I prove this is quantum-mechanically true as well?}
	
	% Now, we can introduce an auxiliary metric and rewrite this action in Polyakov form to get rid of the square root. The  $F$ terms in the Polyakov action would look like
% 	\[
% 		- \frac{T}{2} \int \sqrt{-h} h^{a b} F_{a b} - \frac{i}{2} \d_a \lambda \epsilon^{abc} F_{bc}.
% 	\]
% 	Then, we can integrate out $F$ directly, performing the gaussian integral and getting
% 	\[
% 		- \frac{1}{2 T} h^{a b} \d_a \lambda \d_b \lambda
% 	\]
	
	\item We are looking at the transformation $\tau \to -1/\tau$. We see that 
	\[
		C_0 + i e^{- \Phi} \to \frac{-1}{C_0 + i e^{- \Phi}} = \frac{-C_0 + i e^{- \Phi}}{C_0^2 + e^{-2\Phi}}
	\]
	So we see $C_0 \to - \frac{C_0}{C_0^2 + e^{-2\phi}}$ and $e^{-\Phi} \to \frac{e^{-\Phi}}{C_0^2 + e^{-2\Phi}}$. On the other hand, $C_0$ will not affect the $C_2, B_2$ transformations. Nor will it affect $C_4$, which remains invariant
	
	In the Einstein frame the metric is invariant. That means that $e^{-\Phi/2} g_{string}$ is invariant, which means $g_{string}$ transforms as $e^{-\Phi/2}$ times the Einstein frame metric. Consequently, in the string frame $g_{string}' = e^{-\Phi} g_{string}$ (I think Kiritsis is wrong here, and Polchinski agrees with this)

	\textbf{Am I missing anything with that last one?}
	
	\item There's effectively nothing to derive. Translating the Einstein frame means multiplying all lengths by $e^{-\Phi/4}$. At fixed dilaton this is $g_s^{-1/4}$. Given $\ell_s^2$ in the denominator will then contribute a factor $\sqrt{g_s}$ overall, that's exactly what was done here.  
	
	\item We have that $C_4$ is invariant. That means that objects charged under $C_4$ remain charged under $C_4$, with the same charge. These are precisely the D3/anti-D3 branes. Now recall the DBI action has coupling constant
	\[
		g_{YM}^2 = \frac{1}{(2\pi \ell_s^2)^2 T_3} = 2 \pi g_s
	\]
	note that this is dimensionless, as it should be for a gauge theory in 4D. At low energies, the closed strings decouple we can reliably trust the DBI action, considering the D-brane gauge theory on its own. In the absence of axion, the $\SL(2, \ZZ)$ of IIB takes $g_s \to 1/g_s$. This corresponds to
	\[
		g^2_{YM} \to \frac{4\pi^2}{g_{YM}^2}
	\]
	So this is the Weak-Strong Montonen-Olive duality of $\mathcal N=4$ SYM.
	
	The only subtlety is that one must take care to include the Chern-Simons terms in the DBI action in order to get the full duality, specifically 
	\[
		\int C_0 \Tr [F \wedge F].
	\]
	At fixed $C_0 = \theta/2\pi$ this produces the instanton number. The duality $C_0 \to C_0 + 1$ is a bona-fide duality of the $\mathcal N=4$ theory, a consequence of the fact that instanton charge is quantized. 
	
	\textbf{Is there anything else that I can say that constitutes any form of ``showing'' that this fact is true?}
	\textbf{The only thing is I think I'm assuming that the D3 brane is the only object charged under $C_3$ at leading order in $\ell_s$. Can I safely assume this?}
	
	\item I'll start from the F1 string rather than the D1, not that it matters. Let us look at the macroscopic solution in the Einstein frame, so we multiply the string frame solution obtained in the chapter 8 exercises by $H^{1/4}$. We get:
	\[
		ds^2_E = H^{-3/4}(-dt^2 + (dx^1)^2)+ H^{1/4} d \vec x \cdot d \vec x, \qquad H = 1 + \frac{L^6}{r^6}
	\]
	Here $L^6 = \frac{2 \kappa_{10}^2 T_{F1}}{6 \Omega_7} = 32 \ell_s^6 g_s^2 \pi^2$ % where $N \in \ZZ$ is the $B_2$ charge, interpretable as the number of fundamental strings.
	Note this is the same metric as the D1 solution, and indeed the metric will stay the same for all $(p,q)$ strings.
	
	The $C_0$ field has been set to zero. For F1 the dilaton and $B$-field have the profile
	\[
		e^{\Phi} = g_s H^{-1/2}, \quad B_{01} = H^{-1}
	\]
	and indeed the dilaton has the inverse of this for the D1 while $B$ and $C$ exchange. Indeed, consider the $\SL_2(\ZZ)$ action 
	\[
		\Lambda = \begin{pmatrix}
			a & b \\  c & d
		\end{pmatrix}.
	\]
	Here, we have $ad-bc = 1$, implying $c, d$ are relatively prime. This will correspond to the fact that $(p,q)$ bound states only exist for $p,q$ relatively prime, since otherwise there is a decay process of marginal instability allowing the $(p,q)$ system to separate into two or more sub-systems. Further $\mathcal S = C_0 + i e^{-\phi}$ and $C_2, B_2$ transform as
	\[
		\mathcal S \to \frac{a \mathcal S + b}{c \mathcal S + d}, \qquad \begin{pmatrix}
			B_2\\C_2
		\end{pmatrix} \to (\Lambda^T)^{-1} \cdot \begin{pmatrix}
			B_2 \\ C_2
		\end{pmatrix} = \begin{pmatrix}
			d & -c\\ -b & a
		\end{pmatrix}   \begin{pmatrix}
			B_2 \\ C_2
		\end{pmatrix}
	\]
	There is a subtlety in the problem, which resolved the ambiguity in our choice of $\Lambda$. I learned of it from reading arXiv:hep-th/9508143. The subtlety is as follows: We need to fix the dilaton's asymptotic value as $r \to \infty$ so as to define the vacuum of our string theory. First, consider $\phi, C_0 = 0$ asymptotically, i.e. $\mathcal S \to i$. We then stay within the $\SO(2) \subset  \SL_2(\RR)$ that fixes $\mathcal S = i$. We want to take $(1,0)$ to the string $p, q$. This is now uniquely determined:
	\[
		\Lambda = \frac{1}{\sqrt{p^2 + q^2}} \begin{pmatrix}
			p & -q\\
			q & p
		\end{pmatrix}
	\]
	Applying this to $B_2, C_2$ given that we start with only NS charge $(1,0)$ gives
	\[
		\begin{pmatrix}
					B_2\\ C_2
				\end{pmatrix} = \frac{H^{-1}}{\sqrt{p^2 + q^2}} \begin{pmatrix}
					p\\q
				\end{pmatrix}
	\]
	Upon doing this, the $B_2, C_2$ fluxes will have coefficients that get modified from just $p, q$ by a factor of $\frac{1}{\sqrt{p^2 + q^2}}$, so will no longer be integers satisfying the quantization condition. We can fix this by modifying $T \to T_{p,q} = \sqrt{p^2 + q^2} T$. Since this only serves to modify $L$, which was an arbitrary parameter of the classical solution, this still remains a valid solution.
	
	This means: $H_{p,q} = 1 + \frac{L^6_{p,q}}{r^6}$, $L_{p,q}^6 = \frac{2 \kappa^2 T_{p,q}}{6 \Omega_7} = \sqrt{q_1^2 + q_2^2}\,  \frac{2 \kappa^2 T_{1,0}}{6 \Omega_7} = \sqrt{q_1^2 + q_2^2} L^6$.
	
	Our solution is now:
	\[
		ds^2_E = H_{p,q}^{-3/4} (-dt^2 + (dx^1)^2) + H_{p,q}^{1/4} d\vec x \cdot d \vec x \qquad \begin{pmatrix}
			B_2\\ C_2
		\end{pmatrix} = \frac{H_{p,q}^{-1}}{\sqrt{p^2 + q^2}} \begin{pmatrix}
			p\\q
		\end{pmatrix}\qquad \mathcal S = \chi_0 + i e^{- \phi} = \frac{i p H_{p,q}^{1/2} - q}{i q H_{p,q}^{1/2} + p}
	\]
	Note that as $r \to \infty, \mathcal S \to i$ as we expect. 
	Now, let us generalize this for different asymptotic values of the dilaton and axion. After applying $\Lambda$, we can further apply
	\[
		\Lambda' = \begin{pmatrix}
			e^{-\phi_0 / 2} & \chi_0 e^{\phi_0/2}\\
			0 & e^{\phi_0/2}
		\end{pmatrix}
	\]
	$\mathcal S$ now asymptotes to
	\[
		\frac{e^{-\phi_0/2} i + \chi_0 e^{\phi_0/2}}{0 + e^{\phi_0/2}} = \chi_0 + i e^{-\phi_0}
	\]
	exactly as we want. To get the right final field strengths, take $\Lambda$ initially arbitrary:
	\[
		(\Lambda^T)^{-1} = \begin{pmatrix}
			\cos \theta & -\sin \theta\\
			\sin \theta & \cos \theta
		\end{pmatrix}
	\]
	Again, applying this will break our quantization condition. Now, the electric charges transform \emph{contragradiently} from the fields strengths, which means that
	\[
		\begin{pmatrix}
			Q_B\\
			Q_C
		\end{pmatrix}
		= e^{\phi_0/2} \begin{pmatrix}
			e^{-\phi_0} \cos \theta + \chi_0 \sin \theta\\
			\sin \theta
		\end{pmatrix}  =: \frac{1}{\sqrt{\Delta_{p,q}}} \begin{pmatrix}
			p\\q
		\end{pmatrix}
	\]
	We can solve this to get 
	\[
		\sin \theta = \frac{e^{\phi_0/2}}{\sqrt{\Delta_{p,q}}} e^{-\phi_0} q \Rightarrow \cos \theta = \frac{e^{\phi_0/2}}{\sqrt{\Delta_{p,q}}} (p - \chi_0 q) \Rightarrow e^{i \theta} = \frac{e^{\phi_0/2}}{\sqrt{\Delta_{p,q}}} (p - \overline{\mathcal S} q )
	\]
	The asymptotic value of the charges of $B_2, C_2$ is thus given by $(p, q)/\Delta^{1/2}_{p,q}$. Unimodularity gives:
	\[
		1 = e^{i \theta} e^{-i\theta} \Rightarrow \Delta_{p,q} = e^{\phi_0} |p - q \mathcal S|^2 = e^{\phi_0} (p-q \chi_0)^2 + e^{-\phi_0} q^2
	\]
	This coincides with the invariant
	\[
		\begin{pmatrix}
			p & q
		\end{pmatrix} \;
		\mathcal S_2^{-2}
		\begin{pmatrix}
			|\mathcal S|^2 & \mathcal S_1\\
			\mathcal S_1 & 1
		\end{pmatrix}\begin{pmatrix}
			p \\ q
		\end{pmatrix}  = e^{\phi_0} (p - q \chi_0)^2 + e^{-\phi_0} q^2
	\]
	So in full generality we get the tension:
	\[
		T_{p,q} = \sqrt{e^{\phi_0} (p - q \chi_0)^2 + e^{-\phi_0} q^2} \; T_{F1}
	\]
	Where $T_{F1} = \frac{1}{2 \pi \ell_s^2}$ is the tension in the string frame. 
	
	Because (aside from redefining $L$) the metric is unchanged, the singularity structure of $(p,q)$ strings is no different from $(1,0)$ or $(0,1)$ strings. Neither of these has a regular horizon. \textbf{Confirm}
	
	
	\item First, by naive reasoning - there is no reason to write the full effective action to see what $\beta$ should look like. From the perspective of IIA in the string frame, we have coupling $g_A = \tilde R_{11}/\ell_s = R_{11} \tau_2/ \ell_s = R_{11}/\ell_{s} g_B$. Recognizing $R_B = \ell_s^2/R_11$ we can write this as $\frac{\ell_s}{R_B g_B}$. Because the translation of metrics between the 11-D frame and the standard string frame in IIA involves the factor $g_A^{2/3}$ we get $\beta = \left(\frac{\ell_s}{R_B g_B} \right)^{2/3}$. \textbf{What about conversion to the Einstein frame?}
	
	\begin{center}
		\includegraphics[scale=0.13]{"Drawings/M IIB"}
	\end{center}
	
	Now let's do it the long way. 
	The 11-D SUGRA Lagrangian is
	\[
		\mathcal L_{D=11} = \frac{1}{2\kappa_{11}^2}\left[R - \frac12 |G_4|^2 + G_4 \wedge G_4 \wedge \hat C_3 \right]
	\]
	In this problem we'll ignore the Chern-Simons terms. 
	
	Let's take M-theory to $9$ dimensions. The metric takes the form
	\[
		G_{\hat \mu \hat \nu} = \begin{pmatrix}
			g_{\mu \nu} + G_{\alpha \beta} A_\mu^\alpha A_\nu^\beta & G_{\alpha \beta} A^\beta_\mu \\
			G_{\alpha \beta} A_\nu^\beta  & G_{\alpha \beta}
		\end{pmatrix}
	\]
	Here
	\[
		G_{\alpha \beta} = \frac{e^{\sigma}}{\tau_2} \begin{pmatrix}
			1 & \tau_1\\
			\tau_1 & |\tau|^2
		\end{pmatrix}
	\]
	with $e^{\sigma} = \sqrt{\det G}$ the K\"ahler parameter (area) of the torus. 
	The metric's $R$ term becomes:
	\[
		\frac{1}{2 \kappa_{11}^2} e^{\sigma} \Big[R + \d^\mu \sigma \d_\mu \sigma + \frac14 \d_\mu G_{\alpha \beta} \d^\mu G^{\alpha \beta} - \frac14 G_{\alpha \beta} {F_{\mu \nu}^A}^\alpha {F^{A\, \mu \nu}}^\beta \Big].
	\]
	Here $F_{\mu \nu}^A = \d_\mu A_\nu^\alpha - \d_\nu A^\mu \alpha$. We now have \emph{two} $U(1)$ field strengths. Using the fact that we have the explicit form of the torus metric, we can further write this as:
	\begin{equation}\label{eq:MframeLag}
		\frac{(2\pi R_{11})^2 \tau_2}{(2\pi)^8 \ell_{11}^9} e^\sigma \left[R + \frac12 (\d\sigma)^2 - \frac12 \frac{(\d \tau_1)^2}{\tau_2^2} - \frac12 \frac{(\d\tau_2)^2}{\tau_2^2} - \frac{1}{2\cdot 2!} \frac{e^{\sigma}}{\tau_2} \begin{pmatrix}
			F^{A,1} & F^{A,2}
		\end{pmatrix}\begin{pmatrix}
			1 & \tau_1\\ \tau_1 & |\tau|^2
		\end{pmatrix}\begin{pmatrix}
			F^{A,1}\\ F^{A,2}
		\end{pmatrix} \right]
	\end{equation}
	The kinetic 3-form potential yields a four-form, two three-form, and a two-form field strength:
	\[
		\frac{R_{11}^2 \tau_2}{(2 \pi)^6 \ell_{11}^9} e^{\sigma} \Big[-\frac{1}{2 \cdot 4!} F_{\mu \nu \rho \sigma}^{(4)} F^{(4)\, \mu \nu \rho \sigma} - \frac{1}{2 \cdot 3!} G^{\alpha \beta} F_{\mu \nu \rho \alpha}^{(3)} F_{\qquad\; \beta}^{(3)\, \mu \nu \rho} - \frac{1}{2 \cdot 2!} G^{\alpha \beta} G^{\gamma \delta} F_{\mu \nu \alpha \gamma} F_{\beta \delta}^{(2)\, \mu \nu}]
	\]
	Again, using the explicit form of the metric we can write this as
	\[
	\begin{aligned}
		&\frac{R_{11}^2 \tau_2}{(2 \pi)^6 \ell_{11}^9} e^{\sigma} \Big[-\frac{1}{2} |F_4|^2 
		 - \frac{1}{2 \cdot 3!}\frac{e^{-\sigma}}{\tau_2} \begin{pmatrix}
			F^{(3)} & H^{(3)}
		\end{pmatrix}\begin{pmatrix}
			1 & \tau_1\\ \tau_1 & |\tau|^2
		\end{pmatrix}\begin{pmatrix}
			F^{(3)}\\ H^{(3)}
		\end{pmatrix}
		 - \frac{1}{2 \cdot 2!} \frac{e^{-2\sigma}}{\tau_2^2} (|\tau|^2 - \tau_1^2) F_{\mu \nu 12}^{(2)} F_{\qquad12}^{(2)\, \mu \nu} ]\\
		 &= \frac{R_{11}^2 \tau_2}{(2 \pi)^6 \ell_{11}^9} \Big[-\frac{e^{\sigma}}{2} |F_4|^2 
		 - \frac{1}{2 \cdot 3!}\frac{1}{\tau_2} \begin{pmatrix}
			F^{(3)} & H^{(3)}
		\end{pmatrix}\begin{pmatrix}
			1 & \tau_1\\ \tau_1 & |\tau|^2
		\end{pmatrix}\begin{pmatrix}
			F^{(3)}\\ H^{(3)}
		\end{pmatrix}
		 - \frac{e^{-\sigma}}{2} |F_2|^2 \Big]
	\end{aligned}
	\]
	Here the last 2-form field strength is defined as $F^{(2)}_{\mu \nu} := F^{(2)}_{\mu \nu 12}$.
	
	This action not in any standard frame. Let's take it to the Einstein frame $g_{11} = e^{-2/7 \sigma} g_{E}$: 
	\[
	\begin{aligned}
		 \frac{R_{11}^2 \tau_2}{(2 \pi)^6 \ell_{11}^9} \Big[R - \frac37 (\d\sigma)^2 - \frac12 \frac{(\d \tau_1)^2}{\tau_2^2} &- \frac12 \frac{(\d\tau_2)^2}{\tau_2^2} - \frac{1}{2\cdot 2!} \frac{e^{9\sigma/7}}{\tau_2} \begin{pmatrix}
			F^{A,1} & F^{A,2}
		\end{pmatrix}\begin{pmatrix}
			1 & \tau_1\\ \tau_1 & |\tau|^2
		\end{pmatrix}\begin{pmatrix}
			F^{A,1}\\ F^{A,2}
		\end{pmatrix} \\
		& - \frac{e^{-12\sigma/7}}{2} |F_2|^2
		 - \frac{1}{2 \cdot 3!}\frac{e^{-3\sigma/7}}{\tau_2} \begin{pmatrix}
			F^{(3)} & H^{(3)}
		\end{pmatrix}\begin{pmatrix}
			1 & \tau_1\\ \tau_1 & |\tau|^2
		\end{pmatrix}\begin{pmatrix}
			F^{(3)}\\ H^{(3)}
		\end{pmatrix}
		  -\frac{e^{6\sigma/7}}{2} |F_4|^2  \Big]
	\end{aligned}
	\]
	
	The IIB SUGRA Lagrangian \textbf{in the string frame} is
	\[
		\frac{1}{2\kappa_{10}^2}  \left[ e^{-2\Phi} \left[R + 4 (\nabla \Phi)^2 - \frac12 |H_3|^2 \right] - \frac12 |F_1|^2 - \frac12 |F_3|^2 - \frac14 |F_5|^2  \right]
	\]
	supplemented by $\star F_5 = F_5$. Taking this to 9 dimensions, the NSNS terms become
	\[
		\frac{2\pi R_B}{(2\pi)^7 \ell_s^8 g_B^2} e^{-2 \phi} \Big[R + 4 (\nabla \phi)^2 - (\d \rho)^2  - \frac{e^{2\rho}}{2} |F^A|^2  - \frac{1}{2} |H_3|^2 - \frac{e^{-2\rho}}{2} |H_2|^2 \Big]
	\]
	with $G_{10,10} = e^{2 \rho}$, $\phi = \Phi - \frac12 \rho$. The RR forms give
	\[
		\frac{R_B}{(2\pi)^6 \ell_s^8 g_B^2} e^{\rho} \Big[ -\frac12 F_1^2 - \frac{1}{2 \cdot 3!} F_3^2 - \frac{e^{-2 \rho}}{4} F_2^2 - \underbrace{\frac{1}{4 \cdot 5!} F_5^2}_{\text{dualize}} \, -  \, \frac{e^{-2\rho}}{4 \cdot 5!} F_4^2 \Big].
	\]
	Here $F_2$ comes from $F_3$ and $F_4$ from $F_5$. We can dualize the 9D $F_5$ to give the canonical normalization to the $F_4$ term.
	\[
		\frac{R_B}{(2\pi)^6 \ell_s^8 g_B^2} \Big[ -\frac{e^\rho}{2} |F_1|^2 - \frac{e^{\rho}}{2} |F_3|^2 - \frac{e^{-\rho}}{2} |F_2|^2 - \frac{e^{-\rho}}{2} |F_4|^2 \Big].
	\]
	It is important to T-dualize this to get to IIA. This takes $\phi \to \phi, \rho \to - \rho, G^{(9)} \to G^{(9)}$ and also swaps $H_2$ and $F^A$. Lastly, we have $g_B^2/R_B = g_A^2/R_A$. We then get
	\[
	\begin{aligned}
		\mathcal L_{IIA} = \frac{R_A}{(2\pi)^6 \ell_s^8 g_A^2} \Bigg[ e^{-2\phi} \Big[R  + 4 (\nabla \phi)^2 - (\d \rho)^2 
		&- \frac{e^{2\rho}}{2} |F^A|^2  - \frac{1}{2} |H_3|^2 - \frac{e^{-2\rho}}{2} |H_2|^2 \Big]\\
		 &-\frac{e^{-\rho}}{2} |F_1|^2 - \frac{e^{-\rho}}{2} |F_3|^2 - \frac{e^{\rho}}{2} |F_2|^2 - \frac{e^{\rho}}{2} |F_4|^2 \Bigg]
	\end{aligned}
	\]
	Now let's take this to the Einstein frame $g_S = e^{4/7 \phi} g_{E}$: 
	\[
	\begin{aligned}
		\mathcal L_{IIA}^E = \frac{R_A}{(2\pi)^6 \ell_s^8 g_A^2} \Bigg[ R  - \frac47 (\nabla \phi)^2 - (\d \rho)^2 
		&- \frac{e^{2\rho - 4 \phi/7}}{2} |F^A|^2  - \frac{e^{-8\phi/7}}{2} |H_3|^2 - \frac{e^{-2\rho-4 \phi/7}}{2} |H_2|^2 \\
		 &-\frac{e^{-\rho + 2 \phi}}{2} |F_1|^2 - \frac{e^{\rho + 10 \phi/7}}{2} |F_2|^2  - \frac{e^{-\rho + 6 \phi/7}}{2} |F_3|^2 - \frac{e^{\rho + 2 \phi/7}}{2} |F_4|^2 \Bigg]
	\end{aligned}
	\]
	Comparing $|\tau_1|$ with $|F_1|^2$ since these are the only two scalars that aren't minimally coupled, we get $-\rho_A + 2 \phi = - 2 \log(\tau_2)$. T-dualizing to get back to IIB gives $\rho_B + 2 \phi = 2 \Phi_B = - 2 \log \tau_2$ implying that $\tau_1 = C_0$ and $\tau_2 = e^{- \Phi}$ in IIB as required. 
	
	Comparing the $F_4$ coefficient gives $\rho_A + 2 \phi/7 = 6 \sigma/7$. This gives $\sigma = \frac43 \rho_A + \frac13 \Phi_B = -\frac43 \rho_B + \frac13 \Phi_B$. This gives $A^{3/2} g^{-1/2} \sim R_{B}^{-2}$, close to what is desired. Expressing the relevant quantities in terms of the fundamental units of their respective frames, this gives our desired relationship 
	\[
		\frac{\ell_s^2}{R_B^2} = \frac{A^{3/2}}{(2 \pi \ell_{11})^3 g^{1/2}} \Rightarrow \frac{1}{R_{B}^2} = \frac{R_{11}^3}{\ell_s^5 g^{5/2}}
	\]
	\textbf{Off by a factor of $g^{1/2}$}
	
	Note that in the IIA action, $F^A$ and $F_2$ have coefficients that differ by $-\rho_A + 2 \phi = 2 \Phi_B$. We should thus identify them with $e^{9 \sigma/7 \pm \Phi_B}$ of the M theory action. This implies that $9 \sigma / 7 = \frac32 \rho_A + 3 \phi/7$, exactly what we got from the $F_4$ coefficient. The same argument for the $F_3, H_3$ terms in both theories gives the same difference between them, and their average gives the same relationship. Finally, the lone $H_2$ term in IIA compared to the $F_2$ gives the same dependence as well, giving three nontrivial checks that what we've done is correct. 

	Finally let's get the conversion factor. To go from 11D to the string frame we must do $e^{4/7 \phi} e^{2/7 \sigma}$. We now understand $\sigma = -\frac{4}{3} \rho_B + \frac13 \Phi_B$ and $\phi = \Phi - \frac{\rho_B}{2}$ we get the relationship
	\[
		\frac27 \sigma + \frac47 \phi = - \frac23 (\rho_B - \Phi_B) \Rightarrow \beta = \left(\frac{\ell_s}{R_B g_s}\right)^{2/3}
	\]
	as required. 
	\textbf{The dilaton dependence is flipped, fix!}
	
	\item There is a subtlety in this problem involving the form of the metric. Recall that the Einstein frame metric $g_E$ gets mapped to itself under S-duality $g_E = g_E'$. This implies that the string frame metric $g_S = e^{\Phi/2} g_S$ is related to its S-dual by:
	\[
		g_S = e^{-\Phi'} g_S'
	\]
	We can verify this at the level of the solutions to the string equations of motion:
	\[
		\begin{aligned}
			\text{\textbf{D5}}: \; ds_E^2 &= H^{-1/4} dx_{\parallel} + H^{3/4} dx_\perp, \quad ds_S^2 = H^{-1/2} dx_\parallel + H^{1/2} dx_\perp, \quad &e^{\Phi} = g_s H^{-1/2} \\
			\text{\textbf{NS5}}: \; ds_E^2 &= H^{-1/4} dx_{\parallel} + H^{3/4} dx_\perp, \quad ds_S^2 =  dx_\parallel + H dx_\perp, \quad &e^{\Phi} = g_s H^{1/2}
		\end{aligned}
	\]
	We see that the string frame metric are related in this way \emph{except for the issue of rescaling by $g_s$}. This means we should redefine length so that $ds_S^2$ asymptotes to $g_s \eta_{\mu \nu}$ for the NS5 metric \textbf{why don't we modify D5 instead?}. 
	
	\begin{center}
		\includegraphics[scale=0.18]{"Drawings/D5 S NS5"}
	\end{center}
	
	First, let's calculate the energy of the F1 string stretched between two $D5$ branes. % At weak coupling, the backreaction is minimal and the metric in the string frame is just purely flat space.
	Directly from the Nambu-Goto action, noting that the parallel $X^\mu$ will be along the $\tau$ direction while the transverse $X^i$ will be along the $\sigma$ direction we can write 
	\[
	\begin{aligned}
		S_{NG} = -T_{F1} \int d^2 \xi \sqrt{\det(G_{ab} + B_{ab})} &= -\frac{1}{2\pi \ell_s^2} \int d^2 \xi \sqrt{\begin{vmatrix}
			\d_\tau X^\mu \d_\tau X_\mu & \d_\tau X^\mu \d_\sigma X_i (G_{\mu i} + B_{\mu i})\\
			\d_\tau X^\mu \d_\sigma X_i (G_{\mu i} + B_{\mu i}) & \d_\sigma X^i \d_\sigma X_i
		\end{vmatrix}}\\
		&= -\frac{1}{2\pi \ell_s^2} \int d \sigma d\tau \sqrt{H^{-1/2} \d_\tau X^\mu \d_\tau X^\mu} \sqrt{H^{1/2} \d_\sigma X^i \d_\sigma X^i} \\
		&= -\underbrace{\frac{1}{2\pi \ell_s^2} \int_0^\pi d\sigma |\d_\sigma X^i|}_{m_S} \int d\tau |\d_\tau X^\mu| 
	\end{aligned}
	\]
	This gives the string and Einstein frame mass:
	\[
		m_S = \frac{1}{2\pi \ell_s^2} \int_0^\pi |\d_\sigma X^1| d\sigma \Rightarrow m_E = \frac{g^{1/4}}{2\pi \ell_s^2} \, \Delta x^1
	\]
	
	For the D1 stretching the two NS5s, we apply the same logic to the DBI action:
	\[
	\begin{aligned}
		S_{DBI} &= -\int d^2 \xi\, T_{D1} \sqrt{-\det(G_{ab} + B_{ab})} = - \frac{1}{2\pi \ell_s^2}  \int d\sigma e^{-\Phi(x^i)} \sqrt{\d_\sigma X^i \d_\sigma X_i}  \int d\tau  \sqrt{\d_\tau X^\mu \d_\tau X_\mu}\\
		&= - \underbrace{\frac{\sqrt{g}}{2\pi \ell_s^2 g} \int d\sigma |\d_\sigma X^i| \int d\tau}_{m_S}  \sqrt{\d_\tau X^\mu \d_\tau X_\mu}
	\end{aligned}
	\] 
	Again we get string and  Einstein frame mass: 
	\[
		m_S = \frac{1}{2\pi \ell_s^2 \sqrt g} \int d\sigma |\d_\sigma X^1| \Rightarrow m_E = \frac{1}{2 \pi \ell_s^2 g^{1/4}} \Delta x^1
	\]
	The masses agree under S duality: $g \to 1/g$.
	
	\item The argument will go very similar to how it did for the string-like objects. Again, call $(p,q) = (1,0)$ the NS5 brane (magnetically charged under $B_2$) with $(p,q) = (0,1)$ the D5 brane (magnetically charged under $C_2$). Again, first take the axio-dilaton $\mathcal S$ to asymptote to $i$. The NS5 solution in the Einstein frame is:
	\[
		ds_E^2 = H^{-1/4} \eta_{\mu \nu} dx^\mu  dx^\nu + H^{3/4} d\vec x \cdot d\vec x, \qquad H = 1 + \frac{L^2}{r^2}
	\]
	Here $L^2 = Q \frac{2 \kappa^2_{10} T_{NS5}}{2 \Omega_3} = Q \ell_s^2$. We also have
	\[
		e^{\Phi} = g_s H^{1/2}, \qquad (dB)_{\theta \phi \psi} = -\d_r H
	\]
	
	This time, the magnetic charges transform in the same way as the field strengths (since they are associated with the Bianchi identity, not the EOMs), giving 
	\[
		\begin{pmatrix}
			Q_B\\
			Q_C
		\end{pmatrix}
		=  \begin{pmatrix}
			e^{\phi_0/2} \cos \theta\\
			\chi_0 e^{\phi_0/2} \cos \theta + e^{-\phi_0/2} \chi_0 \sin \theta
		\end{pmatrix}  =: \frac{1}{\sqrt{\Delta_{p,q}}} \begin{pmatrix}
			p\\q
		\end{pmatrix}
	\]
	Solving this gives
	\[
		\cos \theta = \frac{e^{- \phi_0/2}}{\sqrt{\Delta_{p,q}}} p \Rightarrow \sin \theta = \frac{e^{\phi_0/2}}{\sqrt{\Delta_{p,q}}} (q + p \chi_0) \Rightarrow e^{i\theta} = i \frac{e^{\phi_0/2}}{\sqrt{\Delta_{p,q}}} (q + \overline {\mathcal S} p)
	\]
	Unimodularity gives 
	\[
		\Delta_{p,q} = e^{\phi_0} |q+p \mathcal S|^2 = e^{\phi_0} (q + p \chi_0)^2 + e^{-\phi_0} p^2
	\]
	
	We thus get that the 5-brane tension in the Einstein frame satisfies a similar relation to the case of 1-branes:
	\[
		T_{p,q} = \sqrt{e^{-\phi_0} p^2 + e^{\phi_0} (q + p \chi_0)^2}\; T
	\]
	with $T = \frac{1}{(2\pi)^5 \ell_s^6}$ the appropriate dimensionful constant. 
	
	For the general $(p,q)$-brane solution, we get $L_{p,q} = \sqrt{\Delta_{p,q}} L = \sqrt{\Delta_{p,q}}\, \ell_s^2$
	
	\item We're going to work in the Einstein frame. After compactifying on $T^2$ we will get scalars not just from the $\phi$ and $C_0$ term but also from $C_2, B_2$, and the 3 metric components $G_{\alpha \beta}$. 
	
	The torus moduli in $\frac14 \d_\mu G_{\alpha \beta} \d^\mu G^{\alpha \beta}$ will take the same form as in Equation~\eqref{eq:MframeLag}, namely
	\[
		\frac{(\d T)^2}{T^2} + \frac14 \d_\mu G_{\alpha \beta} \d^\mu G^{\alpha \beta} = -\frac12 \frac{(\d \tau_1^2)}{\tau_2^2} - \frac12 \frac{(\d \tau_1^2)}{\tau_2^2} + \frac12 \frac{(\d T)^2}{T^2}
	\]
	Here $T = \sqrt{\det G_{\alpha \beta}}$ is the K\"ahler modulus and does not belong to the $\SL(2, \RR)/U(1)$ coset. We have that the axio-dilaton is $-\frac12 \frac{|\d \mathcal S|^2}{\mathcal S_2^2}$. 
	The scalars coming from $B_2, C_2$ give kinetic terms:
	\[
		-\frac12 \frac{G^{11} G^{22} - (G^{12})^2}{\mathcal S_2} (\mathcal S_2 \d_\mu B_{12})^2 = - \frac12 \frac{\mathcal S_2}{T^2} (\d_\mu B_{12})^2, \qquad - \frac12 \frac{(\d_\mu C_{12})^2}{T^2 \mathcal S_2} 
	\]
	Altogether the scalars have appear as:
	\[
		\int d^{8} x \sqrt{-g}\, T \, \Big[R \underbrace{-\frac12 \frac{(\d \tau_1^2)}{\tau_2^2} - \frac12 \frac{(\d \tau_1^2)}{\tau_2^2}}_{\SL(2, \RR)/U(1)} + \underbrace{ \frac12 \frac{(\d T)^2}{T^2} - \frac12 \frac{|\d \mathcal S|^2}{\mathcal S_2^2} - \frac12 \frac{ \mathcal S_2}{T^2}  (\d_\mu B_{12})^2 - \frac12 \frac{(\d_\mu C_{12} + C_0 H_3)^2}{T^2 \mathcal S_2} }_{\SL(3, \RR)/\SO(3, \RR)} \Big]
	\]
	Taking things to the new Einstein frame will get rid of the $T$ out front, and modify $\frac12\frac{(\d T)^2}{T^2} \to -\frac23 \frac{(\d T)^2}{T^2}$
	
	It remains to find the metric for $\SL(3, \RR)/\SO(3)$. Because $\SO(3)$ is maximally compact, we can write the metric on this space in a set of global coordinates known as \emph{Borel gauge}. This is given by taking the Einbein on $T^3$ symmetric space to be the exponentiation of the $\SL(3,\ZZ)$ Borel sub-algebra: $L = \exp[\chi^i E_i]\exp[\phi^i H_i]$. From this, the $T^3$ metric is $\mathcal M = L L^T$, and the kinetic terms are then $\Tr [\d_\mu \mathcal M\, \d^\mu \mathcal M^{-1}]$. 
	By choosing the $\chi_i$ and $\phi_i$ judiciously we see
	\begin{center}
		\hspace{-.3in} \includegraphics[scale=0.48]{"Figures/Borel Gauge"}
	\end{center}
	here $e^t = T, e^\Phi = \mathcal S_2^{-1}$. It is worth stressing that this \emph{exactly} recovers our kinetic terms. Everything matches perfectly. 
	
	\item The the field strengths coming from the two-forms yield the following terms in the Einstein frame Lagrangian
	\[
		\int d^{8} x \sqrt{-g}\, T^{2/3} \Big[ -\frac12 \frac{|F_3 + C_0 H_3|^2}{\mathcal S_2} - \frac12 \mathcal S_2 |H_3|^2- \frac12 \frac{|F_{3 \leftarrow 5}|^2}{T^2} \Big]
	\]
	The two-form field strengths $F_3, H_3$ are unaffected by dualities of the torus. $F_5$ can be dualized to an $F_3$ as well, and we also get a further $F_{3 \leftarrow 5}$ by wrapping the D3 around the torus, which will combine with the $F_{3 \leftarrow 5}^2$ to give a single (canonically normalized) field strength invariant under symmetries of the torus. Thus, the $F_3, F_{3 \leftarrow 5}, H_3$ are invariant under the $\SL(2, \ZZ)$ part of the U-duality group involving $\tau_1, \tau_2$. 
	
	We can indeed write these terms in a manifestly $\SL(3, \RR)$-invariant form, namely as
	\[
		-\frac12 \begin{pmatrix}
			H_3 & F_3 & F_{3 \leftarrow 5}'
		\end{pmatrix} \mathcal M \begin{pmatrix}
			H_3 \\ F_3 \\ F_{3 \leftarrow 5}' 
		\end{pmatrix}
	\]
	Here though, we should take care that it is really $F_{3 \leftarrow 5} + B_{12} F_3 + C_{12} H_3$ that forms the kinetic term of the action. \textbf{Understand this, as well as the $C_0 H_3$ in the Einstein frame generally.}
	\begin{center}
		\includegraphics[scale=0.6]{"Figures/SL(3,R) triplet"}
	\end{center}
	
	\item The metric will contribute $6$ scalars while the $3$-form $C_3$ will contribute a seventh. We understand how to generally build $T^3$ metrics from the last problem. Indeed, $L$ there is the einbein not on the symmetric space itself but on the \emph{torus $T^3$}. Given a Borel subgroup of $\SL(3, \RR)$, the einbein for the \emph{unit} torus is specified by three twist ``axion'' parameters $\chi_1, \chi_2, \chi_3$ and two dilaton parameters $\phi_1, \phi_2$ as:
	\[
		L = \exp[\chi^i E_i] \exp[\phi^i H_i] =  \begin{pmatrix}
			1 & \chi_1 & \chi_2 \\
			0 & 1 & \chi_3\\
			0 & 0 & 1
		\end{pmatrix} \begin{pmatrix}
			e^{\phi_1/3 - \phi_2/2} & 0 & 0\\
			0 & e^{\phi_1/3 + \phi_2/2} & 0\\
			0 & 0 & e^{-\phi_2/2}
		\end{pmatrix}
	\]
	We see directly that the parameters of this three-torus coincide exactly with the scalars $C_0, C_{12}, B_{12}, T, \Phi$ in IIB compactified on $T^2$ from the prior problem. 
	
	The 3-torus volume parameter, which we will call $T$ (not to be confused with $T$) in the prior problem, together will have kinetic terms
	\[
		\int d^8 \sqrt{-g}\, T [R + \frac{(\d T)^2}{T^2} - \frac13 \frac{(\d T)^2}{T^2} - \frac12 \frac{|C_{0 \leftarrow 3}|^2}{T^2}] = \int d^8 \sqrt{-g}\, T [R + \frac23 \frac{(\d T)^2}{T^2} - \frac12 \frac{|C_{0 \leftarrow 3}|^2}{T^2}]
	\]
	Taking this to the Einstein frame:
	\[
		\int d^8 \sqrt{-g} [R - \frac12 \frac{(\d T)^2}{T^2} - \frac12 \frac{|C_0|^2}{T^2}]
	\]
	This is exactly the $\SL(2, \ZZ)$-invariant action, which came from the perturbative $T$-duality in the earlier problem. We see they are neutral under $\SL(3, \ZZ)$, while the other 5 belonging to the $\SL(3, \RR)/\SO(3)$ coset are neutral under this $\SL(2, \ZZ)$. This re-derives the results for scalars of \textbf{Section 11.6}.
	
	From the M-theory perspective, the three distinct 2-form potentials come from wrapping the $C_3$ around different $T^3$ cycles from 11D. 
	
	\item From M-theory, the $3$-form $C^{(3)}$ descends directly down to a 3-form in the 8D picture. This has a field strength $G_4$ with kinetic term
	\[
		\int d^8 x \sqrt{g} \, T \left[- \frac12 |G_4|^2 \right] 
	\]
	which is the same in \emph{both} the original \emph{and} the Einstein frames. We could have started with $\star G_4$ in 11D, giving the 8D action:
	\[
		\int d^8 x \sqrt{g} \, T^{-1} \left[- \frac12 |G_4|^2 \right] 
	\]
	The Chern-Simons term further contributes a further topological piece:
	\[
		 \int d^8 x \sqrt{g}\, C_0 T \, G_4 \wedge G_4
	\]
	Summing these all together gives the standard $\SL(2, \RR)$ invariant bilinear form.  Thus, $\SL(2,\RR)$ acts by electric magnetic duality, transforming the tuple $(G, \star G)$ in the \textbf{2} representation. 
	
	\textbf{Slightly incomplete, understand the origin of the action better. Look at 9506011}
	
	\item Taking IIB down to 5D and looking at conserved vectors (coupling to point-like objects). First, note we can wind any of the $(p,q)$ strings around any of the $5$ cycles of $T^5$, giving $10$ vector currents. We also get 5 KK currents from the dimensional reduction that are $T$-dual to the string modes and together forms a $10$ of $\SO(5,5)$. We also have the D3 brane winding around any ${5 \choose 3} = 10$ cycles. Finally, the D5-brane \emph{and} NS5 can wrap the torus giving an additional $2$ charges. The NS5 is a singlet of $\SO(5,5)$. The D-branes are all T-dual and give a 16-dimensional representation, which is either the spinor or conjugate spinor depending on whether we start from IIA or IIB. 
	
	Altogether we get \textbf{1+10+16}. This is exactly the \textbf{27} representation of $E_6$ under $U$-duality. This gives a total of $27$ point-like charges, which are the 27 different electric charges than can be carried by black holes in 5D.
	
	\item Note that rescaling the string length by $e^\gamma$ will correspond rescaling the metric by $e^{-2\gamma}$. So relationships between the string lengths are inverse-square-root proportional to the relationships between the metrics.
	
	At the level of the supergravity theory, we have $g^I = e^{-\Phi^{het}} g^{het}$ is a symmetry of the theory. Then the string length scales must obey
	\[
		\ell^I_s = \ell^{het}_s \sqrt{g^{het}_s} \Rightarrow M^I_s = \frac{M^{het}_s}{\sqrt{g^{het}_s}}
	\] 
	
	\item The effective action will look like
	\[
		\frac{V}{(2\pi)^7 \ell_s^8 g_s^2} \int d^4 x \sqrt{g}\, e^{6 \sigma} e^{-2\phi} [R + \dots - \frac{e^{-2 \sigma}}{2} |H_2|^2 - \frac{1}{4} \Tr[F^2]]
	\]
	Taking this to the proper string frame requires $g \to e^{-6\sigma} g$. This gives 
	\[
		\frac{V}{(2\pi)^7 \ell_s^8 g_s^2}\int d^4 x \sqrt{g} e^{-2\phi} [R + \dots - \frac{e^{4 \sigma}}{2} |H_2|^2 - \frac{e^{6 \sigma}}{4} \Tr[F^2]]
	\]
	Because $\sigma$ is such that $V e^{6 \sigma}$ is strictly larger than the order of $\ell_s$, then both of the gauge fields will have coupling constants that go as $O(\ell_s^8 g_s^2/V e^{4 \sigma})$ or $O(\ell_s^8 g_s^2/V e^{6 \sigma})$. This is only going to be $O(1)$ if $g_s \gg 1$. In this case, we can (after a possible $T$-duality that doesn't change the coupling substantially, esp. if one of the dimensions is reasonably close to the order of the string length already) apply the type I - heterotic O duality to get a weakly coupled type I description. 

	\item Since the B-field strength $H^{het}$ gets mapped directly to the RR field strength $H^{I}$, we expect that the objects electrically charged between them should get mapped to one another. This means the heterotic fundamental string gets mapped to the D1 brane in type I. Their magnetic cousins should also be swapped, which will interchange the heterotic NS5 with the type I D5 brane. 
	At the classical level this is easy to see, since the two branes have the same supergravity solution. Clearly this is not enough, eg in IIA vs IIB the worldvolume theories of the NS5 are radically different.
	
	 To understand the quantum mechanical equivalence, we need to understand the origin of the $\Sp(2)$ on the D5 in type I and the NS5 in the heterotic picture. This question is answered (using nontrivial arguments involving ADHM) first in Witten ``Small Instantons in String Theory''. \textbf{Return and understand this when you know more $\mathcal N =2$ SUSY}.
	
	
	
	\item Certainly we see that heterotic-type I together with $T$-duality will relate both heterotic strings together, and connect this with type I which, after T-dualizing and moving the orientifold plane appropriately, will connect with the other type II string theories. 
	
	It remains to look at the self-duality of type IIB. For this, we took a leaf from Sen's paper. Let's look at IIB on a $\ZZ_2$ orientifold $T^2/(-1)^{F_L} \cdot \Omega \cdot \mathcal I$ where $\mathcal I$ is the inversion $z \to -z$ on the torus and $\Omega$ is worldsheet parity inversion. This manifold has 4 singular points that each carry $-4$ RR charge \textbf{why}. Since it is compact, we must cancel this by placing $4$ D6 branes at each of the 4 points for a total of 16. In this case, the geometry of the tetrahedron is flat everywhere except for the $4$ deficit angles of $\pi$ at each vertex. The singularities at the verticies are of $D_4 = SO(8)$ type, so this theory has an unbroken $\SO(8)^4$ gauge symmetry. The torus has moduli $T, \tau$ together with axiodilaton $\mathcal S$. There is no $B$ field in the orientifold.
	
	Now, let's T-dualize \emph{both} cycles of the torus. This keeps us in IIB, but takes us to $(T^2)'/\Omega$, undoing the effects of $(-1)^{F_L} \mathcal I$. Type IIB on this space is just Type I on $(T^2)'$, but with $\SO(32)$ broken down to $\SO(8)^4$. Now it is time to dualize to heterotic O theory. We see that we have heterotic string theory on $(T^2)'$ with gauge group broken down to $\SO(8)^4$. 
	
	Let's match the moduli: 
	\begin{itemize}
		\item The $\tau$ modulus is the same in IIB and the heterotic theory.
		\item The torus in the heterotic picture now has a $B_{89}$ scalar that gets mapped to the axion $C_0$ in IIA. $B_{89}$ can combine with the heterotic torus volume to provide another modular parameter $\rho = B + i V_{het}$. 
		\item The standard parameter in compactification on a torus is $\Psi_{het} = \Phi_{het} - \frac14 \log \det G_{\alpha \beta}^{het} = \Phi_{het} - \frac12 \log V_{het}$. This will be mapped to $-\frac12 \Phi_{IIB} + \log V_{IIB}$ where $V_{IIB}$ is the original $T^2$ radius. 
	\end{itemize} 
	
	We know that heterotic on $T^2$ has T-duality $\O(18, 2; \ZZ)$. This has a subgroup $\SO(2,2) \sim \SL(2, \ZZ) \times \SL(2, \ZZ)'$ that does not affect the Wilson lines but acts only on the torus parameters. Both $\tau$ and $\rho$ transform under fractional linear transformations of the two $\SL(2, \RR)$ separately, while $\Psi_{het}$ remains unaffected.
	
	Now, taking $V_{IIB} \to \infty$, the two $\SL(2, \ZZ)$ symmetries remain unbroken. One of these can be identified with large diffeomorphisms of the torus, and so combines with spacetime diffeomorphisms in the large $V$ limit. The remaining $\SL(2, \ZZ)$ then becomes the S-duality group. The $\SO(8)$ gauge theory living at each of the vertices is not seen, since the singularities and accompanying D7 branes have ``flown off'' to infinity. 
	
	That $\Psi_{het}$ remains unaffected means that $G_{IIB} e^{-\Phi_{IIB}/2}$ is an invariant under $\SL(2, \ZZ)$. So the volume as measured in the frame of that modified metric is an invariant. This is exactly the Einstein frame metric. 
	
	We have also seen in the chapter that the M theory - heterotic E duality can be obtained through a chain of dualities involving heterotic O - type I together with the M theory - type IIA. We are only asked to reproduce dualities between \emph{string theories} in this question however. 
	
	\item The D9 brane is orthogonally projected, as we know from tadpole conditions on it from chapter 7, and the same argument with the cylinder gives a $\frac{1}{\sqrt2}$ reduction of tension relative to type II. 
	
	For a D1 brane interacting with itself, the gravitational contribution in the cylinder amplitude also has a extra $\frac12$ factor due to the orientation-projection. Thus, the total tension of the D1 brane is lowered by a factor of $\frac1{\sqrt{2}}$ relative to type II as required. 
	
	Naively we could apply the same argument to D5 branes, which would then violate the D1-D5 Dirac quantization by a factor of 2.
	
	However, from an analysis of the cylinder amplitude for 59 and 95 strings with orientation projection, we get the constraint $\epsilon^2_{59} \zeta_5 \zeta_9 = 1$. By consistency of interactions of 59 strings with 55 and 99 strings, we get $\epsilon_{59}^2 = \epsilon_{55}^2 = \epsilon_{99}^2 = -1$. Consequently, the D5 brane will have opposite orientation projection than the D9 brane, namely the symplectic one. Taking the determinant of $\gamma = \zeta \gamma^T$ however gives $\zeta^N = 1$, so $\zeta = -1$ will only work for $N$ even. Another way to say this is: ``symplectically projected branes must move in pairs''. 
	
	Thus, the ``fundamental'' D5 brane should be thought of as a D5 with $\Sp(2)$ index $a=1,2$. Repeating the cylinder amplitude calculation gives a factor of $2^2$, which translates to a tension of $2 \times \frac{T_5^{II}}{\sqrt2} = \sqrt2 T_5^{II}$.
	
	
	\item The crucial component of this is to note that at D$p$ worldvolume theory contains a CP-odd term coupling to the lower-dimensional forms going as:
	\[
		i T_p \int d^{p+1} x C \wedge \Tr [e^{\mathcal F}] \wedge \mathcal G \supset i T_p (2\pi \ell_s^2)^2 \int d^{p+1} x\,  C_{p-3}  \Tr[F \wedge F]
	\]
	in the absence of an NS-NS background. 
	
	Consider the $9$-brane with an instanton background in the 5678 directions with instanton number obtained from integrating over $x^{5,6,7,8}$
	\[
		 \int d^4 x \frac{\Tr[F \wedge F]}{(2\pi)^2} = k
	\]
	For the case of $k=1$, the CP-odd term simplifies to
	\[
		i T_9 (2 \pi)^2 (2 \pi \ell_s^2)^2  \int d^{6} x\, C_{6} = i T_{5} \int d^{6} x \,  C_{6} 
	\] 
	This is exactly the CP-odd term for a D5 brane. In the limit of vanishing instanton size, this sources RR fields in the same way with the exact same RR charge. It is also a BPS state, so has the same mass as a D5 brane. This satisfies all the criteria to qualify as a D5 brane. 
	
	We can extend this to $k$ localized $D$-branes and see the exact same coupling
	\[
		\sum_{i=1}^k i T_5 \int_{x^{5 \dots 8} = x_i} d^6 x\, C_6
	\]
	as $k$ distinct D5 branes. For nonvanishing instanton size, this describes D5 branes ``dissolved'' in the D9.
	
	This argument can be carried over for an arbitrary pair $(p, p-4)$. 
	
	\item We have already seen by general arguments that we need the number of Newman-Dirichlet conditions to be a multiple of 4 so that the NS and R sectors have a chance of having degeneracy. I will repeat the argument here. 
	
	In the R sector, the zero-point energy is always zero because of the equal number of periodic fermions and bosons. The excitations above this will have integer or half-integer weights. 
	
	In the NS sector, the NN and DD fermions and bosons contribute zero point energies $-\frac{1}{24}$ and $-\frac{1}{48}$, so $-\frac{1}{16}$ total. The ND sector bosons and fermions contribute $\frac{1}{48}$ and $\frac{1}{24}$, ie the opposite. Altogether for $\nu$ ND boundary conditions we get:
	\[
		- \frac{(8 - \nu)}{16} + \frac{\nu}{16} = - \frac12 + \frac{\nu}{8}
	\]
	This ground state and its excitations above it will have half-integer weight when $\nu = 0$ mod $4$.
	
	 Since type I string theory necessitates 32 D9 branes to cancel out the O9 tension, we are only allowed $\nu=8,4,0$ giving D1, D5, and D9 brane configurations preserving supersymmetry in the theory.
	In the text, we have seen that D1, D5, D9 all lead to consistent worldvolume excitations that respect GSO and $\Omega$-projection 
	
	\item Let's review the logic so far. For supersymmetric open strings in the NS sector, we are principally interested in the $\psi_r$ states. Orientation projection acts as on the NN string as $\Omega \psi_r = i^{2r} \psi_r$ (in both NS and R sectors) and on the DD string as $\Omega \psi_r = - i^{2r} \psi_r$. For the R sector ground states, supersymmetry requires that for all directions NN (D9 brane) $\epsilon_R = -1$: that is, $\Omega \ket{R} = - \ket{R}$. 
	
	When we add indices, writing the NS state as $\ket{p, ij}$, for NN strings the massless levels are given by $\psi_{-1/2}^\mu \lambda_{ij} \ket{p, ij}$. We get the constraint $\lambda = -i \epsilon_{NS} \gamma \lambda^T \gamma$. WLOG we can either have $\gamma = 1$ for $\SO(N)$ with $\zeta = 1$ or $\gamma = i \omega$ for $\Sp(N)$ with $N$ even, $\zeta = -1$. In \emph{either case} the Jacobi identity require $\epsilon_{NS} = - i$. This gives that $\lambda = - \gamma^T \lambda \gamma^{-1}$ for the massless level. In both cases this corresponds to the \emph{adjoint representation}. In the DD case, we get an extra minus sign, giving $\lambda = \gamma^T \lambda \gamma^{-1}$. This corresponds to the \emph{symmetric traceless} representation plus a \emph{singlet}.
	
	For the D1 brane, the above discussion already shows us that in the 1-1 NS sector, we get the $8$ DD scalars transforming the symmetric traceless plus single representation of $\SO(N)$ together with the $2$ NN scalars transforming in the adjoint. 
	
	For the 1-1 R sector, before orientation projection we have the $16_+$ ground state from GSO. The orientation projection acts as
	\[
		\Omega \ket{S_\alpha, i, j} = - e^{i \pi (s_1 + s_2 + s_3 + s_4)} \gamma \ket{S_\alpha, i, j} \gamma^{-1}
	\]
	\textbf{What Kiritsis writes can't be the adjoint for $N=1$. We need it to have dim 1 in that case, but it would have dim 0.} I believe that the correct thing is that we have 8 fermions forming the $8_-$ (ie left-moving) and in the \emph{symmetric representation} of $\SO(N)$ while we have 8 forming the $8_+$ (ie right-moving) but in the \emph{adjoint} of $\SO(N)$ (these disappear for $N = 1$).
	
	In the 1-9 sector, we have 2 NN and 8 DN boundary conditions. The NS ground state energy is positive, so this will not contribute. The massless states come from the R ground state in the DN part combined with the $O(1,1)$ spinor from the R sector of the NN part. The fermions are right-moving (chirality +) as before. We get $32$ indices from the D9 brane and $N$ from the D1 brane. The orthogonal projection guarantees that these transform in the $(N, 32)$ bi-fundamental representation. Orientation projection disallows for the second copy of this spectrum (ie the 1-9 string with the orientation reversed).
	
	\item To get to the D5-brane from the D9-brane we T-dualize four times. In this problem we \emph{focus only on the 5-5 strings}.
	 Again, the R-sector contains the GSO-projected $16_+$ spinor before orientation projection. We must decompose under $\SO(5,1)_\parallel \times \SO(4)_\perp$. The projection condition in the R sector reads:
	\[
		\Omega \lambda_{ij} \ket{S_\alpha, ij} = - e^{i \pi (s_1 + s_2)} \gamma \lambda_{ij} \gamma^{-1} \ket{S_\alpha, ij}
	\]
	Here $\gamma = i \omega$, since we have the symplectic $\Sp(2)$ projection for the D5. Then, for $s_1 + s_2$ odd, the 6D fermion is negative chirality, and we require $\lambda = - \gamma \lambda^T \gamma^{-1}$. This gives a negative-chirality fermion in the adjoint representation of $\Sp(2)$, completing the vector multiplet. 
	
	For $s_1 + s_2$ even, the 6D fermion is positive chirality and we require $\lambda = \gamma \lambda^T \gamma^{-1}$ which will leave the skew-traceless antisymmetric  representation plus a singlet. For $\Sp(2)$ this is just the singlet, so we get a single positive chirality fermion, completing the hypermultiplet. 
	
	\textbf{I think I'm off by a sign?}
	% In the R sector, we decompose the positive helicity spinor $8_+$ under $\SO(4)_\parallel \times \SO(4)_\perp$ as
% 	\[
% 		8_+ = 2 \otimes 2 + \bar 2 \otimes \bar 2
% 	\]
% 	We still have $\epsilon_R = -1$
	
	
	\item From the D5-D5 analysis of the previous problem, we immediately see the generalization to general $\Sp(2N)$. The R sector yields fermions in the $\Sp(2)$ adjoint combining with the vectors $\psi_{-1/2}^\mu$  in the adjoint, yielding the vector multiplet. The DD boundary conditions reverse the projection sign for $\psi_{-1/2}^i \lambda_{ij} \ket{p, ij}$ yielding a sum of the skew-traceless antisymmetric representation plus a singlet. I assume this is the same as the two-index symmetric rep, by analogy to $\SO(N)$, where a similar thing happens. We also know that the R sector also provides (positive chirality) fermions to combine with this to form the hypermultiplet. 
	
	\begin{center}
		\includegraphics[scale=0.13]{"Drawings/D5"}
	\end{center}
	
	Finally, we must look at the D5-D9 spectrum. We have 4 ND boundary conditions and 6 NN ones. 
	For 4 ND boundary conditions, the NS sector ground states \emph{also} contribute to the massless spectrum. The ND conditions these consists of ground states transforming in the \textbf{4} of $\SO(4)$, combining with the singlet NS ground state of the 6 NN coordinates. This yields 4 scalars. 
	
	In the R sector, the massless states come from the bosonic ND ground state combining with one of the 4 NN R sector states giving an $\SO(5,1)$ spinor. After GSO projection, this gives a chirality $+$ fermion, completing the hypermultiplet.
	\textbf{This part is a bit shifty, thing about it}
	
	Each of these states has 32 labels from the D9 brane, and 2N labels from the D5 brane. Therefore, we get that this hypermultiplet in fact transforms in the $(2N, 32)$ bi-fundamental. Again, orientation projection simply restricts us to not have a second copy of this spectrum from 5-9 strings of opposite orientation.
	
	Say we pull apart $m$ D5 branes. Because the D5 branes move in pairs in type I, we must have $m$ an even integer. The 5-9 strings now all have positive zero-point energy and will not contribute to the massless spectrum. The 5-5 strings remain the same, but transforming in $\Sp(2N - m)$ instead of $\Sp(2N)$.
 	
	% The transverse components gives a symmetric tensor product of representations $S(\mathbf{2N} \otimes \mathbf{2N})$.  \textbf{Show}. This contains the irreducible symmetric representation together with the singlet representation.
	
	\item We can focus on the purely chiral left-moving CFT, since this is the only part that the orbifold acts on nontrivially. Immediately, we see that the untwisted sector corresponds to the NS states, which are the same between IIA and IIB. 
	
	In the twisted sector, we again have NS and R fermions. Because the NS fermions are taken to minus themselves, they are now \emph{integrally modded} while the R fermions become half-integral. Again, the $R$ fermions will be projected out by the $(-1)^{\mathbf{F}_L}$. The $8$ NS fermions will give two (unprojected) ground states $8 + \bar 8$ of fermion numbers $1, -1$ respectively. In Polchinski's convention, the original $\ket{0}$ NS ground state has fermion number $-1$, so the only the $C$ operator on top of this will give something that is unprojected. In Kiritsis' convention, the NS ground state has fermion number $1$ but we we take $(-1)^F = -1$ for GSO. In either case, we can only keep the $C$ operator. In the original IIA we kept the $S$ on the left and the $C$ on the right. Now we keep $C$ on both sides giving IIB (we could have done the same with $(-1)^{\mathbf{F}_R}$, and $C,C$ or $S,S$ both yield IIB, since they are related by parity).
	
	Orbifolding IIB by this symmetry is the same as orbifolding twice. This necessarily must return us back to IIA. 
	
	The M theory parity orbifold differs from this $(-1)^{F_L}$ orbifold primarily in that it includes fixed points, on which the twisted sectors localize. 
	
	\item Start with the heterotic E theory and compactify on a circle. $n$ units of KK momentum on this circle will be T-dualized to $n$ units NS flux in the O(32) theory, ie a string wrapping the circle $n$ times. Upon S-duality, this will correspond to a D1 brane wrapping the circle in type I $n$ times. We T-dualize again to get a D0 brane in the type I' theory carrying $n$ units of charge. In the strong coupling limit, this is understood as $n$ units of momentum in the eleventh direction.  
	
	\item 
	The bosonic part of the vector multiplet on a single boundary is given by
	\[
		-\frac{1}{4 \lambda^2} \int d^{10} x \sqrt{-g_{10}} \,\Tr[F^2]
	\]
	At first glance, $\lambda$ would appear arbitrary. Anomaly cancelation will yield an exact value for it in terms of the eleven-dimensional gravitational coupling. Kiritsis writes explicitly $\lambda^2 = 2 \pi (4 \pi \kappa_{11}^2)^{2/3}$, which gives that the dimensionless ratio $\lambda^6/\kappa_{11}^4 = (2 \pi)^3 (4 \pi)^2 = 128 \pi^5$. This is as in Horava and Witten, but it is not obvious that this is how $\lambda$ is determined from $\kappa_{11}$ from first principles. 
	
	Let's recall anomaly cancelation in 10D. Recall that for the type I supergravity theories, it was crucial to have enough 10D vector multiplets to cancel the $\Tr[R^6]$ terms, giving $n=496$
	
	We can view the M-theory orbifold $\RR^{11}/\ZZ_2$ as giving rise to two twisted sectors (as in string theory). In 11D we have
	\[
		\Gamma^1 \dots \Gamma^{11} = 1
	\]
	 The supersymmetries preserved by the $\ZZ_2$ action are those that satisfy (WLOG) $\Gamma^{11} \epsilon = \epsilon$. This means that in the 10D perspective, this gives rise to chiral fermions in the $16_+$. Although in the smooth part of the bulk, there cannot be a gravitational anomaly, the incorporation of a boundary (or more) can. A general diffeomorphism in the bulk will not lead to any anomalous variation $\delta \Gamma$ of the effective action. WLOG, take a diffeomorphism on $\RR^{10}$ and pull it back to the orbifold by making it constant along the interval $S^1/\ZZ_2$. The anomaly is the standard one in 10D. The boundaries must therefore contribute massless multiplets. The only such candidate is a vector multiplet. By symmetry, each must contribute the same number of vector multiplets. The prior paragraph then shows that each must contribute 248.  
	 
	 In order to apply Green-Schwarz, we crucially need a two-form $B$, that we are guaranteed in all 10D string theories. The answer here comes from the pullback of the 3-form $A_3$ to the boundaries $H_1, H_2$ giving $B_2, B_2' = A_{\mu \nu 11}\big|_{H_1, H_2}$ respectively. 	The anomaly polynomial in 10D of the form $I_4 \wedge X_8$ leads to $\int B \wedge X_8$, but one can see that $X_8$ for $E_8 \times E_8$ involves no cross terms from either $E_8$, and can in fact be written as 
	 \begin{align}
	 	I_{12}(R, F_1, F_2)&= \hat I_{12}(R, F_1) + \hat I_{12}(R, F_2)\\
		\hat I_{12}(R, F_i) &= \hat I_4(R, F_i) \hat I_8(R, F_i)\\
		\hat I_4(R, F) &= \frac12 \tr R^2 - \frac12 \tr F^2\\
		\hat I_8(R, F) &= - \frac12 \hat I_4 (R, F)^2 + \big(-\frac18 \tr R^4 + \frac{1}{32} (\tr R^2)^2 \big) \label{eq:I8}
	 \end{align}
	 Here $\tr = \frac{1}{30} \Tr_{adj}$ for $E_8$.
	 This gives Chern-Simons terms of the form:
	 \[
	 	\int_{H_1} B_2 \wedge I_8(R, F_1) + \int_{H_2} B_2' \wedge I_8(R, F_2)
	 \]
	 This does not quite go far enough, in that we would not be able to recover $\lambda^6/\kappa_{11}^2$ from this. 
	 
	\textbf{This next part I got from Horava and Witten 9603142}
	 First note that we would like our boundary theory to be \emph{locally} supersymmetric. As it stands, it is not. The standard way in SUGRA (although I did not know this because I don't know enough SUGRA at this point) is to add the gravitino interaction with the supercurrent to the Lagrangian: 
	 \[
		-\frac{1}{4 \lambda^2} \int d^{10} x \sqrt{-g} \bar \psi \mathcal S_{YM} = -\frac{1}{4 \lambda^2} \int d^{10} x \sqrt{-g} \bar \psi_A \Gamma^A \slashed{F}^a \chi^a
	 \]
	 After some work, we see that the only term with uncanceled supersymmetric gauge variation is
	 \[
	 	\frac{1}{16 \lambda^2} \int d^{10} x\sqrt{-g} \bar \psi_A  \Gamma^{ABCDE} F_{BC}^a F_{DE}^a \epsilon
	 \]
	 The only way to cancel this is to modify the 11D Bianchi identity. The reason is that, in checking invariance under local supersymmetry for the 11D Lagrangian, there is an integration by parts that involves the Bianchi identity $dG = 0$. If one instead modifies it to
	 \[
	 	dG_{11 \, A B C D} = - 3 \sqrt2 \frac{\kappa^2}{\lambda^2} \delta(x^11) \tr (F_{[AB} F_{CD]})
	 \]
	 We can then locally write 
	 \begin{align}
	 	G_{11, ABC} &= \d_{11} C_{ABC} + \frac{\kappa^2}{\sqrt2 \lambda^2} \delta(x^{11}) \Omega_{ABC}^{CS} \label{eq:anom1}\\
		 \Omega_{ABC}^{CS} &= \tr(A_A F_{BC} + \frac23 A_A [A_B, A_C] + \perms) \label{eq:anom2} \\
		 &\Rightarrow \dd \Omega^{CS} = 6 \tr (F_{[AB} F_{CD]}) \label{eq:anom3}\\
		 \delta_{\epsilon} \Omega^{CS} &= \dd \tr [\epsilon F] \label{eq:anom4}\\
		&\Rightarrow \delta_{\epsilon} C_{11, AB} = - \frac{\kappa^2}{6\sqrt2 \lambda^2} \delta(x^{11}) \tr (\epsilon F) \label{eq:anom5}
	 \end{align}
	 
	 Note that the anomalous variation in equation~\eqref{eq:anom5} is similar to the $\delta B \sim \tr [\epsilon F]$ for the string theory 2-form. This gives that the 11D Chern-Simons 11D interaction has variation 
	 \[
	 	\frac{1}{3!} \int  C \wedge G \wedge G \to \frac{1}{2!} \frac{\kappa^2}{6\sqrt2 \lambda^2}  \int_{H} \tr[\epsilon F] \wedge G \wedge G
	 \]
	 The value of $G_{ABCD}$ on one of the hyperplanes is quickly seen to be 
	 \[
	 	G_{ABCD}|_H = -\frac{3 \kappa^2}{\sqrt{2}\lambda^2} \tr[ F_{[AB} F_{CD]}]
	 \]
	 Altogether the anomalous variation looks like
	 \[
	 	- \frac{\kappa^4}{128 \lambda^6} \int \tr[\epsilon F] \tr [F^2]^2
	 \]
	 On the other hand, the variation from the 10D chiral fermions in the vector multiplet gives
	 \[
	 	\frac12 \frac{1}{(4 \pi)^5 5!} \int \Tr [\epsilon F^5]
	 \]
	 where the trace is taken in the adjoint. 
	 
	 It is \emph{only} for $E_8$ that we have the nice identity $\Tr X^6 = \frac{(\Tr X^2)^2}{7200}$ and similarly $\Tr[\epsilon F^5] = \frac{\Tr \epsilon \Tr F^5}{7200}$. After identifying $\tr = \Tr/30$, we can write $\Tr X^6 = \frac{15}{4} (\tr X^2)^3$ etc. We get
	 \[
	 	\frac{15}{8 (4 \pi)^5 5!} \int \Tr [\epsilon F^5]
	 \]
	 This will cancel exactly when $\kappa$ and $\lambda$ are related as required. 
	 
	 	 %
	 % Another way to say this as that the Bianchi identity for the 11-dimensional 4-form is violated. Its violation is a five-form supported on both boundaries $x^{11}_i = 0, \pi$.
	 % \[
	 % 	dG \propto \sum_{i=1,2} \delta(x^{11} - x^{11}_i) dx^{11} \hat I_4
	 % \]
	 %
	 % \textbf{Get $X_8$ from $I_8$}
	 Thus, we see that the Green-Schwarz term already present in 11D SUGRA plays a crucial role in canceling the gauge anomaly. We needed the GS terms to be bulk objects, as if they were simply $\delta$-function supported on the boundary this would a) not seem very much like quantum gravity, and b) give gauge variations of boundary interactions proportional to $\delta(0)$.
	 
	 Note that the CS term for the gauge field in the supergravity action was classical (as opposed to the 10D superstrings, where they are 1-loop effects). Consequently, a classical theory with the SUGRA multiplet in the bulk and the gauge multiplet on each boundary is \emph{classically inconsistent.}
	 
	 Given our understanding of 10D anomalies, we then expect (correctly) that the full gauge \emph{and} gravity variation will modify the Bianchi identity as:
	 \[
	 dG_{11 \, A B C D} = - 3 \sqrt2 \frac{\kappa^2}{\lambda^2} \delta(x^11) \Big(\tr (F_{[AB} F_{CD]}) - \frac12 \tr (R_{[AB} R_{CD]}) \Big)	
	 \]
	These $\tr R^2$ terms are not required from classical 11D SUGRA, so they must arise as \emph{quantum effects} of M-theory.
	 
	 The anomaly cancelation term usually takes the form:
	 \[
	 	\int C \wedge I_8
	 \]
	 Note the suggestive way $I_8$ is written in equation \eqref{eq:I8}.
	Because we have seen that $G \propto I_4$ on the boundary, the Chern-Simons term $C \wedge G \wedge G$ on the boundary reduces to $C \wedge I_4^2$. This is a part of $I_8$. We thus expect the remaining part to take the form:
	\[
		\frac{\sqrt{2}}{(4 \pi)^3 (4 \pi \kappa^2)^{1/3}} \int C \wedge \left(-\frac18 R^4 + \frac{1}{32} (\tr R^2)^2 \right)
	\]
	Again, this is a purely quantum effect of M-theory. 
		%
	% Upon dimensional reduction to IIA, this gives the interaction
	% \[
	% 	\int B \wedge X_8
	% \]
	% which is indeed a 1-loop effect in that theory (c.f. question \textbf{35}).
	
	\item Here $M$ is a $20 \times 20$ matrix. It is quick to see that $M L M = L$ for the $20 \times 20$ matrix 
	\[
		L = \begin{pmatrix}
			0 & 1_4 & 0\\
			1_4 & 0 & 0\\
			0 & 0 & 1_{16}
		\end{pmatrix}.
	\]
	This means that $M$ is an element of $O(4,20)$. We can act on it as a bi-fundamental representation (on left and right). \textbf{This is more subtle, because not all $O(4, 20)$ matrices have the form of $M$. Showing that $M$ keeps the same form would take too much time}.  This ensures that the last term is invariant.
	
	The $4+4+16 = 24$ gauge fields from the compactification can be directly seen to transform in the contragradient representation of $O(4,20)$. This ensures that the second-to-last term is invariant. All other terms are invariant.
	
	\item The heterotic action
	\[
	\begin{aligned}
		\int d^6 x \sqrt{-G} [R - \d^\mu \Phi \d_\mu \Phi - &\frac{e^{-2 \Phi}}{2} |H|^2 - \frac{e^{-\Phi}}{4} M^{-1}_{ij} F^i_{\mu \nu} F^{j\, \mu \nu} + \frac18 \Tr[\d_\mu M \d^\mu M^{-1}]]\\
		 H_{\mu \nu \rho} &= \d_\mu B_{\nu \rho} - \frac12 L_{ij} A^i_\mu F^j_{\nu \rho} + 2 \perms
	\end{aligned}
	\]
	and the IIA action:
	\[
	\begin{aligned}
		\int d^6 x \sqrt{-G} [R - \d^\mu \Phi \d_\mu \Phi - \frac{e^{-2 \Phi}}{2} |H|^2 - \frac{e^{\Phi}}{4} M^{-1}_{ij}& F^i_{\mu \nu} F^{j\, \mu \nu} + \frac18 \Tr[\d_\mu M \d^\mu M^{-1}]] + \frac12 \int d^6 x \, L_{ij}\, B \wedge F^i \wedge F^j
		 % \frac{1}{16} \int d^6 x \epsilon^{\mu \nu \rho \sigma \tau \lambda} B_{\mu \nu} F_{\rho \sigma}^i L_{ij} F_{\tau \lambda}^j
		 \\
		 \quad H_{\mu \nu \rho} &= \d_\mu B_{\nu \rho} + 2 \perms
	\end{aligned}
	\]
	I will take the shorthand $\frac12 M^{-1}_{ij} F^i_{\mu \nu} F^{j\, \mu \nu} = |F|^2$
	
	
	The EOMs for $G$ give respectively
	\[
	\hspace{-.3in}
		\begin{aligned}
			&R_{\mu \nu} - \frac12 g_{\mu \nu} R - 2 (\nabla_{\mu} \Phi \nabla_\nu \Phi - g_{\mu \nu} (\nabla \Phi)^2 ) - \frac12 R g_{\mu \nu} - e^{-2 \Phi} (H^2_{\mu \nu} - \frac12 g_{\mu \nu} |H|^2) - e^{-\Phi} (F_{\mu \nu}- \frac12 g_{\mu \nu} |F|^2) + (M \text{ terms})\\
			&R_{\mu \nu} - \frac12 g_{\mu \nu} R - 2 (\nabla_{\mu} \Phi \nabla_\nu \Phi - g_{\mu \nu} (\nabla \Phi)^2 ) - \frac12 R g_{\mu \nu} - e^{-2 \Phi} (H^2_{\mu \nu} - \frac12 g_{\mu \nu} |H|^2) - e^{\Phi} (F_{\mu \nu}- \frac12 g_{\mu \nu} |F|^2) + (\text{same } M \text{ terms})\\
		\end{aligned}
	\]
	here $H^2_{\mu \nu} = \frac{1}{4} H_{\mu \rho \sigma} H^{\nu}_{\rho \sigma}$ etc. All terms are invariant under $\Phi \to - \Phi$, including the terms involving $H^2$, since we will have
	\[
		e^{-2\Phi'} {H'}^2 = e^{2 \Phi} (e^{-2\Phi} \star H)^2 = e^{-2 \Phi} |H|^2
	\]
	and the same for $H_{\mu \nu}$.

	The EOMs for $\Phi$ give respectively:
	\[
	\begin{aligned}
		&\text{Het E:} & \nabla^2 \Phi + e^{-2\Phi} |H|^2  + \frac{e^{-\Phi}}{2} |F|^2 = 0\\
		&\text{IIA:} & \nabla^2 \Phi + e^{-2\Phi} |H|^2  - \frac{e^{\Phi}}{2} |F|^2 = 0
	\end{aligned}
	\]
	
	The EOMs for the $A_\mu$ give respectively:
	\[
	\begin{aligned}
		& \text{Het E:}& e^{-2 \Phi} (\star H) \wedge F - \dd(e^{-\Phi} M_{ij} \star F^j) = 0\\
		& \text{IIA:}&-\dd(e^{\Phi} M_{ij} \star F^j) + H \wedge F  = 0
	\end{aligned}
	\]
	This is equivalent under $\Phi \to - \Phi, e^{-2\Phi} H = \star H'$.
	
	The EOMs \emph{and} Bianchi identity for the $B_{\mu \nu}$ give respectively 
	\[
	\begin{aligned}
		& \text{Het E:}& -\dd (e^{-2 \Phi} \star H) = 0, &\qquad &\dd H - F \wedge F = 0 &\\
		& \text{IIA:}& -\dd (e^{-2 \Phi} \star H) + F \wedge F = 0, &\qquad  &\dd H = 0 &
	\end{aligned}
	\]
	The fact that the $H$ duality exchanges the Bianchi identity and the EOMs speaks to the fact that it is an \emph{electric-magnetic duality of strings}.

	The EOMs for the $M$ terms are $\Phi, A, B$ independent. Consequently, the $M$ matrices can be directly identified between the two theories.
	
	\item Consider just the 3D space of the $x^i$. Note that $V$ is harmonic, and consequently $F := - \star d V$ is a closed 2-form on that space. We can view $F$ as a curvature 2-form on a principal $U(1)$ bundle, and can thus write (upon picking a trivialization of the $U(1)$) a potential $A$ giving $d A = F$. Call the the $U(1)$ bundle $X$. We will write the connection as $\mathcal A = A + d \gamma$.
	
	For each of the three $x^i$ there is a symplectic form on the 4D $U(1)$ bundle given by: 
	\[
		\omega_i := \mathcal A \wedge dx^i + V \star dx^i \Rightarrow d \omega = - \star dV \wedge dx^i + dV \wedge \star dx^i = 0
	\]
	Here we have used that all the $dx^i$ forms are (canonically) pulled back from $\RR^3$ and in $\RR^3$, $\star \alpha \wedge \beta = \alpha \wedge \star \beta$. Now, define a different basis of symplectic forms on $X$ by
	\[
		\Omega_1 = \omega_2 + i \omega_3 = \mathcal A \wedge (dx^2 + i dx^3) + i V dx^1 \wedge (dx^2 + i dx^3) 
	\]
	defining $z^1 = x^2 + i x^3$ this gives:
	\[
		\Omega_i = A \wedge dz^i + i V dx^i \wedge dz^i = V (\underbrace{V^{-1} \mathcal A + i dx^1}_{\alpha_1}) \wedge dz^i.
	\]
	The kernel of this form on the $U(1)$-bundle is 2D. For $\Omega_1$ it is spanned by 
	\[
		\tilde \d_{x^2} + i \tilde \d_{x^3}, \quad V \d_\gamma + i \tilde \d_{x^1}
	\]
	Here each $\d_{x^i}$ is lifted to the $U(1)$ tangent space by using the connection $A$ to identify the appropriate horizontal subspace. We identify this as a \emph{holomorphic tangent space}. Similarly $\bar \Omega_1$ would complete the basis of $T_p M$ and give the anti-holomorphic tangent space. Thus, each $\Omega_i$ gives a distinct stratification into holomorphic and anti-holomorphic tangent spaces. The closedness of $\Omega$ guarantees integrability. Defining 3 separate complex structures $I_j$ to act as $+i$ on the $j$th holomorphic tangent space and as $-i$ on the $j$th anti-holomorphic tangent space, we can easily check that pointwise they reproduce the quaternion algebra. This makes the manifold hyper-K\"ahler, with metric given by:
	\[
	\begin{aligned}
		ds^2 &= V (\Re(\alpha_1)^2 + \Im(\alpha_1)^2) + V (\Re(dz_1)^2 + \Im(dz_1)^2) \\
		&= V^{-1} (\mathcal A)^2 + V |d\vec x|^2 \\
		&=  V^{-1} (A dx + d \gamma)^2 + V |d\vec x|^2 
	\end{aligned}
	\]
	In particular, $V$ can take the form of the multi-center potential in the problem.

	\textbf{Could we not have just exhibited a 3 Killing spinors? Are there such? In any case, this was more instructive}
	
	Lastly, to see the asymptotic limit, we can take the $x_i$ to collide. At a distance, $V$ will look like $\frac{N}{r}$. This corresponds to an $F$ with $N$ units of flux asymptotically. The circle bundle over the $\RR^3$ will asymptotically looks like an $S^1$ fibration over $S^2$. For $N = 1$, this is simply the Hopf fibration. For higher $N$, the connection is $N$ times larger, which makes the $U(1)$ circle $N$ times smaller, and corresponds to a fiberwise quotient of $S^3 \to S^2$ by $\ZZ_N$.
	
	\textbf{Did Kiritsis mean to write $N$?}
	

	\item We have seen that to apply the GS mechanism, the heterotic B-field must have a modification of its Bianchi identity from
	\[
		H = dB - \Omega_3^{YM}(A) + \frac{\kappa^2}{g^2} \Omega_3^{GR}(\omega), \quad \Omega_3(A) = \Tr \Big[A \wedge dA - \frac{2i}3 A \wedge A \wedge A\Big]
	\]
	I have absorbed the factor of $\frac{\kappa^2}{g^2}$ into these fields. This will cancel the anomalous change in B
	\[
		\delta B = \frac{\kappa^2}{g^2} \Tr[\Lambda F_0 - \Theta R_0]
	\]
	required to have a GS term $\int B \wedge X_8$ cancel the full anomaly. The modified Bianchi identity is
	\[
		dH = -\tr[F \wedge F] + \tr[R \wedge R]
	\]
	Here $\tr$ is taken always in the fundamental. The second addition important for the cancelation of \emph{gravitational anomalies}.
		%
	% When we compactify the heterotic string the 6D, $F$ gets additional indices, but in any case the pontryagin class represented by $dH$ must be trivial.
	% In 6D, this anomaly polynomial is simpler. We take $\int B \wedge X_8$ and reduce on K3. Only the $\tr B \wedge (R \wedge R)$ term survives.
	%
	Following the logic in question \textbf{33}, by duality, a modification of the Bianchi identity in heterotic string requires a modification in the $B$ equations of motion on the heterotic side corresponding to the addition to the action of a term:
	\[
		-\int B \wedge R \wedge R
	\]
	The CS terms are thus:
	\[
		% \frac{1}{2 \pi \ell_s^2} \int \hat L_{ij} \; B \wedge \frac{F^i \wedge F^j}{(4\pi)^2} - \int B \wedge \frac{R \wedge R}{(4 \pi)^2} \sim
		\int B \wedge (\tr(F \wedge F) - \tr(R \wedge R))
	\]
	 Further, Vafa and Witten confirm this term from a 1-loop calculation using the elliptic genus on the IIA side in \href{https://arxiv.org/pdf/hep-th/9505053.pdf}{9505053}. This gives a nontrivial 1-loop test of the 6D string-string duality. This problem only asks me to assume the duality in performing the match.
	
	\item In our case, wrapping 3-branes around 2-cycles give rise to two-forms. As one 2-cycle shrinks $B$ to zero size, we get a tensionless string, of tension approximately $|\text{Vol}(B)|/g_s$. For each isolated singularity of K3 (type ADE) there is such a tensionless string theory. Note that this is not yet the (2,0) SCFT, since we have not taken any sort of IR limit that would lead us to expect that the theory is conformal. We still have mass scales. This is an interacting QFT of light strings. 
	
	Upon compactifying on an $S^1$, we can T-dualize to type IIA, where now we have the familiar appearance of massless states associated to a 3-cycle shrinking in K3. The IIA theory sees massless particles emerge at this transition, corresponding to the tensionless strings of IIB wrapping the $S^1$.
	
	\item Here, our cycle is $C = n_i B^i$. Take a \emph{euclidean} D2 brane wrapping this cycle. The total volume (counting orientation) will be $|n_i \int_{B^i} \Omega| = |n_i Z^i|$. 
	
	Because of the BPS property of the 3-cycle, we will still have $M = T_p |Z|$, giving us
	\[
		S_{inst} = \frac{1}{(2\pi)^2 \ell_s^3 g} |n_i Z^i|.
	\]
	It is worth remarking that we get contributions from all winding numbers of D instantons in this case, while in the IIB case, it looks like only the singly-wrapped D3 brane is stable.
	
	\textbf{Is there anything else I should say? Reproduce Vafa+Ooguri's calculation?}
	
	\item In IIB, we have seen that as a three-cycle shrinks, a (BPS) D-brane wrapping this cycle contributes a hypermultiplet that becomes massless as the volume goes to zero. At the conifold point, we get a new massless multiplet. Resolving this singularity by expanding the 2-cycle corresponds to giving an expectation value to the massless hypermultiplet from the D-brane. In general, these hypermultiplets will have a potential. See the discussion on page \textbf{378}.
	
	From this POV, condensation of D-branes has the interpretation of topology change! For IIA the (instantonic) D2 branes instead serve to smooth out the singularity, which corresponds to the hypermultiplet moduli space receiving quantum corrections. 
	
	This does not answer the question, though - which was about the resolution of the two-cycles. However, using the tool of mirror symmetry, we can posit a guess. A two-cycle shrinking in IIA causes a singularity in the vector multiplet, and maps to the familiar three-cycle shrinking in IIB. In IIA, then, we expect a wrapped D2 brane to contribute to a massless hypermultiplet. On the other hand, we expect quantum effects in IIB to smooth out this singularity. 
	
	\textbf{Check against literature.}
	
	\item 
	To simplify this problem, I will reduce the heterotic theory directly from 10D and keep in mind that reduction on tori commutes. In the string frame I get
	\[
		\int d^4 x \sqrt{-g} e^{-2\phi} \left[R + 4 (\d \phi)^2 - \frac12 |H_3|^2 - \frac14 {M^{-1}}_{ij} F_{\mu \nu}^i F^{j\, \mu \nu} + \frac18 \Tr[\d_\mu M^{ij} \d^\mu M_{ij}] \right]
	\]
	Upon taking $g \to e^{2\phi} g$ we get the Einstein frame:
	\[
		\int d^4 x \sqrt{-g} \left[R - 2 (\d \phi)^2 - \frac{e^{-4 \phi}}{2} |H_3|^2 - \frac{e^{-2 \phi}}{4} {M^{-1}}_{ij} F_{\mu \nu}^i F^{j\, \mu \nu} + \frac18 \Tr[\d_\mu M^{ij} \d^\mu M_{ij}] \right]
	\]
	Here the $M_{ij}$ scalar matrix lives in the $\SO(6, 22)$ coset, and there are $28$ fields $F^i$. We can dualize the $H_3$ to a scalar axion through the relation (as in \textbf{9.1.12)}
	\[
		e^{-4\phi} H = \star \d C_0 \Rightarrow e^{-4 \phi} |H|^2 = e^{4 \phi} (\d C_0)^2
	\]
	In performing this dualization, the $B$ equations of motion are automatically satisfied.
	\[
		\nabla^\mu (e^{-4 \phi} H_{\mu \nu \rho}) = \nabla^\mu(\varepsilon_{\mu \nu \rho}^{\ \ \ \ \sigma} \d_\sigma C_0) = 0
	\]
	The Bianchi identity for $H$ must now be imposed by hand 
	\[
		\varepsilon^{\mu \nu \rho \sigma} \d_\mu H_{\nu \rho \sigma} = - L_{ij} F^i_{\mu \nu} \tilde F^{j\ \mu \nu} 
	\]
	This corresponds to adding the term 
	\[
		\frac14 C_0\, L_{ij} F^i_{\mu \nu} \tilde F^{j\ \mu \nu}
	\]
	to the Lagrangian. 
	We then combine the axion with $\phi$ to give:
	\[
		\int d^4 x \sqrt{-g} \left[R - \frac12 \frac{\d \S \d \bar \S}{\S_2^2}  - \frac{1}{4} \S_2\, {M^{-1}}_{ij} F_{\mu \nu}^i F^{j\, \mu \nu} + \frac14 S_1 L_{ij} F_{\mu \nu}^i \tilde F^{j\ \mu \nu} + \frac18 \Tr[\d_\mu M^{ij} \d^\mu M_{ij}] \right]
	\]
	where here $\S = C_0 + i e^{-2\phi}$ (note the $2 \phi$, by contrast with 10D).
	
	% I'll start from the Einstein frame in 6D. Taking $\sigma = \frac12 \det G_{\alpha \beta}$ we get for heterotic:
% 	\[
% 	\begin{aligned}
% 		\int d^4 x \sqrt{-G} e^{\sigma } \Big(R &+ |\d \sigma|^2 + \frac14 \d_\mu G_{\alpha \beta} \d^\mu G^{\alpha \beta} - |\d \Phi|^2\\
% 		&- \frac{e^{-2\Phi}}{2} |\mathcal S_1|^2 - \frac{e^{-2\Phi}}{4} G^{\alpha \beta} H_{2 \alpha} H_{2\beta} - \frac{e^{-2\Phi-2\sigma}}{2} |H_1|^2\\
% 		& - \frac{e^{-\Phi}}{2} |F_2|^2 - e^{-\Phi} G^{\alpha \beta} F_{1 \alpha} F_{1\beta}+ \frac18 \Tr[\d M \d M^{-1}]\\
% 		& + \frac12  \mathcal S_1 L_{IJ} F^I \wedge \tilde F^J \Big)
% 	\end{aligned}
% 	\]
% 	The $M$ matrix will combine with the $G_{\alpha \beta}$ to parameterize a $\SO(6,22)$ coset. Rescaling $G \to e^{-\sigma} G$ we get:
% 	\[
% 		\int d^4 x \sqrt{-G} \left(R  - |\d \Phi|^2 - \frac{e^{-2\Phi +  2 \sigma}}{2} |H_3|^2 - \frac{e^{-\Phi - \sigma}}{2} |F_2|^2 + \frac18 \Tr[\d M \d M^{-1}] \right)
% 	\]
	
	For the IIA side we get:
	\[
	\begin{aligned}
		\int d^4 x \sqrt{-G} e^{\sigma } \Big(&R + |\d \sigma|^2 + \frac14 \d_\mu G_{\alpha \beta} \d^\mu G^{\alpha \beta} - |\d \Phi|^2 - \\
		&\quad  \frac{e^{-2\Phi}}{2} |H_3|^2 - \frac{e^{-2\Phi}}{4} G^{\alpha \beta} H_{\alpha\, \mu \nu} H_\beta^{\ \mu \nu} - \frac12 e^{-2\Phi -2 \sigma} (\d_\mu B_{\alpha \beta})^2\\
		& \quad - \frac{e^{\Phi}}{2} {M^{-1}}_{ij} F_{\mu \nu}^i F^{j\, \mu \nu}
		- \frac{e^{\Phi}}{2} {M^{-1}}_{ij} G^{\alpha \beta} F_{\mu \alpha}^i F^{j\, \mu \beta}
		+ \frac18 \Tr[\d M \d M^{-1}]  \Big)
		% = \int d^4 x \sqrt{-G} \Big(&R - \frac12 |\d \sigma|^2 + \frac14 \d_\mu G_{\alpha \beta} \d^\mu G^{\alpha \beta} - |\d \Phi|^2 - \\
		% & \quad  \frac{e^{-2\Phi+2\sigma}}{2} |H_3|^2 - \frac{e^{\Phi + \sigma}}{2}{M^{-1}}_{ij} F_{\mu \nu}^i F^{j\, \mu \nu} + \frac18 \Tr[\d M \d M^{-1}] + \dots \Big)
	\end{aligned}
	\]
	We take this to the Einstein frame $g \to e^{-\sigma} g$ giving
	\[
	\begin{aligned}
		\int d^4 x \sqrt{-g} \Big(&R - \frac12 |\d \sigma|^2 + \frac14 \d_\mu G_{\alpha \beta} \d^\mu G^{\alpha \beta} - |\d \Phi|^2 - \\
		&\quad  \frac{e^{-2\Phi + 2 \sigma}}{2} |H_3|^2 - \frac{e^{-2\Phi + \sigma}}{4} G^{\alpha \beta} H_{\alpha\, \mu \nu} H_\beta^{\ \mu \nu} - \frac12 e^{-2\Phi -2 \sigma} (\d_\mu B_{\alpha \beta})^2\\
		& \quad - \frac{e^{\Phi + \sigma}}{2} {M^{-1}}_{ij} F_{\mu \nu}^i F^{j\, \mu \nu}
		- \frac{e^{\Phi}}{2} {M^{-1}}_{ij} G^{\alpha \beta} F_{\mu \alpha}^i F^{j\, \mu \beta}
		+ \frac18 \Tr[\d M \d M^{-1}]  \Big)
	\end{aligned}
	\]
	
	In this case, the dilaton and axion fields enter the mass matrix $M$, as does the torus complex modulus, whose kinetic term is the third term in the action above. The complexified K\"ahler modulus parameters of the torus remain \textbf{Why? This part needs elaboration}. Altogether, including the CS term, this reduces to
	\[
		\int d^4 x \sqrt{-g} \Big(R - \frac12 |\d \sigma|^2 - \frac12 e^{-2\sigma} (\d_\mu B)^2  - \frac{e^\sigma}{4} \, {M^{-1}}_{ij} F_{\mu \nu}^i F^{j\, \mu \nu}  + \frac18 \Tr[\d_\mu M^{ij} \d^\mu M_{ij}] + \frac14 B L_{ij} F_{\mu \nu}^i \tilde F^{j\ \mu \nu}  \Big)
	\] 
	Defining the parameter $T = B + i e^\sigma$ we get
	\[
		\int d^4 x \sqrt{-G} \Big(R - \frac12 \frac{\d T \d \bar T}{T_2^2} - \frac{1}{4} T_2\, {M^{-1}}_{ij} F_{\mu \nu}^i F^{j\, \mu \nu}  + \frac14 T_1\, L_{ij} F_{\mu \nu}^i \tilde F^{j\ \mu \nu}  + \frac18 \Tr[\d_\mu M^{ij} \d^\mu M_{ij}] \Big)
	\]
	
	Comparing the above action with the heterotic one we identify $\S$ with $T$, giving (unprimed indicates heterotic, primed indicates IIA)
	\[
	\begin{aligned}
		C_0 = B_{\alpha \beta}' \text{ (external SL$_2$)}, &\quad C_{0}' = B_{\alpha \beta}' \text{ (internal)}\\
		e^{-2\phi} = e^{\sigma'}  \text{ (external SL$_2$)}, &\quad e^{-2 \phi'} = e^{\sigma} \text{ (internal)}
	\end{aligned}
	\]
	This makes us also identify $g = g'$ and $M = M'$ as well as all the gauge fields descending from 6D $A_\mu^i = (A_\mu^i)'$ and the two descending from the $T^2$ metric $A_\mu^\alpha = (A_\mu^\alpha)'$.
	
	Electric-magnetic duality acts least trivially on $H_{\mu \nu \rho}$ in 6D. Consequently, it will act nontrivially on its axion and scalar descendants $B, C_0$ (as we have seen). The 2-form field strength $F^B$ remains. From 6D we have $e^{-2\phi} H = \star H'$. This descends to
	\[
		e^{-2\phi} G^{\alpha \beta} (F_{\beta, \mu \nu}^B  - L_{ij} Y^i_{\beta} F^j_{\mu \nu} - C_{\beta \gamma} F^{A, \gamma}_{\mu \nu} ) - \frac12 a \star F^A = \frac12 \frac{\epsilon_{\mu \nu}^{\ \ \rho \sigma} \varepsilon^{\alpha \beta}}{\sqrt{g}} (F^B_{\beta, \rho \sigma})'
	\]
	where we see on the heterotic side we have additional terms due to how $\hat H_{\mu \nu \rho}$ is defined in that case \textbf{account for additional axion term there}. 
	
	\textbf{Relatively unfinished.}
	
	%  This can be seen by the fact that they act as $M_{ij}^{-1}$ factors on $F_2$. Parameterizing the torus by complex modulus $\tau = \tau_1 + i \tau_2$ we get the action:
	% \[
	% 	\int d^4 x \sqrt{-G} \left(R - \frac12 \frac{\d \tau \d \bar \tau}{\tau_2^2} - \frac{1}{2} |F_2|^2 + \frac18 \Tr[\d M \d M^{-1}] + \dots \right)
	% \]
	%
	% In the Einstein frame, we expect duality to preserve the metric. The simplest way for the $M$ matrix terms to match is if we identify the matrices directly $M = M'$. This implies that we identify the internal Wilson lines $Y_{\alpha}^i$ and $G_{\alpha \beta}, B_{\alpha \beta}$. as well as gauge fields $A_\mu^i, A_\mu^\alpha$.
	%
	% Duality now requires
	% \[
	% 	-\Phi + \sigma/2 = \Phi' + \sigma'/2 \Rightarrow e^{-2 \Phi} \sqrt{G_{\alpha \beta}} = e^{2 \Phi'} \sqrt{G_{\alpha \beta}'}
	% \]
	
	\item I will just demonstrate this on the Bosonic sector. $(-1)^{\mathbf{F}_L}$ sends the RR fields to minus themselves (ie $C_0, C_2, C_4$), while $S$ swaps $B_2, C_2$ and flips the axion part of the axio-dilaton $\tau \to -\bar \tau$. Conjugating $(-1)^{\mathbf{F}_L}$ by $S$ will flip the sign of $C_4$ and $B_2$ and also take $\tau \to -\bar \tau$. The untwisted sector will thus be without $B_2, C_0, C_4$ leaving only $G, \phi, C_2$. This is the closed-string sector of type I. 
	
	On the other hand, we have shown that orbifolding IIB by just $(-1)^{\mathbf{F}_L}$ yields just IIA. At the level of bosonic fields, we already see that these operations do not commute. 
	
	\item It is worth appreciating that this duality was known before the Horava-Witten construction.
	
	\begin{center}
		\includegraphics[scale=0.13]{"Drawings/M Type I"}
	\end{center}
	
	First note that the moduli space of the heterotic string on $T^3$ is given by the coset space 
	\[
		\RR^+ \times \SO(19,3;\ZZ)\backslash \SO(19,3)/(\SO(19) \times \SO(3))
	\]
	with $\RR^+$ parameterizing the dilaton. Now, in string compactifications K3 has a moduli space coming from cosets of $\SO(4, 20)/\SO(4) \times \SO(20)$. This includes the \emph{complexified} K\"ahler modulus, which takes into account the NSNS $B$-field. M theory lacks this parameter, and consequently the K\"ahler component of moduli space involves only \emph{real} moduli (ie metrics). This gives $\SO(3, 19)/\SO(3) \times \SO(19)$. The volume gives another factor of $\RR$.
	
	The low-energy effective actions also match. Wrapping the M-theory $A_3$ on K3 gives one 3-form $C_3$ and 22 1-forms $A_1^i$. We get an action:
	\[
		\frac{1}{2 \kappa_{11}^2}\int d^{11} \sqrt{G_{11}} (R + \frac12 |dA_3|^2)  \to \int   d^{7} \sqrt{G_{7}} [e^{4 \sigma} (R + (\d \sigma)^2 - \sum_i \frac12 |dA_1^i|^2  + \text{moduli} ) - \frac12 |dC_3|^2]
	\]
	
	Upon rescaling $g \to e^{-4\sigma} g$ and taking $\phi = 3 \sigma$ we arrive at:
	\[
		\int   d^{7} \sqrt{G_{7}} [e^{-2 \phi} (R + (\d \sigma)^2 - \sum_i \frac12 |dA_1^i|^2  + \text{moduli} - \frac12 |dC_3|^2) ]
	\]
	This exactly matches with the heterotic theory. We go to strong heterotic coupling by taking $\sigma \to \infty$, ie taking the volume of the K3 to be large. 
	
	
	\item
	 Because $A_3$ is odd under the $\ZZ_2$ transformation, we must wrap it on either a 1-cycle or a 3-cycle to have things survive. There are 5 1-cycles giving 5 vectors and ${5 \choose 2} = 10$ 2-cycles giving 10 $0$-forms in 6D. Further, the internal metric has $5 \times 6/2 = 15$ even terms that survive. Altogether we get 5 2-forms, 25 scalars, and 10 vectors. 
	
	This is $N=(2,0)$ (chiral) supergravity consisting of the supergravity multiplet and five tensor multiplets, each of which contains an anti-self-dual two-form field (the 5 self-dual parts are part of the SUGRA multiplet). Cancelation of anomalies \textbf{(prove and understand compared to the $N = (1,0)$ case)} require $N_T = 21$. We are missing \emph{sixteen} tensor multiplets.  
	
	
	The orbifold has $2^5 = 32$ fixed points which we expect will lead to twisted sectors. We shouldn't be too sure of how things go, though, because we don't know how to deal with twisted sectors of M-theory. It initially looks paradoxical that we need 16 extra tensor multiplets but have 32 fixed points. This is resolved in \href{https://arxiv.org/pdf/hep-th/9512219.pdf}{Witten 9512219}. The solution is to recognize the fixed points as 32 magnetic sources of charge $-1/2$ for the $G_4$ field. The constraint that the total charges should vanish is satisfied when 16 of these sources have a five-brane on top of them of $+1$ magnetic charge. Each fivebrane can be seen to support a single tensor multiplet, giving our desired 16. We interpret the five scalar in each multiplet as describing the position of the fivebrane inside $T^5/\ZZ_2$. 
	
	This now gives the massless spectrum of type IIB on K3. 
	
	The question is how to arrange the fivebranes in such a way that we can see the duality to IIB on K3. The equivalence implies that when \emph{any} circle shrinks \textbf{show more rigorously}, we would expect to recover weakly coupled IIB on K3. The naive guess I have is to arrange them in an alternating ``checkerboard'' pattern. \textbf{(Witten confirms this.)} Now, in the limit where any circle shrinks to zero size, the opposite charge sources cancel, giving zero 4-form field strength in the 6D spacetime, consistent with the fact that the 3-form has been projected out on the M-theory side and doesn't exist on the IIB side. 
	
	As a second check, we can further compactify on $S^1$. We get IIA on $T^5/\ZZ_2$ vs IIB on $S^1 \times K^3$. T-dualizing the latter along $S^1$ (the only 1-cycle!) gives IIA on $S^1 \times K^3$ which is equivalent to heterotic on $S^5$. S-dualizing this gives type I on $S^5$ which is T-dual to IIA on $T^5/\ZZ_2$  as an \emph{orientifold} \textbf{(why do we need to act with $\Omega$ too?)}
	
	% First, we compare the moduli spaces. For IIB on K3 it is
	% \[
	% 	\SO(5,21,\ZZ)\backslash\SO(5, 21)/(\SO(5) \times \SO(21))
	% \]
	% This will match with M-theory on $T^5/\ZZ_2$ \textbf{prove}.
	
	\item I will work with covariant derivatives and take the axiodilaton fields in terms of the $\mathcal S, \bar{\mathcal S}$ basis. I will write the equations of motion for the axiodilaton as:
	\[
		\bar \nabla \left(\frac{\d \S}{(\S- \bar \S)^2} \right) - 2 \frac{\d \S \bar \d \bar {\S}}{(S - \bar{\S})^3} = \frac{\d \bar \d \mathcal S}{(S-\bar S)^2} - 2 \frac{\d \S \bar \d \bar {\S}}{(S - \bar{\S})^3} + 2 \frac{\d \S \bar \d \bar {\S} - \d \S \bar \d \S}{(S - \bar{\S})^3} \Rightarrow \d \bar \d \S + 2 \frac{\d \S \bar \d \S}{\bar \S - \S} = 0
	\]
	Where it is important to note that we can write $\d \bar \d \S$ for the laplacian in complex 2D coordinates instead of $\nabla \nabla \S$. We thus get our desired EOM. 
	
	\item Recall that as a holomorphic function of $z$, $\mathcal S$ should have positive imaginary part, and have its image restricted to the fundamental domain $\mathcal F$. This mapping should be finite energy density. From the effective action we compute the energy density by pulling back as:
	\[
		\mathcal E = -\frac{i}{\kappa_{10}^2} \int d^2 z \frac{\d \mathcal S \bar \d \bar {\mathcal S}}{(\mathcal S - \bar {\mathcal S})^2} = \frac{i}{\kappa_{10}^2} \int_{\mathcal F} d^2 {\mathcal S} \, \d \bar \d \log(\mathcal S - \bar {\mathcal S})
	\]
	At this point we apply Stokes' theorem to get a boundary integral:
	\[
		\frac{i}{\kappa_{10}^2} \int_{\d \mathcal F} d \mathcal S \d \log(\mathcal S - \bar{\mathcal S}) = \frac{i}{\kappa_{10}^2} \int_{\d \mathcal F} \frac{d \mathcal S}{S - \bar{\mathcal S}}
	\]
	The vertical lines of the fundamental domain have the same values but are traversed in opposite orientation \textbf{picture}. Therefore, only the semicircle counts. This integral is readily evaulated:
	\[
		\int_{2 \pi /3}^{\pi/3} \frac{d \theta\, i e^{i \theta}}{e^{i \theta} - e^{-i \theta}} = -\frac{i \pi}{6}
	\]
	This gives our desired final answer of $\frac{\pi}{6 \kappa_{10}^2}$: the angular defect due to a D7 brane.
	
\end{enumerate}

% section chapter_11_duality_connections_and_nonperturbative_effects (end)
\end{document}
	