\documentclass[11pt, class=article, crop=false]{standalone}
\usepackage{amsmath,amssymb,amsfonts,amsthm}
\usepackage{enumitem}
\usepackage{fancyhdr}
\usepackage{tikz-cd}
\usepackage{mathabx}
\usepackage{geometry}
\usepackage{natbib}
\usepackage{braket}
\usepackage{graphicx}
\usepackage{simpler-wick}
\usepackage{hyperref}
\usepackage{ytableau}
\usepackage{cancel}
\usepackage{listings}
\usepackage{relsize}
\usepackage{xcolor}
\usepackage{stmaryrd}
\usepackage{slashed}
\usepackage{tikz-feynman}
\usepackage{kiritsis}
\geometry{margin = 0.5in}


\begin{document}
\section*{Chapter 9: Compactification and Supersymmetry Breaking} % (fold)
\label{sec:chapter_9_compactification_and_supersymmetry_breaking}
\begin{center}
	\vspace{-.05in}
	\textbf{In collaboration with Alek Bedroya}
	\vspace{-.05in}
\end{center}
\begin{enumerate}
	\item We compactify the heterotic string along just one dimension, making it a compact circle of radius $R$ with all $16$ Wilson lines turned on. 
	
	Each noncompact boson contributes
		\[
			\frac{1}{\sqrt \tau_2 \eta \bar \eta }
		\]
	The fermions on the supersymmetric side contribute
		\[
			\sum_{a,b=0}^1 (-1)^{a+b+ab} \frac{\theta\twist{a}{b}^4}{\eta^4}
		\]
	The $(p,p)$ compact bosons and $16$ complex right-moving fermions that can be written as the pair $\psi^I(\bar z), \bar \psi^I(\bar z)$ have the action as in \textbf{E.1} (setting $\ell_s = 1$)
		\[		\hspace{-.3in}
			\frac{1}{4\pi} \int d^2 \sigma \sqrt{\det g} g^{ab} G_{\alpha \beta} \d_a X^\alpha \d_b X^\beta + \frac{1}{4\pi} \int d^2 \sigma \epsilon^{ab} B_{\alpha \beta}\d_a X^\alpha \d_b X^\beta + \frac{1}{4\pi} \int d^2 \sigma \sqrt{- \det g} \sum_I \psi^I [\bar \nabla + Y_\alpha^I \bar \d X^\alpha] \bar \psi^I
		\]
		Here $\alpha, \beta$ are the toral coordinates for the compact spacetime and $Y^I_\alpha$ is the Wilson line along torus cycle $\alpha$. To evaluate the path integral, as we did in the purely bosonic case, we have a factor of
		\[
			\frac{\sqrt{\det G}}{\tau_2^{p/2} (\eta \bar \eta)^{p}}
		\] 
		coming from evaluating the determinant $(\det \nabla^{2})^{-1/2}$ of the bosons. This multiplies a sum over instanton contributions labelled by $m^\alpha, n^\alpha$ taking values in a $(p,p)$-signature lattice with classical action 
		\[
			\sum_{m^\alpha, n^\alpha} e^{-\frac{\pi}{\tau} (G+B)_{\alpha \beta} (m + \tau n)^\alpha (m + \bar \tau n)^\beta} \times \text{fermions}.
		\]
		The fermion contribution depends via the Wilson lines on the configuration of the $X^\alpha$. In each such instanton sector, the fermion path integral with a constant background Wilson line is equivalent to a free fermion with twisted boundary conditions. For simplicity, let's compactify just on $S^1$, and denote $\theta^I = Y^I n, \phi^I = -Y^I m$. We get boundary conditions: 
		\[
		\begin{aligned}
			\psi^I (\sigma + 1, \sigma_2) &= -(-1)^{a} e^{2 \pi i \theta^I} \\
			\psi^I (\sigma, \sigma_2 + 1) &= -(-1)^{b} e^{-2\pi i \phi^I}
		\end{aligned}
		\]
		where $a,b = 0,1$ denotes anti-periodic/periodic boundary conditions respectively. We know that (in the absence of Wilson lines) the determinant of $\d$ acting on complex fermions is:
		\[
			\text{det}_{a,b}\, \d = \frac{\theta \twist{a}{b}}{\eta}
		\]
		Let us now investigate the twisted boundary conditions. For simplicity its enough to take $a= b = 0$ (all antiperiodic). We have two different ways to write the partition function. As a product over modes, we have $\psi_m, \bar \psi_m$ modes, with respective weights $m-\frac12 -\theta, m-\frac12+\theta$ \textbf{Check against Polch 16.1.16} and respective fermion numbers $\pm1$ \emph{relative to the ground state}. The fermion number of the ground state has no canonical value (as far as I can see). On the other hand, the ground state energy is given by the standard mneumonic to be $-\frac{1}{24} + \frac12 \theta^2$. This gives:
		\[
			\Tr_{\theta} [e^{2 i \pi \phi F} q^{H}] = q^{\frac{\theta^2}{2} - \frac{1}{24}} \prod_{m=1}^\infty (1 + q^{m -1/2 + \theta} e^{2\pi i \phi}) (1 + q^{m -1/2 - \theta} e^{-2\pi i \phi}) = q^{\theta^2/2}\frac{\theta\twist00 (\phi + \theta \tau | \tau)}{\eta}
		\]
		For other boundary conditions, we can apply the same logic to get 
		\[
			q^{\theta^2/2} \frac{\theta \twist a b (\phi + \theta \tau | \tau)}{\eta}
		\]
		The overall phase is still a mystery. Writing $\theta \twist a b \twist \theta \phi$ as a new theta function, we can fix the phase by requiring modular invariance 
		\begin{equation}\label{eq:thetaproperties}
			\footnotesize
		\begin{aligned}
			\theta \twist 0 0 \twist \theta \phi (\tau+1) &= \theta \twist 0 0 \twist{\theta}{\phi+\theta} (\tau)
			&\quad&
			 \theta \twist 0 1 \twist \theta \phi (\tau+1) = \theta \twist 0 0 \twist{\theta}{\phi+\theta}(\tau)\\
			 \theta \twist10 \twist \theta \phi (\tau+1) &= e^{i \pi /4} \theta \twist11 \twist{\theta}{\phi+\theta}(\tau)
			 &\quad&
			 \theta \twist11 \twist \theta \phi (\tau+1) = e^{i\pi/4} \theta \twist10 \twist{\theta}{\phi+\theta}(\tau)
		\end{aligned}
		\end{equation}
		Even from the first of these conditions, we see that we need a term going as $e^{i \theta \phi}$ out front. After adding this in, all other transformations will hold automatically. The $\tau \to -1/\tau$ transformation will thus hold automatically. \textbf{Interpret this as an anomaly? Yes, Narain, Witten do this in Section 3 of their paper. It seems careful anomaly analysis is not enough and one must indeed impose modular invariance by hand.} 
		
		Altogether then the 16 complex antiholomorphic fermions contribute in each instanton sector: 
		\[
			e^{-i \pi \sum_I \theta^I (\phi^I + \bar \tau \theta^I)} \frac12 \sum_{a,b=0}^1 \prod_{i=1}^{16} \frac{\bar \theta \twist ab (\phi + \bar \tau \theta  | \bar \tau)}{\bar \eta}
		\]
		Giving a total partition function as in the second (unnumbered) equation of \textbf{Appendix E}:
		\[
			\left[\frac{R}{\sqrt{\tau_2} \eta \bar \eta^{17}} \sum_{m,n} e^{-\frac{\pi R^2}{\tau_2} |m + n \tau|^2} \underline{e^{-i \pi \sum_I n Y^I (m + n \bar \tau) Y^I Y^I}\frac12 \sum_{a,b=0}^1  \prod_{i=1}^{16} \bar \theta \twist ab (Y^I (m + \bar \tau n) | \bar \tau)} \right] \times \frac{1}{\tau_2^{7/2} \eta^7 \bar \eta^7} \frac12 \sum_{a,b = 0}^1 \frac{\theta^4 \twist a b}{\eta^4}
		\]
		From the properties of the theta functions in Equation~\eqref{eq:thetaproperties}, the underlined fermionic sum has the exact same transformation properties as a sum of $\theta^{16}$ terms and thus makes the full partition function modular invariant.
		
		Each theta function can be written in sum form as: 
		\[
			\theta \twist ab \twist \theta \phi = e^{\pi i \theta \phi} q^{\theta^2/2} \sum_{n\in \ZZ} q^{\frac12 (n- \frac a2)^2} e^{2\pi i (n- \frac a2) (\phi + \tau \theta - \frac b2)} = \sum_{n \in \ZZ} q^{\frac12 (n + \theta - \frac a2)^2} e^{2\pi i  \phi (n + \frac12 \theta - \frac a2) - \pi i b (n- \frac a2)}
		\]
		Then we get the following expression for the underlined fermionic term: 
		\[
		\begin{aligned}
			&\frac12 \sum_{a,b=0}^1 \prod_{I = 1}^{16} \sum_{k\in \ZZ} \bar q^{\frac12 (k + n Y^I - \frac a2)^2} e^{-2\pi i m Y^I  (k + \frac12 n Y^I - \frac a2) + \pi i b (k- \frac a2)} \\
			&= \frac12 \sum_{a,b=0}^1
			 \sum_{q^I \in \ZZ^{16}} \bar q^{\frac12 (q^I + n Y^I - \frac a2)^2} e^{-2\pi i m Y^I  (q^I + n Y^I - \frac a2) + \pi i b (k- \frac a2)} \\
			&= \frac12 \sum_{q^I \in \ZZ^{16}} \left[ \bar q^{\frac12 (q^I + n Y^I)^2} e^{-2\pi i m Y^I  (q^I + \frac12 n Y^I) } (1 + (-1)^{\sum_I q^I})
			+ \bar q^{\frac12 (q^I + n Y^I- \frac12)^2} e^{-2\pi i m Y^I  (q^I + \frac12 n Y^I - \frac12) } (1 + (-1)^{\sum_I (q^I - \frac12)}) \right]\\
			&= \sum_{q^I \in \Lambda^{16}} q^{(q^I + n Y^I)^2} e^{-2 \pi i m Y^I (q^I + \frac12 n Y^I)}
		\end{aligned}
		\]
		We note that the second-to last line is indeed the sum over the roots of $O(32)$ augmented with one of the spinor weight lattices. Altogether the compact dimensions contribute:
		\[
			\frac{R}{\sqrt{\tau_2} \eta \bar \eta^{17}} \sum_{m\in \ZZ,n \in \ZZ ,q^I \in \Lambda^{16}} \exp\left[\frac{\pi R^2}{\tau_2} (m + n \tau) (m + n \bar \tau) + \pi i \tau (q^I + n Y^I)^2 - 2 \pi i m Y^I (k + \frac12 Y^I) \right]
		\]
		To put this whole thing into Hamiltonian form, we proceed as in the bosonic case and perform a Poisson summation over $m$. The terms that contribute are:
		\[
		\hspace{-.3in}
		\begin{aligned}
			e^{-\frac{\pi R^2}{\tau_2} n^2 \tau_1^2 - n^2 \pi R^2 \tau_2}&\sum_{m} e^{-\frac{\pi R^2}{\tau_2} m^2 - 2\pi i m Y^I (q^I + \frac12 n Y^I)- i\frac{n R^2 \tau_1}{\tau_2}} \\
			&= e^{-\frac{\pi R^2}{\tau_2} n^2 \tau_1^2 - n^2 \pi R^2 \tau_2}\frac{\sqrt{\tau_2}}{R} \sum_{m} e^{-\frac{\pi \tau_2}{R^2} (m + Y^I (q^I + \frac12 n Y^I) - i n \frac{R^2 \tau_1}{\tau_2} )^2}\\
			&= e^{-\frac{\pi R^2}{\tau_2} n^2 \tau_1^2 - n^2 \pi R^2 \tau_2} \frac{\sqrt{\tau_2}}{R} \sum_{m} e^{-\frac{\pi \tau_2}{R^2} (m + Y^I (q^I + \frac12 n Y^I) )^2 
			+ \pi R^2 \frac{\tau_1^2}{\tau_2} n^2
			+ 2 \pi i (m + q^I + \frac12 n Y^I) n  \tau_1}\\
			&= e^{- n^2 \pi R^2 \tau_2} \frac{\sqrt{\tau_2}}{R} \sum_{m}
			 e^{-\frac{\pi \tau_2}{R^2} (m + Y^I (q^I + \frac12 n Y^I) )^2 
			+ 2 \pi i (m + q^I + \frac12 n Y^I) n  \tau_1}\\
		\end{aligned}
		\]
		Together with the other terms this gives us
		\[
		\begin{aligned}
			&\frac{1}{\eta \bar \eta^{17}} \sum_{n, m, q^I} q^{\frac12 (q^I + n Y^I)^2}  e^{- n^2 \pi R^2 \tau_2}  e^{-\frac{\pi \tau_2}{R^2} (m + Y^I (q^I + \frac12 n Y^I) )^2 
			+ 2 \pi i (m + q^I + \frac12 n Y^I) n  \tau_1 } \\
			&= \frac{1}{\eta \bar \eta^{17}} \sum_{n, m, q^I} q^{\frac12 (q^I + n Y^I)^2} q^{\frac12 (\frac{1}{R} (m - Y^I (q^I + \frac12 n Y^I) + n R)^2} \bar q^{\frac12 ( \frac{1}{R} (m - Y^I (q^I + \frac12 n Y^I) - n R)^2}
		\end{aligned}
		\]
		where I've flipped $m \to -m$ at the end there. We get momenta 
		\[
		\begin{aligned}
			k_L &= \frac{1}{R}(m - q^I Y^I - \frac12 n Y^I Y^I) + n R = \frac{m}{R} + n (R - \frac12 Y^I Y^I) - q^I Y^I \\
			k_R &= \frac{1}{R}(m - q^I Y^I - \frac12 n Y^I Y^I ) - n R = \frac{m}{R} - n (R + \frac12 Y^I Y^I) - q^I Y^I\\
			k_R^I &= q^I + n Y^I 
		\end{aligned}
		\]
		consistent with Polchinski with $m \leftarrow n_m, n \leftarrow w^n, Y^I \leftarrow R A^I$ and $\alpha' = 0$ \textbf{(might be off by a factor of $2$ for $k_R^I$ rel. to Polchinski but I think I'm consistent with Ginsparg)}. We only care about the $SO(1,1, \ZZ)$ T-duality group coming from the compact $x^9$. This does not act on the $Y^I$ as far as I can see \textbf{CHECK}
		
		The $\mathrm{SO}(16, \ZZ)$ on the other hand acts on the $Y^I$ as in the standard vector representation. 
	
	
	\item I am going to re-do the computations of appendix F Hatted indices denote the 10D terms. Greek indices from the start of the alphabet denote compact 10-$D$-dimensional indices while greek indices from the middle of the alphabet denote noncompact $D$-dimensional indices.
	
	The 10D action is
	\[
		\int d^{10} x \sqrt{-\hat G_{10}} \,  e^{-2\hat \Phi} [\hat R + 4 (\nabla \hat \Phi)^2 - \frac{1}{12} \hat H^2 - \frac14 \Tr \hat F^2] + O(\ell_s^2)
	\]
	with $\hat F^I_{\mu \nu} = \d_\mu \hat A^I_\nu - \d_\nu \hat A^I_\mu$ and $\hat H_{\mu \nu \rho} = \d_\mu \hat B_{\nu \rho} - \frac12 \sum_I \hat A_{\mu}^I \hat  F^I_{\nu \rho} + 2 \perms $. Here $I$ is the internal $16$-dimensional index for the heterotic string. 
	
	We take the 10-bein ($r, a$ denote $D$ and $10-D$ 10-bein indices, hatted indices $\hat r, \hat \mu$ should not be confused for 10-bein indices!!)
	\[
		e_{\hat \mu}^{\hat r} = \begin{pmatrix}
			e^r_\mu & A^\beta_\mu E^a_\beta\\
			0 & E^a_\alpha
		\end{pmatrix}\qquad
		e_{\hat r}^{\hat \mu} = \begin{pmatrix}
			e_r^\mu & -e_r^\nu A^\alpha_\nu\\
			0 & E_a^\alpha
		\end{pmatrix}
	\]
	This gives us the metric:
	\[
		G_{\hat \mu, \hat \nu} = \begin{pmatrix}
			G_{\mu \nu} - A_\mu^\alpha G_{\alpha \beta} A_\nu^\beta &  G_{\alpha \beta} A_\mu^\beta\\
			G_{\alpha \beta} A_\nu^\beta & G_{\alpha \beta}
		\end{pmatrix}
	\]
	As we've done before in chapter 7, we then define 
	\[
		\phi = \Phi - \frac14 \log \det G_{\alpha \beta}, \qquad F^A_{\mu \nu} = \d_\mu A_\nu - \d_\nu A_\mu
	\]
	With this, the compactification of $R + 4 (\nabla \phi)^2$ is clear:
	\[
		\int d^D \sqrt{g} e^{-2\phi} [R + 4 \d_\mu \phi \d^\mu \phi + \frac14 \d_\mu G_{\alpha \beta} \d^\mu G^{\alpha \beta} - \frac14 G_{\alpha \beta} {F^A_{\mu \nu}}^\alpha {F^A_{\mu \nu}}^\beta]
	\]
	The first and second terms are clear. The third term makes up for the redefinition of $\Phi$ in terms of $\phi$ while the last term is the standard KK mechanism generating a gauge field strength from the compact dimensions. 
	
	Next, let's look $\hat H$. Because we have no sources for the $H$ field, $\hat H$ is on the compact cycles. We can define the $D$-dimensional fields using the 10-bein as:
	\begin{align}
		H_{\mu \alpha \beta} &= e^r_\mu e_r^{\hat \mu} \hat H_{\hat \mu \alpha \beta} = \hat H_{\mu \alpha \beta} \label{eq:H1}\\
		H_{\mu \nu \alpha} &= e^r_\mu e^s_\nu e_r^{\hat \mu} e_s^{\hat \nu} H_{\hat \mu \hat \nu \alpha} = \hat H_{\mu \nu \alpha} - A_\mu^\beta \hat H_{\nu \alpha \beta} + A_\nu^\beta \hat H_{\mu \alpha \beta} \label{eq:H2}\\
		H_{\mu \nu \rho} &= e^r_\mu e^s_\nu  e^t_\rho e_r^{\hat \mu} e_s^{\hat \nu} e_t^{\hat \rho} \hat H_{\hat \mu \hat \nu \hat \rho} = 
		\hat H_{\hat \mu \hat \nu \hat \rho} + [- A_\mu^\alpha \hat H_{\alpha \nu \rho} + A_{\mu}^\alpha A_\nu^\beta \hat H_{\alpha \beta \rho} + 2 \perms] \label{eq:H3}
	\end{align}
	The point of defining these coordinates in terms of the 10-bein coordinate is that now, we can just directly separate the $\hat H_{\hat \mu \hat \nu \hat \rho} \hat H^{\hat \mu \hat \nu \hat \rho}$ sum into terms without worrying about the metric, and yield directly:
	\[
		\int d^D \sqrt{-g} e^{- 2 \phi} [-\frac{1}{12} H_{\mu \nu \rho} H^{\mu \nu \rho} - \frac{3}{12} H_{\mu \nu \alpha} H^{\mu \nu \alpha} - \frac{3}{12} H_{\mu \alpha \beta} H^{\mu \alpha \beta} ]
	\]
	
	The method is the same for the $F$ tensor. We define new Wilson lines and field strengths:
	\[
	\begin{aligned}
		Y^I_\alpha = A^I_\alpha, \qquad A^I_\mu = e_\mu^r e^{\hat \mu}_r \hat A^I_{\hat \mu} = \hat A_\mu^I - Y^I_\alpha A^\alpha_\mu
	\end{aligned}
	\]
	I can define $F$ in the standard $F_{\mu \nu}^I = \d_\mu A_\nu^I - \d_\nu A_\mu^I$, $\tilde F_{\mu \alpha}^I = \d_\mu Y_\alpha^I$. This gives me
	$\hat F_{\mu \nu}^I = F_{\mu \nu}^I + \d_\mu (Y_\alpha^I A_\nu^\alpha) - \d_\nu (Y_\alpha^I A_\nu^\alpha)$. By redefining 
	\[
		\tilde F^I_{\mu \nu} = F^I_{\mu \nu} + Y^I_\alpha F^{A, \alpha}_{\mu \nu}
	\]
	we can equate this with $\hat F_{\mu \nu}^I$. For the compact coordinates its more simple and I take $\tilde F_{\mu \alpha} = \d_\mu Y^I_\alpha$. Again $\tilde F_{\alpha \beta}$ vanishes since we cannot have internal sources. This yields directly
	\[
		\int d^D x \sqrt{-g} e^{-2\phi}[-\frac14 \sum_I^{16} \tilde F^{I}_{\mu \nu} \tilde F^{I, \mu \nu} - \frac{2}{4} \tilde F^{I}_{\mu \alpha} \tilde F^{I, \mu \alpha}]
	\]
	
	Its not good enough for us to write everything in terms of an abstract $H$ 3-form. We want to relate $H$ to $B$ and $Y$. From our relationship in $10$D we can directly write:
	\[
	H_{\mu \alpha \beta} = \d_\mu B_{\alpha \beta} + \frac12 \sum_{I} (Y^I_\alpha \d_\mu Y^I_\beta - Y^I_\beta \d_\mu Y^I_\alpha)
	\]
	Taking $C_{\alpha \beta} = \hat B_{\alpha \beta} - \frac12 \sum_I Y^I_\alpha Y^I_\beta$ we get
	\[
		H_{\mu \alpha \beta} = \d_\mu C_{\alpha \beta} + \sum_I Y^I_\alpha \d_\mu Y^I_\beta
	\]
	Next
	\[
		H_{\mu \nu \alpha} = \d_\mu B_{\nu \alpha} - \d_\nu B_{\mu \alpha} + \frac12 \sum_I (\hat A_\nu^I \d_\mu Y^I_\alpha - \hat A^I_\mu \d_\nu Y^I_\alpha - Y^I_\alpha F^I_{\mu \nu})
	\]
	We define the $B$ field using not just the vielbein but also the gauge connection:
	\[
		B_{\mu \alpha} := \hat B_{\mu \alpha} + B_{\alpha \beta} A^\beta_\mu + \frac12 \sum_I Y^I_\alpha A^I_\mu, \qquad F^B_{\mu \nu} = \d_\mu B_\nu - \d_\nu B_\mu
	\]
	Then using \eqref{eq:H2} we get
	\[
		H_{\mu \nu \alpha} = F^B_{\alpha \mu \nu} - C_{\alpha \beta} {F^{A}_{\mu \nu}}^\beta - \sum_I Y^I_\alpha F^I_{\mu \nu}
	\]
	Finally, using both vielbein and connection
	\[
		B_{\mu \nu} = \hat B_{\mu \nu} + \frac12 [A_\mu^\alpha B_{\nu \alpha} + \sum_{I} A_\mu^I A^\alpha_\nu Y^I_\alpha - (\nu \leftrightarrow \mu)] - A_\mu^\alpha A_\nu^\beta B_{\alpha \beta}
	\]
	And this gives us 
	\[
		H_{\mu \nu \rho} = \d_\mu B_{\nu \rho} - \frac12 L_{ij} A^i_\mu F^j_{\nu \rho} + 2 \perms
	\]
	where $L_{ij}$ is the $(10-D, 26 - D)$-invariant metric and we have combined $A^\alpha_\mu, B_{\alpha \mu}, A^I_\mu$ into a length $36-2D$ vector.
	
	Now the full action is:
	\[
	\begin{aligned}
		\int d^D \sqrt{g} e^{-2\phi} [&R + 4 \d_\mu \phi \d^\mu \phi  -\frac{1}{12} H_{\mu \nu \rho} H^{\mu \nu \rho}\\
		& 
		- \frac{1}{4} G^{\alpha \beta} H_{\mu \nu \alpha} H^{\mu \nu \beta} - \frac14 G_{\alpha \beta} {F^{A}_{\mu \nu}}^{\alpha} {F^A}^{\mu \nu \beta}  -\frac14 \tilde F^{I}_{\mu \nu} \tilde F^{I, \mu \nu} \\
		&  - \frac{1}{4} H_{\mu \alpha \beta} H^{\mu \alpha \beta} + \frac14 \d_\mu G_{\alpha \beta} \d^\mu G^{\alpha \beta} - \frac{1}{2} \tilde F^{I}_{\mu \alpha} \tilde F^{I, \mu \alpha}]
	\end{aligned}
	\] 
	Using our expressions for $H_{\mu \nu \alpha}$ and $\tilde F_{\mu \nu}^A$, the middle line can be combined into
	\[
		-\frac14 \begin{pmatrix}
			G + C^T G^{-1} C + Y^T Y & -C^T G^{-1}  &  C^T G^{-1} Y^T + Y^T\\
			-G^{-1} C & G^{-1} & -G^{-1} Y^T  \\ 
			Y G^{-1} C + Y & -Y G^{-1} & 1 + Y G^{-1} Y^T
		\end{pmatrix}_{ij} F_{\mu \nu}^i F^{\mu \nu\, j}
	\]
	here $F^i = ({F^A}^\alpha, {F^B}_\alpha, F^I)$. Call the matrix $M^{-1}$ and notice that $L M L = M^{-1}$, and indeed we get $M$ transforms in the adjoint of $\SO(26-D, 10-D)$. 
	
	Similar arguments would give that the last line becomes $\frac18 \Tr \d_\mu M \d^\mu M^{-1}$ (Too much algebra).
	
	From this, its immediate that any $\SO(10-D, 26-D)$ transformation on the scalar matrix (adjoint rep) and array of vector bosons (vector rep) will preserve both of these last two terms. It will also preserve $H$ since it depends on the invariant $B_{\nu \rho}$ and $\SO$-invariant combination $L_{ij} A^i_\mu F_{\nu \rho}^j$. 
	
	\item The action for IIA in the string frame is
	\[
		\frac{1}{2 \kappa_{10}^2} \int d^{10} x \sqrt{-\hat G} \left[e^{-2\hat \Phi} [\hat R + 4 (\nabla \hat \Phi)^2  - \frac{1}{12} \hat H_{\hat \mu \hat \nu \hat \rho} \hat H^{\hat \mu \hat \nu \hat \rho}] - \frac{1}{4} F_2^2 - \frac{1}{2 \cdot 4!} F_4^2 \right] + \frac{1}{4 \kappa^2} \int B_2 \wedge \dd C_3 \wedge \dd C_3
	\]
	Doing the same reduction as before, the $\hat R + 4 (\nabla \hat \Phi)^2 - \frac{1}{12} H^2$ term becomes:
	\[
	\begin{aligned}
		& \int d^4 \sqrt{-g} e^{-2\phi} \Big[R + 4 \d_\mu \phi \d^\mu \phi  - \frac14  {F^{A}_{\mu \nu}}^{\alpha} {F^A}^{\mu \nu}_\alpha  + \frac14 \d_\mu G_{\alpha \beta} \d^\mu G^{\alpha \beta}  -\frac{1}{12} H_{\mu \nu \rho} H^{\mu \nu \rho}   - \frac{1}{4} H_{\mu \alpha \beta} H^{\mu \alpha \beta} - \frac{1}{4} G^{\alpha \beta} H_{\mu \nu \alpha} H^{\mu \nu \alpha} \Big]\\
		&= \int d^4 \sqrt{-g} e^{-2\phi} \Big[R + 4 \d_\mu \phi \d^\mu \phi  -\frac{1}{12} H_{\mu \nu \rho} H^{\mu \nu \rho}  - \frac14 M_{ij}^{-1} F_{\mu \nu}^{i} F^{\mu \nu\, j} + \frac18 \Tr[\d_\mu M \d^\mu M^{-1}]\Big]
	\end{aligned}
	\] 
	Here we used $H$ as in the last problem and the matrix $M$ consisting of the 21 $G_{\alpha \beta}$ and 15 $B_{\alpha \beta}$. The $F^i$ are the field strengths of the $6+6$ $U(1)$ vectors coming from $G$ and $B$ compactification.
	\[
		H_{\mu \nu \rho} = \d_\mu B_{\mu \rho} - \frac12 L_{ij} A^i_\mu  F^j_{\nu \rho} + 2 \perms
		 \qquad M^{-1} = \begin{pmatrix}
			G + B^T G^{-1} B & - B^T G^{-1}\\
			- G^{-1} B G^{-1} & G
		\end{pmatrix}
	\]
	
	
	The $H_{\mu \nu \rho}$ can be dualized to provide a \emph{sixteenth} scalar coming from the $B$ field. By analogy to \textbf{9.1.13}, in the string frame I would expect to write:
	\[
		e^{-2\phi} H_{\mu \nu \rho} = E_{\mu \nu \rho \sigma} \nabla^\sigma a
	\]
	The $B_{\mu \nu}$ equations $\nabla^\mu (e^{-2\phi} H_{\mu \nu \rho})$ are now automatically satisfied. The axion EOMs come from the Bianchi identity:
	\[
		E^{\mu \nu \rho \sigma} \d_\mu H_{\nu \rho \sigma } = - \frac12 L_{ij} E^{\mu \nu \rho \sigma} F^i_{\rho \sigma} F^j_{\mu \nu} = - L_{ij} \tilde F^{i}_{\mu \nu} F^{j\, \mu \nu}, \qquad \tilde F_{\mu \nu}^i  = \frac12 E^{\mu \nu \rho \sigma} F_{\rho \sigma}
	\]
	Here we have defined the dual 2-form as required. This can now be recast as the equation of motion for the axion (contracting the $E$s gives a $4$):
	\[
		\nabla^\mu (e^{2 \phi} \nabla_\mu a) = -\frac14 L_{ij} F^i_{\mu \nu} \tilde F^{j\, \mu \nu}
	\]
	With this, we can dualize the action in terms of the axion to yield:
	\[
		\int d^4 \sqrt{-g} e^{-2\phi} \Big[R + 4 \d_\mu \phi \d^\mu \phi  -\frac12 e^{4\phi} (\d a)^2+ \frac14 e^{2\phi} a L_{ij} F^i_{\mu \nu} \tilde F^{j\, \mu \nu}  - \frac14 M_{ij}^{-1} F_{\mu \nu}^{i} F^{\mu \nu\, j} + \frac18 \Tr[\d_\mu M \d^\mu M^{-1}]\Big]
	\]
	We could also do this in the Einstein frame and get \emph{exactly} the same action as in \textbf{9.1.15} with the $M$ matrix as we have it (no sum over heterotic internals). 
	
	The only thing left is the RR fields. We follow Kiritis' treatment of the 4-form field strength. We use the 10-bein to get: 
	\[
		\begin{aligned}
			C_{\alpha \beta \gamma} &= \hat C_{\alpha \beta \gamma}\\
			C_{\mu \alpha \beta} &= \hat C_{\mu \alpha \beta} - C_{\alpha \beta \gamma} A^\gamma_\mu\\
			C_{\mu \nu \alpha} &= \hat C_{\mu \nu \alpha} + \hat C_{\mu \alpha \beta} A^\beta_\nu  - \hat C_{\nu \alpha \beta} A^\beta_\mu + C_{\alpha \beta \gamma} A^\beta_\mu A^\alpha_\nu\\
			C_{\mu \nu \rho} &= \hat C_{\mu \nu \rho} - (A_{\mu}^\alpha \hat C_{\nu \rho \alpha} + A_\mu^\alpha A_\nu^\beta C_{\alpha \beta \rho} + 2 \perms) - C_{\alpha \beta \gamma} A^\alpha_\mu A^\beta_\nu A^\gamma_\rho
		\end{aligned}
	\]
	Let's now define the field strengths. Now we must have $F_{\alpha \beta \gamma \delta}= 0$ since the internal dimensions do not contain sources for the field. What remains is
	\[
	\begin{aligned}
		F_{\mu \alpha \beta \gamma} &= \d_\mu C_{\alpha \beta \gamma}\\
		F_{\mu \nu \alpha \beta} &= \d_\mu C_{\nu \alpha \beta} - \d_\nu C_{\mu \alpha \beta} + C_{\alpha \beta \gamma} F_{\mu \nu}^\gamma\\
		F_{\mu \nu \rho \alpha} &= \d_\mu C_{\nu \rho \alpha} + C_{\mu \alpha \beta} F^{\beta}_{\nu \rho} + 2 \perms\\
		F_{\mu \nu \rho \sigma} &= (\d_\mu C_{\alpha \beta \gamma} + 3 \perms) + (C_{\sigma \rho \alpha} F^\alpha_{\mu \nu} + 5 \perms)\\
	\end{aligned}
	\]
	Then this gives the contribution (here all two-lower one-upper index $F_{\mu \nu}^\alpha$ are taken to mean $F^A$):
	\[
		S_{RR}^{(4)} = -\frac{1}{2 \cdot 4!} \int d^4 \sqrt{-g} \sqrt{\det G_{\alpha \beta}} [F_{\mu \nu \rho \sigma} F^{\mu \nu \rho \sigma} + 4 F_{\mu \nu \rho \alpha} F^{\mu \nu \rho \alpha} + 6 F_{\mu \nu \alpha \beta} F^{\mu \nu \alpha \beta} +4 F_{\mu \alpha \beta \gamma} F^{\mu \alpha \beta \gamma}]
	\]
	It is important to realize that in 4-D the 4-form field strength coming from the 3-form has \emph{no} dynamical degrees of freedom. It plays the role of a cosmological constant \textbf{Check w/ Alek}.
	
	The two-spacetime-index term can be directly dualized. It corresponds to $6 \times 5/3= 15$ vectors.	The three-spacetime-index term can be dualized to become the kinetic term for $6$ scalar axions $a_\alpha$ with no interaction term.
	
	The $F_{\mu \alpha \beta \gamma}$ correspond to kinetic terms of the $6 \times 5 \times 4/3! = 20$ scalars $C^{(4)}_{\alpha \beta \gamma}$.
	
	Let's do a similar thing for the $2$-form field strength. There, we get $C_\alpha = \hat C_\alpha, C_\mu = \hat C_\mu - C_{\alpha} A^\alpha_\mu$. The corresponding field strength is $F_{\alpha \beta} = 0, F_{\mu \alpha} = \d_\mu C_\alpha$ and $F_{\mu \nu} = \d_\mu C_{\nu} - \d_\nu C_{\mu} + C_{\alpha} F^\alpha_{\mu \nu}$. We then get contribution 
	\[
		S_{RR}^{(2)} = -\frac{1}{4} \int d^4 \sqrt{-g} \sqrt{\det G_{\alpha \beta}} [F_{\mu \nu} F^{\mu \nu} + 2 F_{\mu \alpha} F^{\mu \alpha}]
	\]
	Again $F_{\mu \nu}$ can be written in terms of dual fields $\tilde F^{(2)}_{\mu \nu} = E_{\mu \nu \rho \sigma} F^{(2) \, \rho \sigma}$. This is one gauge fields and six further scalars. 
	
	\textbf{Return and think about the effect of the CS terms. I bet they  make the RR field equations non-free.}

	\item First note that using the OPE
	\[
		\Sigma^I(z) \bar \Sigma^J(w) = \frac{\delta^{IJ}}{(z-w)^{3/4}} + (z-w)^{1/4} J^{IJ}(w)
	\]
	 the $\braket{J^{II} \Sigma^J \bar \Sigma^J}$ correlator can be evaluated as
	\[
		\braket{J^{II}(z_1) \Sigma^J(z_2) \bar \Sigma^J(z_3)} =  (\delta^{IJ} - \tfrac14) \frac{z_{23}^{1/4}}{z_{12} z_{13}}
	\]
	Taking $z_1 \to z_2$ we see a singularity going as $\frac{(\delta^{IJ} - \tfrac14)}{z_{12}} z_{23}^{-3/4}$. Meanwhile taking the $J \Sigma$ OPE gives
	\[
		q \frac{\braket{\Sigma(z_2) \bar \Sigma(z_3)}}{z_{12}} = \frac{q}{z_{12}} z_{23}^{-3/4}
	\]
	So we see that under $J^I$ the charge of $\Sigma^J$ is $3/4$ if $I=J$ and $-1/4$ otherwise. We have $4$ $J^{II}$, and notice that the total charge under all four of each $\Sigma^I$ is always zero. Consider the following combination of charges, which provides a basis for the $\Sigma^I$ charge space
	\[
		\begin{aligned}
			\tilde J^1 &= J^{11} + J^{22} - J^{33} - J^{44}\\
			\tilde J^2 &= J^{11} - J^{22} + J^{33} - J^{44}\\
			\tilde J^3 &= J^{11} - J^{22} - J^{33} + J^{44}\\
		\end{aligned}
	\]
	Under each of $\tilde J^i$ we have the following charges 
	\[
	\begin{aligned}
		&\Sigma^1 \to (\tfrac12, \tfrac12, \tfrac12), \quad &\Sigma^2 \to (\tfrac12, -\tfrac12, -\tfrac12), \quad  &\Sigma^3 \to (-\tfrac12, \tfrac12, -\tfrac12), \quad &\Sigma^4 \to (-\tfrac12, -\tfrac12, \tfrac12)& \\
		&\bar \Sigma^1 \to (-\tfrac12,-\tfrac12,-\tfrac12), \quad &\bar \Sigma^2 \to (-\tfrac12, \tfrac12, \tfrac12), \quad &\bar \Sigma^3 \to (\tfrac12, -\tfrac12, \tfrac12), \quad &\bar \Sigma^4 \to (\tfrac12, \tfrac12, -\tfrac12) &\\
	\end{aligned}
	\]
	These are exactly all combinations, and we can define the three bosonic fields $\phi_i$ with $T = \sum_i \frac12(\d \phi_i)^2$ so that
	\[
		\Sigma^1 = \exp\left[i (\tfrac12 \phi_1 + \tfrac12 \phi_2 + \tfrac12 \phi_3) \right], \quad \Sigma^2 = \exp\left[i (\tfrac12 \phi_1 - \tfrac12 \phi_2 - \tfrac12 \phi_3) \right], \quad \text{etc.}
	\]
	Each of these $\Sigma^I, \bar \Sigma^I$ has dimension $3/8$ as required.
	
	Let's look at the supercurrent $G^{int}$. It can be written in terms of an eigenbasis of the commuting $\tilde J^i$. In particular look at $\tilde J^1$. 
	\[
		G^{int} = \sum_q e^{i q \phi_1} T^{(q)} 
	\]
	 Now consider the OPEs $G^{int} \cdot \Sigma^1$ and $G^{int} \cdot \bar \Sigma^1$. As observed in the chapter, both of these have only the singular term going as $(z-w)^{-1/2}$. Together both of these require that $q$ in $G$ can only be $\pm 1$. We can repeat this argument for $\tilde J^2, \tilde J^3$ to see that $G^{int}$ must be a sum of $6$ terms:
	 \[
	 	 e^{i q_1 \phi_1} Z_1 + e^{-i q_1 \phi_1} \bar Z_1 + e^{i q_2 \phi_2} Z_2 + e^{-i q_2 \phi_2} \bar Z_2 + e^{i q_3 \phi_3} Z_3 + e^{-i q_3 \phi_3} \bar Z_3
	 \]
	 Each $Z_i, \bar Z_i$ must be dimension one operators, so they are themselves bosonic fields $i \d X^i_\pm$. We thus have that $G^{int} = \sum_{i=1, \pm}^3 \psi_i^\pm \d X^i_{\pm}$. This is exactly the supercurrent for six free boson-fermion systems and will give (under anticommutator) the stress tensor of a six free boson-fermion systems. This is exactly a toroidal CFT.
	\item The relevant partition function is not difficult to compute, as we can follow $9.4$'s example but not do the twist on the internal $(0,16)$ part. Firstly the fermions on the left-moving (SUSY) side have orbifold blocks under the shifts as before:
	\[
		Z_{\psi}\twist{h}{g} = \frac12 \sum_{a,b=0}^1 (-1)^{a+b+ab} \frac{\theta^2 \twist{a}{b} \theta \twist{a+h}{b+g} \theta \twist{a-h}{b-g}}{\eta^4}
	\]
	Similarly we've already constructed the bosonic blocks before. They are given by \textbf{4.12.10} as:
	\[
		Z_{4,4}\twist00 = \frac{\Gamma_{4,4}}{\eta^4 \bar \eta^4}, \quad Z_{4,4} \twist{h}{g} = 2^4 \frac{\eta^2 \bar \eta^2}{\theta^2 \twist{1-h}{1-g} \bar \theta^2 \twist{1-h}{1-g}}
	\]
	Then the $(2,2)$ part is untouched, yielding $\frac{\Gamma_{2,2}}{\eta^2 \bar \eta^2}$ as is the $(0,16)$ part. We get the partition function
	\[
		Z^{het} = \underbrace{\frac{\Gamma_{2,2}}{\eta^2 \bar \eta^2}}_{\mathbf{1}}
		 \times \underbrace{\frac12 \sum_{h,g=0}^1 \frac{Z_{4,4} \twist{h}{g}}{\tau_2 \eta^2 \bar \eta^2}}_{\mathbf{2}} 
		\times \underbrace{\frac12 \sum_{a,b=0}^1 (-1)^{a+b+ab} \frac{\theta^2 \twist{a}{b} \theta \twist{a+h}{b+g} \theta \twist{a-h}{b-g}}{\eta^4}}_{\mathbf{3}} 
		\times \underbrace{\frac{\left(\frac12 \sum_{a,b=0}^1 \bar \theta \twist ab^8 \right)^2}{\bar \eta^{16}}}_{\mathbf{4}}
	\]
	Let's see how each term transforms under $\tau \to -1/\tau$. \textbf{1} stays invariant. \textbf{2} have $Z_{4,4} \twist{h}{g} \to Z_{4,4} \twist{g}{h}$ with  $\tau_2 \eta^2 \bar \eta^2$ invariant. \textbf{3} is the only nontrivial one. We will do it explicitly in the next step. % it will have $\theta^2 \twist ab \to \theta^2 \twist ba$ and $\theta \twist{a+h}{b+g} \theta \twist{a-h}{b-g} \to \theta \twist{b+g}{-a-h} \theta \twist{b-g}{-a+h}$. Redefining $g \leftrightarrow h$, $a\leftrightarrow b$ we get the sum
 % 	\[
 % 		\sum_{a,b} (-1)^{a+b+ab} \frac{\theta^2 \twist ab \theta \twist{a+h}{-b-g} \theta \twist {a-h}{-b+g}}{\eta^4}  = \sum_{a,b} (-1)^{a+b+ab} \frac{\theta^2 \twist ab \theta \twist{a+h}{b+g} \theta \twist {a-h}{b-g}}{\eta^4} \cancel{(-1)^{(a+h)(b+g) + (a-h)(-b+g)}}
 % 	\]
 % 	In this last step we recognize that $-b-g$ is always $0,-1,-2$. To get $b+g$ we must take add $2$ to $-b-g$ when it's odd, and this gives an extra $(-1)^{(a+h)(b-g)}$, with the other theta function contributing similarly.
	\textbf{4} will remain invariant. 
		
	Under $\tau \to \tau + 1$, we must be careful, as $\theta\twist10$ picks up an $e^{i \pi /4}$ while $\theta \twist{-1}{0}$ picks up $e^{-3 i \pi /4}$. The other two nonzero theta functions simply do $\theta \twist{a}{b} \to \theta \twist{a}{a+b-1}$
	
	\textbf{1}, \textbf{2}, remain invariant, with \textbf{2} making us change variables $g',h' = g, h+g-1$. The $\eta$ functions in the denominators of \textbf{3} and \textbf{4} leave over an $1/\bar \eta^{12}$ which contributes a $-$ sign. 
	
	Let's look at $3$. First when $h=0,g=0$ we have $(-1)^{a+b+ab} \theta^4 \twist ab$ and $\tau + \tau +1$ will send this to $-$ itself as required to cancel the $\bar \eta^{12}$ $-$ sign. 
	
	The other terms looks like (after canceling $\theta \twist11$)
	\[
	\footnotesize
	\begin{aligned}
		h=0, h=0:\;& \theta\twist00^4 - \theta\twist10^4 - \theta\twist01^4 - \theta\twist11^4 = 0\\
		h=1, g=0:\;& \theta \twist00^2 \theta\twist10 \theta \twist{-1}{0} - \theta \twist10^2 \theta \twist20 \theta \twist00 - \cancel{\theta \twist01^2 \theta \twist11 \theta \twist{-1}1} - \cancel{\theta \twist11^2 \theta \twist21 \theta \twist01} = \theta \twist00^2 \theta\twist10^2 - \theta \twist10^2 \theta \twist00^2 = 0 \\
		h=0, g=1:\;& \theta \twist00^2 \theta\twist01 \theta \twist0{-1} - \cancel{\theta \twist10^2 \theta \twist11 \theta \twist1{-1}} - \theta \twist01^2 \theta\twist02 \theta \twist00 - \cancel{\theta\twist11^2 \theta\twist12 \theta\twist10} = \theta \twist00^2 \theta\twist01 \theta \twist01 -   \theta \twist01^2 \theta \twist00^2  = 0 \\
		h=1, g=1:\;& \cancel{\theta \twist00^2 \theta\twist11 \theta \twist{-1}{-1}} - \theta \twist10^2 \theta \twist21 \theta \twist0{-1} - \theta \twist01^2 \theta \twist12 \theta \twist{-1}0 - \cancel{\theta \twist11^2 \theta\twist22 \theta\twist00} = - \theta\twist10^2 \theta\twist01^2 + \theta\twist01^2 \theta\twist10^2 = 0
	\end{aligned}
	\]
	Ok, so in fact this partition function is zero. This should not be surprising, since naively we are just breaking supersymmetry in half, and so we should still expect fermions and bosons to run in loops such that the vacuum energy vanishes. Naively, then we would again say ``zero is modular invariant'' and be done with it- but not so fast. There are still phases we can pick up, say from  $\tau \to \tau + 1$ that would not be visible given the vanishing of the partition function, but would nonetheless spoil modular invariance. 
	
	One way around this is to turn on the chemical potential $\nu_i$ in the theta functions to prevent vanishing. Effectively, then, we ignore the Jacobi identity and don't just set $\theta\twist11 = 0$. Then, let's look at how each term transforms under $\tau \to \tau+1$. Again, the terms not involving $\theta\twist11$ will cancel independently of $\nu_i = 0$ or not, and after simplifying things ,we have
	\[
	\footnotesize
	\hspace{-.5in}
	\begin{aligned}
		(0,0):\;&  \theta\twist00^4 - \theta\twist10^4 - \theta\twist01^4 - \theta\twist10^4 \to \theta\twist01^4 + \theta\twist10^4 - \theta\twist00^4 + \theta\twist11^4 \Leftarrow - (0,0)\\
		(1,0):\;&   - 2 \theta \twist01^2  \theta \twist11^2 \to -2 i\, \theta \twist00^2  \theta \twist11^2  \Leftarrow i \times (1,1) \\
		(0,1):\;&  2 \theta \twist10^2 \theta \twist11^2
		\to -2 \theta \twist10^2 \theta \twist11^2 \Leftarrow -(0,1) \\
		(1,1):\;&  -2\theta \twist00^2 \theta\twist11^2  \to -2 i \theta \twist01^2 \theta\twist11^2 \Leftarrow i \times (1,0)
	\end{aligned}
	\]
	So we see $(0,1)$ (ie the projected part of the untwisted sector) goes to its negative as required. On the other hand, the twisted sector has $(1,0)$ and $(1,1)$ swap, but with a factor of $i$ instead of $-1$. This is not good enough for modular invariance. 
	
	Under $\tau \to -1/\tau$ the sectors appropriately get sent to one another except for the twisted projected sector which picks up a factor of $-1$ from the $\theta \twist11^2$, so this too is not modular invariant.
	
	It is worth adding that Polchinski remarks in 16.1 that for abelian orbifolds (of the type $T^n/H$ with $H$ and abelian group), the only obstruction to modular invariance is $\tau \to \tau+1$ 
	
	Indeed, we see that this twist violates \textbf{16.1.28} of Polchinski, where we hae $r_2=0, r_3=r_4=1$ and so $\sum_{i=2}^4 r_i - \sum_{k=1}^{16} s_k 2 \neq 0 \text{ mod } 2N$ when $N = 2$.
	
	\item Now the partition function is given by
	\[\hspace{-.2in}
		Z^{het}_{N=2} = \underbrace{\frac{\Gamma_{2,2}}{\eta^2 \bar \eta^2}}_{\mathbf{1}} 
		\times \underbrace{\frac12 \sum_{h,g=0}^1 \frac{Z_{4,4} \twist{h}{g}}{\tau_2 \eta^2 \bar \eta^2} }_{\mathbf{2}}
		\times \underbrace{\frac12 \sum_{a,b=0}^1 (-1)^{a+b+ab} \frac{\theta^2 \twist{a}{b} \theta \twist{a+h}{b+g} \theta \twist{a-h}{b-g}}{\eta^4}}_{\mathbf{3}}
		\times \underbrace{\frac12 \sum_{\gamma,\delta=0}^1 \frac{\bar \theta^6 \twist{\gamma}{\delta} \bar \theta \twist{\gamma+h}{\delta+g} \bar \theta \twist{\gamma-h}{\delta-g}}{\bar \eta^8} }_{\mathbf{4}}
		 \times \underbrace{\frac{\frac12 \sum_{a,b=0}^1 \bar \theta \twist ab^8 }{\bar \eta^{8}}}_{\mathbf{5}}
	\]
	Things will still remain invariant under $\tau \to -1/\tau$ for the reasons given above, now applied to both \textbf{3} and \textbf{4}. The only important subtlety is now in the $(1,1)$ sector the $E_8$ $\bar \theta^6 \twist11$ will contribute a $-1$ sign, as necessary to cancel the twisted projected left-moving fermion sector. 
	
	Next, under $\tau \to \tau+1$, the exact same arguments apply to \textbf{3} and \textbf{4}, namely the untwisted sector of the left-handed fermions picks up $-1$ phase as required to cancel with the $\bar \eta$. The twisted sectors look like:
	\[
		\footnotesize
		\hspace{-.5in}
		\begin{aligned}
			(0,0):\;&  \bar \theta\twist00^8 + \bar \theta\twist10^8 + \bar \theta\twist01^8 + \bar \theta\twist10^8 \to \bar \theta\twist01^4 + \bar \theta\twist10^4 + \bar \theta\twist00^4 +\bar \theta\twist11^4 \Leftarrow (0,0)
			\\
			(1,0):\;& \bar \theta\twist00^6 \bar \theta \twist10^2 + \bar \theta\twist10^6 \bar \theta \twist 00^2 + \bar\theta \twist01^6  \bar \theta \twist11^2 + \bar \theta \twist11^6 \bar \theta\twist01^2 \to -i \bar \theta\twist01^6 \bar \theta \twist10^2 - i \bar \theta\twist10^6 \bar \theta \twist 01^2 - i \bar\theta \twist00^6  \bar \theta \twist11^2 - i \bar \theta \twist11^6 \bar \theta\twist00^2  \Leftarrow i \times (1,1) 
			\\
			(0,1):\;&  \bar \theta\twist00^6 \bar \theta \twist01^2 
			- \bar \theta\twist10^6 \bar \theta \twist 11^2
			+ \bar\theta \twist01^6  \bar \theta \twist00^2 
			- \bar \theta \twist11^6 \bar \theta\twist10^2 
			\to  \bar \theta\twist01^6 \bar \theta \twist00^2 
			- \bar \theta\twist10^6 \bar \theta \twist 11^2 
			+ \bar\theta \twist00^6  \bar \theta \twist01^2 
			- \bar \theta \twist11^6 \bar \theta\twist10^2  \Leftarrow  (0,1)
			\\
			(1,1):\;&  - \bar \theta\twist00^6 \bar \theta \twist11^2 - \bar \theta\twist10^6 \bar \theta \twist 01^2 
			- \bar\theta \twist01^6  \bar \theta \twist10^2 
			- \bar \theta \twist11^6 \bar \theta\twist00^2 
			\to i \bar \theta\twist01^6 \bar \theta \twist11^2 
			+i \bar \theta\twist10^6 \bar \theta \twist 01^2 
			+ i \bar\theta \twist00^6  \bar \theta \twist11^2 
			+ i \bar \theta \twist11^6 \bar \theta\twist00^2  \Leftarrow i \times (0,1) 
		\end{aligned}
	\]
	So we get that the untwisted sector remains the same, while each of the two twisted sector components change by a factor of $i$. This combines with what we know about the left-moving fermions to make \emph{every} combined contribution change with a $-$ phase which exactly cancels the $\eta$-functions. The result is modular invariant.
	
	To verify the spectrum, as remarked in the text when we act by orbifold on the $E_8 \times E_8$ we break down $[120]\oplus [128]$ of $\mathrm O(16)$. We get: $[120] \to [3,1,1] \oplus [1,3,1] \oplus [1,1,66] \oplus [2,1,12] \oplus [1,2,12]$ and $128 \to [2,1,32] \oplus [1,\bar 2, 32]$ in $\SU(2) \times \SU(2) \times \mathrm O(12)$.
	
	The $\ZZ_2$ action takes the spinors of the two $\SU(2)$ subgroups to minus themselves, keeping the conjugate spinors in variant. Projecting by this keeps $[3,1,1] \oplus [1,3,1] \oplus [1,1,66], [1,\bar 2, 32]$. This organizes into $[3,1] \oplus [1,133] \oplus [2,56] \in \SU(2) \times E_7$. Here $56$ is the fundamental representation and $133$ is the adjoint representation of $E_7$.
	
	Now let's organize our coordinates into $\mu = 2,3$ indicating the spatial coordinates in lightcone gauge, and pair the remaining $6$ coordinates into $Z^i = \frac1{\sqrt 2} (X^{2i} \pm i X^{2i+1} ), \, i = \{2,3,4\}$. Let's organize the different sector contributions based on how they transform under the $\ZZ_2$ twist:
	\begin{itemize}
		\item Untwisted Sector
			\begin{itemize}
				\item Left-handed side: 
				% For the bosons we have
				% \[
				% 	\begin{aligned}
				% 		&+: \quad \alpha^\mu_n, \alpha^4_n, \alpha^5_n\\
				% 		&-: \quad \alpha^{6,7,8,9}_n
				% 	\end{aligned}
				% \]
				\begin{itemize}
					\item NS - The zero-point energy is $-1/2$ and we thus have massless states coming from single fermion excitations. 
					\[
					\begin{aligned}
						&+: \psi_{-1/2}^\mu, \psi_{-1/2}^{4,5} \\
						&-: \psi_{-1/2}^{6,7,8,9}
					\end{aligned}
					\]
					\item R - The zero-point energy is $0$ from equal number of bosons and fermions and our massless excitation comes from the ground state. Under the rotation $e^{2\pi i (s_2 \phi_2 - s_3 \phi_3)}$ the ground states organize as follows:
					\[
					\begin{aligned}
						&+: \quad \ket{\tfrac12, \tfrac12, \tfrac12, \tfrac12}, 
						\ket{-\tfrac12, -\tfrac12, \tfrac12, \tfrac12}, 
						\ket{\tfrac12, \tfrac12, -\tfrac12, -\tfrac12}, 
						\ket{-\tfrac12, -\tfrac12, -\tfrac12, -\tfrac12}  \\
						&-: \quad \ket{\tfrac12, -\tfrac12, \tfrac12, -\tfrac12}
						\ket{\tfrac12, -\tfrac12, -\tfrac12, \tfrac12}
						\ket{-\tfrac12, \tfrac12, \tfrac12, -\tfrac12}
						\ket{-\tfrac12, \tfrac12, -\tfrac12, \tfrac12} \\
					\end{aligned}
					\]
					Note we only have an even number of $+$ signs in any of the ground states by GSO projection. \emph{These won't matter for the massless bosonic spectrum}.
				\end{itemize}
				\item Right-handed side

				The zero-point energy is $-1$, so we either have a bosonic excitation:
				\[
					\begin{aligned}
						&+: \quad \tilde \alpha_{-1}^\mu, \alpha_{-1}^{4,5}\\
						&-: \quad \tilde \alpha_{-1}^{6,7,8,9}\\
					\end{aligned}
				\]
				Or a weight 1 excitation from the current algebra:
				\[
					\begin{aligned}
						&+:\quad \ket{a^+} \in [3,1,1] \oplus [1,133,1] \oplus [1,1,128]\\
						&-:\quad \ket{a^-} \in [2,56,1]
					\end{aligned}
				\]
			\end{itemize}
			
			So, the untwisted bosonic massless states must be the $\ZZ_2$-invariant combinations of left (NS) and right movers. We get
			\begin{itemize}
				\item $\psi_{-1/2}^\mu \tilde \alpha_{-1}^\nu$: $G_{\mu \nu}, B_{\mu \nu}, \Phi$. 
				\item $\psi_{-1/2}^\mu \ket{a^+}$ - vector boson in the adjoint of $\SU(2) \times E_7 \times E_8$. This combines together with $\psi_{-1/2}^\mu \tilde \alpha_{-1}^{4,5}$ and $\psi_{-1/2}^{4,5} \tilde \alpha_{-1}^\mu$ to produce an extra $U(1)^4$. 
				\item $\psi_{-1/2}^{4,5} \ket a \cup \psi_{-1/2}^{4,5} \tilde \alpha_{1}^{4,5}$ - complex scalar transforming in the adjoint of $U(1)^4 \times \SU(2) \times E_7 \times E_8$
				\item $\psi_{-1/2}^{6,7,8,9} \tilde \alpha_{-1}^{6,7,8,9}$ - 16 neutral real scalars.
				\item $\psi_{-1/2}^{6,7,8,9} \ket{a^-}$ 4 real scalars transforming in the $[2, 56, 1]$ representation of $\SU(2) \times E_7 \times E_8$
			\end{itemize} 
			Here Kiritsis does not mention the presence of the dilaton with the other 16 real scalars. I assume this is an accidental omission. 
		\item Twisted Sector
		
		For the transformation $g$, we have $4$ points on each $T^2$ that are equivalence classes with the transformed point $g x$. This means that we have $4 \times 4$ equivalence classes that we must include in the spectrum for the twisted sector. This will be the same as looking at the spectrum for $1$  class of twist and taking it $16$-fold.
		
		Equivalently, because fixed points correspond to the equivalence classes in this case, note that our transformation has fixed points given by $(0, \frac12, \frac{\tau_2}2, \frac12 + \frac{ \tau_2}2) \times (0, \frac12, \frac{\tau_3}2, \frac12 + \frac{\tau_3}2)$ on the respective $T^2$s. The products give 16 fixed points. So we will have 16 copies of the spectrum at the fixed point $(0,0)$ on our $T^4$ \textbf{Appreciate this. Are you sure its not 32?}
		
		\begin{itemize}
			\item Left side
			The bosonic oscillators will be shifted by $1/2$ 
			
			The fermionic oscillators will also be shifted by $1/2$. 
			\begin{itemize}
				\item NS - The zero-point energy is now $-\frac14 +\frac14 = 0$ and so we get only one ground state - the vacuum.
				
				\item R - The zero-point energy remains zero. The zero modes that give the vacuum are now obtained from $\psi^{2,3,4,5}$. We thus get $2$ ground states after GSO projection, which will end up giving us the two requisite gravitinos
			\end{itemize}
			
			\item Right side:
			
			This is the hardest part. We use complex fermion language for the current algebra. We separate it into two parts $\lambda^{\pm, 1 \dots 8}, \lambda^{\pm, 9 \dots 16}$. We get massless states from the $(R, NS)$ and $(NS, NS)$ states.
			\begin{itemize}
				\item (NS,NS) Here the ground state energy is $-1/2$. We thus get the following states contributing: 
				\[
					\alpha_{-1/2}^{6,7,8,9}, \quad \lambda^{\pm 3\dots 8}_{-1/2}
				\]
				The first one will get GSO projected out (as will anything with an even number of fermions). 
				
				The second one will transform as the $[12]$ of $\SO(12)$. In line with this, we can also construct three other copies of $[12]$ (or $[\overline{12}]$):
				\[
					\lambda^{\pm 3\dots 8}_{-1/2} \lambda_0^{\pm 1} \lambda_0^{\pm 2}
				\]
				\textbf{ISNT THIS 5?}
				
				The other state we can build that does \emph{not} get GSO projected out is:
				\[
					\alpha_{-1/2}^{6,7,8,9} \lambda_0^{\pm, 1,2}
				\]
				This gives $4 \times 2$ copies of the $[2]$ of $\SU(2)$.
				
				\item (R, NS) Here the ground state energy is $0$. We have zero modes coming from the $12$ fermions $\lambda^{\pm, 3 \dots 8}$ giving $2^6$ ground states giving the $32$ and $\overline{32}$ spinors of $\SO(12)$, one of which will get projected out by GSO. 
				
				$\alpha_0$ alone will get GSO projected out, so does not contribute to the spectrum.
			\end{itemize}
			Together the two copies of $[32] + [12] + [12]$ of $\SO(12)$ combine together to form the two copies of the $[56]$ of $E_7$ and we get $8$ copies of the $2$ of $\SU(2)$.
			
			Altogether our gauge multiplets lie in $2 \times [1,56,1]$ and $8 \times [2,1,1]$.
			
		\end{itemize}
		Thus we get the twisted bosonic states coming from $\ket{0}_{NS} \ket{a}$ giving us $32$ scalars in the $[1,56, 1]$ and $128$ scalars in the $[2,1,1]$. 
		
		The zero-point energy calculations are here:
		\begin{center}
			\includegraphics[scale=0.5]{"Figures/Z2 Orbifold"}
		\end{center}
	\end{itemize}
	
	\item Under $\tau \to \tau+1$ its quick to see that compactifying on any $(d, d+16)$ Lorentzian lattice and orbifolding by a $\ZZ_n$ shift symmetry of $\epsilon/N$ will give a transformation 
	\[ 
		\tau \to \tau+1 :  Z^N \twist hg = e^{4\pi i/3} e^{\frac{i \pi h^2 \epsilon^2}{N^2}} Z^N \twist{h}{h+g}
	\]
	where the first exponential factor comes from the $\bar \eta^{-16}$ and the second factor  comes from shifting $p^2_L - p^2_R$ which is otherwise even by $\epsilon h/N$ which gives $\frac{h^2}{N^2} (\epsilon^2_L - \epsilon^2_R) = h^2 \epsilon^2/N^2$. 
	
	The $\tau \to -1/\tau$ phase
	\[
		\tau \to -1/\tau : Z^N\twist hh  \to e^{-\frac{2\pi i h g \epsilon^2}{N}} Z^N \twist{g}{-h}
	\]
	can similarly be proven from straightforward Poisson resummation. 
	
	This problem specializes to $N=2$.
	
	For $\epsilon^2/2=1 \text{ mod } 4$ the twisted sector picks up a phase under $\tau \to \tau +1$ and one can see that this phase is $+i$, just as in the last problem. This is what was necessary to combine with the left-moving fermions to give a modular invariance. Note this happens \emph{only} when $\epsilon^2/2 = 1 \text{ mod } 4$. 
	
	Under $\tau \to -1/\tau$ the twisted sector's projected part picks up a factor of $-1$, exactly what we need to cancel the $-1$ on the left-moving side.
	
	\item The partition function for our general heterotic $\mathcal N=2$ compactification takes the form:
	\[
		Z_{N=2}^{het} = \frac12 \sum_{h,g=0}^1 \frac{\Gamma_{2,18} \twist hg \Gamma_{4,4} \twist hg}{\tau_2 \eta^{8} \bar \eta^{24}} \frac12 \sum_{a,b=0}^1 \frac{\theta^2 \twist ab \theta\twist{a+h}{b+g} \theta\twist{a-h}{b-g}}{ \eta^4}
	\]
	 We seek to compute $\tau_2 B_2$ where $B_2 = \Tr[(-1)^{2\lambda} \lambda^2]$ over our string's Hilbert space. To do this, consider the following \emph{helicity generating} partition function:
	\begin{equation} \label{eq:helgen}
			\mathcal Z(\nu, \bar \nu) = \Tr[q^{L_0} \bar q^{\bar L_0} e^{2\pi i \nu \lambda_L - 2\pi i \bar \nu \lambda_R}] =  \frac12 \sum_{h,g=0}^1 \frac{\Gamma_{2,18} \twist hg \Gamma_{4,4} \twist hg}{\tau_2 \eta^{8} \bar \eta^{24}} \xi(\nu) \bar \xi(\bar \nu) \frac12 \sum_{a,b=0}^1 \frac{\theta\twist ab(\nu) \theta \twist ab \theta\twist{a+h}{b+g} \theta\twist{a-h}{b-g}}{ \eta^4}
	\end{equation}
	Here
	\[
		\xi(\nu) = \prod_{n=1}^\infty \frac{(1-q^n)^2}{(1-q^n e^{2\pi i n \nu}) (1-q^n e^{-2\pi i n \nu})} = \frac{\sin \pi \nu}{\pi} \frac{\theta'_1(\nu)}{\theta_1(\nu)}
	\]
	plays the role of exchanging the traces over the bosons in the non-compact spatial (3,4) directions with traces that involve the helicity. 
	
	I apply formula \textbf{D.21} in Kiritsis to simplify the theta functions to:
	\begin{equation}\label{eq:thetaid}
		\frac12 \sum_{a,b=0}^1 \frac{\theta\twist ab(\nu) \theta \twist ab \theta\twist{a+h}{b+g} \theta\twist{a-h}{b-g}}{ \eta^4} = \frac{\theta^2 \twist11 (\frac \nu2) \theta \twist{1-h}{1-g} (\frac \nu2) \theta \twist{1+h}{1+g} (\frac \nu2)}{\eta^4}
	\end{equation}
	This vanishes at least as fast as $\nu^2$
	
	We must now take \eqref{eq:helgen} this and apply
	\[
		\Big(\frac{1}{2\pi i} \d_\nu - \frac{1}{2\pi i} \bar \d_{\bar \nu}\Big)^2  \mathcal Z(\nu, \bar \nu).
	\]
	Because our generating function \eqref{eq:helgen} vanishes as $\nu^2$ thanks to \eqref{eq:thetaid}, we only need to look at $\d_\nu^2$.
	
	To obtain a nonzero result we thus need to act with $\d_\nu^2$. On these terms for each $h,g$. First note that for $h=g=0$ \eqref{eq:thetaid} vanishes as $\nu^4$ so will not contribute. For $(h,g) \neq (0,0)$, the terms vanish as $\nu^2$ due to the $\theta^2 \twist11$, exactly cancellable by taking two derivatives on that term. Thus, we need only worry about the zeroth order behavior of everything else: $\xi \sim 1$ and $\theta\twist{1-h}{1-g} \theta \twist{1+h}{1+g} (\nu/2) \sim \theta\twist{1-h}{1-g} \theta \twist{1+h}{1+g} (0)$. We are left with 
	\[
	\begin{aligned}
		&-\frac{\pi^2}{4} \frac{1}{(2 \pi)^2} \frac12 \sum_{h,g \neq (0,0)} \frac{\Gamma_{2,18} \twist hg}{\tau_2 \eta^{8} \bar \eta^{20}} \frac{ 16 \eta^2 \bar \eta^2}{\theta^2 \twist{1-h}{1-g} \bar \theta^2 \twist{1-h}{1-g}} \theta\twist{1-h}{1-g} \theta \twist{1+h}{1+g} \big(\d_\nu \theta \twist11_{\nu = 0}\big)^2\\
		&= -\frac{4 \eta^6}{2 \tau_2 \eta^6 \bar \eta^{18}} \left[\frac{\Gamma_{2,18} \twist11 }{\bar \theta^2 \twist00} + \frac{\Gamma_{2,18} \twist10 }{\bar \theta^2 \twist01} -
		\frac{\Gamma_{2,18} \twist01 }{\bar \theta^2 \twist10} \right]
	\end{aligned}
	\]
	Where we have used $\theta \twist 12 = - \theta \twist10$, as well as $\theta_1'|_{\nu = 0} = 2 \eta^3$ 
	We now use the identity
	\[
		\bar \theta_2 \bar \theta_3 \bar \theta_4 = 2 \bar \eta^3
	\]
	
	and recover
	\[
		\tau_2 B_2 = - \frac{\Gamma_{2,18} \twist11 \bar \theta_2 \bar \theta_3}{\bar \eta^{24}} - \frac{ \Gamma_{2,18} \twist 10 \bar \theta_2 \bar \theta_3}{\bar \eta^{24}} +\frac{ \Gamma_{2,18} \twist 01 \bar \theta_3 \bar \theta_4}{\bar \eta^{24}}.
	\]
	
	\item The gravitini can only come from the untwisted left-moving R sector (spinor spacetime index) tensored with an $\tilde \alpha^{2,3}_{-1}$ on the right (vector spacetime index). The zero-point energy of the left-moving R sector is $0$ from equal numbers of bosons and fermions. Because our group acts on the (bosonized) fermions the same way it acts on the bosons, we get that $\ZZ_2^2$ gives the three nontrivial elements given by rotations $e^{2 \pi i (s_1 \phi_1 - s_2 \phi_2)}, e^{2 \pi i (s_1 \phi_1 - s_3 \phi_2)}, e^{2 \pi i (s_2 \phi_1 - s_3 \phi_2)}$, with $\phi_0$ corresponding to the spacetime fermions not appearing. We see that the only spinors which are invariant under these three transformations take the form
	\[
		\ket{\pm \tfrac12, \tfrac12, \tfrac12, \tfrac12}, \ket{\pm \tfrac12, -\tfrac12, -\tfrac12, -\tfrac12}
	\]
	And we must have an even number of signs by GSO projection, so we in fact get two supersymmetries preserved: $\tilde \alpha^{2,3}_{-1} \ket{\frac12,\frac12,\frac12,\frac12}, \tilde \alpha^{2,3}_{-1} \ket{-\frac12,-\frac12,-\frac12,-\frac12}$, providing the $\pm 3/2$ states only \emph{one} gravitino. 
	
	\item 
	
	\item As before, the twist acts the same way on the bosons and (left moving) fermions. Already at this level, we see that the only invariant states $\ket{s_1, s_2, s_3, s_4}$ must satisfy $s_2 = s_3 = s_4$ so we will have the (GSO projected) possibilities:
	\[
		\ket{\tfrac12, \tfrac12, \tfrac12, \tfrac12}, \ket{-\tfrac12, -\tfrac12, -\tfrac12, -\tfrac12}
	\]
	providing again the $\pm 3/2$ states of a \emph{single gravitino}.
	
	To avoid anomaly from ground state energy mismatch, we need the condition of Polchinski \textbf{16.1.28}
	\[
		\sum_{i=2}^4 r_i^2 - \sum_{I=1}^{16} s_I^2 = 0 \text{ mod } 2N
	\]
	Here $N = 6$. Note that our $r_i = (1,1,-2)$ already sums to $6$, so we must have the same for our $s_i$ that determines the $\Gamma_{16}$ action. 
	
	I am confused why Kiritsis is saying there is only one such action of $\ZZ_3$ on $\Gamma_{16}$. As long as $\sum s_i^2= 0 \text{ mod } 6$ we should get a consistent theory, as shown in \textbf{Table 16.1} of Polchinski. 
	
	The simplest such twist (aside from the trivial one that leaves the $E_8 \times E_8$ untouched) would be to act on the first 3 complex fermions of forming the first $E_8$ group in the same way as we act on the complexified bosons and left-moving fermions, namely by
	\[
		\tilde \lambda^{\pm, 1,2,3} \to e^{\pm 2\pi i \beta_{1,2,3}} \tilde \lambda^{\pm, 1,2,3}, \quad \beta_1 = \beta_2 = \tfrac13, \beta_3 = -\tfrac23
	\]
	while the remaining $\lambda^{\pm, 4\dots 16}$ are left untouched. Let's now get the massless spectrum under $\ZZ_3 = \{1, r, r^2\}$
	\begin{itemize}
		\item Untwisted
		\begin{itemize}
			 \item Left-moving
			 The bosons are labeled by
		\end{itemize}
			
			
			
		
		\item Twisted by $r$
		
		\item Twisted by $r^2$
	\end{itemize}
	
	\item The Nijenhuis tensor is defined in terms of the almost-complex structure $(1,1)$ tensor $J^i_j$ as
	\[
		N_{ij}^k = J^l_i (\d_{l} J^k_{j} - \d_j J^k_l) - J_{j}^l (\d_l J^k_i - \d_i J^k_l)
		 = J^l_i (\d_{[l} J^k_{j]}) - J_{j}^l (\d_{[l} J^k_{i]}) = J^l_i (\nabla_{[l} J^k_{j]}) - J_{j}^l (\nabla_{[l} J^k_{i]})
	\]
	We are able to replace partial derivatives with covariant derivatives and vice versa because the non-tensoriality can only enter through the Christoffel symbols $\Gamma$, as follows
	\[
		J^l_i (\Gamma_{r[l}^k J_{j]}^r - \cancel{\Gamma_{[lj]}^q J_q^k} ) - J^l_j (\Gamma_{r[l}^k J_{i]}^r - \cancel{\Gamma_{[li]}^q J_q^k} ) =  \Gamma_{r[l}^k J_{j]}^r J^l_i - \Gamma_{r[l}^k J_{i]}^r  J^l_j = (-)^2 \cancel{(\delta^r_i \Gamma^k_{rj}  - \delta^r_j \Gamma^k_{ri})} + \cancel{\Gamma_{rl}^k (J^r_j J^l_i - J^l_j J^r_i)} =0
	\]
	So we see because everything is antisymmetrized that the Nijenhuis tensor is indeed a tensor. 
	
	
	\item It is easiest to directly construct coordinate patches on $\CP^N$. We define $N$ such patches to consist of $N$ complex coordinates $z_1 = Z_1/Z_i \dots z_{i-1} = Z_{i-1}/Z_{i}, z_{i+1} = Z_{i+1}/Z_i, \dots z_N = Z_N/Z_i$ that are valid for all parts of $\CP^N$ where $Z_i \neq 0$. It is clear that these coordinates cover the whole manifold, since one $Z_i$ is always not equal to zero, so any given point is always in a coordinate patch. 
	
	Moreover, the transition functions between different patches $U, U'$ are simply fractional linear transformations of the $z_i, z_i'$, so are holomorphic. This is enough to give a globally defined complex structure (vanishing $N_{ij}^k$) the the manifold. 
	
	\item We begin with the existence of a Killing spinor 
	\[
		[\nabla_m, \nabla_n]\xi = R_{rs, mn} \gamma^{rs} \xi  =0
	\]
	Now, multiplying by $\gamma^n$, we get
	\[
		0 = \gamma^{n} \gamma^{rs} R_{rs, mn} \xi = (\gamma^{nrs} + g^{nr} \gamma^s - g^{ns} \gamma^r) R_{rs, mn} \xi 
	\]
	The first term vanishes by the Bianchi identity. The other two terms give:
	\[
		2 R_{ns} \gamma^s \xi = 0
	\]
	Now we can multiply by $\bar \xi \gamma_r$ to get
	\[
		0 = R_{ns} \bar \xi \gamma_r \gamma^s \xi = R_{ns} J^r_s
	\]
	Since the complex structure is invertible, this gives $R_{ns} = 0$, so indeed our space is Ricci flat.
	
	\item To verify the masslessness of the graviton, it is enough to look at linearized gravity and confirm that the perturbations satisfy the massless spin 2 condition. 
	
	To
	
	\item 
	
	\item 
	
	\item The NSNS fields give a graviton, an antisymmetric tensor, and 81 scalars. From the RR sector we get another scalar from the axion $C_2$, another 2-index antisymmetric tensor and 22 scalars from $C_2$. Finally the self-dual 4-form $C_4$ gives 19 anti-self-dual and 3 self-dual two-index antisymmetric tensors. 
	
	The supergravity multiplet contains two left-handed Weyl gravitini. The tensor multiplet contains two Weyl fermions of opposite chirality from the gravitini. 
	
	In order to cancel anomalies, we must be able to apply the Green-Schwartz mechanism
	
	Thus we get $N_{T} = 21$. Now IIB has a self-dual 5-form. For each of 20 2-cycles wrapped we get 
	
	
	 % Its easiest to do this by induction. We already know $\CP^1$, the Riemann sphere, has complex structure most easily constructed through two complex coordinate charts patching the northern hemisphere with the southern hemisphere.
	
\end{enumerate}

% section chapter_9_compactification_and_supersymmetry_breaking (end)
\end{document}
	