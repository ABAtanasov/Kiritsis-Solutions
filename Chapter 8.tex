\documentclass[11pt, class=article, crop=false]{standalone}
\usepackage{amsmath,amssymb,amsfonts,amsthm}
\usepackage{enumitem}
\usepackage{fancyhdr}
\usepackage{tikz-cd}
\usepackage{mathabx}
\usepackage{geometry}
\usepackage{natbib}
\usepackage{braket}
\usepackage{graphicx}
\usepackage{simpler-wick}
\usepackage{hyperref}
\usepackage{ytableau}
\usepackage{cancel}
\usepackage{listings}
\usepackage{relsize}
\usepackage{xcolor}
\usepackage{stmaryrd}
\usepackage{tikz-feynman}
\usepackage{kiritsis}
\geometry{margin = 0.5in}


\begin{document}
\section*{Chapter 8: D-Branes} % (fold)
\label{sec:chapter_8_d_branes}

\begin{enumerate}
	\item First, a simple magnetic monopole for a 1-form gauge field in $D$ spacetime dimensions has a radial magnetic field that $B_r = \frac{\tilde Q_1}{\Omega_{D-2} r^{D-2}}$ where $\Omega_{D-2} = 2 \pi^{d/2} \Gamma(d/2)$ is the volume of a unit $D-2$ sphere. This way, the flux of the solution over any $D-2$ sphere surrounding the (point) monopole will be $\tilde Q_1$.
	
	Upon taking the Hodge star we get the solution is $F = \tilde Q_1 \sin \theta \dd \theta \wedge \dd \phi$. We can write this as $A = \tilde Q_1 (c - \cos \theta) \dd \phi$. Taking $c = 1$ we get $A$ vanishes at $\theta = 0$ (which we need since the $\phi$ coordinate degnerates there) while taking $c = -1$ we get $A$ vanishes at $\theta = \pi$, which we also need.
	
	We cannot have \emph{both} solutions, and so we realize we are dealing with two $A$s, corresponding to local sections of a line bundle over $S^2$ on different hemispheres. Let $A^+$ be well-defined on all points on $S^2$ except $\theta = \pi$. Then $A^+$ is a section of a line bundle on the punctured sphere. The punctured sphere is contractible so any fiber bundle over it is trivial, so $A^+$ is just a \emph{function} on the punctured sphere $S^2 \setminus \{\theta = \pi\}$. So let's define $A^+ = \tilde Q_1 (1- \cos \theta) \dd \phi$. Similarly, we define $A^-$ to be the nonsingular $A$ on the sphere with $\theta = 0$ removed, namely $A^- = \tilde Q_1 (-1 -\cos \theta) \dd \phi$. 
	
	On the overlap, $A^+ - A^-$ differ by an integer, which labels the degree of ``twisting'' of this line bundle over $S^2$. 
	
	For a $p$ form, our monopole will now be spatially extended in $p-1$ directions. Label these (locally), by $x^1 \dots x^{p-1}$. Time is $x^0$. Locally transverse to these coordinates will be $r, \varphi^1 \dots, \varphi^{D-1-p}$, where $\varphi^i$ parameterize a $D-1-p$ sphere enclosing the monopole. The field strength looks like:
	\[
		% B_r = \frac{\tilde Q_p}{\Omega_{D-1-p} r^{D-1-p}} \dd t \wedge \dd r \Rightarrow
		F = \tilde Q_p \, \Omega_{D-p-1}
	\]
	where $\Omega$ is the canonical $D-p-1$-sphere area form:
	\[
		\Omega = \sin^{D-p-2}(\varphi_1) \sin^{D-p-3}(\varphi_2) \dots \sin(\varphi_{D-p-2})\, \dd \varphi_1 \wedge \dots \dd \varphi_{D-p-1}
	\]
	This can be written (unfortunately unavoidably) in terms of a hypergeometric function:
	\[
		A =  {_2 F_1}\left(\frac12, \frac{D-p-1}{2}, \frac{D-p+1}{2}, \sin^2(\varphi_1) \right) \frac{\sin^{D-p-1}(\varphi_1)}{D-p-1} \dd \varphi_2 \wedge \dots \wedge \dd \varphi_{D-p-1}
	\]
	there is no need for an overall constant, as the function above vanishes at both $\varphi_1 = 0$ and $\pi$, \emph{however} this is compensated by the hypergeometric function having a branch cut at $\varphi_1 = \pi/2$. Across this cut, it will have a discontinuity set by an integer depending on the convention of the arcsin function, and again we will have $A^+ - A^-$ differing by an integer.  The same quantization condition follows. 
	
	Again $A^+$ will be defined on the $S^{D-p-1}$ sphere minus the south-pole (this is homeomorphic to the $D-p-1$ ball, and hence contractible, so again the line bundle trivializes and $A^+$ is a bona-fide function for any $D, p$) and $A^-$ is similarly defined on the sphere with the excision of the north pole.
	
	\item Our simply charged point particle with a Wilson line $A_9 = \chi/2\pi R$ turned on will have an action
	\[
		S = \int d \tau \underbrace{\left( \frac12 \dot x^M \dot x_M - \frac{m^2}{2} + q A_9 \dot x^9 \right)}_{\mathcal L}
	\]
	The canonical momentum will be $p_i = \dot x^\mu$ for $\mu = 0 \dots 8$ and $p_9 = \dot x^9 + \frac{q \chi}{2 \pi R}$. Consequently, our hamiltonian is
	\[
	\begin{aligned}
		H &= p_M \dot x^M - \mathcal L = \frac12 p_\mu p^\mu + p_9 (p_9 -  \frac{q \chi}{2 \pi R}) - \left[\frac12 (p_9 - \frac{q \chi}{2\pi R})^2 - \frac{m^2}{2} + (p_9 - \frac{q \chi}{2 \pi R}) \frac{q \chi}{2 \pi R}\right]\\
		&= \frac12 \left(p_\mu p^\mu + (p_9 - \frac{q \chi}{2\pi R})^2 + m^2\right)\\
		&= \frac12 \left(p_\mu p^\mu + \left(\frac{2 \pi n - q \chi}{2\pi R}\right)^2 + m^2\right)
	\end{aligned}
	\]
	
	
	\item For a string satisfying Dirichlet boundary conditions, the total momentum is not conserved (along the directions associated with the D boundary conditions). This is easily interpretable as momentum transfer to the brane that it is attached to.
	
	\item For an open string of state $\ket{ij}$, $A_9$ will act as $\frac{\chi_i - \chi_j}{2\pi i}$. Since this is an open string with no winding, we can only have momentum contribution, and so we will get a mass formula
	\[
		m^2_{ij} = \frac{\hat N - \frac12}{\ell_s^2} + \left(\frac{n}{R} - \frac{\chi_i - \chi_j}{2\pi R} \right)^2
	\]
	In particular at the lowest (massless) level for a string without momentum we will get the desired spectrum
	\[
		m_{ij}^2 =  \left(\frac{\chi_i - \chi_j}{2\pi R} \right)^2
	\]
	
	\item For completeness, we will do both the gauge boson and scalar scattering. These come from the NS sector, and are given by:
	\[
		V_{-1}^{a, \mu} = g_p \lambda^a \psi^\mu e^{-\phi} c e^{i p X}, \qquad
		V_{0}^{a, \mu} = \frac{g_p}{\sqrt 2 \ell_s} \lambda^a (i \dot X + 2 p \cdot \psi \, \psi^\mu) c e^{i p X}
	\]
	We can explicitly scatter four such gauge bosons - two in the $-1$ picture and two in the $0$ picture.
	\[
	\begin{aligned}
		& \qquad i C_{D^2} \sdelta^{10}(\Sigma k) \times g_p^4 \braket{c V_{-1}(y_1) c V_{-1}(y_2) c V_{0}(y_3) c V_{0}(y_4)} + 5 \perms \\
		& = \frac{i \sdelta^{p+1}}{g_p^2 \ell_s^4} \times \frac{g_p^4}{2\ell_s^2} \braket{[\psi^{\mu_1} e^{ik_1 X}]_{y_1}\;
		 [\psi^{\mu_2} e^{ik_2 X}]_{y_2} \;
		[(i {\dot X}^{\mu_3} + 2 k_3 \cdot \psi \psi^{\mu_3}) e^{i k_3 X}]_{y_3}\;
		[(i {\dot X}^{\mu_4} + 2 k_4 \cdot \psi \psi^{\mu_4}) e^{i k_43 X}]_{y_4}\;
		  }  \\ & \qquad  \quad \times \braket{c(y_1) c(y_2) c(y_3)} \braket{e^{-\phi(y_1)} e^{- \phi(y_2)}}
	\end{aligned}
	\]
	Take $y_1 = 0, y_2 = 1, y_3 = \infty$ and integrate $y_4 = y$ from $0$ to $1$ (then we'll have 5 more terms coming from permuatations). There are five contributions. 
	\begin{itemize}
		\item Contracting the $\psi^{\mu_1}(0) \psi^{\mu_2}$ together and allowing the remaining 4 terms at $y_3, y_4$ to contract either amongst themselves or with various vertex operators. 
		\item Not contracting the first two $\psi$ and Contracting the $i \dot X(y_3)$ with any of the vertex operators while contracting the last $\psi$ with the first two
		\item Not contracting the first two $\psi$ and contracting the $i \dot X(y_4)$ with any of the vertex operators while contracting the third $\psi$ with the first two
		\item Forgetting the $i \dot X$, and contracting the $\psi$ at $y_1$ with the various $\psi$ at $y_3$ (consequently the $\psi$ at $y_2$ with the $\psi$ at $y_4$)
		\item Swapping $3$ with $4$ in the above (this gives an overall minus sign by fermionic statistics)
	\end{itemize}
	Integrating this will give two types of terms: $\eta^{ab} \eta^{cd}$ and $\eta^{ab} k^c k^d$. Our shorthand replaces the superscript $\mu_i$ by just  $i$. Below, I underline the terms that contribute to the first type:
	\begin{equation}\label{eq:gauge}
	\begin{aligned}
		\frac{i g_p^2 \sdelta^{p+1}}{\ell_s^4} \int_0^1 dy \frac{y^{2 \ell_s^2 k_1 \cdot k_4} y^{2 \ell_s^2 k_2 \cdot k_4}}{2 \ell_s^2}
		 \Bigg\{
		& -\ell_s^2 \eta^{12} 
		 \left[ \underline{2 \ell_s^2 \eta^{34}} + (2 \ell_s^2)^2 (\underline{-\eta^{34} k_3 \cdot k_4} + k_4^3 k_3^4) + (2\ell_s^2)^2 (k_2^3 + k_4^3) \Big(\frac{k_1^4}{y} + \frac{k_2^4}{y-1}\Big) \right]\\
		& + \ell_s^2 (2 \ell_s^2)^2 
		\big[(k_2^3 + y k_4^3)(-k_4^1 \eta^{24} + k_4^2 \eta^{14}) \big] 
		\\ & + \ell_s^2 (2 \ell_s^2)^2 
		\Big[\Big(\frac{k_1^4}{y} + \frac{k_2^4}{y-1} \Big) (-k_3^1 \eta^{23} + k_3^2 \eta^{13}) \Big]\\
		& + \ell_s^2 \frac{(2 \ell_s^2)^2}{y-1} 
		\left[\underline{\eta^{13} \eta^{24} k_3 \cdot k_4} + \eta^{34} k_3^1 k_4^2 - \eta^{13} k_4^2 k_3^4 - \eta^{24} k_4^3 k_3^1 \right]\\
		& - \ell_s^2 \frac{(2 \ell_s^2)^2}{y} 
		\left[\underline{\eta^{14} \eta^{23} k_3 \cdot k_4 }+ \eta^{34} k_3^2 k_4^1 - \eta^{13} k_4^2 k_3^4 - \eta^{24} k_4^3 k_3^1 \right]  \Bigg\}
	\end{aligned}		
	\end{equation}
	Using $s = -2 \ell_s^2 k_1 \cdot k_2$ etc we see the underlined terms contribute
	\[
	\begin{aligned}
		& \quad \frac{i g_p^2 \sdelta^{p+1}}{\ell_s^2} \left[\frac{\Gamma(1-u) \Gamma(1-t)}{\Gamma(2+s)} (-(1 + s) \eta^{12} \eta^{34}) + \frac{\Gamma(1-u) \Gamma(-t)}{\Gamma(1+s)} \eta^{14} \eta^{23} s + \frac{\Gamma(-u) \Gamma(1-t)}{\Gamma(1+s)} \eta^{14} \eta^{23} s \right][1423]\\
		& = \frac{i g_p^2 \sdelta^{p+1}}{\ell_s^2} \frac{\Gamma(-u) \Gamma(-t)}{\Gamma(1+s)} ( - tu \eta^{12} \eta^{34} - su \eta^{13} \eta^{24} - st \eta^{14} \eta^{23} ) [1423]
	\end{aligned}
	\]
	The non-underlined terms are more involved but end up contributing twelve terms that yield:
	\[
	\begin{aligned}
		& \quad \frac{i g_p^2 \sdelta^{p+1}}{\ell_s^2} \left[\frac{\Gamma(-u) \Gamma(1-t)}{\Gamma(1+s)} 2 \ell_s^2 \eta^{12} k_2^3 k_1^4 + \dots \right][1423] = \frac{i g_p^2 \sdelta^{p+1}}{\ell_s^2} 2 \ell_s^2 \frac{\Gamma(-u) \Gamma(-t)}{\Gamma(1+s)} \left[ t \eta^{12} k_2^3 k_1^4 + 11 \perms \right][1423]
	\end{aligned}
	\]
	Here my Mandelstam variables are dimensionless. The result with dimensionful Mandelstam variables is:
	\begin{equation}\label{eq:4gluons}
		i g_p^2 \sdelta^{p+1} \ell_s^2  \left( \frac{\Gamma(-\ell_s^2 s) \Gamma( -\ell_s^2 u)}{\Gamma(1 + \ell_s^2 t)} K_4(k_i, e_i) ([1234] + [4321]) + 2 \perms\right)
	\end{equation}
	with $K_4 (k_i, e_i) = - t u e_1 \cdot e_2 e_3 \cdot e_4 + 2 s (e_1 \cdot e_3 \, e_2 \cdot k_4 e_4 \cdot k_2 + 3 \perms )  + 2 \perms$
	
	Now for the four transverse scalar amplitude, our vertex operators look like:
	\[
		V_{-1}^{a, \mu} = g_p \lambda^a \psi^\mu e^{-\phi} c e^{i p X}, \qquad
		V_{0}^{a, \mu} = \frac{g_p}{\sqrt 2 \ell_s} \lambda^a (X' + 2 p \cdot \psi \, \psi^\mu) c e^{i p X}
	\]
	here $X' = \partial_\sigma X =  2 i \partial X$. Moreover, we have Dirichlet boundary conditions on both the $X$s and the fermions $\psi$. The $\psi$ are still in the NS sector since we're looking at the (bosonic) scalar field scattering. 
	
	Crucially, the correlators for the $\psi$ field are the same for DD boundary conditions. We had $\braket{\dot X^\mu(z) \dot X^\nu(w)} = -\tfrac{\ell_s^2}{2} \eta^{\mu \nu} (z-w)^{-2}$ while the correlators for $X'$ pick up a minus sign $\braket{{X'}^i (z) {X'}^j(w)} = \tfrac{\ell_s^2}{2} \eta^{ij} (z-w)^{-2}$. This ends up giving the exact same result however, since the vertex operators contain $X'(z)$ while the prior ones contain $i \dot X(z)$. 
	
	Finally, contracting ${X'}^i$ with any of the $e^{i k X}$ will give zero, since the open strings only have momenta parallel to the D$p$ brane while the ${X'}^i$ is transverse. This gives a simpler amplitude than \eqref{eq:4gluons}:
	\begin{equation}\label{eq:4scalars}
		i g_p^2 \sdelta^{p+1} \ell_s^2 \; K_4'  \left( \frac{\Gamma(-\ell_s^2 s) \Gamma( -\ell_s^2 u)}{\Gamma(1 + \ell_s^2 t)} ([1234] + [4321]) + 2 \perms\right)
	\end{equation}
	with $K'_4 = - (t u  \delta_{12} \delta_{34} + s u \delta_{13} \delta_{24} + s t \delta_{14} \delta_{23})$.
	In the case where there are no CP indices we expand the $\Gamma \Gamma/\Gamma$ functions:
	\[
		i g_p^2 \sdelta^{p+1} \ell_s^2 K'_4 \times \left(\frac{2}{\ell_s^4 su} + \frac{2}{\ell_s^4 st} + \frac{2}{\ell_s^4 tu}\right) = i g_p^2 \sdelta^{p+1} \ell_s^2 K'_4 \times \left( \frac{2(s+t+u)}{\ell_s^4 stu}\right)  = 0
	\]
	So to leading order in the string length this is zero. This is consistent with the $U(1)$ DBI action, as the scalars do not directly interact with the $U(1)$ gauge field $A_\mu$ (in general a real scalar cannot be charged under a $U(1)$ gauge field). That is, at leading order the action is free in the $X$ fields. Taking $\xi^\mu = X^\mu$ for $\mu = 0 \dots p$ and $X^i$ independent functions, we get: 
	\[
		 \int d^{p+1} \xi \sqrt{\det{G_{MN} \d_\alpha X^M \d_\beta X^N}} = \int d^{p+1} \xi \sqrt{\det \delta_{\alpha \beta} + \delta_{ij} \d_\alpha X^i \d_\beta X^j} \to \int d^{p+1} \xi\, \frac{\delta^{\alpha \beta}\delta_{ij}}{2} \d_\alpha X^i \d_\beta X^j 
	\]
	This is just a free theory. Its also quick to see that the 3-point function of the transverse scalars vanishes at tree level in string perturbation theory. 

	
	\item I have done the previous problem in full generality, including CP indices. So now let's again look at the $s$ channel. As $s \to 0$ so that $t = -u$ we get from the $\delta_{12} \delta_{34}$ term a pole in $s$ going as:
	\[
		- i g_p^2 \sdelta^{p+1} \ell_s^2 tu \times \left(\frac{1}{\ell_s^4 su} ([1234] + [4321]) + \frac{1}{\ell_s^4 st} ([1243] + [3421])\right) = -i \sdelta^{p+1} \frac{g_p^2}{\ell_s^2} \frac{t}{s} ([1234] + [4321] - [1243] - [3421])
	\]
	We can rewrite this as:
	\[
		-i \sdelta^{p+1} \frac{g_p^2}{\ell_s^2} \frac{t}{s} (\Tr(12[34]) - \Tr([34]21) = -i \sdelta^{p+1} \frac{g_p^2}{\ell_s^2} \frac{t}{s} \Tr ([12] [34])
	\]
	The pole at $s = 0$ corresponds to an exchange of a gluon from the $\frac12 (D_\mu X^I)^2$ term in \textbf{8.6.1}.
	
	We also have a further term that does not involve a pole in $s$. Let's still take $1$ and $2$ equal. Expanding to this order we find:
	\[
		-i \sdelta^{p+1} \frac{g_p^2}{\ell_s^2} \Big( \Tr([12][34])  + \Tr([13][24]) + \Tr([14][23]) \Big) 
	\]
	This comes from exactly the potential term $\frac14 [X^I, X^J]^2$ in the effective action \textbf{8.6.1}. \textbf{Come back to that last term}
	
	\item Momentum conservation will imply $p_{\parallel} = 0$ for the NSNS states. Our vertex operator will take the form $\zeta_{\mu \nu} c \tilde c e^{-\phi} e^{- \tilde \phi} \psi^\mu \tilde \psi^\nu e^{i k_{\perp} X}$. We can use the doubling trick to get $\zeta_{\mu \nu} c(z) c(z^*) e^{-\phi(z)} e^{-\phi(z^*)} \psi^\mu \tilde \psi^\nu e^{i k_{\perp} X}$ and we are automatically in the $-2$ picture. 
	
	The states from in the $p+1$ parallel directions give just the correlator $\braket{\psi^\mu(i) \psi^\nu(-i)} = -\frac{\eta^{\mu \nu}}{2i}$ (importantly NN fermions in NSNS have 2-point function $-1/(z-\bar w)$ c.f. \textbf{4.16.22}). We also get a $\sdelta^{p+1}$ from momentum conservation.
	
	The states in the transverse (Dirichlet) directions give $\frac{\delta^{ij}}{2i}$ correlator. Defining the diagonal matrix $D^{\mu \nu} = (\eta^{\alpha \beta}, \delta^{ij})$ we get a correlator proportional to 
	\[
		-\frac{g_c}{2 \ell_s^2 g_p^2}\sdelta^{p+1}(k_{\parallel})  D^{\mu \nu}  = - \frac{(2 \pi \ell_s)^2 T_p}{2} V_{p+1} D^{\mu \nu}
	\]
	\textbf{Check with Victor}. Confirm the tension relation. This diagonal tensor $D^{\mu \nu}$ allows for a nonvanishing dilaton and graviton tadpole, but will not couple to the antisymmetric Kalb-Ramond $B$-field. 
	
	\item Our RR fields have picture $(r, s)$ for $r,s$ half-integers In order to have total picture $-2$ on the disk, we need to pick this to be the (asymmetric) $(-3/2, -1/2)$ picture. The construction of this operator is complicated. I expect that the $(-1/2,-1/2)$ operator that I am familiar with is basically $e^{- \phi} G_0$ times the $(-3/2,-1/2)$ operator. This means that the $(-3/2, -1/2)$ will be one less power of momentum and one less gamma matrix than the $(-1/2, -1/2)$ operator. Let's set $k = 0$. The $(-1/2, -1/2)$ operator is propotional to (in Blumenhagen's convention)
	\[
		\frac14 \frac{\ell_s}{\sqrt 2} F^{\alpha \beta}  S_{\alpha}(z) \tilde S_\beta(\bar z) e^{-\phi/2 - \bar \phi/2} = \frac14 \frac{\ell_s}{\sqrt 2}
		% \frac14 \sqrt{\frac{\ell_s^2/2}{p!}}
		\frac{F_{\mu_1 \dots \mu_{p+2}}}{(p+2)!} (\Gamma^{\mu_1 \dots \mu_{p+2}})^{\alpha \beta} \overline{S}_{\alpha} \tilde S_{\beta} e^{-\phi/2 - \bar \phi/2}
	\]
	BRST will require that $F$ and $\star F$ be closed. 
	Changing picture means removing one power of momentum and one gamma matrix. This integrating $F$, which must be proportional to the field strength $C$ since $p_{[\mu_1} C_{\mu_2 \dots \mu_{p+2}]} = F_{\mu_1 \dots \mu_p+2}$. It is then reasonable to expect the corresponding $(-3/2, -1/2)$ operator to be proportional to
	\[
		 e^{-3\phi/2 - \bar \phi/2} \frac{C_{\mu_1 \dots \mu_{p}}}{(p+1)!} (\Gamma^{\mu_0 \dots \mu_{p}})^{\alpha \beta} \overline{S}_\alpha \tilde S_\beta
	\]
	Note both $e^{-3\phi/3} S_{\alpha}$ and $e^{-\tilde \phi/2} S_{\beta}$ \emph{remain} primary operators, having dimensions $3/8 + 5/8$, so this is indeed a reasonable guess. From \textbf{5.12.42} of Kiritsis  I expect the leading-order of the $S_\alpha \tilde S_\beta$ correlator to be $C_{\alpha \beta}/(z - \bar z)^{10/8}$ and the $e^{-3\phi/2} \tilde e^{-\phi/2}$ will contribute $(z-\bar z)^{-3/4}$ to make this a primary correlator transforming as $C_{\alpha \beta} (z - \bar z)^{-2}$. For Neumann boundary conditions, $C_{\alpha \beta}$ is the charge conjugation matrix. More generally, I believe this will be $\delta^\perp C$, since $\delta^\perp = \prod_{i=p+1}^9 \delta^i, \delta^i = \Gamma^i \Gamma_{11}$ will reflect the $S_\alpha$ spinor along all the Dirichlet directions. 
	
	Only in the IIB case will $C_{\alpha \beta}$ will be nonzero between $\overline{S}_\alpha$ and $\tilde S_\beta$ since $S_\alpha$ and $\tilde S_\beta$ transform in the same representations. Each $\beta^i$ changes the chirality. So the amplitude in IIB will vanish if we have an even number of $\beta^i$, equivalently $9-p$ is odd, so we will have only odd dimensional branes in IIB as required and even dimensional branes in IIA as required. 
	
	We thus get an amplitude proportional to:
	\[
		\mathcal A = i \frac{g_c \sdelta^{p+1}}{g_p^2 \ell_s^2} \frac{C_{\mu_0 \dots \mu_p}}{(p+1)!} \Tr(\Gamma^{\mu_1 \dots \mu_p} \Gamma^{p+1} \dots \Gamma^9 \Gamma^{11}) = i \frac{g_c \sdelta^{p+1}}{g_o^2 \ell_s^2} \frac{C_{\mu_0 \dots \mu_p} \epsilon_{(p+1)}^{\mu_0 \dots \mu_p}}{(p+1)!} 
	\]
	Comparing with the \textbf{8.4.4}, which should factorize as $\mathcal A(p_\parallel)^2 G_{9-p}(p_\perp) \delta^{p+1}(p_\parallel)$ we see that the normalization of our on-shell amplitude is in fact:
	\[
		\mathcal A = i V_{p+1} \, \sqrt{2 \pi} (2\pi \ell_s)^{3-p} \frac{C_{\mu_0 \dots \mu_p} \epsilon_{(p+1)}^{\mu_0 \dots \mu_p}}{(p+1)!} 
	\]
	This is consistent with other results c.f. Di Veccia, Liccardo \emph{Gauge Theories from D-Branes}, arXiv:0307104 but I think they're not incorporating $1/\alpha_p = 2 \kappa_{10}^2$ in the propagator. Taking this factor into account and dividing by it followed by taking a square root gives us an on-shell amplitude of:
	\[
		\mathcal A = i V_{p+1} \frac{1}{(2 \pi \ell_s)^p \ell_s g_s} \frac{C_{\mu_0 \dots \mu_p} \epsilon_{(p+1)}^{\mu_0 \dots \mu_p}}{(p+1)!}  = i V_{p+1} T_p \, C_{p+1} \wedge \epsilon_{(p+1)}.
	\]
	This is exactly what would come from a minimal coupling term of the form $i T_p \int C_{p+1}$.
	
	\item We take one vertex operator to be in the $(-1, -1)$ picture and gauge fix it to lie at $z=i$, and take the other in the $(0, 0)$ picture, and fix it to range along the line from $0$ to $i$. We wish to calculate the correlator:
	\[
		-\frac{g_c^2}{g_o^2 \ell_s^4} \frac{2}{\ell_s^2} \braket{[\psi^\mu \tilde \psi^{\bar \mu} e^{i k_1 X}]_0\; [(i \d X^\mu + \frac12 k_2 \cdot \psi \psi^\nu) (i \bar \d X^{\bar \nu} + \frac12 k_2 \cdot \bar \psi \bar \psi^{\bar \nu}) e^{i k_2 X}]_y }
	\]
	This will be very similar to the 4-point gauge boson amplitude in \eqref{eq:gauge}. \textbf{Finish.}
	
	
	From this ratio $\Gamma \Gamma/\Gamma$ functions we see that there are open string poles. This corresponds to a closed string splitting in two, with its ends on the D-brane as an intermediate state.
	
	\item In the D9 brane case, we have seen that the open string boundary only preserves the sum of $Q + \tilde Q$. If we T-dualize in the 9 direction, we act on the right-moving sector by spacetime parity, so that necessarily $\bar \d X^9 \to - \bar \d X^9, \tilde \psi^9 \to - \tilde \psi^9, \tilde S_\alpha \to \delta^9 \tilde S^\alpha$ (up to a phase in that last one). Here $\delta^9 = \Gamma^9 \Gamma^{11}$. Our spacetime supersymmetry generator $\tilde Q_\alpha = \frac{1}{2\pi i} \int d\bar z\, e^{-\phi/2} S_{\alpha}$ therefore will be mapped to $\delta^9 \tilde Q$. Thus, in the T-dual picture we preserve the supercharge $Q' + \delta^9 \tilde Q'$.
	
	Iterating this procedure in other directions we get that in general we preserve $Q + \delta^\perp \tilde Q$, with $\delta^\perp = \prod_{i} \delta^i$, where $i$ runs perpendicular to the brane. Note that T-dualities along different directions do not commute! They commute up to a $(-1)^{\mathbf{F}_R}$, and so the order that we do them matters. In this case the $\delta^i$ act by left-action.
	
	\item (As in Polchinski section 13.4) From the previous problem, we see that the first D-brane preserves the supercharges $Q + \delta^\perp \tilde Q$ while the second preserves the supercharges $Q + {\delta^\perp}' \tilde Q = Q + \delta^\perp ({\delta^\perp}^{-1} {\delta^\perp}' \tilde Q)$ so the supersymmetries that will be preserved must be of both forms. This is in one-to-one correspondence with spinors invariant under ${\delta^\perp}^{-1} {\delta^\perp}'$. This operator is a reflection in the direction of the ND boundary conditions (the directions orthogonal to the D$_{p'}$ brane in the D$_p$ brane). Since in either IIA or IIB $p$ and $p'$ must differ by an even integer, the number of mixed boundary conditions--call it $\nu$--must be even. Then we can write ${\delta^\perp}^{-1} {\delta^\perp}'$ as a product of rotations by $\pi$ along each of the $\nu/2$ planes ${\delta^\perp}^{-1} {\delta^\perp}' = e^{i \pi (J_1 + \dots + J_{\nu/2})}$. Each $j$ acts in a spinor representation, so that $e^{i \pi J_i}$ has eigenvalues $\pm i$. If $\nu/2$ is odd, this makes ${\delta^\perp}^{-1} {\delta^\perp}' = -1$ so this will \emph{not} preserve supersymmetry. We thus need $\nu/2$ even, or $\nu = 0$ mod $4$.
	
	
	From this I posit that the static force between two branes vanishes precisely when $\nu = 0 \text{ mod } 4$.
	
	\item Now let's confirm this guess with an amplitude calculation. Take $p' \leq p$. We work in lightcone gauge. We do a trace over an open string with $p'$ NN boundary conditions, $p-p' = \nu$ DN boundary conditions, and $8-p$ DD boundary conditions. 
	
	We begin from the open string point of view in calculating the cylinder amplitude. Chapter 4 has done the NN, DD, and DN boson amplitudes for us. The difficulty lies almost entirely in the fermions.
	Recall the following:
	\[
	\begin{aligned}
		\eta &= q^{1/24} \prod_{n=1}^\infty (1-q^n)\\
		 \sqrt{\frac{\theta[{^0_0}]}{\eta}} = q^{-1/48} \prod_{n=0}^\infty (1 + q^{n+1/2}) \qquad 
		 \sqrt{\frac{\theta[{^1_0}]}{\eta}} &= \sqrt 2 \, q^{1/24} \prod_{n=0}^\infty (1 + q^n) \qquad  \sqrt{\frac{\theta[{^0_1}]}{\eta}} &= q^{-1/48} \prod_{n=0}^\infty (1 - q^{n+1/2})
	\end{aligned}
	\]
	Further from \textbf{4.16.2} recall that for modes $b_{n+1/2}$, $b_n$ corresponding to NS and R sectors  the NN and DD boundary conditions give:
	\begin{itemize}
		\item NN: $\bar b_{n+1/2} = - b_{n+1/2}, \bar b_n = b_n$
		\item DD: $\bar b_{n+1/2} = b_{n+1/2}, \bar b_n = -b_n$
	\end{itemize}
	For DN we have the same result as for DD but now the R sector is half-integrally modded and the NS sector in integrally modded. Now lets compute partition functions. Our final answer will be a sum over spin structures NS+, NS-, R+, R-. Taking $q = e^{- 2 \pi t}$ we see $\Tr[q^{L_0 - c/24}] = $
	\begin{itemize}
		\item NS+: 
		\begin{itemize}
			\item NN: $q^{-1/48}\prod_{n} (1 + q^{n+1/2}) =  \sqrt{\theta[{^0_0}]/\eta}$
			\item DD: $q^{-1/48}\prod_{n} (1 + q^{n+1/2}) = \sqrt{\theta[{^0_0}]/\eta}$
			\item DN: $\sqrt{2} q^{-1/48} q^{1/16} \prod_{n} (1 + q^{n}) = \sqrt{\theta[{^1_0}]/\eta}$ ( $\sqrt 2$ when raised to a power counts ground state degeneracy)
		\end{itemize}
		% The NS sector gives $\frac12 (\theta^4 \twist 00  - \theta^4 \twist 01)/\eta^4$
%
% 		The R sector gives $-\frac12 \theta\twist 10 ^4/\eta^4$
		\item NS-: 
		\begin{itemize}
			\item NN: $q^{-1/48}\prod_{n} (1 - q^{n+1/2}) =  \sqrt{\theta[{^0_1}]/\eta}$
			\item DD: $q^{-1/48}\prod_{n} (1 - q^{n+1/2}) = \sqrt{\theta[{^0_1}]/\eta}$
			\item DN: $0$
		\end{itemize}
		
		\item R+
		\begin{itemize}
			\item NN: $\sqrt 2 q^{1/24}\prod_{n} (1 + q^n) =  \sqrt{\theta[{^1_0}]/\eta}$
			\item DD: $\sqrt 2 q^{1/24}\prod_{n} (1 + q^n) =  \sqrt{\theta[{^1_0}]/\eta}$
			\item DN: $q^{-1/48}  \prod_{n} (1 + q^{n+1/2}) = \sqrt{\theta[{^0_0}]/\eta}$ 
		\end{itemize}
		
		\item R-
		\begin{itemize}
			\item NN: 0
			\item DD: 0
			\item DN: $q^{-1/48} \prod_{n} (1 - q^{n+1/2}) = \sqrt{\theta[{^0_1}]/\eta}$ 
		\end{itemize}
	\end{itemize}
	
	\textbf{Notice} NN vs DD boundary conditions have \emph{no effect} on fermion contribution to partition function. This is because, although the left moving and right-moving modes are identified differently, the mode excitations look exactly the same.
	
	On the other hand for NN and DD the bosons will contribute $1/\eta$ and will contribute $\sqrt{\eta/\theta[{^0_1}]}$ for DN. Thus we have the following contributions to the partition function (here $N$ is the number of NN boundary conditions):
	\[
		\begin{aligned}
		NS+ &= \frac{V_N}{(2 \pi \ell_s)^{N}} \int \frac{dt}{2t} \frac{e^{-2 \pi t \left(\frac{\Delta x}{2 \pi \ell_s} \right)^2}}{(\sqrt{2 t})^{N} \eta^{8-\nu} \left(\theta[{^0_1}]/\eta\right)^{\nu/2}} \left(\frac{\theta\twist00}{\eta} \right)^{(8-\nu)/2} \left(\frac{\theta\twist10}{\eta} \right)^{\nu/2} 
		\\
		NS- &= -\frac{V_N}{(2 \pi \ell_s)^{N}} \int \frac{dt}{2t} \frac{e^{-2 \pi t \left(\frac{\Delta x}{2 \pi \ell_s} \right)^2}}{(\sqrt{2 t})^{N} \eta^{8}} 
		\left(\frac{\theta\twist01}{\eta} \right)^{8} \delta_{\nu=0}
		\end{aligned}
		\]
	\[
	\begin{aligned}
	R+ &= -\frac{V_N}{(2 \pi \ell_s)^{N}} \int \frac{dt}{2t} \frac{e^{-2 \pi t \left(\frac{\Delta x}{2 \pi \ell_s} \right)^2}}{(\sqrt{2 t})^{N} \eta^{8-\nu} \left(\theta[{^0_1}]/\eta\right)^{\nu/2}} 
	\left(\frac{\theta\twist10}{\eta} \right)^{(8-\nu)/2} \left(\frac{\theta\twist00}{\eta} \right)^{\nu/2} 
	\\
	R- &= \frac{V_N}{(2 \pi \ell_s)^{N}} \int \frac{dt}{2t} \frac{e^{-2 \pi t \left(\frac{\Delta x}{2 \pi \ell_s} \right)^2}}{(\sqrt{2 t})^{N} } \delta_{\nu=8} 	
	\end{aligned}
	\]
	All theta and eta functions are evaluated at $it$. The circumference of the cylinder is $2 \pi t$. The relative signs in front of the different contributions come from a combination of defining the NS vacuum to have negative fermion and modular invariance (equivalently spacetime spin-statistics). 
	% Summing this gives
% 	\[
% 		\frac{V_N}{(2 \pi \ell_s)^{N}} \int \frac{dt}{2t} \frac{e^{-t \left(\frac{\Delta x}{2 \pi \ell_s} \right)^2}}{(\sqrt{2 t})^{N} \eta^{8-\nu} \left(\theta[{^0_1}]/\eta\right)^{\nu/2}} \left[ \left(\frac{\theta\twist00}{\eta} \right)^{(8-\nu)/2} \left(\frac{\theta\twist10}{\eta} \right)^{\nu/2} - \left(\frac{\theta\twist01}{\eta} \right)^{8} \delta_{\nu=0} -
%
% 		 \right]
% 	\]
	Note when $\nu = 4$ we only get contributions from NS+ and R+, which exactly cancel. Similarly when $\nu = 4$ or $8$, by the abtruse identity of Jacobi we will get cancelation again. 
	
	We can interpret our result as a one-loop free energy. Differentiating this w.r.t. $\Delta x$ would then give us our force. For $\nu = 0, 4, 8$ we do not get a force, consistent with the D-brane configuration preserving supersymmetry. 
	
	For the sake of completeness, and to clear my own confusion once and for all, I will also do this from the POV of the boundary state formalism (not developed in Kiritsis). For a good reference see the last chapter of Blumenhagen's text on conformal field theory. 
	
	For a single free boson, after the flip $(\sigma, \tau)_{open} \to (\tau, \sigma)_{closed}$ the boundary states $\ket{N}, \ket{D}$ must satisfy 
	\[
		( \alpha_n + \tilde \alpha_{-n}) \ket N = 0, \qquad ( \alpha_n - \tilde \alpha_{-n}) \ket{D_x} = 0,
	\]
	This gives boundary states:
	\[
		\begin{aligned}
			\ket{N} &= \frac{1}{(2\pi \ell_s \sqrt{2})^{1/2}} \prod_n e^{- \frac1n \alpha_{-n} \tilde \alpha_{-n}} \ket{0,0; 0} = \sum_{\vec m = \{m_i\}} \ket{\vec m, \Theta \vec m; 0} \\
			\ket{D_x} &= (2\pi \ell_s/\sqrt{2})^{1/2} \int \frac{dk}{2\pi} e^{i p x} \prod_n e^{- \frac1n \alpha_{-n} \tilde \alpha_{-n}} \ket{0,0; k}\\			
		\end{aligned}
	\]
	The overall normalization came from comparing with cylinder amplitudes. $\Theta$ here is CPT reversal. Similarly for a fermion
	\[
		( \psi_n + \tilde \psi_{-n}) \ket N = 0, \qquad ( \psi_n - \tilde \psi_{-n}) \ket{D_x} = 0,
	\]
	So with GSO projection we get:
	\[
	\begin{aligned}
		\ket{N, \text{NSNS}} = P_L P_R \prod_r e^{\psi_{-r} \tilde \psi_{-r} } \ket{0}, &\qquad \ket{N, \text{RR}} = P_L P_R \prod_n e^{\psi_{-n} \tilde \psi_{-n} } \ket{0}\\
		\ket{D, \text{NSNS}} = P_L P_R \prod_r e^{-\psi_{-r} \tilde \psi_{-r} } \ket{0}, &\qquad \ket{D, \text{RR}} = P_L P_R \prod_n e^{-\psi_{-n} \tilde \psi_{-n} } \ket{0}
	\end{aligned}
	\]
	Here $r$ runs over half-integers in the NSNS sector and $n$ runs over integers in the RR sector. $P_L = \frac12( 1 + (-1)^F), P_R = \frac12 (1 + (-1)^{\tilde F})$ are our GSO projections, defined to project out the tachyon in the NS sector and project out one of the spinors in the R sector. 
	
	For the boson, it is quick to see that ($\ell = 1/t$)
	\[
	\begin{aligned}
		\bra{N} e^{-\pi \ell (L_0 + \tilde L_0 - c/12)} \ket{N} &= \frac{V}{2\pi \ell_s \sqrt 2 \eta(i \ell)} = \frac{V}{(2\pi \ell_s) \sqrt{2 t} \eta(i t)} \\
		\bra{D} e^{-\pi \ell (L_0 + \tilde L_0 - c/12)} \ket{D} &= \frac{2 \pi \ell_s}{\sqrt{2 t} \eta(i t)} \int \frac{dk}{2\pi} e^{i k \Delta x} e^{-\pi \ell_s^2 p^2/2t} = \frac{e^{-2 \pi t \left( \frac{\Delta x}{2 \pi \ell_s} \right)^2 }}{\eta(i t)}\\
		\bra{D} e^{-\pi \ell (L_0 + \tilde L_0 - c/12)} \ket{N} &= \frac{1}{\sqrt 2} \frac{1}{\prod_n (1+q^{2n})} = \sqrt{\frac{\eta(i \ell )}{\theta\twist10 (i \ell)}} = \sqrt{\frac{\eta(i t)}{\theta\twist01 (i t)}}
	\end{aligned}
	\]
	These are exactly what we've already gotten many times before from our trace over the open string bosonic states. The states $\ket{N}, \ket{D}$ must be a sum of both the RR and NSNS sector fermion states. We do not know the relative coefficients. 
	
	Let's look at the NSNS contributions. For the NN boundary conditions, the NSNS sector with projection consists of two terms: 
	\[
		\begin{aligned}
			\bra{N, \text{NSNS}_{unproj}} e^{-\pi \ell (L_0 + \tilde L_0 - c/12)} \ket{N, \text{NSNS}_{unproj}}  &= \left(\frac{\theta\twist00(i \ell)}{\eta(i \ell)}\right)^{\# NN/2}  = \left(\frac{\theta\twist00(i t)}{\eta(i t)}\right)^{\# NN/2}\\
			\bra{N, \text{NSNS}_{unproj}} (-1)^{F_L = F_R} e^{-\pi \ell (L_0 + \tilde L_0 - c/12)} \ket{N, \text{NSNS}_{unproj}}  &= \left(\frac{\theta\twist01(i \ell)}{\eta(i \ell)}\right)^{\# NN / 2} = \left(\frac{\theta\twist10(i t)}{\eta(i t)}\right)^{\# NN / 2}
		\end{aligned}
	\]
	Replacing $N$ with $D$ would give the \emph{exact same} factor in both cases \textbf{WHY?} (explain: bc we need to match on both sides and so both minuses cancel in the exponent).  For DN boundary conditions the NSNS sector give the two terms:
	\[
		\begin{aligned}
			\bra{D, \text{NSNS}_{unproj}} e^{-\pi \ell (L_0 + \tilde L_0 - c/12)} \ket{N, \text{NSNS}_{unproj}}  &= \left(\frac{\theta\twist01(i \ell)}{\eta(i \ell)}\right)^{\nu/2} = \left(\frac{\theta\twist10(it)}{\eta(i t)}\right)^{\nu/2}\\
			\bra{D, \text{NSNS}_{unproj}} (-1)^{F_L = F_R} e^{-\pi \ell (L_0 + \tilde L_0 - c/12)} \ket{N, \text{NSNS}_{unproj}}  &= \left(\frac{\theta\twist00(i \ell)}{\eta(i \ell)}\right)^{\nu/2}  = \left(\frac{\theta\twist00(i t)}{\eta(i t)}\right)^{\nu/2}
		\end{aligned}
	\]
	
	Now let's look at the RR sector. For NN boundary conditions, it contributes:
	\[
		\begin{aligned}
			\bra{N, \text{RR}_{unproj}} e^{-\pi \ell (L_0 + \tilde L_0 - c/12)} \ket{N, \text{RR}_{unproj}}  &= \left(\frac{\theta\twist10(i \ell)}{\eta(i \ell)}\right)^{\# NN/2} = \left(\frac{\theta\twist01(it)}{\eta(i t)}\right)^{\# NN/2} \\ 
			\bra{N, \text{RR}_{unproj}} (-1)^{F_L = F_R} e^{-\pi \ell (L_0 + \tilde L_0 - c/12)} \ket{N, \text{RR}_{unproj}}  &= 0
		\end{aligned}
	\]
	By the argument before, we get the same for DD boundary conditions. Finally, with DN boundary conditions we get
	\[
		\begin{aligned}
			\bra{D, \text{RR}_{unproj}} e^{-\pi \ell (L_0 + \tilde L_0 - c/12)} \ket{N, \text{RR}_{unproj}}  &=  0\\
			\bra{D, \text{RR}_{unproj}} (-1)^{F_L = F_R} e^{-\pi \ell (L_0 + \tilde L_0 - c/12)} \ket{N, \text{RR}_{unproj}}  &= \left(\frac{\theta\twist10(i \ell)}{\eta(i \ell)}\right)^{\nu/2} = \left(\frac{\theta\twist01(i t)}{\eta(i t)}\right)^{\nu/2}
		\end{aligned}
	\]
	Together this is exactly consistent with what we get from tracing over the open string. We can work back to get relative normalizations.
	
	This shows that the massless RR and NSNS fields mediate the force. Moreover the NSNS fields without and with projection correspond respectively to the unprojected NS and R open string states while the RR fields without and with projection correspond to the \emph{projected} NS and R open string states.
	
	\item First recall that for a constant vector potential $A_9 = \frac{\chi_9}{2\pi R}$ corresponds to a $T$-dual picture of a $D$-brane at position $- \chi \tilde R = -2 \pi \ell_s^2 A_9$. Now consider a magnetic flux $F_{12}$ we can write a (nonconstant now) vector potential that gives this flux as $A_2 = F_{12} X^1$. We $T$-dualize along $X^2$ and get $X^2 = -2 \pi \ell_s^2 F_{12} X^1$. Then $\tan \theta = -2 \pi \ell_s^2 F_{12}$.
	
	Although we were working with D1 and D2 branes, we could have done the exact same calculation for $F_{01}$ on a D1 brane and recovered a D0 brane tilted in the $X^0-X^1$ plane (ie boosted). Such a D0 brane has the usual point-particle action:
	\[
		S_{D0} = -T_0 \int dX^0 \sqrt{1 + (\d_0 {X'}^1)^2} 
	\]
	Because the D0 brane and the D1 brane describe the same physics, this action should be identical to the D1 action. Note that $\d_0 {X'}^1$ is infinitesimally exactly $\tan \theta$ calculated above. We get the action
	\[
		S_{D1} = -T_1 \int dX^1 dX^2 \sqrt{1 + (2\pi \ell_s^2 F_{12})^2} 
	\]
	Of course, because the branes couple to strings, the only gauge invariant combination under transformations of the Kalb-Ramond $B$ field is $\mathcal F = B + 2\pi \ell_s^2 F$. We thus get:
	\[
		S_{D1} = -T_1 \int dX^1 dX^2 \sqrt{- \det (G + \mathcal F)} 
	\]
	We can tilt this brane and $T$-dualize to pick up EM field strengths in arbitrary dimension up to $9$.
	
	\item Let's $T$-dualize. See the solution \eqref{eq:partitionfunc} to the next problem and set $\phi_2 = \phi_3 = \phi_4 = 0$. This describes two D4 branes that are tilted only along the $x_1$-$x_5$ plane, and are otherwise parallel in the $x_2, x_3, x_4$ directions. T-dualizing $x_2, x_3, x_4$ makes these into D1 branes tilted in the $x_1 - x_5$ plane. Now setting $\nu_{2,3,4} = 0$ will give poles from the theta function in the denominator. This is to be expected, from the NN boundary conditions that always come with a volume divergence factor in that direction. We regulate this divergence by replacing:
	\[
		\theta\twist11 (i \nu t, i t)^{-1} \to i \frac{L}{\eta(i t)^{-3} 2 \pi \ell_s \sqrt{2 t}}
	\]
	Thus we get a partition function:
	\[
		-i \frac{L}{2 \pi \ell_s^4}\int_0^\infty \frac{dt}{2t} \frac{e^{-2 \pi t \left( \frac{\Delta x}{2\pi \ell_s} \right)^2}}{\sqrt{2t} \eta(it)^9} \frac{\theta \twist11 (i \theta t/2, i t)^3}{\theta \twist11 (i \theta t, it)}
	\]
	\textbf{So what's the force?} Let's take the distance to be large. The small $t$ contributions are then most important and we see a potential of: \textbf{Do it}. We get a potential that decays as $1/y^5$.
	
	\item Following Polchinski, we define variables $Z^i = X^i + i X^{i+4}, i= 1, \dots, 4$. Let the $\sigma = 0$ endpoint be on the untilted string. Then at $\sigma = 0$ we have $\d_1 \Re Z^a = \Im Z^a = 0$ and at $\sigma = \pi$ on the tilted string we have $\d_1 \Re(e^{i \theta_a} Z^a) = \Im(e^{i \theta_a} Z^a)$. 
	
	We see that the field that satisfies this is given by $Z^a(w, \bar w) = \mathcal Z^a(w) + \bar {\mathcal Z}^a (-\bar w)$ for $w = \sigma^1 + i \sigma^2$. Using the doubling trick we have $\mathcal Z^a(w + 2 \pi) = e^{2 i \theta_a} \mathcal Z^a(w)$ (and similarly for the conjugate). This gives a mode expansion with $\nu_a = \theta_a / \pi$
	\[
		\mathcal Z^a = i \frac{\ell_s}{\sqrt 2} \sum_{r \in \mathbb Z + \nu_a} \frac{\alpha^a_r}{r} e^{i r w}
	\]
	Then taking $q = e^{-2\pi t}$ as usual for open string partition functions, and restricting $0 < \phi_a < \pi$ we get:
	\[
		\frac{q^{\frac1{24} - \frac12 (\nu_a - \frac12)^2}}{\prod_{m=0}^\infty (1 - q^{m + \nu_a}) (1 - q^{m + 1 - \nu_a})} = -i \frac{q^{-\nu_a^2/2} \eta(it)}{\theta\twist11 (i \nu_a t, it)}
	\]
	Interesting... so the angles act like chemical potentials to make the theta functions nonzero. Now its time for the fermions (oh boy!). In each NS and R sector (projected and unprojected) the boundary conditions shift by $\nu_a$. We thus get e.g. for NS unprojected:
	\[
		Z\twist00 = q^{-\frac1{24} + \nu_a^2/2} \prod_{m=1}^\infty (1-q^{m+1/2 + \nu_a}) (1-q^{m + 1/2 - \nu_a}) = q^{\nu_a^2/2} \frac{\theta\twist00 (i \nu_a t , it)}{\eta(i t)}
	\]
	In total, then we will see that the fermion part gives us
	\[
		\frac12 \left[\prod_{a=1}^4 Z\twist00 (\phi_a, it) - \prod_{a=1}^4 Z\twist10 (\phi_a, it) - \prod_{a=1}^4 Z\twist01 (\phi_a, it) - \prod_{a=1}^4 Z\twist11 (\phi_a, it) \right]= \prod_{a=1}^4 Z\twist11 (\phi_a', it)
	\]
	This last equality follows from the full abstruse identity of Jacobi, where $\phi_1' = \frac12 (\phi_1 + \phi_2 + \phi_3 + \phi_4)$, $\phi_2' = \frac12 (\phi_1 + \phi_2 - \phi_3 - \phi_4)$, $\phi_3' = \frac12 (\phi_1 - \phi_2 + \phi_3 - \phi_4)$, $\phi_4' = \frac12 (\phi_1 - \phi_2 - \phi_3 + \phi_4)$ and exactly the same relations that 
	We thus get a full potential  as a function of the separation $\Delta_x$
	\begin{equation}\label{eq:partitionfunc}
		V = - \int_0^\infty \frac{dt}{2t} \frac{e^{-2 \pi t \left( \frac{\Delta x}{2\pi \ell_s} \right)^2}}{2\pi \ell_s \sqrt{2t}} \prod_{a=1}^4 \frac{\theta \twist11 (i \nu'_a t, i t)}{\theta \twist11 (i \nu_a t, i t)}
	\end{equation}
	At long enough distances, the exponential factor forces small $t$ to contribute primarily. The complicated ratio of $\theta$-functions becomes a ratio of sines. 

	\item \textbf{Find boundary conditions at $0$ and $\pi$}.
	
	\item We have a vector potential $A_1 = E X^0$. This is a simple as $T$-dualizing in the direction of $X^1$, giving a D-brane lying at $X_1 = 2 \pi \ell_s^2 E X^0$ giving a velocity $2 \pi \ell_s^2 E$. This can also be obtained by analytic continuation of question \textbf{8.13}. 
	
	\item 
	
	\item The action is a factor of two off from Polchinski's. The momenta are $p_i = \frac{2}{g_s \ell_s} (\dot X^i + [A_t, X^i])$. Then we get:
	\[
		H = \int dt \Tr\Big[\frac{g_s \ell_s}{4} p_i p^i + \frac{1}{2 g_s \ell_s (2\pi \ell_s^2)^2} [X^i, X^j]^2\Big]
	\]
	Defining $g^{1/3} \ell_s Y^i = X^i, p_i = p_{Y_i}/g_s^{1/3} \ell_s$, this sets the length scale as desired. Now we have $Y^i$ is dimensionless and get a hamiltonian
	\[
		H = \frac{g_s^{1/3}}{\ell_s} \int dt \Tr\Big[\frac14 p_{Y_i} p_{Y_i} + \frac{1}{2 (2\pi )^2} [Y^i, Y^j]^2\Big]
	\]
	So the hamiltonian is dimensionless, with $g, \ell_s$ appearing only in the overall normalization, which gives an energy scale of $g_s^{1/3}/\ell_s$. For strong coupling $g_s > 1$ this probes deeper than the string scale. 
	
	\item This is pretty direct. Since the metric $G_{\mu \nu}$ does not depend on $X^i$ for $i=p+1 \dots 9$, we have that each $X^i$ is killing, in particular the metric takes a block-diagonal form where only the first $(p+1)^2$ $G_{\mu \nu}$ entries have nontrivial coordinate dependence and the remaining metric is just the identity matrix $\delta_{ij}$ along the $X_i$ directions (we didn't even have to do this since 8.5.1 has $\eta_{\mu \nu}$ the flat metric. Is my logic here even right?). 
	
	Take the ansatz $A \to (A_\mu, \Phi_i)$. We thus get $F_{\mu \nu}$ in the first $(p+1)^2$ entries and $\partial_\mu \Phi^i$ in the off-block-diagonal piece. We can rewrite this as a determinant of just the $(p+1)$ piece \textbf{Justify this step}
	\[
		\sqrt{-\det(G_{\mu \nu} + 2 \pi F_{\mu \nu} + \d_\mu \Phi^i \d_\nu \Phi^i)}
	\]
	
	\item The bosonic part of this is immediate. Write the fields $A_i$ in the dimensionally reduced dimensions as $X^I$ and we immediately get $\Tr F^2_{10} \to \Tr [F_{d+1}^2 + 2 [D_\mu, X^I]^2 + \sum_{I, J} [X^I, X^J]^2]$. The fermionic part will reduce to:
	\[
		(\Tr \bar \chi \Gamma^\mu D_\mu \chi)_{10D} \to \Tr [\bar \chi \Gamma^\mu D_\mu \chi + \bar \lambda_a \Gamma^{i} [X_i , \lambda^b] ] 
	\] 
	Where now the $\chi_i$ are fermions that break the \textbf{16} representation of $\SO(9,1)$ into a representation of $SO(d-1, 1) \times SO(10-d)$. For $d = 3$ we get $\mathcal N = 4$ SYM and this is $(2, 4) + (\bar 2, \bar 4)$, corresponding to four Weyl spinors. 
	\item At the minimum of the potential, all $X^I$ lie in a cartan and mutually commute. The $A_{ij}$ correspond to open strings moving between the D-branes at positions $X_I$. The ground states of these open strings have a mass squared of $(X_I - X_J)^2/2 \pi \ell_s^2$, so indeed the mass is linear in the separation. \textbf{Confirm. Understand Lie-Theoretic perspective}.
	
	\item The worldvolume coupling to the RR 2-form looks like $i T_2 \int G \wedge C_2$

	\item 
	
	\item 
	
\end{enumerate}

% section chapter_8_d_branes (end)	
\end{document}
	