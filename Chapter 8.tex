\documentclass[11pt, class=article, crop=false]{standalone}
\usepackage{amsmath,amssymb,amsfonts,amsthm}
\usepackage{enumitem}
\usepackage{fancyhdr}
\usepackage{tikz-cd}
\usepackage{mathabx}
\usepackage{geometry}
\usepackage{natbib}
\usepackage{braket}
\usepackage{graphicx}
\usepackage{simpler-wick}
\usepackage{hyperref}
\usepackage{ytableau}
\usepackage{cancel}
\usepackage{listings}
\usepackage{relsize}
\usepackage{xcolor}
\usepackage{stmaryrd}
\usepackage{slashed}
\usepackage{tikz-feynman}
\usepackage{kiritsis}
\geometry{margin = 0.5in}


\begin{document}
\section*{Chapter 8: D-Branes} % (fold)
\label{sec:chapter_8_d_branes}

\begin{enumerate}
	\item First, a simple magnetic monopole for a 1-form gauge field in $D$ spacetime dimensions has a radial magnetic field $B_r = \frac{\tilde Q_1}{\Omega_{D-2} r^{D-2}}$ where $\Omega_{D-2} = 2 \pi^{d/2} \Gamma(d/2)$ is the volume of a unit $D-2$ sphere. This way, the flux of the solution over any $D-2$ sphere surrounding the (point) monopole will be $\tilde Q_1$.
	
	Upon taking the Hodge star we get the solution is $F = \tilde Q_1 \sin \theta \dd \theta \wedge \dd \phi$. We can write this as $A = \tilde Q_1 (c - \cos \theta) \dd \phi$. Taking $c = 1$ we get $A$ vanishes at $\theta = 0$ (which we need since the $\phi$ coordinate degnerates there) while taking $c = -1$ we get $A$ vanishes at $\theta = \pi$, which we also need.
	
	We cannot have \emph{both} solutions, and so we realize we are dealing with two $A$s, corresponding to local sections of a line bundle over $S^2$ on different hemispheres. Let $A^+$ be well-defined on all points on $S^2$ except $\theta = \pi$. Then $A^+$ is a section of a line bundle on the punctured sphere. The punctured sphere is contractible so any fiber bundle over it is trivial, so $A^+$ is just a \emph{function} on the punctured sphere $S^2 \setminus \{\theta = \pi\}$. So let's define $A^+ = \tilde Q_1 (1- \cos \theta) \dd \phi$. Similarly, we define $A^-$ to be the nonsingular $A$ on the sphere with $\theta = 0$ removed, namely $A^- = \tilde Q_1 (-1 -\cos \theta) \dd \phi$. 
	
	On the overlap, $A^+ - A^-$ differ by an integer, which labels the degree of ``twisting'' of this line bundle over $S^2$. 
	
	For a $p$ form, our monopole will now be spatially extended in $p-1$ directions. Label these (locally), by $x^1 \dots x^{p-1}$. Time is $x^0$. Locally transverse to these coordinates will be $r, \varphi^1 \dots, \varphi^{D-1-p}$, where $\varphi^i$ parameterize a $D-1-p$ sphere enclosing the monopole. The field strength looks like:
	\[
		% B_r = \frac{\tilde Q_p}{\Omega_{D-1-p} r^{D-1-p}} \dd t \wedge \dd r \Rightarrow
		F = \tilde Q_p \, \Omega_{D-p-1}
	\]
	where $\Omega$ is the canonical $D-p-1$-sphere area form:
	\[
		\Omega = \sin^{D-p-2}(\varphi_1) \sin^{D-p-3}(\varphi_2) \dots \sin(\varphi_{D-p-2})\, \dd \varphi_1 \wedge \dots \dd \varphi_{D-p-1}
	\]
	This can be written (unfortunately unavoidably) in terms of a hypergeometric function:
	\[
		A =  {_2 F_1}\left(\frac12, \frac{D-p-1}{2}, \frac{D-p+1}{2}, \sin^2(\varphi_1) \right) \frac{\sin^{D-p-1}(\varphi_1)}{D-p-1} \dd \varphi_2 \wedge \dots \wedge \dd \varphi_{D-p-1}
	\]
	there is no need for an overall constant, as the function above vanishes at both $\varphi_1 = 0$ and $\pi$, \emph{however} this is compensated by the hypergeometric function having a branch cut at $\varphi_1 = \pi/2$. Across this cut, it will have a discontinuity set by an integer depending on the convention of the arcsin function, and again we will have $A^+ - A^-$ differing by an integer.  The same quantization condition follows. 
	
	Again $A^+$ will be defined on the $S^{D-p-1}$ sphere minus the south-pole (this is homeomorphic to the $D-p-1$ ball, and hence contractible, so again the line bundle trivializes and $A^+$ is a bona-fide function for any $D, p$) and $A^-$ is similarly defined on the sphere with the excision of the north pole.
	
	\item Our simply charged point particle with a Wilson line $A_9 = \chi/2\pi R$ turned on will have an action
	\[
		S = \int d \tau \underbrace{\left( \frac12 \dot x^M \dot x_M - \frac{m^2}{2} + q A_9 \dot x^9 \right)}_{\mathcal L}
	\]
	The canonical momentum will be $p_i = \dot x^\mu$ for $\mu = 0 \dots 8$ and $p_9 = \dot x^9 + \frac{q \chi}{2 \pi R}$. Consequently, our hamiltonian is
	\[
	\begin{aligned}
		H &= p_M \dot x^M - \mathcal L = \frac12 p_\mu p^\mu + p_9 (p_9 -  \frac{q \chi}{2 \pi R}) - \left[\frac12 (p_9 - \frac{q \chi}{2\pi R})^2 - \frac{m^2}{2} + (p_9 - \frac{q \chi}{2 \pi R}) \frac{q \chi}{2 \pi R}\right]\\
		&= \frac12 \left(p_\mu p^\mu + (p_9 - \frac{q \chi}{2\pi R})^2 + m^2\right)\\
		&= \frac12 \left(p_\mu p^\mu + \left(\frac{2 \pi n - q \chi}{2\pi R}\right)^2 + m^2\right)
	\end{aligned}
	\]
	\begin{center}
		\includegraphics[scale=0.1]{"Drawings/Wilson"}
	\end{center}
	
	\item For a string satisfying Dirichlet boundary conditions, the total momentum is not conserved (along the directions associated with the D boundary conditions). This is easily interpretable as momentum transfer to the brane that it is attached to.
	
	\item For an open string of state $\ket{ij}$, $A_9$ will act as $\frac{\chi_i - \chi_j}{2\pi i}$. Since this is an open string with no winding, we can only have momentum contribution, and so we will get a mass formula
	\[
		m^2_{ij} = \frac{\hat N - \frac12}{\ell_s^2} + \left(\frac{n}{R} - \frac{\chi_i - \chi_j}{2\pi R} \right)^2
	\]
	In particular at the lowest (massless) level for a string without momentum we will get the desired spectrum
	\[
		m_{ij}^2 =  \left(\frac{\chi_i - \chi_j}{2\pi R} \right)^2
	\]
	
	\item For completeness, we will do both the gauge boson and scalar scattering. These come from the NS sector, and are given by:
	\[
		V_{-1}^{a, \mu} = g_p \lambda^a \psi^\mu e^{-\phi} c e^{i p X}, \qquad
		V_{0}^{a, \mu} = \frac{g_p}{\sqrt 2 \ell_s} \lambda^a (i \dot X + 2 p \cdot \psi \, \psi^\mu) c e^{i p X}
	\]
	We can explicitly scatter four such gauge bosons - two in the $-1$ picture and two in the $0$ picture.
	\[
	\begin{aligned}
		& \qquad i C_{D^2} \sdelta^{10}(\Sigma k) \times g_p^4 \braket{c V_{-1}(y_1) c V_{-1}(y_2) c V_{0}(y_3) c V_{0}(y_4)} + 5 \perms \\
		& = \frac{i \sdelta^{p+1}}{g_p^2 \ell_s^4} \times \frac{g_p^4}{2\ell_s^2} \braket{[\psi^{\mu_1} e^{ik_1 X}]_{y_1}\;
		 [\psi^{\mu_2} e^{ik_2 X}]_{y_2} \;
		[(i {\dot X}^{\mu_3} + 2 k_3 \cdot \psi \psi^{\mu_3}) e^{i k_3 X}]_{y_3}\;
		[(i {\dot X}^{\mu_4} + 2 k_4 \cdot \psi \psi^{\mu_4}) e^{i k_43 X}]_{y_4}\;
		  }  \\ & \qquad  \quad \times \braket{c(y_1) c(y_2) c(y_3)} \braket{e^{-\phi(y_1)} e^{- \phi(y_2)}}
	\end{aligned}
	\]
	Take $y_1 = 0, y_2 = 1, y_3 = \infty$ and integrate $y_4 = y$ from $0$ to $1$ (then we'll have 5 more terms coming from permuatations). There are five contributions. 
	\begin{itemize}
		\item Contracting the $\psi^{\mu_1}(0) \psi^{\mu_2}$ together and allowing the remaining 4 terms at $y_3, y_4$ to contract either amongst themselves or with various vertex operators. 
		\item Not contracting the first two $\psi$ and Contracting the $i \dot X(y_3)$ with any of the vertex operators while contracting the last $\psi$ with the first two
		\item Not contracting the first two $\psi$ and contracting the $i \dot X(y_4)$ with any of the vertex operators while contracting the third $\psi$ with the first two
		\item Forgetting the $i \dot X$, and contracting the $\psi$ at $y_1$ with the various $\psi$ at $y_3$ (consequently the $\psi$ at $y_2$ with the $\psi$ at $y_4$)
		\item Swapping $3$ with $4$ in the above (this gives an overall minus sign by fermionic statistics)
	\end{itemize}
	Integrating this will give two types of terms: $\eta^{ab} \eta^{cd}$ and $\eta^{ab} k^c k^d$. Our shorthand replaces the superscript $\mu_i$ by just  $i$. Below, I underline the terms that contribute to the first type:
	\begin{equation}\label{eq:gauge}
	\begin{aligned}
		\frac{i g_p^2 \sdelta^{p+1}}{\ell_s^4} \int_0^1 dy \frac{y^{2 \ell_s^2 k_1 \cdot k_4} y^{2 \ell_s^2 k_2 \cdot k_4}}{2 \ell_s^2}
		 \Bigg\{
		& -\ell_s^2 \eta^{12} 
		 \left[ \underline{2 \ell_s^2 \eta^{34}} + (2 \ell_s^2)^2 (\underline{-\eta^{34} k_3 \cdot k_4} + k_4^3 k_3^4) + (2\ell_s^2)^2 (k_2^3 + k_4^3) \Big(\frac{k_1^4}{y} + \frac{k_2^4}{y-1}\Big) \right]\\
		& + \ell_s^2 (2 \ell_s^2)^2 
		\big[(k_2^3 + y k_4^3)(-k_4^1 \eta^{24} + k_4^2 \eta^{14}) \big] 
		\\ & + \ell_s^2 (2 \ell_s^2)^2 
		\Big[\Big(\frac{k_1^4}{y} + \frac{k_2^4}{y-1} \Big) (-k_3^1 \eta^{23} + k_3^2 \eta^{13}) \Big]\\
		& + \ell_s^2 \frac{(2 \ell_s^2)^2}{y-1} 
		\left[\underline{\eta^{13} \eta^{24} k_3 \cdot k_4} + \eta^{34} k_3^1 k_4^2 - \eta^{13} k_4^2 k_3^4 - \eta^{24} k_4^3 k_3^1 \right]\\
		& - \ell_s^2 \frac{(2 \ell_s^2)^2}{y} 
		\left[\underline{\eta^{14} \eta^{23} k_3 \cdot k_4 }+ \eta^{34} k_3^2 k_4^1 - \eta^{13} k_4^2 k_3^4 - \eta^{24} k_4^3 k_3^1 \right]  \Bigg\}
	\end{aligned}		
	\end{equation}
	Using $s = -2 \ell_s^2 k_1 \cdot k_2$ etc we see the underlined terms contribute
	\[
	\begin{aligned}
		& \quad \frac{i g_p^2 \sdelta^{p+1}}{\ell_s^2} \left[\frac{\Gamma(1-u) \Gamma(1-t)}{\Gamma(2+s)} (-(1 + s) \eta^{12} \eta^{34}) + \frac{\Gamma(1-u) \Gamma(-t)}{\Gamma(1+s)} \eta^{14} \eta^{23} s + \frac{\Gamma(-u) \Gamma(1-t)}{\Gamma(1+s)} \eta^{14} \eta^{23} s \right][1423]\\
		& = \frac{i g_p^2 \sdelta^{p+1}}{\ell_s^2} \frac{\Gamma(-u) \Gamma(-t)}{\Gamma(1+s)} ( - tu \eta^{12} \eta^{34} - su \eta^{13} \eta^{24} - st \eta^{14} \eta^{23} ) [1423]
	\end{aligned}
	\]
	The non-underlined terms are more involved but end up contributing twelve terms that yield:
	\[
	\begin{aligned}
		& \quad \frac{i g_p^2 \sdelta^{p+1}}{\ell_s^2} \left[\frac{\Gamma(-u) \Gamma(1-t)}{\Gamma(1+s)} 2 \ell_s^2 \eta^{12} k_2^3 k_1^4 + \dots \right][1423] = \frac{i g_p^2 \sdelta^{p+1}}{\ell_s^2} 2 \ell_s^2 \frac{\Gamma(-u) \Gamma(-t)}{\Gamma(1+s)} \left[ t \eta^{12} k_2^3 k_1^4 + 11 \perms \right][1423]
	\end{aligned}
	\]
	Here my Mandelstam variables are dimensionless. The result with dimensionful Mandelstam variables is:
	\begin{equation}\label{eq:4gluons}
		i g_p^2 \sdelta^{p+1} \ell_s^2  \left( \frac{\Gamma(-\ell_s^2 s) \Gamma( -\ell_s^2 u)}{\Gamma(1 + \ell_s^2 t)} K_4(k_i, e_i) ([1234] + [4321]) + 2 \perms\right)
	\end{equation}
	with $K_4 (k_i, e_i) = - t u e_1 \cdot e_2 e_3 \cdot e_4 + 2 s (e_1 \cdot e_3 \, e_2 \cdot k_4 e_4 \cdot k_2 + 3 \perms )  + 2 \perms$
	
	Now for the four transverse scalar amplitude, our vertex operators look like:
	\[
		V_{-1}^{a, \mu} = g_p \lambda^a \psi^\mu e^{-\phi} c e^{i p X}, \qquad
		V_{0}^{a, \mu} = \frac{g_p}{\sqrt 2 \ell_s} \lambda^a (X' + 2 p \cdot \psi \, \psi^\mu) c e^{i p X}
	\]
	here $X' = \partial_\sigma X =  2 i \partial X$. Moreover, we have Dirichlet boundary conditions on both the $X$s and the fermions $\psi$. The $\psi$ are still in the NS sector since we're looking at the (bosonic) scalar field scattering. 
	
	Crucially, the correlators for the $\psi$ field are the same for DD boundary conditions. We had $\braket{\dot X^\mu(z) \dot X^\nu(w)} = -\tfrac{\ell_s^2}{2} \eta^{\mu \nu} (z-w)^{-2}$ while the correlators for $X'$ pick up a minus sign $\braket{{X'}^i (z) {X'}^j(w)} = \tfrac{\ell_s^2}{2} \eta^{ij} (z-w)^{-2}$. This ends up giving the exact same result however, since the vertex operators contain $X'(z)$ while the prior ones contain $i \dot X(z)$. 
	
	Finally, contracting ${X'}^i$ with any of the $e^{i k X}$ will give zero, since the open strings only have momenta parallel to the D$p$ brane while the ${X'}^i$ is transverse. This gives a simpler amplitude than \eqref{eq:4gluons}:
	\begin{equation}\label{eq:4scalars}
		i g_p^2 \sdelta^{p+1} \ell_s^2 \; K_4'  \left( \frac{\Gamma(-\ell_s^2 s) \Gamma( -\ell_s^2 u)}{\Gamma(1 + \ell_s^2 t)} ([1234] + [4321]) + 2 \perms\right)
	\end{equation}
	with $K'_4 = - (t u  \delta_{12} \delta_{34} + s u \delta_{13} \delta_{24} + s t \delta_{14} \delta_{23})$.
	In the case where there are no CP indices we expand the $\Gamma \Gamma/\Gamma$ functions:
	\[
		i g_p^2 \sdelta^{p+1} \ell_s^2 K'_4 \times \left(\frac{2}{\ell_s^4 su} + \frac{2}{\ell_s^4 st} + \frac{2}{\ell_s^4 tu}\right) = i g_p^2 \sdelta^{p+1} \ell_s^2 K'_4 \times \left( \frac{2(s+t+u)}{\ell_s^4 stu}\right)  = 0
	\]
	So to leading order in the string length this is zero. This is consistent with the $U(1)$ DBI action, as the scalars do not directly interact with the $U(1)$ gauge field $A_\mu$ (in general a real scalar cannot be charged under a $U(1)$ gauge field). That is, at leading order the action is free in the $X$ fields. Taking $\xi^\mu = X^\mu$ for $\mu = 0 \dots p$ and $X^i$ independent functions, we get: 
	\[
		 \int d^{p+1} \xi \sqrt{\det{G_{MN} \d_\alpha X^M \d_\beta X^N}} = \int d^{p+1} \xi \sqrt{\det \delta_{\alpha \beta} + \delta_{ij} \d_\alpha X^i \d_\beta X^j} \to \int d^{p+1} \xi\, \frac{\delta^{\alpha \beta}\delta_{ij}}{2} \d_\alpha X^i \d_\beta X^j 
	\]
	This is just a free theory. Its also quick to see that the 3-point function of the transverse scalars vanishes at tree level in string perturbation theory. 

	
	\item I have done the previous problem in full generality, including CP indices. So now let's again look at the $s$ channel. As $s \to 0$ so that $t = -u$ we get from the $\delta_{12} \delta_{34}$ term a pole in $s$ going as:
	\[
		- i g_p^2 \sdelta^{p+1} \ell_s^2 tu \times \left(\frac{1}{\ell_s^4 su} ([1234] + [4321]) + \frac{1}{\ell_s^4 st} ([1243] + [3421])\right) = -i \sdelta^{p+1} \frac{g_p^2}{\ell_s^2} \frac{t}{s} ([1234] + [4321] - [1243] - [3421])
	\]
	We can rewrite this as:
	\[
		-i \sdelta^{p+1} \frac{g_p^2}{\ell_s^2} \frac{t}{s} (\Tr(12[34]) - \Tr([34]21) = -i \sdelta^{p+1} \frac{g_p^2}{\ell_s^2} \frac{t}{s} \Tr ([12] [34])
	\]
	The pole at $s = 0$ corresponds to an exchange of a gluon from the $\frac12 (D_\mu X^I)^2$ term in \textbf{8.6.1}.
	
	We also have a further term that does not involve a pole in $s$. Let's still take $1$ and $2$ equal. Expanding to this order we find:
	\[
		-i \sdelta^{p+1} \frac{g_p^2}{\ell_s^2} \Big( \Tr([12][34])  + \Tr([13][24]) + \Tr([14][23]) \Big) 
	\]
	This comes from exactly the potential term $\frac14 [X^I, X^J]^2$ in the effective action \textbf{8.6.1}. \textbf{Come back to that last term}
	
	\item Momentum conservation will imply $p_{\parallel} = 0$ for the NSNS states. Our vertex operator will take the form $\zeta_{\mu \nu} c \tilde c e^{-\phi} e^{- \tilde \phi} \psi^\mu \tilde \psi^\nu e^{i k_{\perp} X}$. We can use the doubling trick to get $\zeta_{\mu \nu} c(z) c(z^*) e^{-\phi(z)} e^{-\phi(z^*)} \psi^\mu \tilde \psi^\nu e^{i k_{\perp} X}$ and we are automatically in the $-2$ picture. 
	
	The states from in the $p+1$ parallel directions give just the correlator $\braket{\psi^\mu(i) \psi^\nu(-i)} = -\frac{\eta^{\mu \nu}}{2i}$ (importantly NN fermions in NSNS have 2-point function $-1/(z-\bar w)$ c.f. \textbf{4.16.22}). We also get a $\sdelta^{p+1}$ from momentum conservation.
	
	The states in the transverse (Dirichlet) directions give $\frac{\delta^{ij}}{2i}$ correlator. Defining the diagonal matrix $D^{\mu \nu} = (\eta^{\alpha \beta}, \delta^{ij})$ we get a correlator proportional to 
	\[
		-\frac{g_c}{2 \ell_s^2 g_p^2}\sdelta^{p+1}(k_{\parallel})  D^{\mu \nu}  = - \frac{(2 \pi \ell_s)^2 T_p}{2} V_{p+1} D^{\mu \nu}
	\]
	\textbf{Check with Victor}. Confirm the tension relation. This diagonal tensor $D^{\mu \nu}$ allows for a nonvanishing dilaton and graviton tadpole, but will not couple to the antisymmetric Kalb-Ramond $B$-field. 
	
	\item Our RR fields have picture $(r, s)$ for $r,s$ half-integers. In order to have total picture $-2$ on the disk, we need to pick this to be the (asymmetric) $(-3/2, -1/2)$ picture. The construction of this operator is complicated. I expect that the $(-1/2,-1/2)$ operator that we are familiar with is basically $e^{- \phi} G_0$ times the $(-3/2,-1/2)$ operator. This means that the $(-3/2, -1/2)$ will be one less power of momentum and one less gamma matrix than the $(-1/2, -1/2)$ operator. Picking out the $p+2$ form part of the $(-1/2, -1/2)$ operator (in Blumenhagen's convention \textbf{16.21}) gives
	\[
		\frac14 \frac{\ell_s}{\sqrt 2} 
		F^{\alpha \beta}  \overline{S}_{\alpha}(z) \tilde S_\beta(\bar z) e^{-\phi/2 - \bar \phi/2} 
		\to 
		\frac14 \frac{\ell_s}{\sqrt 2}
		\frac{F_{\mu_1 \dots \mu_{p+2}}}{(p+2)!} (\Gamma^{\mu_1 \dots \mu_{p+2}})^{\alpha \beta} \overline{S}_{\alpha} \tilde S_{\beta} e^{-\phi/2 - \bar \phi/2}
	\]
	Here $\overline S_\alpha = S_\alpha^\dagger \Gamma^0$ as is standard in a spinor product. BRST will require that $F$ and $\star F$ be closed. 
	
	Changing picture means removing one power of momentum and one gamma matrix. This anti-differentiates $F$, which must be proportional to the potential $C$ since $p_{[\mu_1} C_{\mu_2 \dots \mu_{p+2}]} = F_{\mu_1 \dots \mu_p+2}$. It is then reasonable to expect the corresponding $(-3/2, -1/2)$ operator to be proportional to
	\[
		 e^{-3\phi/2 - \bar \phi/2} \frac{C_{\mu_1 \dots \mu_{p}}}{(p+1)!} (\Gamma^{\mu_0 \dots \mu_{p}})^{\alpha \beta} \overline{S}_\alpha \tilde S_\beta
	\]
	Note both $e^{-3\phi/3} S_{\alpha}$ and $e^{-\tilde \phi/2} S_{\beta}$ \emph{remain} primary operators, having dimensions $3/8 + 5/8$, so this is indeed a reasonable guess. Now, for the D-brane boundary conditions we have
	\[
		 (\delta^\perp  \tilde S)_\beta(\bar z) = S_\beta(z^*), \quad \delta^\perp = \prod_{i=p+1}^9 \delta^i, \quad \delta^i = \Gamma^i \Gamma_{11}
	\]
	This reflects the $S_\beta$ spinor along all the Dirichlet directions and keeps it the same across the Neumann directions. 
	
	From \textbf{5.12.42} of Kiritsis  I expect the leading-order of the $S_\alpha \tilde S_\beta$ correlator to be $C_{\alpha \beta}/(z - \bar z)^{10/8}$ and the $e^{-3\phi/2} \tilde e^{-\phi/2}$ will contribute $(z-\bar z)^{-3/4}$ to make this a primary correlator transforming as $C_{\alpha \beta} (z - \bar z)^{-2}$. For Neumann boundary conditions, $C_{\alpha \beta}$ is the charge conjugation matrix. For the $D$-brane, it will be $\delta^\perp C$.
	
	Only in the IIB case will $C_{\alpha \beta}$ will be nonzero between $\overline{S}_\alpha$ and $\tilde S_\beta$ since $S_\alpha$ and $\tilde S_\beta$ transform in the same representations. Each $\beta^i$ changes the chirality. So the amplitude in IIB will vanish if we have an even number of $\beta^i$, equivalently $9-p$ is odd, so we will have only odd dimensional branes in IIB as required and even dimensional branes in IIA as required. 
	
	We thus get an amplitude proportional to:
	\[
		\mathcal A = i \frac{g_c \sdelta^{p+1}}{g_p^2 \ell_s^2} \frac{C_{\mu_0 \dots \mu_p}}{(p+1)!} \Tr(\Gamma^{\mu_1 \dots \mu_p} \Gamma^{p+1} \dots \Gamma^9 \Gamma^{11}) = i \frac{g_c \sdelta^{p+1}}{g_o^2 \ell_s^2} \frac{C_{\mu_0 \dots \mu_p} \epsilon_{(p+1)}^{\mu_0 \dots \mu_p}}{(p+1)!} 
	\]
	Comparing with the \textbf{8.4.4}, which should factorize as $\mathcal A(p_\parallel)^2 G_{9-p}(p_\perp) \delta^{p+1}(p_\parallel)$ we see that the normalization of our on-shell amplitude is in fact:
	\[
		\mathcal A = i V_{p+1} \, \sqrt{2 \pi} (2\pi \ell_s)^{3-p} \frac{C_{\mu_0 \dots \mu_p} \epsilon_{(p+1)}^{\mu_0 \dots \mu_p}}{(p+1)!} 
	\]
	This is consistent with other results c.f. Di Veccia, Liccardo \emph{Gauge Theories from D-Branes}, arXiv:0307104 but I think they're not incorporating $1/\alpha_p = 2 \kappa_{10}^2$ in the propagator. Taking this factor into account and dividing by it followed by taking a square root gives us an on-shell amplitude of:
	\[
		\mathcal A = i V_{p+1} \frac{1}{(2 \pi \ell_s)^p \ell_s g_s} \frac{C_{\mu_0 \dots \mu_p} \epsilon_{(p+1)}^{\mu_0 \dots \mu_p}}{(p+1)!}  = i V_{p+1} T_p \, C_{p+1} \wedge \epsilon_{(p+1)}.
	\]
	This is exactly what would come from a minimal coupling term of the form $i T_p \int C_{p+1}$.
	
	\item We take one vertex operator to be in the $(-1, -1)$ picture and gauge fix it to lie at $z=i$, and take the other in the $(0, 0)$ picture lying at $iy$, fixing $y$ to range from $0$ to $1$. In doing the doubling trick, we take $\tilde X^\mu = D^\mu_\nu X^\nu, \tilde \psi^\mu = D^\mu_\nu \psi^\nu$ and $\tilde \phi = \phi, \tilde c = c$. 
	
	We wish to calculate the correlator:
	\[
		-\frac{g_c^2}{g_o^2 \ell_s^4} \frac{2}{\ell_s^2} \braket{[\psi^\mu \tilde \psi^{\bar \mu} e^{i k_1 X}]_0\; [(i \d X^\mu + \frac12 k_2 \cdot \psi \psi^\nu) (i \bar \d X^{\bar \nu} + \frac12 k_2 \cdot \bar \psi \bar \psi^{\bar \nu}) e^{i k_2 X}]_y }
	\]
	
	The simplest way to do this problem is to recognize that after the doubling trick has been applied, we are calculating the a correlator of four fields \emph{at collinear insertions on} the Riemann sphere. 
	\[
		\varepsilon^1_{\mu \bar \mu} \varepsilon^2_{\nu \bar nu} D^{\bar \mu}_{\mu'} D^{\bar \nu}_{\nu'} \braket{V_{-1}^\mu(p_1), V_{-1}^{\mu'}(D p_1), V_0^\nu(p_2), V_0^{\nu'}(D p_2)}
	\]
	 We can map these four points to $0,1,y, \infty$ and take the integral to be from $y = 0$ to $y = 1$, with appropriate jacobian. 
	 
	This is nothing more than the 4-gluon amplitude calculated previously, which gave
	\[
		2 i \frac{g_o^2}{g_p^2} \sdelta^{p+1} \ell_s^2 \frac{\Gamma(-\ell_s^2 s) \Gamma(-\ell_s^2 u)\Gamma(-\ell_s^2 s)}{\Gamma(1 + 2 \ell_s^2 t)} \to  K_4(k_i, e_i)
	\]
	Here we do not have three terms for $y$ in different regions - we only have one. Our momenta are $p_1, p_2, D p_1, D p_2$ giving:
	% \[
	% 	2 i \frac{g_o^2}{g_p^2} \sdelta^{p+1} \ell_s^2 \frac{\Gamma(-2\ell_s^2 p_1 \cdot p_2) \Gamma(-\ell_s^2 u)\Gamma(-\ell_s^2 s)}{\Gamma(1 + 2 \ell_s^2 t)} \to  K_4(k_i, e_i)
	% \]
	 The only caveat is that in that case, we had boundary normal ordering. We can view this as $\ell_s^{here} = \ell_s^{there}/2$. This is reflected by halving the momenta of the open strings.
 	\[
 		\begin{aligned}
 			p_1^{there} \to p_1^{here}/2, &\quad& p_2^{there} \to p_2^{here}/2\\
 			p_3^{there} \to D\cdot p_1^{here}/2, &\quad& p_4^{there} \to D \cdot p_2^{here}/2\\
 		\end{aligned}
 	\]
 	This gives us (here we do not have three terms for $y$ in different regions - we only have one)
 	\[
 		2 i \frac{g_o^2}{g_p^2} \sdelta^{p+1} \ell_s^2 \frac{\Gamma(-\ell_s^2 s) \Gamma(-\ell_s^2 u)}{\Gamma(1 + 2 \ell_s^2 t)} \to  K_4(k_i, e_i)
 		\to 2 i \frac{g_o^2}{g_p^2} \sdelta^{p+1} \ell_s^2
 		\frac{\Gamma(\ell_s^2 p_1 \cdot p_2/2) \Gamma(\ell_s^2 p_1 \cdot D p_1 /2)}{\Gamma(1 + \ell_s^2 p_1 \cdot p_2 / 2 + \ell_s^2 p_1 \cdot D p_1 /2)}
 	\]
	Following Myers, we can write $t = -2 p_1 \cdot p_2$ as the momentum transfer to the brane, and $q^2 = p_1 \cdot D p_1/2$ as the momentum flowing parallel to the brane. The Gamma ratio then looks like:
	\[
		\frac{\Gamma(-\ell_s^2 t/4) \Gamma(\ell_s^2 q^2)}{\Gamma(1 + \ell_s^2 t / 4 + \ell_s^2 q^2)}
	\]

		%
	% \textbf{TODO: FINISH} Its just the kinematic factors that are annoying.. Myers 9603194 has the exact expression, as does Klebanov. You can always argue that the 4-point open string amplitude has the same kinematic structure.
	If we held $t$ fixed and took the large $q$ limit we would get a series of open string poles. This corresponds to a closed string splitting in two, with its ends on the D-brane as an intermediate state.
	
	Now let's hold $q^2$ fixed and take the large $t$ limit. This is the limit of large energy transfer - the Regge limit. From the ratio of $\Gamma \Gamma/\Gamma$ we see that there are closed string poles. This can be interpreted as the closed strings interacting with the long-range background fields generated by the presence of the D$p$ brane. 
	
	\begin{center}
		\includegraphics[scale=0.15]{"Drawings/Dbrane Scatter"}
	\end{center}
	
	\textbf{Comment about pole structure}
	
	\item In the D9 brane case, we have seen that the open string boundary only preserves the sum of $Q + \tilde Q$. If we T-dualize in the 9 direction, we act on the right-moving sector by spacetime parity, so that necessarily $\bar \d X^9 \to - \bar \d X^9, \tilde \psi^9 \to - \tilde \psi^9, \tilde S_\alpha \to \delta^9 \tilde S^\alpha$ (up to a phase in that last one). Here $\delta^9 = \Gamma^9 \Gamma^{11}$. Our spacetime supersymmetry generator $\tilde Q_\alpha = \frac{1}{2\pi i} \int d\bar z\, e^{-\phi/2} S_{\alpha}$ therefore will be mapped to $\delta^9 \tilde Q$. Thus, in the T-dual picture we preserve the supercharge $Q' + \delta^9 \tilde Q'$.
	
	Iterating this procedure in other directions we get that in general we preserve $Q + \delta^\perp \tilde Q$, with $\delta^\perp = \prod_{i} \delta^i$, where $i$ runs perpendicular to the brane. Note that T-dualities along different directions do not commute! They commute up to a $(-1)^{\mathbf{F}_R}$, and so the order that we do them matters. In this case the $\delta^i$ act by left-action.
	
	\item (As in Polchinski section 13.4) From the previous problem, we see that the first D-brane preserves the supercharges $Q + \delta^\perp \tilde Q$ while the second preserves the supercharges $Q + {\delta^\perp}' \tilde Q = Q + \delta^\perp ({\delta^\perp}^{-1} {\delta^\perp}' \tilde Q)$ so the supersymmetries that will be preserved must be of both forms. This is in one-to-one correspondence with spinors invariant under ${\delta^\perp}^{-1} {\delta^\perp}'$. This operator is a reflection in the direction of the ND boundary conditions (the directions orthogonal to the D$_{p'}$ brane in the D$_p$ brane). Since in either IIA or IIB $p$ and $p'$ must differ by an even integer, the number of mixed boundary conditions--call it $\nu$--must be even. Then we can write ${\delta^\perp}^{-1} {\delta^\perp}'$ as a product of rotations by $\pi$ along each of the $\nu/2$ planes ${\delta^\perp}^{-1} {\delta^\perp}' = e^{i \pi (J_1 + \dots + J_{\nu/2})}$. Each $j$ acts in a spinor representation, so that $e^{i \pi J_i}$ has eigenvalues $\pm i$. If $\nu/2$ is odd, this makes ${\delta^\perp}^{-1} {\delta^\perp}' = -1$ so this will \emph{not} preserve supersymmetry. We thus need $\nu/2$ even, or $\nu = 0$ mod $4$.

	From this I posit that the static force between two branes vanishes precisely when $\nu = 0 \text{ mod } 4$.
	
	\item Now let's confirm this guess with an amplitude calculation. Take $p' \leq p$. We work in lightcone gauge. We do a trace over an open string with $p'$ NN boundary conditions, $p-p' = \nu$ DN boundary conditions, and $8-p$ DD boundary conditions. 
	
	We begin from the open string point of view in calculating the cylinder amplitude. Chapter 4 has done the NN, DD, and DN boson amplitudes for us. The difficulty lies almost entirely in the fermions.
	Recall the following:
	\[
	\begin{aligned}
		\eta &= q^{1/24} \prod_{n=1}^\infty (1-q^n)\\
		 \sqrt{\frac{\theta[{^0_0}]}{\eta}} = q^{-1/48} \prod_{n=0}^\infty (1 + q^{n+1/2}) \qquad 
		 \sqrt{\frac{\theta[{^1_0}]}{\eta}} &= \sqrt 2 \, q^{1/24} \prod_{n=0}^\infty (1 + q^n) \qquad  \sqrt{\frac{\theta[{^0_1}]}{\eta}} &= q^{-1/48} \prod_{n=0}^\infty (1 - q^{n+1/2})
	\end{aligned}
	\]
	Further from \textbf{4.16.2} recall that for modes $b_{n+1/2}$, $b_n$ corresponding to NS and R sectors  the NN and DD boundary conditions give:
	\begin{itemize}
		\item NN: $\bar b_{n+1/2} = - b_{n+1/2}, \bar b_n = b_n$
		\item DD: $\bar b_{n+1/2} = b_{n+1/2}, \bar b_n = -b_n$
	\end{itemize}
	For DN we have the same result as for DD but now the R sector is half-integrally modded and the NS sector in integrally modded. Now lets compute partition functions. Our final answer will be a sum over spin structures NS+, NS-, R+, R-. Taking $q = e^{- 2 \pi t}$ we see $\Tr[q^{L_0 - c/24}] = $
	\begin{itemize}
		\item NS+: 
		\begin{itemize}
			\item NN: $q^{-1/48}\prod_{n} (1 + q^{n+1/2}) =  \sqrt{\theta[{^0_0}]/\eta}$
			\item DD: $q^{-1/48}\prod_{n} (1 + q^{n+1/2}) = \sqrt{\theta[{^0_0}]/\eta}$
			\item DN: $\sqrt{2} q^{-1/48} q^{1/16} \prod_{n} (1 + q^{n}) = \sqrt{\theta[{^1_0}]/\eta}$ ( $\sqrt 2$ when raised to a power counts ground state degeneracy)
		\end{itemize}
		% The NS sector gives $\frac12 (\theta^4 \twist 00  - \theta^4 \twist 01)/\eta^4$
%
% 		The R sector gives $-\frac12 \theta\twist 10 ^4/\eta^4$
		\item NS-: 
		\begin{itemize}
			\item NN: $q^{-1/48}\prod_{n} (1 - q^{n+1/2}) =  \sqrt{\theta[{^0_1}]/\eta}$
			\item DD: $q^{-1/48}\prod_{n} (1 - q^{n+1/2}) = \sqrt{\theta[{^0_1}]/\eta}$
			\item DN: $0$
		\end{itemize}
		
		\item R+
		\begin{itemize}
			\item NN: $\sqrt 2 q^{1/24}\prod_{n} (1 + q^n) =  \sqrt{\theta[{^1_0}]/\eta}$
			\item DD: $\sqrt 2 q^{1/24}\prod_{n} (1 + q^n) =  \sqrt{\theta[{^1_0}]/\eta}$
			\item DN: $q^{-1/48}  \prod_{n} (1 + q^{n+1/2}) = \sqrt{\theta[{^0_0}]/\eta}$ 
		\end{itemize}
		
		\item R-
		\begin{itemize}
			\item NN: 0
			\item DD: 0
			\item DN: $q^{-1/48} \prod_{n} (1 - q^{n+1/2}) = \sqrt{\theta[{^0_1}]/\eta}$ 
		\end{itemize}
	\end{itemize}
	
	\textbf{Notice} NN vs DD boundary conditions have \emph{no effect} on fermion contribution to partition function. This is because, although the left moving and right-moving modes are identified differently, the mode excitations look exactly the same.
	
	On the other hand for NN and DD the bosons will contribute $1/\eta$ and will contribute $\sqrt{\eta/\theta[{^0_1}]}$ for DN. Thus we have the following contributions to the partition function (here $N$ is the number of NN boundary conditions):
	\[
		\begin{aligned}
		NS+ &= \frac{V_N}{(2 \pi \ell_s)^{N}} \int \frac{dt}{2t} \frac{e^{-2 \pi t \left(\frac{\Delta x}{2 \pi \ell_s} \right)^2}}{(\sqrt{2 t})^{N} \eta^{8-\nu} \left(\theta[{^0_1}]/\eta\right)^{\nu/2}} \left(\frac{\theta\twist00}{\eta} \right)^{(8-\nu)/2} \left(\frac{\theta\twist10}{\eta} \right)^{\nu/2} 
		\\
		NS- &= -\frac{V_N}{(2 \pi \ell_s)^{N}} \int \frac{dt}{2t} \frac{e^{-2 \pi t \left(\frac{\Delta x}{2 \pi \ell_s} \right)^2}}{(\sqrt{2 t})^{N} \eta^{8}} 
		\left(\frac{\theta\twist01}{\eta} \right)^{8} \delta_{\nu=0}
		\end{aligned}
		\]
	\[
	\begin{aligned}
	R+ &= -\frac{V_N}{(2 \pi \ell_s)^{N}} \int \frac{dt}{2t} \frac{e^{-2 \pi t \left(\frac{\Delta x}{2 \pi \ell_s} \right)^2}}{(\sqrt{2 t})^{N} \eta^{8-\nu} \left(\theta[{^0_1}]/\eta\right)^{\nu/2}} 
	\left(\frac{\theta\twist10}{\eta} \right)^{(8-\nu)/2} \left(\frac{\theta\twist00}{\eta} \right)^{\nu/2} 
	\\
	R- &= \frac{V_N}{(2 \pi \ell_s)^{N}} \int \frac{dt}{2t} \frac{e^{-2 \pi t \left(\frac{\Delta x}{2 \pi \ell_s} \right)^2}}{(\sqrt{2 t})^{N} } \delta_{\nu=8} 	
	\end{aligned}
	\]
	All theta and eta functions are evaluated at $it$. The circumference of the cylinder is $2 \pi t$. The relative signs in front of the different contributions come from a combination of defining the NS vacuum to have negative fermion and modular invariance (equivalently spacetime spin-statistics). 
	% Summing this gives
% 	\[
% 		\frac{V_N}{(2 \pi \ell_s)^{N}} \int \frac{dt}{2t} \frac{e^{-t \left(\frac{\Delta x}{2 \pi \ell_s} \right)^2}}{(\sqrt{2 t})^{N} \eta^{8-\nu} \left(\theta[{^0_1}]/\eta\right)^{\nu/2}} \left[ \left(\frac{\theta\twist00}{\eta} \right)^{(8-\nu)/2} \left(\frac{\theta\twist10}{\eta} \right)^{\nu/2} - \left(\frac{\theta\twist01}{\eta} \right)^{8} \delta_{\nu=0} -
%
% 		 \right]
% 	\]
	Note when $\nu = 4$ we only get contributions from NS+ and R+, which exactly cancel. Similarly when $\nu = 4$ or $8$, by the abstruse identity of Jacobi we will get cancelation again. 
	
	We can interpret our result as a one-loop free energy. Differentiating this w.r.t. $\Delta x$ would then give us our force. For $\nu = 0, 4, 8$ we do not get a force, consistent with the D-brane configuration preserving supersymmetry. 
	
	For the sake of completeness, and to clear my own confusion once and for all, I will also do this from the POV of the boundary state formalism (not developed in Kiritsis). For a good reference see the last chapter of Blumenhagen's text on conformal field theory. 
	
	For a single free boson, after the flip $(\sigma, \tau)_{open} \to (\tau, \sigma)_{closed}$ the boundary states $\ket{N}, \ket{D}$ must satisfy 
	\[
		( \alpha_n + \tilde \alpha_{-n}) \ket N = 0, \qquad ( \alpha_n - \tilde \alpha_{-n}) \ket{D_x} = 0,
	\]
	This gives boundary states:
	\[
		\begin{aligned}
			\ket{N} &= \frac{1}{(2\pi \ell_s \sqrt{2})^{1/2}} \prod_n e^{- \frac1n \alpha_{-n} \tilde \alpha_{-n}} \ket{0,0; 0} = \sum_{\vec m = \{m_i\}} \ket{\vec m, \Theta \vec m; 0} \\
			\ket{D_x} &= (2\pi \ell_s/\sqrt{2})^{1/2} \int \frac{dk}{2\pi} e^{i p x} \prod_n e^{- \frac1n \alpha_{-n} \tilde \alpha_{-n}} \ket{0,0; k}\\			
		\end{aligned}
	\]
	The overall normalization came from comparing with cylinder amplitudes. $\Theta$ here is CPT reversal. Similarly for a fermion
	\[
		( \psi_n + \tilde \psi_{-n}) \ket N = 0, \qquad ( \psi_n - \tilde \psi_{-n}) \ket{D_x} = 0,
	\]
	So with GSO projection we get:
	\[
	\begin{aligned}
		\ket{N, \text{NSNS}} = P_L P_R \prod_r e^{\psi_{-r} \tilde \psi_{-r} } \ket{0}, &\qquad \ket{N, \text{RR}} = P_L P_R \prod_n e^{\psi_{-n} \tilde \psi_{-n} } \ket{0}\\
		\ket{D, \text{NSNS}} = P_L P_R \prod_r e^{-\psi_{-r} \tilde \psi_{-r} } \ket{0}, &\qquad \ket{D, \text{RR}} = P_L P_R \prod_n e^{-\psi_{-n} \tilde \psi_{-n} } \ket{0}
	\end{aligned}
	\]
	Here $r$ runs over half-integers in the NSNS sector and $n$ runs over integers in the RR sector. $P_L = \frac12( 1 + (-1)^F), P_R = \frac12 (1 + (-1)^{\tilde F})$ are our GSO projections, defined to project out the tachyon in the NS sector and project out one of the spinors in the R sector. 
	
	For the boson, it is quick to see that ($\ell = 1/t$)
	\[
	\begin{aligned}
		\bra{N} e^{-\pi \ell (L_0 + \tilde L_0 - c/12)} \ket{N} &= \frac{V}{2\pi \ell_s \sqrt 2 \eta(i \ell)} = \frac{V}{(2\pi \ell_s) \sqrt{2 t} \eta(i t)} \\
		\bra{D} e^{-\pi \ell (L_0 + \tilde L_0 - c/12)} \ket{D} &= \frac{2 \pi \ell_s}{\sqrt{2 t} \eta(i t)} \int \frac{dk}{2\pi} e^{i k \Delta x} e^{-\pi \ell_s^2 p^2/2t} = \frac{e^{-2 \pi t \left( \frac{\Delta x}{2 \pi \ell_s} \right)^2 }}{\eta(i t)}\\
		\bra{D} e^{-\pi \ell (L_0 + \tilde L_0 - c/12)} \ket{N} &= \frac{1}{\sqrt 2} \frac{1}{\prod_n (1+q^{2n})} = \sqrt{\frac{\eta(i \ell )}{\theta\twist10 (i \ell)}} = \sqrt{\frac{\eta(i t)}{\theta\twist01 (i t)}}
	\end{aligned}
	\]
	These are exactly what we've already gotten many times before from our trace over the open string bosonic states. The states $\ket{N}, \ket{D}$ must be a sum of both the RR and NSNS sector fermion states. We do not know the relative coefficients. 
	
	Let's look at the NSNS contributions. For the NN boundary conditions, the NSNS sector with projection consists of two terms: 
	\[
		\begin{aligned}
			\bra{N, \text{NSNS}_{unproj}} e^{-\pi \ell (L_0 + \tilde L_0 - c/12)} \ket{N, \text{NSNS}_{unproj}}  &= \left(\frac{\theta\twist00(i \ell)}{\eta(i \ell)}\right)^{\# NN/2}  = \left(\frac{\theta\twist00(i t)}{\eta(i t)}\right)^{\# NN/2}\\
			\bra{N, \text{NSNS}_{unproj}} (-1)^{F_L = F_R} e^{-\pi \ell (L_0 + \tilde L_0 - c/12)} \ket{N, \text{NSNS}_{unproj}}  &= \left(\frac{\theta\twist01(i \ell)}{\eta(i \ell)}\right)^{\# NN / 2} = \left(\frac{\theta\twist10(i t)}{\eta(i t)}\right)^{\# NN / 2}
		\end{aligned}
	\]
	Replacing $N$ with $D$ would give the \emph{exact same} factor in both cases \textbf{WHY?} (explain: bc we need to match on both sides and so both minuses cancel in the exponent).  For DN boundary conditions the NSNS sector give the two terms:
	\[
		\begin{aligned}
			\bra{D, \text{NSNS}_{unproj}} e^{-\pi \ell (L_0 + \tilde L_0 - c/12)} \ket{N, \text{NSNS}_{unproj}}  &= \left(\frac{\theta\twist01(i \ell)}{\eta(i \ell)}\right)^{\nu/2} = \left(\frac{\theta\twist10(it)}{\eta(i t)}\right)^{\nu/2}\\
			\bra{D, \text{NSNS}_{unproj}} (-1)^{F_L = F_R} e^{-\pi \ell (L_0 + \tilde L_0 - c/12)} \ket{N, \text{NSNS}_{unproj}}  &= \left(\frac{\theta\twist00(i \ell)}{\eta(i \ell)}\right)^{\nu/2}  = \left(\frac{\theta\twist00(i t)}{\eta(i t)}\right)^{\nu/2}
		\end{aligned}
	\]
	
	Now let's look at the RR sector. For NN boundary conditions, it contributes:
	\[
		\begin{aligned}
			\bra{N, \text{RR}_{unproj}} e^{-\pi \ell (L_0 + \tilde L_0 - c/12)} \ket{N, \text{RR}_{unproj}}  &= \left(\frac{\theta\twist10(i \ell)}{\eta(i \ell)}\right)^{\# NN/2} = \left(\frac{\theta\twist01(it)}{\eta(i t)}\right)^{\# NN/2} \\ 
			\bra{N, \text{RR}_{unproj}} (-1)^{F_L = F_R} e^{-\pi \ell (L_0 + \tilde L_0 - c/12)} \ket{N, \text{RR}_{unproj}}  &= 0
		\end{aligned}
	\]
	By the argument before, we get the same for DD boundary conditions. Finally, with DN boundary conditions we get
	\[
		\begin{aligned}
			\bra{D, \text{RR}_{unproj}} e^{-\pi \ell (L_0 + \tilde L_0 - c/12)} \ket{N, \text{RR}_{unproj}}  &=  0\\
			\bra{D, \text{RR}_{unproj}} (-1)^{F_L = F_R} e^{-\pi \ell (L_0 + \tilde L_0 - c/12)} \ket{N, \text{RR}_{unproj}}  &= \left(\frac{\theta\twist10(i \ell)}{\eta(i \ell)}\right)^{\nu/2} = \left(\frac{\theta\twist01(i t)}{\eta(i t)}\right)^{\nu/2}
		\end{aligned}
	\]
	Together this is exactly consistent with what we get from tracing over the open string. We can work back to get relative normalizations.
	
	
	\begin{center}
		\includegraphics[scale=0.15]{"Drawings/RRNSNS"}
	\end{center}
	
	
	This shows that the massless RR and NSNS fields mediate the force. Moreover the NSNS fields without and with projection correspond respectively to the unprojected NS and R open string states while the RR fields without and with projection correspond to the \emph{projected} NS and R open string states.
	
	\item First recall that for a constant vector potential $A_9 = \frac{\chi_9}{2\pi R}$ corresponds to a $T$-dual picture of a $D$-brane at position $- \chi \tilde R = -2 \pi \ell_s^2 A_9$. Now consider a magnetic flux $F_{12}$ we can write a (nonconstant now) vector potential that gives this flux as $A_2 = F_{12} X^1$. We $T$-dualize along $X^2$ and get $X^2 = -2 \pi \ell_s^2 F_{12} X^1$. Then $\tan \theta = -2 \pi \ell_s^2 F_{12}$.
	
	\begin{center}
		\includegraphics[scale=0.13]{"Drawings/Magnetic"}
	\end{center}
	
	Although we were working with D1 and D2 branes, we could have done the exact same calculation for $F_{01}$ on a D1 brane and recovered a D0 brane tilted in the $X^0-X^1$ plane (ie boosted). Such a D0 brane has the usual point-particle action:
	\[
		S_{D0} = -T_0 \int dX^0 \sqrt{1 + (\d_0 {X'}^1)^2} 
	\]
	Because the D0 brane and the D1 brane describe the same physics, this action should be identical to the D1 action. Note that $\d_0 {X'}^1$ is infinitesimally exactly $\tan \theta$ calculated above. We get the action
	\[
		S_{D1} = -T_1 \int dX^1 dX^2 \sqrt{1 + (2\pi \ell_s^2 F_{12})^2} 
	\]
	Of course, because the branes couple to strings, the only gauge invariant combination under transformations of the Kalb-Ramond $B$ field is $\mathcal F = B + 2\pi \ell_s^2 F$. We thus get:
	\[
		S_{D1} = -T_1 \int dX^1 dX^2 \sqrt{- \det (G + \mathcal F)} 
	\]
	We can tilt this brane and $T$-dualize to pick up EM field strengths in arbitrary dimension up to $9$.
	
	\item Let's $T$-dualize. This describes two D4 branes that are tilted only along the $x_1$-$x_5$ plane, and are otherwise parallel in the $x_2, x_3, x_4$ directions. T-dualizing $x_2, x_3, x_4$ makes these into D1 branes tilted in the $x_1 - x_5$ plane. See the solution \eqref{eq:partitionfunc} to the next problem. Now setting $\nu_{2,3,4} = 0$ will give poles from the theta function in the denominator. This is to be expected, from the NN boundary conditions that always come with a volume divergence factor in that direction. We regulate this divergence by replacing:
	\[
		\theta\twist11 (i \nu t, i t)^{-1} \to i \frac{L}{\eta(i t)^{-3} 2 \pi \ell_s \sqrt{2 t}}
	\]
	Thus we get a potential:
	\[
		-i\frac{L^3}{(2 \pi \ell_s)^4}\int_0^\infty \frac{dt}{t} \frac{e^{-2 \pi t \left( \frac{\Delta x}{2\pi \ell_s} \right)^2}}{(2t)^{4/2} \eta(it)^9} \frac{\theta \twist11 (i \nu t/2, i t)^4}{\theta \twist11 (i \nu t, it)}
	\]
	Let's take the distance to be large. The small $t$ contributions are then most important. The $\theta$-function ratio contributes a factor of $e^{-3 \pi /4 t} t^{-3/2}$ while the $\eta^{-9}$ contributes $e^{3 \pi /4 t} t^{9/2}$ \textbf{Finish the details here}
	
	\begin{center}
		\includegraphics[scale=0.13]{"Drawings/Tilt"}
	\end{center}
	
	 We get a potential that decays as $-\frac{L^3 \ell_s}{(\Delta x)^5} \frac{\sin^4 (\theta/2)}{\sin(\theta)}$ giving another attractive force going as $1/(\Delta x)^6$. 
	
	\item Following Polchinski, we define variables $Z^i = X^i + i X^{i+4}, i= 1, \dots, 4$. Let the $\sigma = 0$ endpoint be on the untilted string. Then at $\sigma = 0$ we have $\d_1 \Re Z^a = \Im Z^a = 0$ and at $\sigma = \pi$ on the tilted string we have $\d_1 \Re(e^{i \theta_a} Z^a) = \Im(e^{i \theta_a} Z^a)$. 
	
	We see that the field that satisfies this is given by $Z^a(w, \bar w) = \mathcal Z^a(w) + \bar {\mathcal Z}^a (-\bar w)$ for $w = \sigma^1 + i \sigma^2$. Using the doubling trick we have $\mathcal Z^a(w + 2 \pi) = e^{2 i \theta_a} \mathcal Z^a(w)$ (and similarly for the conjugate). This gives a mode expansion with $\nu_a = \theta_a / \pi$
	\[
		\mathcal Z^a(w) = i \frac{\ell_s}{\sqrt 2} \sum_{r \in \mathbb Z + \nu_a} \frac{\alpha^a_r}{r} e^{i r w}.
	\]
	The $a^\dagger$ modes are then linearly independent. Taking $q = e^{-2\pi t}$ as usual for open string partition functions, and restricting $0 < \phi_a < \pi$ we get:
	\begin{equation*}
		\frac{q^{\frac1{24} - \frac12 (\nu_a - \frac12)^2}}{\prod_{m=0}^\infty (1 - q^{m + \nu_a}) (1 - q^{m + 1 - \nu_a})} = -i \frac{q^{- \nu_a^2/2}\eta(it)}{\theta\twist11 (i \nu_a t, i t)}
	\end{equation*}
	Where we have used 
	\[
		\theta\twist11 (i \nu t , it) = i q^{\frac{1}{8}} (q^{\nu/2} - q^{-\nu/2}) \prod_{m=1}^\infty (1-q^m) (1- q^{m+\nu}) (1- q^{m-\nu}) =i  \eta(i t) q^{\frac18 - \frac{1}{24} + \frac{\nu}{2}} \prod_{m=0}^\infty (1- q^{m+\nu}) (1- q^{m+1-\nu})
	\]
	So the angles act like chemical potentials to make the theta functions nonzero. Now its time for the fermions (oh boy!). In each NS and R sector (projected and unprojected) the boundary conditions shift by $\nu_a$. We thus get e.g. for NS unprojected:
	\[
		Z\twist00 = q^{-\frac1{24} + \nu_a^2/2} \prod_{m=1}^\infty (1-q^{m+1/2 + \nu_a}) (1-q^{m + 1/2 - \nu_a}) = q^{\nu_a^2/2} \frac{\theta\twist00 (i \nu_a t , it)}{\eta(i t)}
	\]
	In total, then we will see that the fermion part gives us
	\[
		\frac{\prod_a q^{\nu_a^2/2}}{2 \eta(it)^4} \left[\prod_{a=1}^4 \theta\twist00 (i \nu_a t, it) - \prod_{a=1}^4 \theta\twist10 (i \nu_a t, it) - \prod_{a=1}^4 \theta\twist01 (i \nu_a t, it) - \prod_{a=1}^4 \theta\twist11 (i \nu_a t, it) \right]= \prod_{a=1}^4 \frac{e^{\nu_a^2/2}\theta\twist11 (i \nu_a' t, it)}{\eta(it)}
	\]
	This last equality follows from the full abstruse identity of Jacobi, where $\phi_1' = \frac12 (\phi_1 + \phi_2 + \phi_3 + \phi_4)$, $\phi_2' = \frac12 (\phi_1 + \phi_2 - \phi_3 - \phi_4)$, $\phi_3' = \frac12 (\phi_1 - \phi_2 + \phi_3 - \phi_4)$, $\phi_4' = \frac12 (\phi_1 - \phi_2 - \phi_3 + \phi_4)$ and the $\nu_a'$ are defined identically.
	Inserting the DD conditions along the $9$ direction that denotes separation, we get the full potential  as a function of the separation $\Delta x$
	\begin{equation}\label{eq:partitionfunc}
		V = - 2 \times \int_0^\infty \frac{dt}{2t} \frac{e^{-2 \pi t \left( \frac{\Delta x}{2\pi \ell_s} \right)^2}}{2\pi \ell_s \sqrt{2t}} \prod_{a=1}^4 \frac{\theta \twist11 (i \nu'_a t, i t)}{\theta \twist11 (i \nu_a t, i t)}
	\end{equation}
	The initial overall factor of two comes from two orientations of the open string.
	At long enough distances, the exponential factor forces small $t$ to contribute primarily. The complicated ratio of $\theta$-functions becomes a ratio of sines. Then we get
	\[
		\prod_{a=1}^4 \frac{\sin(\pi \nu_a')}{\sin(\pi \nu_a)} \int_0^\infty \frac{dt}{2 \pi \ell_s \sqrt{2 t} t} e^{- \frac{(\Delta x)^2}{2\pi \ell_s^2} t}
	\]
	Taking the integral and analytically continuing, we get a potential that looks like $- \frac{|\Delta x|}{2 \pi \ell_s^2} \prod_{a=1}^4 \frac{\sin(\pi \nu_a')}{\sin(\pi \nu_a)}$, giving an attractive, constant force, of $-\frac{1}{2 \pi \ell_s^2} \prod_{a=1}^4 \frac{\sin(\pi \nu_a')}{\sin(\pi \nu_a)}$.
	
	\item Let the first brane at $\sigma = 0$ have no electric field and put an electric field $F_{01}$ on the second brane at $\sigma = \pi$. The endpoints of the string are charged. We have the following action (take $e$, the charge of the string endpoint to be $1$)
	\[
		-\frac{1}{4\pi \ell_s^2} \int d\sigma d\tau [(\d_\sigma X)^2 + (\d_\tau X)^2] + \int_{\sigma = \pi} d\tau A_\mu \d_{\tau} X^\mu
	\]
	Upon variation, we get a boundary term:
	\[
	\begin{aligned}
		&\quad -\frac{1}{2\pi \ell_s^2} \int d\tau \d_\sigma X^\mu \delta X_\mu  + \int d\tau \delta(A_\nu \d_\tau X^\nu)
		\\
		&= -\frac{1}{2\pi \ell_s^2} \int d\tau \d_\sigma X^\mu \delta X_\mu  + \int d\tau \d_\mu A_\nu \delta X^\mu \d_\tau X^\nu - \d_\tau A_\nu \\
		&= -\frac{1}{2\pi \ell_s^2} \int d\tau \d_\sigma X_\mu \delta X^\mu  + \int d\tau F_{\mu \nu} \delta X^\mu \d_\tau X^\nu\\
		& \Rightarrow \d_\sigma X_\mu - 2\pi \ell_s^2 F_{\mu \nu} \d_\tau X^\nu = 0
	\end{aligned}
	\]
	This gives mixed boundary conditions on the $X^0$ and $X^1$ which can be written as
	\[
		\begin{pmatrix}
			\d_\sigma X^0\\
			\d_\sigma X^1
		\end{pmatrix}= 2 \pi \ell_s^2 E \begin{pmatrix}
			0 & 1 \\
			1 & 0
		\end{pmatrix}
		\begin{pmatrix}
			\d_\tau X^0\\
			\d_\tau X^1
		\end{pmatrix}
	\]
	with $E = F_{10}$.
	Note that we have been careful in raising the $\mu$ index. I will define $Z^{\pm} = X^0 \pm X^1$ and have $\d_\sigma Z^+ = \d_\sigma Z^- = 0$ at $\sigma = 0$ and $(\d_\sigma - 2\pi \ell_s^2 E \d_\tau) Z^+ = (\d_\sigma +2\pi \ell_s^2 E \d_\tau) Z^- = 0 $ at  $\sigma = \pi$. The modes thus satisfy Neumann-Mixed boundary conditions. Following a modification of exercise \textbf{2.14} and solving these boundary conditions we get that the modes must be labeled by $\nu = -i u/\pi + \mathbb Z$. Here $u = \mathrm{atanh}{v}$ is the \emph{rapidity}. Now let's compute a cylinder diagram. Let's assume for now that we are scattering D1 branes (the problem does not explicitly give $p, p'$). It will look like the wick-rotation of the integral considered in the previous two questions. We have an amplitude
	\[
		-i V_p \times 2 \times \int_0^\infty \frac{dt}{2t} \frac{e^{-t\frac{(\Delta x)^2}{2\pi \ell_s^2}}}{(2 \pi \ell_s)^p (2 t)^{p/2}} \frac{\theta\twist11 (u t / 2 \pi, it)^4}{\eta(i t)^9 \theta\twist11(u t/\pi, i t)}
	\]
	Note however that since the first argument of the $\theta$ functions is real, we have poles at $t = \pi n / u$, $\nu = u/\pi$ for $n$ an integer. Upon deforming the integration contour, we can use the identity $\frac{1}{x-i\epsilon} = \pi \delta(x) + \mathrm P(x)$ to pick up poles at $t =  n / \nu$, at \emph{odd} integers $n$ (so that the four-order zero in the numerator doesn't cancel them), giving:
	\begin{equation}\label{eq:amplitude}
		\pi V_p  \sum_{n \in \mathbb Z^{odd}} \frac{e^{- n \frac{(\Delta x)^2}{2\pi \ell_s^2 \nu}}}{(2\pi \ell_s)^p (2 n/\nu)^{(p+2)/2}} \frac{\theta \twist11 (\frac{n}{2}, it)^4}{2 \eta(i n/\nu)^{12}}
	\end{equation}
	The imaginary part of the amplitude can be interpreted (after $T$-duality) as resonances (ie bound states) of the D-branes.
	
	\item From the last problem we can write:
	In terms of $\d_+, \d_-$, we have:
		\begin{equation}\label{eq:electric}
		\begin{pmatrix}
			\d_+ X^0\\
			\d_+ X^1
		\end{pmatrix}=
			\begin{pmatrix}
				\frac{1+\mathcal E^2}{1- \mathcal E^2} & \frac{2\mathcal E}{1-\mathcal E^2}\\
				\frac{2\mathcal E}{1-\mathcal E^2} & \frac{1+\mathcal E^2}{1- \mathcal E^2}\\
			\end{pmatrix}
			\begin{pmatrix}
				\d_- X^0\\
				\d_- X^1
			\end{pmatrix}
		\end{equation}
		Here $\mathcal E = 2\pi \ell_s^2 E$. Taking $\mathcal E$ close to zero recovers NN boundary conditions on $X^0$, $X^1$.
		
		\begin{center}
			\includegraphics[scale=0.13]{"Drawings/Electric"}
		\end{center}
		
		 (It's worth noting that that the speed of light here will translate to a maximum electric field $|E| < T$ on the brane. This provides one motivation for the necessity of a nonlinear electrodynamics, namely DBI). 
	 taking $E= 0$ give N boundary conditions on $X^0, X^1$. T-dualizing $X^1$ gives $D$ boundary conditions on $\tilde X^1$. Now, boosting the brane along $X^0$ and  
	\[
	\begin{aligned}
	&\begin{pmatrix}
		\d_+ X^0\\
		\d_+ \tilde X^1
	\end{pmatrix}=
		\begin{pmatrix}
			\gamma & v \gamma\\
			v \gamma & \gamma
		\end{pmatrix}
		\begin{pmatrix}
			1 & 0\\
			0 & -1
		\end{pmatrix}
		\begin{pmatrix}
			\gamma & -v \gamma\\
			-v \gamma & \gamma
		\end{pmatrix}
		\begin{pmatrix}
			\d_- X^0\\
			\d_- \tilde X^1
		\end{pmatrix}\\
		&\Rightarrow \begin{pmatrix}
		\d_+ X^0\\
		\d_+  X^1 
	\end{pmatrix}
	=\begin{pmatrix}
			\gamma & v \gamma\\
			v \gamma & \gamma
		\end{pmatrix}
		\begin{pmatrix}
			1 & 0\\
			0 & -1
		\end{pmatrix}
		\begin{pmatrix}
			\gamma & -v \gamma\\
			-v \gamma & \gamma
		\end{pmatrix}
		\begin{pmatrix}
			1 & 0\\
			0 & -1
		\end{pmatrix}
		\begin{pmatrix}
			\d_- X^0\\
			\d_- \tilde X^1
		\end{pmatrix} = \begin{pmatrix}
			\frac{1+v^2}{1-v^2} & \frac{2v}{1-v^2}\\
			\frac{2v}{1-v^2} & \frac{1+v^2}{1-v^2}
		\end{pmatrix}
		\begin{pmatrix}
			\d_- X^0\\
			\d_- \tilde X^1
		\end{pmatrix} 
	\end{aligned}
	\]
	This exactly what we had before, with $v = 2 \pi \ell_s E$.
	
	This argument is quite simple from the abstract picture: taking $A_1 = E X^0$, $T$-dualizing in the direction of $X^1$ gives a D-brane lying at $X_1 = 2 \pi \ell_s^2 E X^0$ giving a velocity $2 \pi \ell_s^2 E$. This can also be obtained by analytic continuation of question \textbf{8.13}. 
	
	\item Using the fact that this scattering problem is exactly T-dual to the electric field problem mentioned before, we return to \eqref{eq:amplitude} and consider $b= \Delta x$ to be small. In this case the large $t$ regime dominates (corresponding to a loop of light open strings). First we perform a modular transformation to get
	\[
		\mathcal A = \frac{V_p}{(8\pi^2 \ell_s^2)^{p/2}} \int_0^\infty \frac{dt}{t} t^{(6-p)/2} e^{-\frac{t b^2}{2 \pi \ell_s^2}} \frac{\theta \twist11(i \nu/2, i/t)^4}{\eta(i/t)^9 \theta \twist11 (i \nu, i/t)}
	\]
	Now we follow Polchinski and rewrite $\mathcal A$ in terms of an integral over the particle's path $r(\tau)^2 = b^2 + v^2 \tau^2$, $\mathcal A = -i \int d\tau V(r(\tau), v)$. Then we get $V$ from reversing a Gaussian integral to be
	\[
		V(r, v) = i \frac{2V_p}{(8 \pi^2 \ell_s^2)^{(p+1)/2}} \int_0^\infty dt t^{(5-p)/2} e^{-\frac{t r^2}{2 \pi \ell_s^2}} \frac{\tanh \pi \nu \, \theta \twist11(i \nu/2, i/t)^4}{\eta(i/t)^9 \theta \twist11 (i \nu, i/t)}
	\]
	The large-$t$ limit is now direct:
	\[
		V(r, v) = - \frac{2V_{p+1}}{(8 \pi^2 \ell_s^2)^{(p+1)/2}} \int_0^\infty \frac{dt}{t^{(1+p)/2}} e^{-\frac{t r^2}{2 \pi \ell_s^2}} \frac{\tanh u \, \sin^4(u t/2)}{\sin(u t)}
	\]
	Using steepest descent at zeroth order, the $t$ that dominates is of order $2 \pi \ell_s^2/r^2$ so that $u t \approx 2 \pi \ell_s^2 v/r^2$ is the leading contribution. \textbf{Justify why $u t$ is $O(1)$}. This then gives that $r \sim \ell_s \sqrt{v}$. If we go at very small velocities we can probe below the string scale. 
	
	\begin{center}
		\includegraphics[scale=0.13]{"Drawings/Dbrane Dbrane Scatter"}
	\end{center}
	
	On the other hand, a slower velocity means that the time it takes to probe this distance is longer $\delta t = r/v = \ell_s/\sqrt{v}$. This implies that $\delta x \delta t \geq \ell_s^2$. This looks like a type of \emph{noncommutative geometry} with $\alpha'$ playing the role of Planck's constant now.
	
	Combining this with the usual position-momentum uncertainty relation 
	\[
		1 \leq \delta x \, m \delta v = g \ell_s \delta x \delta v \Rightarrow \Delta x = \frac{g \ell_s}{\delta v},
	\]
	 we can minimize simultaneously $\ell_s \sqrt v, g \ell_s/v$ by having $v \sim g^{2/3}$ giving $\delta x = \ell_s g^{1/3}$. At weak coupling this is smaller than the string scale. 
	
	\item The action is a factor of two off from Polchinski's. The momenta are $p_i = \frac{2}{g_s \ell_s} (\dot X^i + [A_t, X^i])$. Then we get:
	\[
		H = \int dt \Tr\Big[\frac{g_s \ell_s}{4} p_i p^i + \frac{1}{2 g_s \ell_s (2\pi \ell_s^2)^2} [X^i, X^j]^2\Big]
	\]
	Defining $g^{1/3} \ell_s Y^i = X^i, p_i = p_{Y_i}/g_s^{1/3} \ell_s$, this sets the length scale, which coincides with what we got in the previous question by less rigorous arguments. Now we have $Y^i$ is dimensionless and get a hamiltonian
	\[
		H = \frac{g_s^{1/3}}{\ell_s} \int dt \Tr\Big[\frac14 p_{Y_i} p_{Y_i} + \frac{1}{2 (2\pi )^2} [Y^i, Y^j]^2\Big]
	\]
	So the only dimensionful content of this hamiltonian comes from $g, \ell_s$ appearing in the overall normalization. This gives an energy scale of $g_s^{1/3}/\ell_s$. For strong coupling $g_s > 1$ this probes deeper than the string scale. 
	
	\item This is pretty direct. Since the metric $G_{\mu \nu}$ does not depend on $X^i$ for $i=p+1 \dots 9$, we have that each $X^i$ is killing, in particular the metric takes a block-diagonal form where only the first $(p+1)^2$ $G_{\mu \nu}$ entries have nontrivial coordinate dependence and the remaining metric is just the identity matrix $\delta_{ij}$ along the $X_i$ directions (we didn't even have to do this since 8.5.1 has $\eta_{\mu \nu}$ the flat metric. Is my logic here even right?). 
	
	Take the ansatz $A \to (A_\mu, \Phi_i)$. We thus get $F_{\mu \nu}$ in the first $(p+1)^2$ entries and $\partial_\mu \Phi^i$ in the off-block-diagonal piece. We can rewrite this as a determinant of just the $(p+1)$ piece \textbf{Justify this step}
	\[
		\sqrt{-\det(G_{\mu \nu} + 2 \pi F_{\mu \nu} + \d_\mu \Phi^i \d_\nu \Phi^i)}
	\]
	
	\item The bosonic part of this is immediate. Write the fields $A_i$ in the dimensionally reduced dimensions as $X^I$ and we immediately get $\Tr F^2_{10} \to \Tr [F_{d+1}^2 + 2 [D_\mu, X^I]^2 + \sum_{I, J} [X^I, X^J]^2]$. The fermionic part will reduce to:
	\[
		(\Tr \bar \chi \Gamma^\mu D_\mu \chi)_{10D} \to \Tr [\bar \chi \Gamma^\mu D_\mu \chi + \bar \lambda_a \Gamma^{i} [X_i , \lambda^b] ] 
	\] 
	Where now the $\chi_i$ are fermions that break the \textbf{16} representation of $\SO(9,1)$ into a representation of $SO(d-1, 1) \times SO(10-d)$. For $d = 3$ we get $\mathcal N = 4$ SYM and this is $(2, 4) + (\bar 2, \bar 4)$, corresponding to four Weyl spinors. 
	\item At the minimum of the potential, all $X^I$ lie in a cartan and mutually commute. The $A_{ij}$ correspond to open strings moving between the D-branes at positions $X_I$. The ground states of these open strings have a mass squared of $(X_I - X_J)^2/2 \pi \ell_s^2$, so indeed the mass is linear in the separation. \textbf{Confirm. Understand Lie-Theoretic perspective}.
	
	\item The worldvolume coupling to the RR 2-form looks like $i T_2 \int C_2$. For the brane tilted in the $x^1, x^2$ plane we can write this explicitly as:
	\[
		i \int dx^0 d x^1 (C_{01} + C_{02} \tan(\theta))
	\]
	Now T-dualizing in the $x^2$ direction changes the $C_{01}$ form to the RR 3-form $C_3$, giving the standard $i \int C_3$ term. On the other hand, the second term gets reduced to $-2 \pi \ell_s^2  \int dx^0 dx^1 dx^2 C_0 F_{12}$, where I have used exercise \textbf{8.13} to write $\tan \theta = -2\pi\ell_s^2 F_{12}$. So we get a leading coupling to the three-form and a sub-leading coupling (in $\ell_s^2$) to the one-form. This is a hint of the \emph{Meyers effect}.

	\item For the D2 brane the CP odd terms are $C_3$, $C_1 \wedge \mathcal F$ and $-C_1 \wedge \frac{(2\pi \ell_s)^4}{48} [p_1(\mathcal T) - p_1(\mathcal N)]$. In the frame described by exercise \textbf{7.26}, $C_3$ transforms trivially under $A$ transformations, and under its own gauge transformations it only adds an exact term which does not modify the CS action. 
	
	$\mathcal F$ transforms trivially under $A$ transformations and under $B$ transformations only modifies the action by a closed term again. 
	
	Equation \textbf{I.14} is not in any standard frame. The Dilaton plays no role here. The 
	
	\textbf{I feel I am missing something.}
	
	\item Gauge transformations of the axion $C_0$ are just shifts $C_0 \to C_0 + a$. $C_0$ couples to $F_2$ through the Chern-Simons term:
	\[
		\int C_0 \wedge \Tr e^{\mathcal F} \wedge \mathcal G
	\]
	Because of the Bianchi identity, $\dd \mathcal F = 0$, and the same holds for any trace of any polynomial of $F$. Similarly $\mathcal G$ is also a closed form. Therefore shifting $C_0$ gives an integrand term $\Tr e^{\mathcal F} \wedge \mathcal G$ which is closed.
	
	\item For trivial flat-space background $\eta_{\mu \nu}$, we have $g_{ab} = \d_a X_\mu \d_b X^\mu$. Take $M_{ab} = \d_a X_\mu \d_b X^\mu +  2\pi \ell_s^2 F_{ab}$ and $M = \det M_{ab}$. Taking the DBI variation w.r.t. $X_\mu^a$ and $A_a$ respectively gives:
	\[
	\begin{aligned}
		\frac{T}{2} \d^a \left( \sqrt{-M} M^{-1}_{ab} \d_b X^\mu \right) = 0\\
		 \frac{2 \pi \ell_s^2 T}{2} \d^a \left(\sqrt{-M} M^{-1}_{ab} \right) = 0
	\end{aligned}
	\]
	Its rather nasty to evaluate that inverse matrix. On the other hand, taking $X^9$ to be the only nontrivial function of the $\xi$, and depending only on the radial distance $r$ from a central point, and setting all $A_i = 0$ with $A_0$ a function of $r$ alone, we get $M = 1 + \delta_{a=r, b=r} (\d_r X^9)^2 + 2 \pi \ell_s^2 (\delta_{a=r, b=0} - \delta_{a=0, b=r}) E$. We take $E = \d_r X^9$. The determinant is then $(\nabla X^9)^2 (1 + 2\pi \ell_s^2)$. 
	
	Note that if the second equation of motion holds, the first equation of motion implies that  we would want for $\d^r \d_r X^9 = 0$, namely that $X^9$ is a harmonic function of $r$. On a $p$ brane this is $X^9 = \frac{C_p}{r^{p-2}}$. In this case, the determinant, as well as $M^{-1}$ will vanish when covariantly differentiated by $\d^r$, giving us our desired second equation of motion. 
	
	This solution is known as a BI-on (BI for Born-Infeld). It represents an infinitely long open string ending on our $p$-brane. 
	
	\item Let's have $G, B, \Phi, C_2$ trivial. We get
	\[
		S = \frac{1}{2\pi \ell_s^2 g} \int d^2 \xi \sqrt{1-(2\pi \ell_s^2 F_{01})^2} + \frac{1}{2\pi \ell_s^2} \int d^2 \xi \, C_0 (2 \pi \ell_s^2) F_{01}
	\]
	Pick the gauge $A_0 = 0$. Our variable is then $A_1$. From this, we get a canonical momentum conjugate to $A_1$ equal to:
	\[
		-\frac{1}{2 \pi \ell_s^2 g} \frac{(2 \pi \ell_s^2)^2 F_{01}}{\sqrt{1-(2\pi \ell_s^2 F_{01})^2}} + C_0 F_{01}
	\]
	Consider putting the D1 brane in a circle. Now since $C_0$ acts as a $\theta$ term, consider putting an integer $m$ for $C_0$. The momentum is quantized, and in particular there is a gap between the zero momentum ground state and the next state up. We get:
	\[
		\frac{2\pi \ell_s^2 F_{01}}{\sqrt{1- (2 \pi \ell_s^2 F_{01})^2}} = g m \Rightarrow 2 \pi \ell_s^2 F_{01} = \frac{g m}{\sqrt{1 + m^2 g^2}}
	\]
	We have a Hamiltonian
	\[
		\mathcal H = \frac{1}{2 \pi \ell_s^2 g} \frac{1}{\sqrt{1-(2\pi \ell_s^2 F_{01})^2}} 
	\]
	And from the quantization condition on the electric field we obtain from the Hamiltonian a set of quantized tensions
	\[
		T = \frac{1}{2\pi \ell_s^2 g} \sqrt{1 + m^2 g^2}.
	\]
	\begin{center}
		\includegraphics[scale=0.13]{"Drawings/Dissolve F1"}
	\end{center}
	
	\item The D$_3$-D$_{-1}$ system has $\#$ND$=4$ and so preserves $1/4$ supersymmetry ($1/2$ the SUSY of the D3 brane itself). Similarly, the instanton configurations satisfying $\star F_2 = \pm F_2$. The supersymmetric variation of the gaugino is $\delta \lambda \propto F_{\mu \nu} \Gamma^{\mu \nu}$. The $\Gamma^{\mu \nu}$ are generators of $\SO(4) = \SU(2) \times \SU(2)$, and the (A)SD conditions on $F_2$ will ensure that only half the generators (the first or second $\SU(2)$) will appear in the variation. Thus instanton configurations are also 1/2 BPS on the worldvolume.
	
	To confirm that these instantons really \emph{are} D$_{-1}$ branes, note that the CS term contains $\frac12 (2\pi \ell_s^2)^2 T_3 \int C_0 F_2 \wedge F_2$. For a nontrivial instanton configuration we get $\int F_2 \wedge F_2 = 8 \pi^2$. Thus the instanton coupling to $C_0$ is $(2 \pi \ell_s)^4 T_3 = T_{-1}$, exactly the charge of the D$_{-1}$ brane. \textbf{Does this exclude the possibility of objects with the same charge and BPS properties as $D_{-1}$ branes, but that don't have interpretations as endpoints of open strings?}
	
	 I expect the moduli space to have dimension $4 n$, corresponding to the space (technically Hilbert scheme) of $n$ points on $\mathbb R^4$.
	
	\item This configuration is invariant under $x^1$ translations as well as under time $x^0$. The exact same BPS properties discussed in the previous question apply here. The state is half-BPS on the worldvolume both from the POV of string theory and from the POV of the low energy SYM theory having half the gaugino variations vanish. 
	
	The same instanton action argument in the previous question gives us that $\frac12 (2\pi \ell_s^2)^2 T_4 \int C_1 F_2 \wedge F_2$ yields a coupling $(2\pi \ell_s)^4 T_4 = T_0$ to the $C_1$ form.  
	
	For $N$ D5 branes the low-energy effective theory is $\SU(N)$ SYM, and we obtain the moduli space of $SU(N)$ instantons. The dimension now becomes $4 N k$ \textbf{justify} . I expect that the moduli spaces of D1-D5 bound states are identical to the moduli space in the previous problem. 
	
	\item First, the pair of 5-branes in the 12345 and 12367 dimensions are parallel in the 123 directions and $90\degree$ rotated in two directions. This gives 2 sets of $ND$ and 2 sets of $DN$ boundary conditions, on the strings which gives us $\nu = 4$. In this case, following Polchinski the spinor $\beta = {\beta^{\perp}}^{-1} {\beta^\perp}'$ has an equal number of eigenvalues $-1$ and $1$. So half of the original $16$ spinors preserved by the first D-brane will be preserved by the combination of both. 
	
	Now take a third $D$-brane in the 12389 direction, perpendicular to both the first two. The same argument shows that we brane another half of the supersymmetry, giving $4$ supersymmetries left in this configuration. In other words it is $\frac18$ BPS. \textbf{Confirm}
	
	\item Adding a D1 string gives 4 ND boundary conditions with each of the other D-branes. This breaks the supersymmetry in half again, preserving $2$ supersymmetries now. It is $\frac{1}{16}$ BPS. 
	
	If I were to add it along direction 4 it would have 5 ND boundary conditions with the second two branes, which preserves no SUSY, so the latter configuration has nothing preserved. 
	
	\item Note this is $\mathrm O(2)$ and not $\SO(2)$, so instead of getting one $D$-brane at $\theta \tilde R$ and the other at $-\theta \tilde R$, we get one ``half'' D8 brane at $x^9 = 0$ (the location of one orientifold plane) and its image at $\pi \tilde R$ (the location of the other).
	
	
	\item % Any symplectic matrix has a Bloch-Messiah decomposition $A = O Z O'$, where $Z$ is a diagonal matrix of the form $(D, D^{-1})$.
	We work with the compact real form $\mathrm{USp}(2N) = \mathrm{Sp}(2N) \cap \mathrm U(2N)$. In this case any symplectic matrix can again be diagonalized to be of the form $(e^{i \theta_1}, e^{-i\theta_1}, \dots )$. Again we interpret this as D-branes on both sides $\pm \theta_i \tilde R$ of the orientifold plane. The generic gauge group is $U(1)^{2N}$. If $m$ branes lie at either orientifold plane we get an enhancement $\mathrm{Sp}(2m)$. When all $N$ branes and their images lie on one of the orientifold planes, we recover the full symmetry. 
	
	\item Due to the negative tension, an excitation on it has even lower energy, corresponding to a negative norma state which is forbidden in a unitary theory by positivity. \textbf{What more can I say? }
	
	\item There is a mistake in Kiritsis' equation \textbf{G.8}. We should have $A = -H^{-1}(\rho)$ not $-H(\rho)$. The way to see that is by noting that $-H^{-1}(\rho) = -\frac{\rho}{\rho+Q} = -\frac{r-Q}{r} = 1 + \frac{Q}{r}$. The constant $1$ is gauge and hence irrelevant, while the second term is the proper electric potential that will give rise to a $F_{tr} = \frac{Q}{r^2}$.
	
	Also, this problem asks us to work in $\mathcal N =2, D = 4$ SUGRA, so the appropriate equations should be that the variation of \emph{each} dilatino by a Killing spinor is zero. In this SUGRA, there are two Majorana gravitinos $
\psi_{\mu, A}, A=1,2$ with four components each, for a total of $8$ SUSYs. Consequently, the variations involve two Majorana spinor parameters $\epsilon_{A}, A=1,2$. We will use lower indices to indicate chiral and upper indices to indicate anti-chiral fermions. The gravitino variation is then (c.f. Freedman \emph{Supergravity} Section 22.4)
\begin{equation}\label{eq:SUSYvar}
		\delta \psi_{\mu, A} = \nabla_\mu \epsilon_A - \frac14 F_{\nu \rho} \gamma^{\nu \rho} \gamma_\mu \varepsilon_{AB} \epsilon^B
\end{equation}
	Here we have $\nabla_\mu = \d_\mu + \frac14 \omega_{\mu ab} \Gamma^{ab}$ with $\omega$ spin connection \footnote{This corresponds to setting $\kappa = \sqrt 2$ in Friedman's \emph{Supergravity} \textbf{22.69}.}

	Because of the chirality we can replace $F$ with $F^-$ in the above equation. Now, if we use spatial coordinates $\vec x, |x| = \rho$, the metric takes the form
	\[
		ds^2 = -H^{-2} dt + H^2 d\vec x^2 
	\]
	Take $e^{2A}= H^{-2}$, then we have the frame fields $e^{\hat 0} = e^{A} dt, e^{\hat i} = e^{-A} dx^i$. We will use hats to denote frame indices $a,b$. Our spin connection is then:
	\[
		\omega_{\hat 0 \hat i} = - e^{2A} \d_i A \,dt, \quad \omega_{\hat i \hat j} = - \d_j A dx^i + \d_i A dx^j
	\]
	First let's look at the $\mu= 0$ constraint of equation \eqref{eq:SUSYvar}
	\[
		\d_t \epsilon_A + \frac14 \omega_{0\, ab} \Gamma^{ab} \epsilon_A - \frac14 F_{\nu \rho} \gamma^{\nu \rho} \gamma_0 \varepsilon_{AB} \epsilon^B
	\]
	Now because the solution is static, $\d_t$ is killing and we expect that $\d_t \epsilon_A = 0$. Further, the only contribution to $\omega_{0\, ab}$ is $\omega_{0 \, \hat 0 \hat i}$ since only this has a $dt$ (NB the double sum gives a factor of 2). Similarly for the second term, since there is only an electric field, we only care about $\nu,\rho \in \{0,i\}$ (NB the double gives a factor of 2). Finally, we have only an electric field $F_{0i} = -\d_i A_t$ ($A_t$ is the vector potential, not to be confused with $A$). This yields:
	\[
	\begin{aligned}
		&-\frac12 e^{2A} \, \partial_i A\; \gamma^{\hat 0} \gamma^{\hat i} \epsilon_A  - \frac12 (-\partial_i A_t) \gamma^{0} \gamma^{i} \gamma_0 \varepsilon_{AB} \epsilon^B = 0\\
		&\Rightarrow -\frac12 e^{A} \, \partial_i e^{A}\; \gamma^{\hat i} \gamma^{\hat 0} \epsilon_A  + \frac12 (-\partial_i A_t) \gamma^{i} \varepsilon_{AB} \epsilon^B = 0\\
		&\Rightarrow e^A \, \partial_i e^A \; \gamma^{\hat i} \gamma^{\hat 0} \epsilon_A  - \partial_i A_t \; e^A \gamma^{\hat i} \varepsilon_{AB} \epsilon^B = 0\\
		 &{\Rightarrow \partial_i e^{A} \; \gamma^{\hat 0} \epsilon_A = \partial_i A_0 \,  \varepsilon_{AB} \epsilon^B} 
	\end{aligned}
	\]
	Here, the hatted $\gamma$-matrices are the familiar ones from flat space. 
	We thus need (up to an irrelevant gauge constant) $A_0 = \pm H^{-1}$ and
	\begin{equation}\label{eq:spinorconstraint}
		\epsilon_A = \mp \gamma^{\hat 0} \varepsilon_{AB} \epsilon^B
	\end{equation}
	Since we require $-H^{-1}$ to match the electromagnetic potential $A$, and so that it is asymptotically unity, we have $H = (1 - \frac{Q}{r})^{-1} = \frac{\rho+Q}{\rho} = 1 + \frac{Q}{\rho}$. This verifies the extremal RN solution. 
	
	We have not yet derived the spatial dependence of $\epsilon$. Taking $\mu = i$ we get
	\[
		\d_i \epsilon_A + \frac14 \omega_{i \, ab} \Gamma^{ab} \epsilon_A - \frac14 F_{\nu \rho} \gamma^{\nu \rho} \gamma_i \varepsilon_{AB} \epsilon^B
	\]
	Now we must use that $\omega_{i\, \hat j \hat k} = - \d_n A (\delta_{ij} \delta^n_k - \delta_{ik} \delta^n_j)$. Using the $\mu=0$ constraint we get:
	\[
		\begin{aligned}
			&\d_i \epsilon_A - \frac12 \d_k A \;  \gamma_{i\, \hat k} \epsilon_A - \frac12 F_{0i} \gamma^{0} \cancel{\gamma^i \gamma_i} \varepsilon_{AB} \epsilon^B = 0\\
			&  \d_i \epsilon_A + \frac12 \d_k A\;  e^{-A}\,  \gamma^{i\, \hat k} \epsilon_A \mp  \frac12 \d_i e^{A} H \gamma^{\hat 0} \varepsilon_{AB} \epsilon^B = 0\\
			&\d_i \epsilon_A + \frac12 \d_k A \,  \gamma^{\hat i\, \hat k} \epsilon_A - \frac12 \d_i e^{A} e^{-A} \epsilon_A = 0
		\end{aligned}
	\]
	Now the $\gamma^{\hat i \hat k}$ is nothing more than a \emph{generator of rotations} acting on $\epsilon_A$. Since we are assuming spherical symmetry, $\gamma^{\hat i \hat k} \epsilon_A = 0$ and we are left with the differential equation:
	\[
		\d_i \epsilon_A = \frac12 \d_i A  \epsilon_A \Rightarrow \epsilon_A = e^{1/2 A} \epsilon_0
	\]
	where $\epsilon_0$ is a constant spinor satisfying \eqref{eq:spinorconstraint}.
	
	Because the constraint $\epsilon_A = \mp \gamma^{\hat 0} \varepsilon_{AB} \epsilon^B$ applies to half the space of spinors at any given point, we have that the extremal RN solution is half-BPS. 
	
	\item 
	Again take coordinates $x_i$ so that
	\[
		ds^2 = -\frac{dt^2}{H^2(\rho)} + H^2(\rho) (dx_i^2)
	\]
	 Upon the choice $\epsilon_A = -\mp \gamma^{\hat 0} \varepsilon_{AB}  \epsilon^B$, we had the relation $\d_i H^{-1} = \d_i A_0 = F_{i0}$. The field equation $\d \star F = 0$
	 \[
	 	\d_i \sqrt{-g} g^{00} g^{ii} F_{i0} = \d_i H^2 \d_i H^{-1} = \d_i^2 H(x_i)
	 \]
	 Thus we have that $H$ is a harmonic function of the \emph{flat} Laplacian in transverse space. 
	
	We see that $H$ from the previous problem takes the simple form $1+ \frac{Q}{|x|}$, which is obviously harmonic in 3+1 dimensions.
	
	A more general solution allowing for for multiple charged extremal black holes amounts to nothing more than replacing $H$ with $1 + \sum_i \frac{Q_i}{|x - x_i|}$, which remains harmonic, and thus preserves half supersymmetry. This looks like a bunch of extremal black holes whose pairwise electric repulsions cancel their gravitational attractions.
	
	\item Again we are working in $\mathcal N = 2$ SUGRA. The metric takes the form
	\[
		ds^2 = Q^2 \left[ \frac{-dt^2 + d\rho^2}{\rho^2} + d \Omega_2^2 \right]
	\]
	This corresponds to 2D AdS times a sphere of \emph{constant radius}. Both spaces have radius $Q$. The spinor equation is as before, now with $T_{\mu \nu} = -\frac{1}{L} g_{\mu \nu}$. Consider just AdS$_{2}$. Define the operator 
	\[
		\hat D_{\mu} \epsilon_A = \nabla_\mu \epsilon_A - \frac{1}{2 Q} \gamma_{\mu} \varepsilon_{AB} \epsilon^B
	\]
	Consider now
	\[
		[\hat D_{\mu}, \hat D_\nu] \epsilon_A = (\frac14 R_{\mu \nu a b} \gamma^{a b} + \frac1{2Q^2}) \epsilon
	\]
	And since AdS$_{2}$ is a maximally symmetric space $R_{\mu \nu a b} = - (e_{a\mu} e_{b\nu} - e_{a \nu} e_{b \mu})/L^2$ and so the commutator vanishes identically. This is the integrability condition we need. At each point, the spinor bundle is dimension $\mathcal N \times 2^{[D/2]}$ - for $\mathcal N=2$ AdS2 this is $4$. We see that any spinor can be transported by the connection $\hat D_{\mu}$ to define a spinor field on all of AdS$_{2}$, and thus we get that the space is maximally supersymmetric.
	
	The exact same arguments (with $Q \to -Q$) apply for the positively curved 2-sphere of the same radius. 
	
	The product of two maximally supersymmetric spaces is maximally supersymmetric \textbf{Confirm}. We get $8$ Killing spinors. 
	We now see that the Bertotti-Robertson universe preserves full supersymmetry, and thus the extremal RN black hole plays the role of a \emph{half-BPS soliton} that interpolates between two fully supersymmetric backgrounds (flat space and AdS$_{2} \times S^2$). 
	
	% \[
	% 	\delta \psi_A = D_\mu \epsilon_A  - \frac{1}{2 Q} \gamma_\mu \varepsilon_{A B} \epsilon^{B} = (\d_\mu + \frac14 \omega_{\mu ab} \gamma^{ab}) \epsilon_A - \frac{1}{2 Q} \gamma_\mu \epsilon_{AB} \epsilon^B
	% \]
	
	
	
	\item The variation of $\frac{1}{2(p+2)!} F_{p+2}^2 = F_{p+2} \wedge \star F_{p+2}$ directly gives $\star F_{p+2} = 0$. 
	
	Varying the dilaton gives 
	\[
	\begin{aligned}
		0 = -2 e^{-2 \Phi} [R + 4 (\nabla \Phi)^2] - \nabla (e^{-2 \Phi} 8 \nabla \Phi) &= -2 e^{-2 \Phi} [R + 4 (\nabla \Phi)^2] + 16 e^{-2 \Phi} (\nabla \Phi)^2 - 8 e^{-2 \Phi} \Box \Phi\\
		&\Rightarrow R = 4 (\nabla \Phi)^2 - 4 \Box \Phi
	\end{aligned}	
	\]
	Finally, writing the metric explicitly in the action
	\[
		\sqrt{-g} e^{-2\Phi} [g^{\mu \nu} R_{\mu \nu} + 4 g^{\mu \nu} \d_\mu \Phi \d_\nu \Phi] - \frac{1}{2(p+2)!} \sqrt{-g} F^2_{p+2}
	\]
	Let's look how each term changes when we vary $\frac{1}{\sqrt{-g}} \frac{\delta}{\delta g^{\mu \nu}}$.
	\begin{itemize}
		\item $\sqrt{-g} e^{-2\Phi} R$
		 \[
		 \begin{aligned}
		 &\to (R_{\mu \nu} + g_{\mu \nu} \Box - \nabla_\mu \nabla_\nu) e^{-2\Phi}  - \frac12 g_{\mu \nu} e^{-2\Phi} R\\
		 &= e^{-2\Phi} \Big(R_{\mu \nu} - \frac12 g_{\mu \nu} R + g_{\mu \nu} (-2 \Box \Phi + 4 (\d \Phi)^2) - (-2 \nabla_\mu \nabla_\nu \Phi + 4 \d_\mu \Phi \d_\nu \Phi) \Big) 
		 \end{aligned}
		\]
		\item $\sqrt{-g} e^{-2\Phi} 4 g^{\mu \nu} \d_\mu \Phi \d_\nu \Phi \to 4 e^{-2\Phi} \d_\mu \Phi \d_\nu \Phi - 2 e^{- 2 \Phi} (\d \Phi)^2$
		\item $- \frac{1}{2(p+2)!} \sqrt{-g} g^{\mu_1 \nu_1} \dots g^{\mu_{p+2} \nu_{p+2}} F_{\mu_1 \dots \mu_{p+2}} F_{\nu_1 \dots \nu_{p+2}} \to - \frac{1}{2 (p+1)!} F_{\mu \nu}^2 + \frac{1}{4(p+2)!} g_{\mu \nu} F^2$. Here $F_{\mu \nu}^2 = F_{\mu \dots} F^{\nu \dots}$
	\end{itemize}
	Combining these all together and using the dilaton equations of motion gives
	\[
		e^{-2\Phi} R_{\mu \nu} + 2 \nabla_\mu \nabla_\nu \Phi - \frac{1}{2 (p+1)!} F_{\mu \nu}^2 + \frac{1}{4(p+2)!} g_{\mu \nu} F^2 = 0 \Rightarrow R_{\mu \nu} + 2 \nabla_\mu \nabla_\nu \Phi = \frac{e^{2\Phi}}{2(p+1)!} \Big(F_{\mu \nu}^2 - \frac{g_{\mu \nu}}{2 p+2} F^2\Big)
	\]
	exactly as required. 
	
	\item
		 This problem was very time-consuming to do out explicitly. The only resource that was of any help for cross-checking was Or\'tin's ``\emph{Gravity and Strings}''. 
	
	First consider the possibility of $\Phi = 0$, $F = 0$. In this case we have no stress tensor and are left with static vacuum Einstein equations, spherically symmetric in $10-p$ dimensions and translationally invariant in $p$ dimensions. In that case we know that our solution is nothing more than the Schwarzschild solution in $10-p$ dimensions times a transverse $p$-dimensional space:
	\[
		ds^2 = - f(r) dt^2 + dx_i^2 + \frac{1}{f(r)} dr^2 + r^2 d\Omega_{8-p}^2
	\]
	Reproducing the arguments from black holes in 4D, we see that 
	\[
		\frac{d}{dr} (r^{8-p} f(r) \frac{d}{dr} \log(f)) = \frac{d}{dr}(r^{8-p} f'(r)) = 0 
	\]
	ie $f(r)$ must be harmonic in the transverse dimension. After rescaling coordinates to have appropriate asymptotic behavior, we  get:
	\[
		f(r)  = 1 - \frac{r_0^{7-p}}{r^{7-p}}
	\]
	for some constant $r_0$ related to the ADM mass of the solution. So we see that $H(r) = 1$ when the dilaton and $p+2$-form field strength are turned off. The curvature is $R = 0$. Now let's turn on $\Phi$. We expect small $\Phi$ will correspond to small $H$. 
	
	To do this, let's look at the simplest case, $p=0$. Take the Ansatz (which you can convince yourself to be completely general given the symmetry of the problem)
	\[
		ds^2 = - \lambda(r) dt^2 + \frac{dr^2}{\mu(r)} + R(r)^2 d\Omega_{d-2}^2
	\]
	We will later set $d=10, \lambda = \mu = f(r)/\sqrt{H}$. Let's explicitly calculate the Christoffel symbols. There are three categories: $\Gamma$s involving just $r, t$ $\Gamma$s involving mixed $r, \Omega$, and $\Gamma$s involving just the $\Omega$ variables. I will use $a,b,c$ to index the angular variables $\psi_a$, whose metric is $R^2 d\Omega^2_{ab} = q_{a} \delta_{ab}$, and I use $'$ to denote ordinary partial differentiation w.r.t. $r$. 
	\begin{center}
			$\Gamma^r_{tt} = \frac12 \mu \lambda'$  \qquad $\Gamma^t_{tr} = \frac12 \lambda^{-1} \lambda'$ \qquad $\Gamma^r_{rr} = -\frac12 \mu^{-1} \mu'$\\
			$\Gamma^a_{rb} = \delta^a_b \frac{R'}{R}$ \qquad $\Gamma^r_{ab} = \delta_{ab} \mu (R^2)' \frac{q_a}{R^2}$ \\
			
		$\Gamma^{a}_{bc} = \theta_{b>c} \delta^a_b \cot \psi_b +  \delta_{c>b} \theta_{ac} \cot \psi_c - \theta_{b>a} \delta_{bc} \cot \psi_a \frac{q_b}{q_a}$
	\end{center}
	That last Christoffel symbol looks particularly nasty. Thankfully, by using the fact that the sphere is a symmetric space, we will not need to use it explicitly. 
	
	Now let's directly compute the Ricci tensor.
	\[
		R_{\mu \nu} = \d_{\rho} \Gamma^{\rho}_{\mu \nu} - \d_\mu \Gamma^{\rho}_{\rho \nu} + \Gamma^{\rho}_{\rho \sigma} \Gamma^\sigma_{\mu \nu} - \Gamma_{\mu \rho}^\sigma \Gamma^\rho_{\sigma \nu}
	\]
	In what follows, it is useful to recall the identity $\Gamma^\mu_{\mu \nu} = \d_\nu \log \sqrt{-g}$. 
	The nonzero terms will be $R_{tt}, R_{rr}, R_{ab}$. Respectively they are:
	\[
	\begin{aligned}
		R_{tt} &= \d_{r} \Gamma^{r}_{tt} - \cancel{\d_t \Gamma^{\rho}_{\rho t}} + \Gamma^{\rho}_{\rho r} \Gamma^r_{t t} - \Gamma_{t \rho}^\sigma \Gamma^\rho_{\sigma t}\\
		& = \frac12 (\mu \lambda' )' + \d_r \log \sqrt{g} \, (\mu \lambda') - 2 \frac12 \frac{\lambda'}{\lambda} \frac12 \mu \lambda' \\
		&= \frac12 \frac{\lambda}{\sqrt g} \d_r \Big(\sqrt{-g} \mu \frac{\lambda'}{\lambda} \Big) = \frac12 \lambda \nabla^2 \log \lambda
	\end{aligned}
	\]
	
	It's important for this next one to note that $q_a/R^2$ is independent of $r$. It's equally important to appreciate that the final combination of $\Gamma$ symbols is the only thing that would appear in the absence of $r$ dependence in $R$. In this case, because the $d-2$ sphere is a symmetric space, we'd have $R_{ab} = \frac{d-3}{R^2} g_{ab}$. Indeed, this is exactly what the final term gives. We thus get
	\[
	\begin{aligned}
		R_{ab} &= \d_{r} \Gamma^{r}_{ab} - \cancel{\d_a \Gamma^{\rho}_{\rho b}} + \Gamma^{\rho}_{\rho r} \Gamma^r_{a b} - \Gamma_{a \rho}^\sigma \Gamma^\rho_{\sigma b} 
		\\ &= -\frac12 \delta_{ab} \d_r \Big(\mu (R^2)' \frac{q_a}{R^2}\Big) + \d_r \log \sqrt{-g} \, \delta_{ab} \mu (R^2)' \frac{q_a}{R^2} + \frac{d-3}{R^2} q_a \delta_{ab} \\
		&= g_{ab} (- \nabla^2 \log R  +  \frac{d-3}{R^2}) 
	\end{aligned}
	\]
	The next one is a bit different. Less cancelation. The last term will sum over $(\sigma, \rho) = (r,r), (t,t), (a,a)$. Also note $\sqrt{-g} = \sqrt{\lambda/\mu} R^{d-2}$.
	\[
	\begin{aligned}
		R_{rr} &= \d_r \Gamma^{r}_{rr} - \d_r \Gamma^{\rho}_{\rho r} + \Gamma^{\rho}_{\rho \sigma} \Gamma^\sigma_{r r} - \Gamma_{r \rho}^\sigma \Gamma^\rho_{\sigma r} 
		\\ &= \cancel{- \frac12 (\mu^{-1} \mu')'} - \d_r^2 \log(\sqrt{\lambda/\cancel{\mu}} R^{d-2}) - \frac12 \d_r \log(\sqrt{\lambda/\mu} R^{d-2})  \; (\mu^{-1} \mu')  -\frac14 \frac{(\lambda')^2}{\lambda^2}  - \frac14 \mu^2 (\mu')^2 - (d-2) \frac{(R')^2}{R^2}\\
		&= - \frac12 \d_r^2 \log(\lambda) - \frac{d-2}{R} R'' -\frac12 (d-2) \frac{R'}{R} \d_r \log\sqrt{\lambda/\mu}\\
		&= -\frac12 \mu^{-1} \nabla^2 \log \lambda - \frac{d-2}{R} \sqrt{\frac{\lambda}{\mu}} \left(R' \sqrt{\frac{\mu}{\lambda}} \right)'
	\end{aligned}
	\]
	Altogether we get a Ricci scalar: 
	\[
		R = -\nabla^2 \log(\lambda R^{d-2}) + \frac{(d-2)(d-3)}{R^2} - \frac{d-2}{R} \sqrt{\lambda \mu} \left(R' \sqrt{\frac{\mu}{\lambda}} \right)'
	\]
	Now let's take $\lambda = \mu = f(r)/\sqrt{H}$ and $R = H^{1/4}$, $\sqrt{-g} = H^{(d-2)/4}$. 
	\[
		 -\nabla^2 \log(f(r) r^{d-2} H^{\frac{d-4}{4}}) + \frac{(d-2)(d-3)}{r^2 H^{1/2}} - (d-2) \frac{f(r)}{r H^{3/4}} (r H^{1/4})''
	\]
	The Laplacian takes the form $\frac{\d_r[ H^{(d-4)/4} r^8 f(r) \d_r ]}{r^{d-2} H^{(d-2)/2}}$ which simplifies the above to:
	\[
		 - \frac{\d_r^2 (f(r) r^{d-2} H^{(d-4)/4})}{r^{d-2} H^{(d-2)/2}} + \frac{(d-2) (d-3)}{r^{d-2} H^{1/2}} % + \frac{(d-2)}{2} \frac{f(r) H'(r)}{r H^{3/2}}
	\]
	The dilaton equation of motion is
	\[
		R = 4 (\nabla \Phi)^2 - 4 \nabla^2 \Phi
	\]
	Since a nonzero $\Phi$ is what gives an $H$ away from $1$, we might hypothesize a relationship $\log H \propto \Phi$, meaning we should replace $\Phi$ with $\log(H^{\alpha})$ in the dilaton equation. Let's also take $f(r)$ to be \emph{not different} from the Schwarzschild solution: $f(r) = 1- \frac{r_0^{d-3}}{r^{d-3}}$. 
	So far we will not be so bold as to assume \emph{anything} about $H$. We also at this point need to specialize to $d=10$, otherwise no nice simplification occurs. Straightforward algebra then gives:
	\begin{center}
		\includegraphics[scale=0.5]{"Figures/nonextremal"}
	\end{center}
	
	To get rid of the term quadratic in $H'(r)$ we need $3 + 8(1-2\alpha) \alpha = 0 \Rightarrow \alpha = 3/4$. 

	After that, these will only be equal if
	\[
		8 H' - r H'' = 0 \Rightarrow H = 1 - \frac{L^7}{r^7}
	\]
	The above solution is the most general given that $H \to 1$ as $r \to \infty$. 
	
	Now let us generalize this to higher dimensions. We add $p$ $x_i$ in the parallel dimensions of the solution.
	\[
		ds^2 = - \lambda(r) dt^2 + \nu(r) d\vec x_i^2 + \frac{dr^2}{\mu(r)}  + R(r)^2 d\Omega_{d-2}^2
	\]
	 This gives two new Christoffels. Here is a complete list
	\begin{center}
			$\Gamma^r_{tt} = \frac12 \mu \lambda'$  \qquad $\Gamma^t_{tr} = \frac12 \lambda^{-1} \lambda'$ \qquad $\Gamma^r_{rr} = -\frac12 \mu^{-1} \mu'$\\
			$\Gamma^a_{rb} = \delta^a_b \frac{R'}{R}$ \qquad $\Gamma^r_{ab} = \delta_{ab} \mu (R^2)' \frac{q_a}{R^2}$ \\
			
		$\Gamma^{a}_{bc} = \theta_{b>c} \delta^a_b \cot \psi_b +  \delta_{c>b} \theta_{ac} \cot \psi_c - \theta_{b>a} \delta_{bc} \cot \psi_a \frac{q_b}{q_a}$\\
		$\Gamma^r_{ij} = -\frac12 \delta_{ij} \mu \nu'$  \qquad $\Gamma^i_{rj} = \frac12 \delta^i_j \nu^{-1} \nu'$
	\end{center}
	Our nonzero Ricci components are now $R_{tt}, R_{rr}, R_{ab}, R_{ij}$. The primary way that the new dimensions will contribute is by modifying $\sqrt{-g}$ We get:
	\[
		R_{tt} = R_{tt}^{(10-p)} - \frac14 p \mu \lambda' (\log \nu)'
	\]
	\[
		R_{ab} = R_{ab}^{(10-p)} - \frac12 p g_{ab} \mu (\log \nu)' (\log R)'
	\]
	\[
		R_{rr} = R_{rr}^{(10-p)} + \frac12 p (\mu \nu)^{-1/2} ((\mu \nu)^{1/2} (\log \nu)')'
	\]
	\[
		R_{ij} =  \frac12 \delta_{ij} \nu \nabla^2 \nu
	\]
	This gives a Ricci curvature of:
	\[
		R = R^{(10-p)} + \frac12 p (-\nabla^2_{10-p} \log \nu - \nu^{-1} \nabla^2 \nu + \frac12 \mu ((\log \nu)')^2)
	\]
	Making the necessary replacements we get
	\[
	\begin{aligned}
		&- \frac{\d_r^2 (f(r) r^{d-2} H^{(d-4)/4})}{r^{d-2} H^{(d-2)/2}} + \frac{(d-2) (d-3)}{r^{d-2} H^{1/2}}\\
		& + \frac12 p \Bigg(-\frac{\d_r [f r^{8-p} H^{(6-p)/4} \d_r \log H^{-1/2}]}{r^{8-p} H^{(8-p)/4}}  - \frac{\sqrt{H}}{r^{8-p} H^{(8-2p)/4}} \d_r [f r^{8-p} H^{(6-2p)/4} \d_r H^{-1/2} ]  + \frac12 \frac{f}{H^{1/2}} (\d_r \log H^{-1/2})^2  \Bigg)
	\end{aligned}
	\]
	It makes sense to take the ansatz $f(r) = 1 - \frac{r_0^{7-p}}{r^{7-p}}$ and $H = 1 + \frac{L^{7-p}}{r^{7-p}}$. Further, the relationship between $H$ and $\Phi$ can be guessed from reasoning in the $p = 0$ case to go as $\Phi \propto H^{(3-p)/4}$, or alternatively we can establish this from first principles by algebra
	\begin{center}
		\includegraphics[scale=0.5]{"Figures/nonextremal2"}
	\end{center}
	This immediately gives that the dilaton term will equal the scalar curvature only when $\alpha = (3-p)/4$.
	
	We have thus proved the form of $f, H, \Phi$. Let's finally look at the RR field.
		%
	% This problem is really difficult if you try to calculate the scalar curvature directly. Moreover, it is made more difficult because by contrast to the the ansatz taken in almost all papers (eg Horowitz, Strominger):
	% \[
	% 	e^{A(r)} dx_{p+1}^2 + e^{B(r)} dx_i^2
	% \]
	% which makes calculation of all of these quantities a good deal easier.

	
	
		%
	%  The trick to do some clever dimension hopping combined with conformal transformation of the metric, given that we know from \textbf{C.17} that in $D$ dimensions when $g' = e^{\phi} g$, that
	% \[
	% 	R' = e^{-\phi} [R - (D-1) \nabla^2 \phi - \frac{(D-1)(D-2)}{4} (\nabla \phi)^2]
	% \]
	% Now there are four steps. Starting from the ansatz \textbf{8.8.4} that the metric takes the form
	% \[
	% 	ds_{10}^2 = \frac{-f(r) dt^2 + d\vec x^2}{\sqrt{H_p(r)}} + \sqrt{H_p(r)} \Big(\frac{dr^2}{f(r)} + r^2 d\Omega^2 \Big)
	% \]
	% Taking $H = e^{2\phi}$ we do a conformal transformation by $\sqrt{H} = e^{\phi}$, which gives
	% \[
	% 	ds_{10}^2 = -f(r) dt^2 + d\vec x^2+ H_p(r) \Big(\frac{dr^2}{f(r)} + r^2 d\Omega^2_{8-p} \Big)
	% \]
	% And the curvature of this metric is
	% \[
	% 	R_{\mathrm{\mathbf{ii}}} = e^{-\phi} [R_{\mathrm{\mathbf{i}}} - 9 \nabla^2 \phi  - 18 \nabla \phi^2 ]
	% \]
	% given that this is a product of a nontrivial geometry with a spacelike $p$-plane, we can ignore those $p$ dimensions and retain the same scalar curvature. Then upon doing a conformal transformation again by $H^{-1} = e^{-2\phi}$ (now in $10-p$ dimensions) we obtain a vaguely familiar-looking form:
	% \[
	% 	- \frac{f(r)}{H} dt^2 + \frac{dr^2}{f(r)} + r^2 d \Omega^2_{8-p}
	% \]
	% The scalar curvature of the original 10D solution is a differential function of $H$ added to the scalar curvature of this solution.
	%
	%
	% Keeping this form for $f$, letting $\phi$ be arbitrary gives the curvature for the solution $-\frac{f(r) dt}{e^{2\phi}} + \frac{dr^2}{f(r)} + r^2 d \Omega_{8-p}^2$ to be:
	% \[
	% 	2 \nabla^2 \phi - \nabla^2 \log f
	% \]
	% Note this laplacian is taken w.r.t. the metric involving $f, \phi$ so e.g. there will be terms quadratic in $\phi$.
	%
	% % \[
	% % 	\frac{\phi'(r)}{r} \left(2 (8-p) + \frac{r_0^{7-p}}{r^{7-p}} \right) + 2 f(r) [\phi''(r) - (\phi'(r))^2]
	% % \]
	% In terms of $10D$, the curvature of this is
	% \[
	% \begin{aligned}
	% 	R_{\mathrm{\mathbf{iii}}} &= e^{2\phi} \left[ e^{-\phi} [R_{\mathrm{\mathbf{i}}} - 9 \nabla^2 \phi  - 18 (\nabla \phi)^2 ] + 2 (9-p) {\nabla'}^2 \phi - 4 \frac{(9-p)(8-p)}{4} (\nabla' \phi)^2 \right]\\
	% 	&= e^{\phi} \left[ R_{10D} - 9 \nabla^2 \phi  - 18 (\nabla \phi)^2 + 2(9-p) {\nabla}^2 \phi - 2 (9-p)(8-p) (\nabla \phi)^2 - 4 \frac{(9-p)(8-p)}{4} (\nabla \phi)^2 \right]\\
	% 	&= e^{\phi} [R_{10D} + (9-2p) \nabla^2 \phi - 3 [(9-p)(8-p) + 6] (\nabla \phi)^2 ]
	% \end{aligned}
	% \]
	%
	% Look at the $\nabla^2 \phi$ term. Up to corrections higher order in $\phi$ this is exactly $2 \nabla^2 \phi$ in this geometry. Performing a Weyl transformation by $H = e^{2\phi}$ into \textbf{ii} will (again to leading order in $\phi$) gives
	% \[
	% 	R_{\mathrm{\mathbf{ii}}} = e^{-2\phi} (2 \nabla^2 \phi - 2 (9-p) \nabla^2 \phi)
	% \]
	% Then performing another Weyl transformation by $\sqrt{H}^{-1} = e^{-\phi}$ gives
	% \[
	% 	R_{\mathrm{\mathbf{iii}}} = e^{-\phi} (2 \nabla^2 \phi - 2 (9-p) \nabla^2 \phi + 9 \nabla^2 \phi) = - (7-2p) e^{-\phi} \nabla^2 \phi
	% \]
	%  	\textbf{I want $6-2p$ what is going wrong?}
	%
	% The equation for the dilaton in 10D reads
	% \[
	% 	R_{\mathbf{{i}}} =  4 f(r) e^{-\phi} \Phi'(r)^2 - 4  f(r) e^{-\phi} \left(\Phi''(r) + \frac{f'(r)}{f(r)} \Phi'(r) + (3-p) \phi'(r) \Phi'(r) + \frac{8-p}{r} \Phi'(r) \right)
	% \]
	% At leading order in $\Phi, \phi$ we see this goes as $-4 \nabla^2 \Phi$.
	% Comparing these two, we see that the only way we can have equality of these two expressions is if
	% \[
	% 	\Phi = \frac{3-p}{2} \phi + c_0 \Rightarrow e^{2 \Phi} = g_s^2 H^{(3-p)/2}(r)
	% \]
	% In fact Mathematica confirms in all relevant dimensions that these match exactly so long as $H = 1 + \frac{L^{7-p}}{r^{7-p}}$. \textbf{I am unhappy with this. Although it is true that Mathematica confirms it, I want to see how to get it from first principles}.
	
	For now let us ignore the issues with self-duality at $p=3$. Take the the $p+1$ form has flux in the radial but not angular directions in transverse space.  The only nonzero component of $F_{p+2}$ is given by $F_{r 0 \dots p}$. The equation of motion gives:
	\[
		\d_r\left(\sqrt{-g} g^{\mu_0 \nu_0} g^{\dots} F_{\mu_0 \dots} \right) = 0 
	\]
	Now we already have $\sqrt{g} = r^{8-p} H^{(4-p)/2}$, while we will have raising for each index $0 \dots p$ as well as $r$, giving a factor of $f(r) H^{(p+1)/2} H^{-1/2} / f(r) = H^{p/2}$. Altogether the differential equation becomes:
	\[
		\d_r r^{8-p} H^2 H^r
	\]
	Immediately we must have $F = \frac{\kappa/r^{8-p}}{H^2}$. This means that $F$ is proportional to $H'(r)/H(r)^2$. 
	
	I don't know how to easily get this constant of proportionality without knowing the decay properties of $R_{\mu \nu}$ as $r \to \infty$ \textbf{Return to this}. I know it must scale roughly as a positive power of $L$. It turns out to be:
	\[
		F_{r 0\dots p} = - \sqrt{1- \frac{r_0^{7-p}}{L^{7-p}}} \frac{H'_p(r)}{H^2_p(r)}
	\]
	
	\begin{center}
		\includegraphics[scale=0.25]{"Drawings/pBrane"}
	\end{center}
	
	
	\textbf{Can I do this all by somehow ``boosting'' Schwarzschild?}
	
	\item From the expression \textbf{8.8.9} of the electric field in terms of $H$ we have (assuming $p < 7$)
	\[
		E_r = (7-p) L^{(7-p)/2} \sqrt{L^{7-p} + r_0^{7-p}} \frac{r^{6+p}}{(r^{7} + L r^{p})^2} \to (7-p) L^{(7-p)/2} \sqrt{L^{7-p} + r_0^{7-p}} r^{p-8}
	\]
	Integrating this over the $8-p$ sphere will give
	\[
		T_p N = \frac{\Omega_{8-p} (7-p)}{2 \kappa_{10}^2} L^{(7-p)/2} \sqrt{L^{7-p} + r_0^{7-p}}.
	\]
	
	\item Using the standard ADM formula (cf, eg Carrol)
	\[
		M = \frac{1}{2 \kappa_{10}^2} \int_{S^8-p} g^{\mu \nu} (g_{\mu \alpha, \nu} - g_{\mu \nu, \alpha}) n^\alpha dS
	\]
	Then for a metric that looks like $g_{\mu \nu} = \eta_{\mu \nu} + h_{\mu \nu}$ with $h_{\mu \nu} = \frac{c_{\mu \nu}}{r^{7-p}}$.
	\[
	\begin{aligned}
		M = T_{00} V_p & = \frac{\Omega_{8-p}}{2 \kappa_{10}^2} ((7-p) c_{00} - \eta_{00} \eta^{ab} c_{ab} )\\
		& = \frac{\Omega_{8-p} V_p}{2 \kappa_{10}^2} ( (7-p) (r_0^{7-p} + \frac12 L^{7-p}) + r_0^{7-p} + \frac12 L^{(7-p)/4} )\\
		&= \frac{\Omega_{8-p} V_p}{2 \kappa_{10}^2} ((8-p) r_0^{7-p} + (7-p) L^{7-p} )
	\end{aligned}
	\]
	\textbf{Revisit- something seems off}
	% And expanding $\sqrt{H}, $ etc. to linear order, we get \textbf{I can't get agreement :( } that the ADM mass per unit volume parallel to the brane is
	% \[
	% 	\frac{\Omega_{8-p}}{2 \kappa_{10}^2} ((8-p) r_0^{7-p} + (7-p) L^{7-p})
	% \]
	% We can compactify the $p$ spatial components and multiply the above through by $V_p$ to get the correct result.
	
	\item Note that
	\[
		f_-(\rho) = 1 - \frac{L^{7-p}}{r^{7-p} + L^{7-p}} = \frac{1}{1+ L^{7-p}{r^{7-p}}} = \frac1{H(r)}
	\]
	Similarly
	\[
		f_+(\rho) = 1 - \frac{r_0^{7-p} + L^{7-p}}{r^{7-p} + L^{7-p}} = \frac{r^{7-p} + r_0^{7-p}}{r^{7-p} + L^{7-p}} = \frac{f(r)}{H(r)}
	\]
	This confirms that the $dt$ and $d\vec x$ terms are indeed consistent, and that the $f^{-1/2}_-(\rho)$ in front of the transverse part is our desired $\sqrt{H}$. Next note that:
	\[
		f_-(\rho)^{\frac{1}{7-p}} = \left[\frac{\rho^{7-p} - L^{7-p}}{\rho^{7-p}} \right]^{1/(7-p)} = \frac{r}{\rho} \Rightarrow \rho^2 f_-(\rho)^{1-\frac{5-p}{7-p}} = \rho^2 \frac{r^2}{\rho^2} = r^2
	\]
	So the angular part is consistent. Lastly, $f_+/f_- = f(r)$ which is the required coefficient for $dr^2$. It remains to cancel the jacobian:
	\[
		dr  = \frac{\rho^{6-p}} {r^{6-p}} d\rho = f_-^{-\frac{6-p}{7-p}} d\rho \Rightarrow dr^2 = f_-^{-\frac{12-2p}{7-p}} d \rho^2 = f_-^{-1-\frac{5-p}{7-p}} d \rho^2
	\]
	So
	\[
		\frac{\sqrt{H(r)}}{f(r)} dr^2 = f_-^{-1/2} \frac{f_-}{f_+} f_-^{-1-\frac{5-p}{7-p}} d \rho^2 = f_-^{-\frac12 - \frac{5-p}{7-p}} \frac{d\rho^2}{f_+(\rho)}
	\]
	
	\item Before doing any supersymmetric manipulations, we should know the spin connection.
	
	Take the extremal $p$-brane metric to be of the form
	\[
		ds^2 = e^{2 A(r)} dx^\mu dx^\nu \eta_{\mu \nu} + e^{2 B(r)} dx^i dx^j \delta_{ij}
	\]
	In this case we have $A = - B = \frac14 \log H(r)$. Take the frame fields
	\[
		e^{\hat \mu} = e^A dx^\mu, \quad e^{\hat i} = e^B dx^{i}
	\]
	Then
	\[
		\begin{aligned}
			\dd e^{\hat \mu} &= \d_r A e^{A} \dd r \wedge \dd x^\mu = \sum_i \d_i A e^A \dd x^i \wedge dx^\mu = e^{\hat i} \wedge \omega^{\hat \mu}_{\;\; \hat i} \Rightarrow  \omega_{\hat \mu \hat \nu} = 0, \; \omega_{\hat \mu \hat i} = (-)^{\mu=0} \d_i A \; e^{A - B} \dd x^\mu \\
			\dd e^{\hat i} &= \d_r B e^{B} \dd r \wedge \dd x^i = \sum_j \d_i B e^B \dd x^j \wedge dx^i = e^{\hat j} \wedge \omega^{\hat i}_{\; \hat j}
			\Rightarrow \omega_{\hat i \hat j} = \d_j B \dd x^i - \d_i B \dd x^j \\
		\end{aligned}
	\]
	Using our extremal form of the solution, I can further write
	\[
		e^{\Phi} = g_s^2 H^{(3-p)/4} = g_s^2 e^{(p-3) A}, \quad F_{r01\dots p} = \mp \frac{H'}{H^2} = \pm 4 A' e^{4A}
	\]
	The $\pm$ corresponds to brane/anti-brane solutions.
	% Here I have appropriately absorbed $g_s^2$ into the definition of $A$. We then will have $F_{p+2} = \mp H'/H^2$ to be proportional (with an appropriate choice of $\kappa$ that we will find) to $4A'/e^{4A}$. $\kappa$ is not to be confused with the gravitational constant, though from Freedman's book, it looks like indeed there is a connection. The sign in the equation above corresponds to a single brane/anti-brane solution.
%
	
	In 10D $\mathcal N = 2$ SUGRA coupled to matter, represent the Killing spinor as $\epsilon = {\epsilon^1 \choose \epsilon^2}$. We have the gravitino and dilatino variations:
	\[
	\begin{aligned}
		0 &=
		 \delta \psi_{\mu, A} & = & \; (\d_\mu + \frac14 \omega_{\mu}^{ab} \Gamma_{ab}) \epsilon + \frac{e^\Phi}{8} \slashed{F} \Gamma_\mu \mathcal P_{p+2} \epsilon \\
		0 &= \delta \lambda & =& \; \slashed{\d} \Phi \epsilon  + \frac{e^{\Phi}}{4} (-1)^p (3-p) \slashed F \mathcal P_{p+2} \epsilon
	\end{aligned}
	\]
	Here we are took what was written the democratic formulation of Kiritsis Appendix \textbf{I.4}, setting all fields equal to zero except for the dilaton and relevant RR $p+2$ field strength. The extra factor of two in the last terms on both lines comes from counting $\hat F_n \cdot \hat F_n$ and $\hat F_{10-n} \cdot \hat F_{10-n}$ on equal footing.
	
	 As written there, for IIA we have $\mathcal P_n = (\Gamma_{11})^{n/2} \sigma^1$ and for IIB we have $\mathcal P_n = \sigma^1$ for $\frac{1+n}{2}$ even and $i \sigma^2$ for $\frac{1+n}{2}$ odd. 
	
	Let's first look at the dilatino variation. We get \footnote{I have set $g_s = 1$ for all of this. I don't understand how any of this could work without being modified for arbitrary $g_s$.}
	\[
	\begin{aligned}
		&(p-3) A' \; \Gamma^r \epsilon \pm e^{A(3-p)} (-1)^{p} (p-3) A'\, e^{4A} \Gamma^{r 0 \dots p} \mathcal P_{p+2} \epsilon = 0\\
		& \Rightarrow (1 \pm (-1)^{p} e^{A (1 + p)} \Gamma^{0 1 \dots p} \mathcal P_{p+2}) \epsilon = 0\\
		& \Rightarrow (1 \pm  (-1)^{p} \Gamma^{\hat 0 \hat 1 \dots \hat p} \mathcal P_{p+2}) \epsilon = 0
	\end{aligned}
	\]
	Here the $\Gamma$ matrices with hatted (vielbein) indices are the familiar 10D Dirac matrices, as in Freedman and Van Proyen \emph{Supergravity}. We need our constant of proportionality $\kappa=2$ in order for the above combination of matrices to have a nontrivial null space. Note we could inversely have taken this as a way to take the profile of $\Phi = e^{(p-3) A}$ and get the profile of $F$ to be  $\pm 4 A' e^{4A}$.

	Locally, then, this is a linear algebraic constraint on the space of spinors at a given point, which half of the spinors will satisfy. 
	
	Now let's look at \emph{longitudinal} the gravitino variation. Similar to the case of the RN black hole, we expect $\d_\mu \epsilon = 0$ since $\d_\mu$ in the longitudinal direction is Killing.
	\[
	\begin{aligned}
		& \d_\mu \epsilon + \frac14 \omega_\mu^{ab} \Gamma_{ab} \epsilon \mp \frac{e^{(p-3)A}}{8} 4 A' e^{4 A} \Gamma^{r 0 \dots p} \Gamma_\mu \mathcal P_{p+2} \epsilon = 0 \\
		\Rightarrow & - \frac12 e^{A-B} A' \, \Gamma_{\hat r \hat \mu} \epsilon \mp \frac12 e^{(p+1)A} A' \Gamma^{r 0 \dots p} \Gamma_\mu \mathcal P_{p+2} \epsilon = 0\\
		\Rightarrow & - \frac12 A' \, \Gamma_{\hat r \hat \mu} \epsilon \mp \frac12 A' \Gamma^{\hat r \hat 0 \dots \hat p} \Gamma_{\hat \mu} \mathcal P_{p+2} \epsilon = 0\\
		\Rightarrow & \Gamma_{\hat \mu} \epsilon \pm \Gamma^{\hat 0 \dots \hat p} \Gamma_{\hat \mu} \mathcal P_{p+2} \epsilon = 0\\
		\Rightarrow & (1 \pm (-1)^p \Gamma^{\hat 0 \dots \hat p} \mathcal P_{p+2}) \epsilon = 0
	\end{aligned}
	\]
	This is exactly the same constraint as the one that the dilatino gave us. This also directly confirms our assumption: $\d_\mu \epsilon = 0$ longitudinally, since we can subtract the dilatino variation from the above gravitino one. 
	
	Meanwhile in the \emph{transverse} directions we no longer expect $\d_i \epsilon = 0$. It is important to note that the components of the spin connection $\omega_{i}^{\; ab} \Gamma_{ab}$ will only be nonvanishing for $a, b = \{j, k\}$ being transverse coordinates, in which case $\Gamma_{jk}$ is proportional to the infinitesimal rotation generator. By assumption of spherical symmetry (just as in the RN case of problem \textbf{35}), this must vanish $\Gamma_{jk} \epsilon = 0$. 
	\[
	\begin{aligned}
		0 &= \d_r \epsilon + \cancel{\frac14 \omega_r^{ab} \Gamma_{ab} \epsilon} \pm \frac{e^{A(p-3)}}{8} 4 A' e^{4A} \Gamma^{r 0 \dots p} \Gamma_r \mathcal P_{p+2} \epsilon\\
		&=  \d_r \epsilon \pm  \frac12 A' \, \Gamma^{\hat r\hat 0 \dots \hat p} \Gamma_r \mathcal P_{p+2} \epsilon \\
		&= \d_r \epsilon \pm  (-1)^{p+1} \frac12 A' \, \Gamma^{\hat 0 \dots \hat p} \mathcal P_{p+2} \epsilon\\
		&= \d_r \epsilon - \frac12 A' \epsilon \Rightarrow \epsilon = e^{A/2} \epsilon_0
	\end{aligned}
	\]
	where $\epsilon_0$ is a constant spinor satisfying the linear algebraic constraints previously given.
	
	We thus have that indeed our configuration is half-BPS.

	\item We write again the spin connection found in the last problem:
	\[
		e^{\hat \mu} = e^A dx^\mu, \quad e^{\hat i} = e^B dx^{i}
	\]
	Hatted indices always denote the vielbein indices.
	
	Then
	\[
		\begin{aligned}
			\dd e^{\hat \mu} &= \d_r A e^{A} \dd r \wedge \dd x^\mu = \sum_i \d_i A e^A \dd x^i \wedge dx^\mu = e^{\hat i} \wedge \omega^{\hat \mu}_{\;\; \hat i} \Rightarrow  \omega_{\hat \mu \hat \nu} = 0, \; \omega_{\hat \mu \hat i} = (-)^{\mu=0} \d_i A \; e^{A - B} \dd x^\mu \\
			\dd e^{\hat i} &= \d_r B e^{B} \dd r \wedge \dd x^i = \sum_j \d_i B e^B \dd x^j \wedge dx^i = e^{\hat j} \wedge \omega^{\hat i}_{\; \hat j}
			\Rightarrow \omega_{\hat i \hat j} = \d_j B \dd x^i - \d_i B \dd x^j \\
		\end{aligned}
	\]
	From this, we can get the Riemann curvature using $\mathbf{R}_{\hat \alpha \hat \beta} = \dd \boldsymbol{\omega}_{\alpha \beta} + \boldsymbol{\omega}_{\alpha \gamma} \wedge \boldsymbol{\omega}^\gamma_\beta$. First $\mathbf{R}_{\hat \mu \hat \nu}$ is the easiest:
	\[
		\mathbf{R}_{\hat \mu \hat \nu} = \cancel{\dd \omega_{\hat \mu \hat \nu}} + \omega_{\hat \mu \hat i} \wedge \omega^{\hat i}_{\; \hat \nu} = e^{2(A-B)} (\d A)^2 dx^\mu \wedge dx^\nu = \mathbf{R}_{\mu}^{\;\; \nu}
	\]
	Note that last expression is unhatted. 
	
	Next is $\mathbf{R}_{\hat \mu \hat i}$
	\[
	\begin{aligned}
		\mathbf{R}_{\hat \mu \hat i} &= \dd \omega_{\hat \mu \hat i} &+& \omega_{\hat \mu \hat j} \wedge \omega^{\hat j}_{\; \hat \nu}\\ 
		&= [\d_j \d_i A e^{A-B} + \d_i A (\d_j A - \d_j B)] dx^j \wedge dx^\mu & - & \d_i A e^{A-B} \d_j B dx^\mu \wedge dx^i + \d_j A e^{A-B} \d_i B dx^\mu \wedge d x^j\\
 	\end{aligned}
	\]
	\[
			= e^{A-B}[(\d_i \d_j A + \d_i A \d_j A - \d_i B \d_j A - \d_j B \d_i A)] dx^j \wedge dx^\mu - e^{A-B} \d_j A \d_j B dx^\mu \wedge d x^i
	\]
	We can get $R_{\mu}^{;i}$ (note unhatted) by multiplying this by $e^{A-B}$ and $R_{i}^{\; \mu}$ by multiplying this by $- e^{B-A}$. 
	
	Finally $\mathbf{R}_{\hat i \hat j}$:
	\[
	\begin{aligned}
		\mathbf{R}_{\hat i \hat j} &= \dd \omega_{\hat i\hat j} \qquad \qquad +  \cancel{\omega_{\hat i \hat \mu} \wedge \omega^{\hat \mu}_{\; \hat j}} &+& \omega_{\hat i \hat k} \wedge \omega^{\hat k}_{; \hat j} \\ 
		&= \d_k \d_j B dx^k \wedge dx^i - \d_k \d_i B dx^k \wedge dx^j & + & \d_k B \d_j B dx^i \wedge dx^k - \d_k B \d_i B dx^j \wedge dx^k - (\d B)^2 dx^i \wedge dx^j \\
 	\end{aligned}
	\]
	\[
		= -(\d B)^2 dx^i \wedge dx^j + (\d_k \d_i B - \d_k B \d_i B) dx^j \wedge dx^k - (\d_k \d_j B - \d_k B \d_j B) dx^i \wedge dx^k = \mathbf{R}_i^{;j}
	\]
	To evaluate $R_{\mu \nu \rho \sigma} R^{\mu \nu \rho \sigma}$ amounts to summing the squares of all the entries in the curvature two form when expressed in only vielbein indices. We can do this in Mathematica:
	
	\textbf{I can't get $c_+, c_-$ exactly right. The best attempt is in \emph{"exact p brane solutions.nb"}. The general $L$ and $r$ dependence in both cases matches though, and I'm not getting any $p-3$ factors, so I can believe this result.}
	
	To get the Ricci tensor, we must to the appropriate contractions. Importantly, if a longitudinal index must be summed over this gives an extra factor of $p+1$ while if a transverse index must be summed over this gives an extra factor of $9-p$.
	\[
		R_{\mu \nu} = R_{\mu \rho \nu}^{\; \quad \rho} + R_{\mu i \nu}^{\; \quad i} = - \eta_{\mu \nu} e^{2(A-B)} \Big( (p+1) (A')^2 + A'' + \frac{8-p}{r} A' + A' B' (9-p - 2) \Big)
	\]
	Here $A' = \d_r A$ is differentiation with respect to the radial coordinate. The Ricci tensor has no components mixing transverse and longitudinal directions: 
	\[
		R_{\mu i} = \cancel{R_{\mu \rho i}^{\; \quad \rho}} + \cancel{ R_{\mu j i}^{\; \quad j}} = 0 
	\]
	Finally the annoying one, for which I looked at Stelle's \emph{Lectures on p-Branes 9701088} : 
	\[
	\begin{aligned}
		R_{ij} = R_{i \mu j}^{\; \quad \mu} + R_{i k j}^{\; \quad k} = & {\textstyle - \delta_{ij} \left(B'' + (p+1) A' B' + (7-p) (B')^2 + \frac{2 (7-p) + 1}{r} B' + \frac{d}{r} A'  \right)}\\
		& \textstyle + \frac{x^i x^j}{r^2} \left( \scriptstyle(7-p) B'' - \frac{7-p}{r} B' + (p+1)A'' - \frac{p+1}{r} A' - 2 (p+1) A' B' + (p+1) (A')^2 - (7-p) (B')^2 \right)
	\end{aligned}
	\]
	In this last part I rewrote $\d_i = \frac{x^i}{r} \d_r$.
	
	We can evaluate this directly in Mathematica. For $R_{\mu \nu} R^{\mu \nu}$ and $R$ we get:
	\begin{center}
		\includegraphics[scale=0.5]{"Figures/Curvature Invariants"}
	\end{center}
	The last line is in agreement with the expression for $R$ in Kiritsis \textbf{8.8.31}
	
	
	\item Exercise 7.7 shows that, upon $T$-dualizing along the $x^9$ direction we get
	\[
		\tilde C^{(p+1)}_{\mu_1 \dots \mu_p 9} = C^{(p)}_{\mu_1 \dots \mu_p}, \quad \tilde C^{(p)}_{\mu_1 \dots \mu_p} = \tilde C^{(p+1)}_{\mu_1 \dots \mu_p 9} 
	\]
	In transverse space, our $(p+1)$-form $C$ has components only along the longitudinal directions. Upon T-dualizing, we pick up the $9$ index in the $C$ form, and thus get that our brane has a $(p+2)$ form charge. We thus expect this to be a $p+1$ brane wrapping that additional $x^9$ direction. I'm unsure if this wants us to explicitly give the form of that solution, since doing it in a compact space seems a bit harder. 
	
	\item Let's assume $p<7$. When $\lambda \gg 1$ the perturbative stringy description is no longer valid. For an extremal p-brane, we know from problem 40 that:
	\[
		L^{7-p} = \frac{2\kappa_{10}^2 T_p N}{(7-p) \Omega_{8-p}} \Rightarrow  \left(\frac{L}{2\pi \ell_s} \right)^{7-p} = \frac{g_s N}{7-p} \frac{\Gamma(\frac{9-p}{2})}{2 \pi^{\frac{9-p}{2}}}
	\]
	So $\lambda = 2\pi g_s N \gg 1$ gives that $L \gg \ell_s$, meaning that the throat size is macroscopic. We can thus probe it without having to see distances smaller than the string scale. 
	
	When $p > 3$ we see from our  calculation of $R$ in problem \textbf{44} that $R$ blows up as $\frac{L^{2(7-p)}}{r^{(p-3)/2}}$ as $r \to 0$. This will become order $\ell_s^{-2}$ at 
	\[
		r \approx \left(\frac{\ell_s^2}{L^{(7-p)/2}}\right)^{2/(p-3)}
	\]	
	When $p < 3$ the formula for $R$ indeed is seen to go to zero. On the other hand the string coupling grows as
	\[
		e^{\Phi} = g_s H^{(3-p)/4} = g_s \left(1 + \frac{L^{7-p}}{r^{7-p}} \right)^{(3-p)/4}
	\]
	So if $g_s$ is the string coupling ``at infinity'' which we can take to initially be small, then it will become appreciable at
	\[
		r = L (-1 + g_s^{-4/(3-p)})^{1/(p-7)}
	\]
	so for $g_s$ sufficiently small, the quantity in parentheses will be quite large and be raised to a negative power, so that this is a small fraction of the throat size. 
	
	\item This is only a slight variant of exercises \textbf{8.38-9}, and in fact is a bit easier. Our action is
	\[
		S = \frac{1}{2 \kappa_{10}^2} \int d^{10} x \sqrt{-g} e^{-2\Phi} \left[R + 4 (\nabla \Phi)^2 - \frac{1}{2 \cdot 3!} (dB)^2 \right]
	\]
	Let's first vary $\Phi$. We get
	\[
	\begin{aligned}
		0 &= - 2 e^{-2 \Phi}  \left[R + 4 (\nabla \Phi)^2 - \frac{1}{2 \cdot 3!} (dB)^2 \right] - \nabla(e^{-\Phi} 8 \nabla \Phi)\\
		&= -2 e^{-2 \Phi} R - 8 e^{-2 \Phi} (\nabla \Phi)^2 + \frac{e^{-2 \Phi}}{3!} (dB)^2 - 8 e^{-2\Phi} \Box \Phi + 16 (\nabla \Phi)^2 e^{-2\Phi}\\
		\Rightarrow & R = 4 (\nabla \Phi)^2 - 4 \Box \Phi + \frac{(dB)^2}{2 \cdot 3!}
	\end{aligned}
	\]
	Next for the $B$-field we will just have
	\[
		\dd \star e^{-2\Phi} dB = 0 
	\]
	Finally, varying $g$ is the hardest, but we've done most of the work already in the other problem:
	\begin{itemize}
		\item $\sqrt{-g} e^{-2\Phi} R$
		 \[
		 \begin{aligned}
		 &\to (R_{\mu \nu} + g_{\mu \nu} \Box - \nabla_\mu \nabla_\nu) e^{-2\Phi}  - \frac12 g_{\mu \nu} e^{-2\Phi} R\\
		 &= e^{-2\Phi} \Big(R_{\mu \nu} - \frac12 g_{\mu \nu} R + g_{\mu \nu} (-2 \Box \Phi + 4 (\d \Phi)^2) - (-2 \nabla_\mu \nabla_\nu \Phi + 4 \d_\mu \Phi \d_\nu \Phi) \Big) 
		 \end{aligned}
		\]
		\item $\sqrt{-g} e^{-2\Phi} 4 g^{\mu \nu} \d_\mu \Phi \d_\nu \Phi \to 4 e^{-2\Phi} \d_\mu \Phi \d_\nu \Phi - 2 e^{- 2 \Phi} (\d \Phi)^2$
		\item $- \frac{e^{-2 \Phi}}{2(p+2)!} \sqrt{-g} g^{\mu_1 \nu_1} \dots g^{\mu_{p+2} \nu_{p+2}} F_{\mu_1 \dots \mu_{p+2}} F_{\nu_1 \dots \nu_{p+2}} \to - \frac{e^{-2 \Phi}}{2 (p+1)!} F_{\mu \nu}^2 + \frac{e^{-2 \Phi}}{4(p+2)!} g_{\mu \nu} F^2$. Here $F_{\mu \nu}^2 = F_{\mu \dots} F^{\nu \dots}$
	\end{itemize}
	Combining these all together, dividing through by $e^{-2\Phi}$ and using the dilaton equations of motion gives
	\[
		R_{\mu \nu} + 2 \nabla_\mu \nabla_\nu \Phi = \frac{1}{2 \cdot 2!} H_{\mu \nu}^2
	\]
	Here $H_{\mu \nu} = H_{\mu \rho \sigma} H_{\nu}^{\; \rho \sigma}$ as we've had before (i.e. in chapter 6). 
	
	Ok next let's take the ansatz as in Kiritsis:
	\[
		ds^2 = - f(R) dt^2 + dx_i^2 + H(r) \left(\frac{dr^2}{f(r)} + r^2 d\Omega_3^2 \right).
	\]
	With dilaton
	\[
		e^{2 \Phi} = g_s^2 H(r)
	\]
	The field strength written is wrong (as you can see by noting that as $r\to \infty$ the magnetic flux integral goes to zero). We can find the correct expression by noting that $\dd \star \dd B = 0$ trivially since $\star \dd B$ has a $\dd r$ component. The only nontrivial equation is the Bianchi identity, giving (by spherical symmetry)
	\[
		dB = 0 \Rightarrow B = c \omega 
	\]
	for $\omega = d\psi \wedge \sin \psi d\theta \wedge \sin \psi \sin \theta d \phi$ the unit volume form on the sphere. Let's see what this constant $c$ should be from the dilaton equations. We get
	\begin{center}
		\includegraphics[scale=0.5]{"Figures/NS5 Dilaton"}
	\end{center}
	So when $c = -  2L\sqrt{1 + r_0^2/L^2}$ we get our dilaton. 
	
	\textbf{By Hodge-dualizing, this also gives credibility for the $\sqrt{1-r_0^2/L^2}$ constant in the $p$-brane solution, which would have required the more complicated $R_{\mu \nu}$ equation.}
	
	Finally the least trivial equation of motion is also straightforward:
	\begin{center}
		\includegraphics[scale=0.5]{"Figures/NS5 Dilaton 2"}
	\end{center}
	
	\item Let's review the extremal near horizon limit first. There, when $r \ll L$ we can just write
	\[
		ds^2 = -dt^2 + d\vec x \cdot d\vec x + L^2 \frac{dr^2}{r^2} + L^2 d \Omega^2_3
	\]
	Defining $\gamma=  \sqrt{N} \ell_s \log\frac{r}{g_s \ell_s \sqrt N}$ gives $d \gamma^2 = L^2/r^2$ giving 
	\[
		ds^2 = - dt^2 +  d\vec x \cdot d\vec x + d\gamma^2 + N \ell_s^2 d \Omega^2_3
	\]
	This looks like flat space times a constant-radius sphere with a linear dilaton background going as $\Phi = \gamma/\sqrt{N} \ell_s$.
	
	\begin{center}
		\includegraphics[scale=0.2]{"Drawings/NS5"}
	\end{center}
	
	
	Next let's look at the near-extremal case. We take $r = r_0 \cosh \sigma$ so that $f(r) = 1 - r_0^2/r^2 = \tanh^{2} \sigma$. Meanwhile 
	\[
		\frac{H(r)}{f(r)} dr^2 = (1 + \frac{L^2}{r_0^2 \cosh \sigma}) \frac{r_0^2 \sinh^2 \sigma d\sigma^2}{\tanh^2 \sigma} = L^2 + r_0^2 \cosh \sigma^2 = H(r) r^2
	\]
	So we get a metric
	\[
		-\tanh^2 \sigma dt^2 +  d\vec x \cdot d\vec x + (N \ell_s^2 + r_0^2 \cosh^2 \sigma) (d\sigma^2 + d\Omega^2_3)
	\]
	At large $N$, rescaling $t$ this looks like
	\[
		-\tanh^2 \sigma N \ell_s^2 dt^2 +  d\vec x \cdot d\vec x + N \ell_s^2 (d\sigma^2 + d\Omega^2_3),
	\]
	which looks like a 2D black hole solution in $\sigma, t$ space, after rescaling 
	
	\item Let's write the spin connection. Take $e^{2A} = H(r)$ so that $\phi - \phi_0 = A$. Our frame fields look like:
	\[
		e^{\hat \mu} = dx^\mu, \quad e^{A(r)} dx^i
	\]
	for $\mu$ parallel and $i$ transverse. It looks like $\omega_{\mu \nu} = \omega_{\mu i} = 0$ while 
	\[
		\omega_{\hat i \hat j} = - \partial_j A dx^i + \partial_i A dx^j
	\]
	similar to what we had before.
	
	We again write the gravitino and dilatino variation in 10D type II SUGRA, neglecting this time the RR forms but incorporating the N 2-form contribution: 
	\[
	\begin{aligned}
		0 &=
		 \delta \psi_{\mu, A} & = & \; (\d_\mu + \frac14 \omega_{\mu}^{ab} \Gamma_{ab}) \epsilon + \frac14 \slashed{H}_\mu \mathcal P \epsilon \\
		0 &= \delta \lambda & =& \; \slashed{\d} \Phi \epsilon  + \frac12 \slashed{H} \mathcal P \epsilon
	\end{aligned}
	\]
	Here $\mathcal P  = \Gamma^{11} \otimes 1_2$ in type IIA and $-1_{32} \otimes \sigma^{3}$ in type IIB. 
	
	The dilatino variation gives
	\[
	\begin{aligned}
		& \d_r \phi \Gamma^r \epsilon \pm \frac12 (-2 L^2) \sin^2 \psi \sin \theta \Gamma^{ \psi \theta \phi} \mathcal P \epsilon \\
		&= \frac{H'}{2 H} \Gamma^r \epsilon \mp \frac{L^2}{H^{3/2} r^3} \Gamma^{ \hat \psi \hat \theta \hat \phi} \mathcal P \epsilon\\
		&= \frac{H'}{2 H^{3/2}} \Gamma^{\hat r} \epsilon \mp \frac{L^2}{H^{3/2} r^3} \Gamma^{ \hat \psi \hat \theta \hat \phi} \mathcal P \epsilon\\
		&\Rightarrow - L^2 (1  \pm  \Gamma^{\hat r \hat \psi \hat \theta \hat \phi} \mathcal P)  \epsilon = 0
	\end{aligned}
	\]
	This is an algebraic constraint that is satisfied by half the space of spinors at any given point. This makes the solution half-BPS, so long as the profile of $\varepsilon$ can be chosen so that the gravitino vanishes.
	
	The $\delta \psi_\mu$ variation longitudinal to the solution is trivial. The transverse variation is
	\[
	\begin{aligned}
		& (\d_i + \frac14 \omega_{i j k} \Gamma^{jk}) \epsilon_i + \frac14 H_{i j k} \Gamma^{j k} \mathcal P \epsilon
	\end{aligned}
	\]
	Crucially, though, $\Gamma^{jk}$ is the generator of rotations. By rotational symmetry we thus reduce this to $\d_i \epsilon = 0$, implying that $\epsilon(r) = \epsilon_0$ is a constant spinor.
	
	It is also worth noting that transverse to the NS5 brane is precisely the extremal BH solution in 5D, which preserves half SUSY by the same arguments as before. Parallel to it is flat space (which preserves all SUSY). The product spacetime therefore preserves half.
	
	\item We have the same equations as when we were solving the for the NS5 brane. 
	This time, the $\dd e^{-2 \Phi} \star dB$ constraint is nontrivial, and we must have a field strength. Because the field is electrically charged under the field, I expect 
	\[
		B \sim H^{-1}(r)
	\]
	For $H = 1 + \frac{L^6}{r^6}$ the relevant harmonic form in transverse space. I don't have much justification for this other than the fact that - in every problem I've seen this seems to hold true. Now let's take the ansatz that the metric and dilaton look like
	\[
		ds^2 = H^{\alpha} (-dt^2 + dx_1^2) + H^{\beta} d_\perp x^2, \qquad e^{\Phi} = H^{\gamma}
	\]
	Then $\sqrt{-g} = H^{\alpha + 4 \beta}$ and we get
	\[
		e^{-2\Phi} \star dB  = H^{-\alpha + 3\beta -2 \gamma - 2} r^7 H'(r)
	\]
	We want $\dd e^{-2\Phi}  \star \dd B = 0$ so we must have
	\[
		-\alpha + 3 \beta - 2\gamma - 2 = 0
	\]
	The simplest guess would be $\alpha = -1, \gamma = -1/2$. This turns out to work. First look at the dilaton EOM: 
	\begin{center}
		\includegraphics[scale=0.5]{"Figures/F1 1"}
	\end{center}
	Next, look at the metric's EOM
	\begin{center}
		\includegraphics[scale=0.5]{"Figures/F1 2"}
	\end{center}
	
	Perhaps the easier thing to do was look for a BPS solution. In either case we are done. My (reasonable) 
guess for the non-extremal version of this would be to keep the dilaton and NS field the same and modify the metric to be
	\[
		H^{-1}(r) (-f(r) dt^2 + dx^2_1) + \frac{dr^2}{f(r)} + r^2 d\Omega_{7}^2
	\]
	where $f(r) = 1 - \frac{r_0^6}{r^6}$. I have not checked this, but it seems right based on experience at this point. s
	
	\item Let's start with IIB. The untwisted sector will contain closed string states that are invariant under the $\mathcal I_4 (-1)^{\mathbf{F}_L}$ combination. The twisted sector will localize to the 5-plane left invariant by the inversion. Let's say that this is labeled by $x_0 \dots x_5$ and $x_6 \dots x_9$ are the coordinates reflected under the orbifold. The supersymmetries $Q_L = Q, Q_R = \tilde Q$ both transform in the $\mathbf{8}_s$ representation. On the 5-plane, this decomposes under $\SO(4)_\parallel \times \SO(4)_\perp$ as
	\[
		\mathbf{8}_s \to [2_\parallel \times 2_\perp] + [\bar 2_{\parallel} \times \bar 2_{\perp}]
	\]
	Here $\mathcal I_4$ acts with $-$ on the $2 \otimes \bar 2$ vector representation of $\SO(4)_\perp$, leaving the $2 \otimes \bar 2$ $\SO(4)_\parallel$ alone. We take $\mathcal I_4$ to flip the sign of only the $2_\perp$ spinor. $(-1)^{\mathbf{F}_L}$  acts with a $-$ sign on only $Q$. Together, this leaves
	\[
		Q \in 2_\parallel \times 2_\perp, \quad \tilde Q \in \bar 2_\parallel \times \bar 2_\perp
	\]
	invariant. These preserved generators give $(1,1)_6$ supersymmetry. The exact same argument would give that the IIA twisted sector has $(2,0)_6$ SUSY. These rigid supersymmetries have a unique massless representation, namely the vector and tensor multiplets respectively, so this is what we would expect to get.
	
	 Let's check this explicitly for the twisted sector of the IIB orbifold. The parallel $\alpha^\mu$ do not get twisted boundary conditions, but the transverse $\alpha^i$ get acted on by a $-$ from the $\mathcal I$, so will get half-integrally modded. For the fermions, the $\psi^\mu$ are affected by the $(-1)^{\mathbf{F}_L}$ and so will become integrally modded in the NS sector and half-integrally modded in the R sector. The $\psi^i$ are additionally affected the $\mathcal I$ and so remain half-integrally modded in the NS sector and integrally modded in the R sector. 

	In both R \emph{and} NS sectors, we have the same number of periodic and anti-periodic bosons and fermions, so the ground state energy vanishes in both sectors. Massless excitations are described purely in terms of the ground states of the system. The bosonic ground state is unambiguous. In the NS sector there are four $\psi^i_0$ transforming in the $\SO(4)_\perp$ vector representation while in the R sector there are four $\psi^\mu_0$ transforming in the $\SO(4)_\parallel$ vector representation. These lead to ground states transforming as $2 + \bar 2$. 
	
	The effect of the $(-1)^{\mathbf{F}_L}$ is to change the \emph{left-moving} GSO projection in the twisted sector (c.f. exercise \textbf{11.29}). Thus in both NS-NS and R-R we get GSO projections:
	\[
		\frac14 (1 - (-)^{\mathbf{F}_L}) (1 + (-)^{\mathbf{F}_R}) 
	\]
	For NS-NS this means:
	\[
		(2_\perp^L + \bar 2_\perp^L) \otimes (2_\perp^R + \bar 2_\perp^R) \to \bar 2_\perp^L \otimes  2_\perp^R
	\]
	% these are the only terms consistent with the action of inversion together with $(-1)^{F_L}$. GSO will project out one of them.
	This transforms in the vector representation of $\SO(4)_\perp$ and can thus be interpreted as 4 scalars with $\SO(4)$ R-symmetry.

	For R-R we get the same:
	\[
		(2_\parallel^L + \bar 2_\parallel^L) \otimes (2_\parallel^R + \bar 2_\parallel^R) \to \bar 2_\parallel^L \otimes  2_\parallel^R
	\]
	This is a vector with little group $\SO(4)$.
	
	So the NS-NS states give 4 scalars and the RR states give the vector. These are consistent with the spectrum of a single NS5 brane in type IIB, and the same logic holds for IIA. \textbf{Understand how this connects with Sen's articles on non-BPS particles} 
	
	
		
\end{enumerate}

% section chapter_8_d_branes (end)	
\end{document}
	