\documentclass[11pt, class=article, crop=false]{standalone}
\usepackage{amsmath,amssymb,amsfonts,amsthm}
\usepackage{enumitem}
\usepackage{fancyhdr}
\usepackage{tikz-cd}
\usepackage{mathabx}
\usepackage{geometry}
\usepackage{natbib}
\usepackage{braket}
\usepackage{graphicx}
\usepackage{simpler-wick}
\usepackage{hyperref}
\usepackage{ytableau}
\usepackage{cancel}
\usepackage{listings}
\usepackage{relsize}
\usepackage{xcolor}
\usepackage{stmaryrd}
\usepackage{slashed}
\usepackage{tikz-feynman}
\usepackage{kiritsis}
\geometry{margin = 0.5in}


\begin{document}
\section*{Chapter 13: Black Holes and Entropy in String Theory} % (fold)
\label{sec:chapter_13_black_holes_and_entropy_in_string_theory}
\begin{enumerate}
	\item We begin with
	\[
		ds^2 = - F(r) C(r) dt^2 + \frac{dr^2}{C(r)} + H(r) r^2 d\Omega_2^2
	\]
	and $C(r)$ vanishes at the horizon $r = r_0$ while all other functions are positive for $r \geq r_0$ and everything asymptotes to $1$ as $r \to \infty$. Now, let's do a wick rotation $t \to i \tau$ with $\tau$ Euclidean time. We get
	\[
		F(r) C(r) dt^2 + \frac{dr^2}{C(r)} + H(r) r^2 d\Omega_2^2
	\]
	Now at $r = r_0 + \epsilon$ we see that the geometry takes the form
	\[
		F(r_0) C'(r_0) (r-r_0) dt^2 + \frac{dr^2}{C'(r_0) (r-r_0)} + r_0^2 d\Omega_2^2
	\]
	The last term is simply the expected metric on a 2-sphere of fixed radius $r_0$. The other two terms give a metric
	\[
		ds^2 = F(r_0) C'(r_0) \epsilon dt^2 + \frac{d \epsilon^2}{C'(r_0) \epsilon}
	\]
	Take $u = \frac{2 \sqrt{\epsilon}}{\sqrt{C'(r_0)}}$ then $du = \frac{d\epsilon}{\sqrt{C'(r_0) \epsilon}}$ giving us
	\[
		ds^2 = F(r_0) \frac{C'(r_0)^2}{4} u^2 dt^2 + du^2
	\]
	This describe a conical deficit geometry in polar coordinates. In order to obtain a smooth geometry, we need the requirement that 
	\[
		\tau + 2 \pi \times \frac{2}{C'(r_0) \sqrt{F(r_0)}}  = \tau 
	\]
	Giving an inverse temperature of 
	\[
		\frac1T = \beta =  \frac{4 \pi}{C'(r_0) \sqrt{F(r_0)}}
	\]
	This formula generalizes directly to higher-dimensional black holes.
	
	\item 
	In what follows, recall the area formula for a general KN Black hole of mass charge and spin $(M, Q, J)$ is:
	\[
		A = 4 \pi (r_+^2 + a^2), \quad r_+ = M + \sqrt{M^2 - a^2 - Q^2}, \quad a = J/M
	\]
	In particular an extremal Kerr black hole has area $8 \pi M^2$. 
	
	\begin{enumerate}
		\item The areas of the individual Schwarzschild black holes are 
	\[
		4 \pi (2 M_i)^2 = 16 \pi M_i^2
	\]
	each. The area of their composite must then be $\geq 16 \pi (M_1^2 + M_2^2)$. Because they start as almost stationary, the total angular momentum in the center of mass frame is zero, so the final black hole will be (essentially) Schwarzschild. So we get 
	\[
		M_f^2 \geq M_1^2 + M_2^2
	\]
	If the initial masses were equal, we'd get $M_f \geq \sqrt{2} M$ so that $E = 2 M - \sqrt 2 M$ and $E/(M_1 + M_2) = 1- 1/\sqrt{2}$. Let's write WLOG $M_2 = \gamma M_1$ with $\gamma \leq 1$ then 
	\[
	\begin{aligned}
		M_f^2 \geq (1+\gamma^2) M_1^2  &\Rightarrow E = (1+\gamma) M_1 - \sqrt{1+\gamma^2} M_1\\ 
		&\Rightarrow \frac{E}{M_1 + M_2} = \frac{(1+\gamma) M_1 - \sqrt{1+\gamma^2} M_1}{(1+\gamma) M_1} = 1 -\frac{\sqrt{1+\gamma^2}}{(1+\gamma)} \geq 1 \leq 1 - \frac{1}{\sqrt 2}
	\end{aligned}
	\]
	as required. 
	\item For two extremal RN black holes we have $r_+ = M$ so each has area $4 \pi M^2$. They will collide to form a neutral (perhaps rotating) black hole. The area law gives us 
	\[
		4 \pi ((M_f + \sqrt{M_f^2 - a^2})^2 + a^2) \geq 2 \times 4 \pi M^2.
	\]
	This bound is sharpest if we take the final state to be extremal Kerr $a = M$, giving
	\[
		 8 \pi M_f^2 \geq 8 \pi M^2 \Rightarrow M_f \geq M
	\]
	We get 
	\[
		E \leq 2M - M \Rightarrow \frac{E}{2M} \leq \frac12.
	\]
	
	\item Such a decay would look like 
	\[
		2 M_f^2  \geq M^2 \Rightarrow \sqrt{2} M_f \geq M  \Rightarrow M - 2 M_f \leq (\sqrt{2} - 2) M_F < 0.
	\]
	This is a contradiction. 
	\end{enumerate}
	
	\item We have that $n = \frac{1}{\sqrt{g_{rr}}} \d_r = \sqrt{f(r)} \d_r$ so that
	\[
		K_{\mu \nu} = \frac12  \frac{1}{\sqrt{g_{rr}}} \d_r G_{\mu \nu} = \frac{\sqrt{f(r)}}{2} \text{diag}\Big(f'(r), -\frac{f'(r)}{f(r)}, 2r, 2r \sin^2 \theta \Big)
	\]
	Contracting with the $3 \times 3$ boundary inverse metric $h^{\mu \nu} = \text{diag}(f(r)^{-1}, r^{-2}, r^{-2} \sin^{-2} \theta)$ which has \emph{no $r$ component} gives
	\[
		K =  \frac{\sqrt{f}}{2} \left(\frac{f'}{f} + \frac{4}{r} \right) = \sqrt{f} \, \frac{r f' + 4 f}{2 r f} \Big|_{r = r_0}
	\]
	where $r_0$ is large and formally infinite. We can then evaluate 
	\[
		\frac{1}{8 \pi G} \int_{\d M} \sqrt{h} K = \frac{4 \pi r^2 \beta \sqrt{f}}{8 \pi G} K\Big|_{r=r_0} = \beta \frac{r}{4G} (r f' + 4 f)\Big|_{r=r_0} 
	\]
	Directly evaluating this for $f(r) = 1 - \frac{2 G M}{r} + \frac{Q^2}{r^2}$ gives
	\[
		\frac{\beta}{2 G} \left(\frac{Q^2}{r_0^2} + 2 r_0 - 3 G M \right)
	\]
	This is the gravitational boundary term contribution to the classical action. 
	 The gravitational bulk term is zero since the Ricci scalar vanishes for the RN solution. The electromagnetic contribution is
	\[
		\frac{1}{16 \pi G} \int_M \sqrt{g} F_{\mu \nu} F^{\mu \nu} = \frac{1}{8 \pi G} \int_0^\beta d\tau \int d\Omega_2 \int_{r_+}^{r_0} r^2 dr \frac{Q^2}{r^4} = \beta \frac{4 \pi Q^2}{8 \pi G} \left(\frac{1}{r_+} - \frac{1}{r_0} \right) = \frac{\beta}{2G} Q^2 \left(\frac{1}{r_+} - \frac{1}{r_0}\right)
	\]
	All together, as $r_0 \to \infty$ we get action:
	\[
		S_{RN} = -\frac{\beta}{2G} \left(2 r_0 - 3 G M + \frac{Q^2}{r_+} \right)
	\]
	Note that there is one divergent term, namely the one linear in $r_0$ in the boundary action, but this is insensitive to the properties of the RN black hole and is also present in flat space. It is then sensible to define a regularized (renormalized) action by subtracting this term off. In doing this subtraction, there is an ambiguity of how we should define the inverse temperature of the reference flat space subtraction. The appropriately redshifted temperature \textbf{Justify} is $\beta \sqrt{f}$, giving reference action:
		\[
			S_{flat} =  -\frac{\beta}{G} r_0 \sqrt{f(r_0)} = - \frac{\beta}{2G} (2 r_0 - 2 G M + O(1/r_0))
		\]
	The renormalized Euclidean action is thus 
	\[
		I_{RN} = S_{RN} - S_{flat} = \frac{\beta}{2} (M - \frac{Q^2}{G r_+}) = \frac{\beta}{2} (M -  \mu Q) = \beta \mathcal F
	\]
	\item The specific heat $C$ is given by the coefficient in
	\[
		dM = M C dT
	\]
	For Schwarzschild, $T = (8 \pi G M)^{-1}$ so this is 
	\[
		dM = - M C \frac{dM}{8 \pi G M^2} \Rightarrow C = -8 \pi G M
	\]
	Which is negative. This should not be so surprising, given that by increasing the energy (ie mass) of the Schwarzschild black hole we make a larger one which thus have \emph{lower} temperature. It is worth noting that, including units, this is proportional to $\frac1\hbar$.
	
	\item First off, at $a = 0$ Kerr-Newman reproduces the RN black hole, which we already know is a solution of the Einstein-Maxwell system. 
	
	Further, it is quick to check using Mathematica that at $Q=0$ the Kerr metric is itself Ricci-Flat: $R_{\mu \nu} = 0$ so is indeed a solution of the vacuum Einstein equations (away from $r = 0$). 
	
	\hspace{-.5in}
	\includegraphics[scale=0.5]{"Figures/Kerr"}
	When $Q \neq 0$ we get a nonzero Ricci tensor (the Ricci scalar still vanishes since classical electrodynamics is conformal). 
	
	\hspace{-.5in}
	\includegraphics[scale=0.5]{"Figures/Kerr-Newman"}
	The Ricci tensor must correspond to an electromagnetic stress-energy tensor. It comes from an electric potential of the form $A_\mu = (\frac{r Q}{\Sigma}, 0, 0, - \frac{a r Q \sin^2 \theta}{\Sigma})$
	\begin{center}
		\includegraphics[scale=0.5]{"Figures/T-Ricc"}
	\end{center}
	
	For $r$ very large we get an electric field going as $q r^2/\Sigma^2 \sim q/r^2$ corresponding to the electric field for a charge $q$, and we also get a magnetic field dying off as $a \cos \theta/r^3$ corresponding to the field from a spinning charged source. Said another way, we see that $\frac{1}{4\pi} \int \star F = q$ and $\frac{1}{4\pi} \int F = 0$ asymptotically, so we have just an electric charge $q$. 
	
	We can verify mass and angular momentum using the killing vectors $\d_t$ and $\d_\phi$ respectively using the formulas in \textbf{Wald 12.3.8-9}
	\[
	\begin{aligned}
		- \frac{1}{8\pi G} \int \epsilon_{abcd} \nabla^c (\d_t)^d &= M
		\frac{1}{16\pi G} \int \epsilon_{abcd} \nabla^c (\d_\phi)^d &= a M
	\end{aligned}
	\]
	\begin{center}
		\includegraphics[scale=0.5]{"Figures/verify M J"}
	\end{center}
	
	For the KN black hole metric, the only singularities can come from $\Sigma = 0$ or $\Delta = 0$. $\Sigma$ is only zero for $a > 0$ when $r= 0, \theta = \pi/2$. This corresponds to the curvature singularity of the black hole (in fact despite deceptive coordinate choice, this takes the form of a ring $S_1 \times \RR$ as is revealed in Kerr-Schild coordinates). The horizons come from $g_{rr}$ becoming singular, namely $\Delta = 0$ which occurs at
	\[
		r^2 - 2 G M r + a^2 + Q^2 = 0 \Rightarrow r_\pm = M \pm \sqrt{M^2 - a^2 - Q^2}.
	\]
	These give the outer and inner horizons. 
	
	The horizon area is given by 
	\[
		\int_0^\pi d\theta \int_0^{2\pi} d\phi \sqrt{g_{\theta \theta} g_{\phi \phi}}\Big|_{r = r_+}  = 2 \pi \int_0^\pi d\theta \sin \theta \sqrt{(r^2_+ + a^2)^2 - \Delta a^2 \sin^2 \theta}
	\]
	But $\Delta = 0$ at the horizon so this trivializes to
	\[
		4 \pi (r_+^2 + a^2) = 4 \pi ((m + \sqrt{m^2 - a^2 - Q^2})^2 + a^2)
	\]
	The entropy of the black hole is then
	\[
		S = \frac{A}{4} = \pi (r_+^2 + a^2)
	\]
	Taking care to write things in terms of  $J$ and not $a$ now, by holding $J, Q$ fixed, let's vary $M$ and get
	\begin{center}
		\includegraphics[scale=0.4]{"Figures/Hawking T"}
	\end{center}
	The Hawking temperature is thus
	\[
		T_H = \frac{1}{2\pi} \frac{\sqrt{M^2 - a^2 - Q^2}}{r_+^2 + a^2}
	\]
	Now let's fix $S$ and $Q$. We get 
	\begin{center}
		\includegraphics[scale=0.4]{"Figures/J pot"}
	\end{center}
	Which gives us that 
	\[
		\Omega = \Big(\frac{\d M}{\d J}\Big)_{Q,S} = -\left(\frac{dS}{dJ}\right)_{Q,M} \left(\frac{dS}{dM}\right)^{-1}_{Q,J} = \frac{a}{r_+^2 + a^2}
	\]
	Finally let's hold $S, J$ fixed and do the same procedure, giving
	\begin{center}
		\includegraphics[scale=0.4]{"Figures/Q pot"}
	\end{center}
	\[
		\mu = \Big(\frac{\d M}{\d Q}\Big)_{J,S} = -\left(\frac{dS}{dQ}\right)_{J,M} \left(\frac{dS}{dM}\right)^{-1}_{Q,J} = \frac{Q r_+}{r_+^2 + a^2}
	\]
	
	The full form of the first law is then
	\[
		dM = T dS + \Omega dJ + \mu dQ
	\]
	
	We obtain an extremal black hole when $M = a^2 + Q^2$, as this is the minimum value of $M$ where $r_+$ is a well-defined radius. At this value, $r_+=  r_-$. 
	
	Thermodynamic stability comes from minimizing the Gibbs free energy:
	\[
		G = M - T S - \Omega J - \mu Q
	\]
	Note that for flat space, $G = 0$, so if $G > 0$ for any of these black holes, thermal fluctuations will eventually drive their decay to flat space.
	
	Plugging in what we have gives  
	\begin{center}
		\includegraphics[scale=0.4]{"Figures/Gibbs BH"}
	\end{center}
	Notice that if $J>0$ then this will \emph{always} be greater than zero, by virtue of the fact that $M > Q$ always. If we take $J = 0$, we get that this is still thermodynamically unstable unless $Q = M$ and the black hole is extremally charged.
	
	\item The Hawking evaporation rate gets modified as
	\[
		\Gamma_H = \frac{\sigma_{abs}(\omega)}{\exp(\beta (\hbar \omega - \vec s \cdot \vec \Omega - q \Phi))\mp 1} \frac{d^3 k}{(2\pi)^3}
	\]
	where $\vec s \cdot \vec \Omega$ is the angular momentum product (orientation of $\vec s$ relative to $\vec \Omega$ matters). The $\mp$ is for bosons and fermions respectively. 
	
	\textbf{Return to understand how this generalizes to systems more broadly}
	
	\item This is direct - we take $M$ theory on $T^6$. Wrap $Q_1$ M2 branes along $x^9-x^{10}$, $Q_2$ M2 branes along $x^7-x^8$, and $Q_3$ M2 branes along $x^5-x^6$. 
	
	Take the M-theory $S^1$ to be along $x^{10}$. Now, taking this to be microscopic first, we get $Q^1$ strings wrapping $x^9$ together with $Q_2, Q_3$ D2 branes in IIA. 
	
	$T$-dualize along $x^5,x^6$ to get $Q_3$ D0 branes, $Q_2$ D4 branes wrapping $x^{5-8}$, and $Q^2$ D2 branes wrapping $x^5, x^6$. The F1 around $x^9$ is unaffected.
	
	Finally, $T$-dualize along $x^9$, taking IIA to IIB and giving $Q_3$ D1 branes and $Q_2$ D5 branes, while replacing the F1 (ie $B$-flux) with KK momentum of the system along the $x^9$. This is exactly the D1-D5 system.
	
	\item The D5 and D5 branes are both BPS. We know that, upon toroidal compactification, 
	
	In $D=10$ have the D1 stretch $x_0 = t, x_5 = \gamma$ and the D5 stretch $x_0, \dots, x_5$, where we write $\gamma^a, a=1 \dots 4$ to be the new D5 directions. These will form the direction of the $T^4$.
	
	Upon compactifying on $T^4 \times S^1$, the logic we used to for the $10$D solution will still carry over to $5$D. We will still write the extremal metric in terms of functions $H_{1,5}$ that must be harmonic w.r.t. the flat metric of the 4D transverse space.
	
	Then the D1 brane solution gives
	\[
		ds^2_{D1}= \frac{-dt^2 + d\gamma^2}{\sqrt{H_1}} + \sqrt{H_1} d \gamma^a \cdot d \gamma^a + \sqrt{H_1}\, dx^i \cdot dx^i, \quad H_1 = 1 + \frac{r_1^6}{r^6}, \quad e^{-2\Phi} = H_1^{-1}, \quad F_{05i} = \d_i (H_1^{-1})
	\]
	While the D5 brane gives
	\[
		ds^2_{D5} = \frac{-dt^2 + d\gamma^2}{\sqrt{H_5}} + \frac{d \gamma^a \cdot d \gamma^a}{\sqrt{H_5}} + \sqrt{H_5} \, dx^i \cdot dx^i, \quad H_5 = 1 + \frac{r_5^2}{r^2}, \quad e^{-2\Phi} = H_5, \quad F_{ijk} = -\epsilon_{ijk} \d_r H
	\]
	I think this problem has a typo and Kiritsis means $D_1, D_5$ not $N_1, N_5$. Further, Kiritsis (likely borrowing from Maldacena's thesis) writes $F_{05i} = -\frac12 \d_i (H^{-1} - 1)$. This factor of $1/2$ is different from what I'm used to seeing both in Kirtsis and Blumenhagen. This stems from a different choice of normalization for the Kalb-Ramond and RR forms in Maldacena's thesis. I am unsure why this different normalization exists, but at any rate I will \emph{ignore} the factor of $1/2$. Finally, the overall sign in $F$ disagrees also with Kiritsis and Blumenhagen, and I think the total D-brane charge in Maldacena counts anti-D-branes in our scheme.
	% We are also assuming here that at $r \to \infty$ we asymptote to the $g_s= 1$ vacuum solution of string theory.
	
	When superimposing a D1 and D5 solution, the dilaton and field strength contributions add while the metric contributions get multiplied. One way to see this is, because the solution remains BPS, we only need to solve the first-order BPS equations.
	
	For a $p$-Brane, as we have seen, the Killing spinors have spatial profile $\epsilon(r) = H^{1/8} \epsilon_0$ regardless of $p$. The linear equations for spinors coincide with the $D$-brane equations $\epsilon_L = \pm \Gamma^0 \dots \Gamma^p \epsilon_R$. We know that for the 1-5 system these can be simultaneously solved, giving a 1/4 BPS state. 
	
	\textbf{I could do this in more detail... but I've computed enough Killing spinors by this point.}
	
	The combined 10D solution thus gives:
	\[
		\frac{-dt^2 + d\gamma^2}{\sqrt{H_1 H_5}} + \sqrt{\frac{H_1}{H_5}} d\gamma^a  \cdot d\gamma^a+ \sqrt{H_1 H_5} dx^i \cdot dx^i, \quad F_{r05} = \d_r H^{-1}, \quad F_{ijk} = - H'(r), \quad, e^{-2\Phi} = \frac{H_5}{H_1}
	\]
	The next step is compactification. Upon wrapping D5 and D1 around a $T^5$, dimensional reduction freezes out $\gamma, \gamma^a$ dependence of the metric and fields.
	The $T^5$ is parallel to the D5, so the D5 solution will look identical to how it looked before. The D1 also wraps a cycle of the $T^5$. Compactifying the other 4 directions will look like a periodic arrangement of $D1$ branes, which effectively serves to remove $\gamma^a$ dependence from the $D1$ contribution to the solution. \textbf{Think about this. Is it really true that the metric warps the same regardless of where on $T^5$ I am? More likely that they are taking $T^5$ small and neglecting it, or we're thinking about a uniform distribution of D1s on $T^5$.}
	
	Finally, the D1 solution can be given momentum. 
	
	\item 
	Ignoring the $-1/2$ discussed before, we can verify the charge from direct integration along a $S^3$. First, the electric charge 
	\[
		\frac{1}{2 \kappa_{10}^2} \int_{S^3 \times T^4} \star F = \frac{1}{2 \kappa_{10}^2} \int_{S^3 \times T^4} \frac{2 r_1^2}{r^3} = -\frac{4\pi^2 (2\pi \ell_s)^4 V}{(2\pi)^7 \ell_s^8 g_s^2} r_1^2 = \frac{Q_1}{2\pi \ell_s^2 g} = Q_1 T_1
	\]
	for $r_1^2 = \ell_s^2 g_s/V$, as required. 
	
	For the magnetic charge, $F_{\theta \phi \psi} = \epsilon_{\theta \phi \psi r} \d_r H_5 = -H_5'$ so we get
	\[
		\frac{1}{2 \kappa_{10}^2} \int_{S^3} F = \frac{1}{2 \kappa_{10}^2} \int_{S^3} d\Omega_3 \frac{2 r_5^2}{r^3} = \frac{4 \pi^2}{(2\pi)^7 \ell_s^8 g_s^2} r_5^2 = \frac{Q_5}{(2\pi)^5 \ell_s^6 g_s} = Q_5 T_5
	\]
	for $r_5^2= \ell_s^2 g_s$, as required.
	
	We can also derive $c_p$ from the KK solution \textbf{Do this}.
	
	In the non-extremal case, this generalizes quite directly. 
	\[
		\frac{Q_1}{2 \pi \ell_s^2 g_s} = \frac{1}{2\kappa_{10}^2} \int_{S^3 \times T^4}  \hspace{-.1in}\star F = \frac{(2 \pi \ell_s)^4 V g_s}{(2\pi)^7 \ell_s^8 g_s^2} \int_{S^3} \coth a_1 \frac{2 r_1^2}{r^3} = \frac{V r_0^2 \sinh^2 a_1 \coth a_1}{2\pi g_s^2 \ell_s^4} = \frac{r_0^2 \sinh 2a_1 }{4\pi g_s \ell_s^2 c_1} \Rightarrow Q_1 = \frac{r_0^2 \sinh 2 a_1}{2 c_1}
	\]
	Similarly: 
	\[
		\frac{Q_5}{(2 \pi)^5 \ell_s^6 g_s} = \frac{1}{2\kappa_{10}^2} \int_{S^3}  \star F = \frac{1}{(2\pi)^7 \ell_s^8 g_s^2} \int_{S^3} \coth a_5 \frac{2 r_5^2}{r^3} = \frac{r_0^2 \sinh^2 a_5 \coth a_5}{(2 \pi)^5 g_s^2 \ell_s^8} = \frac{r_0^2 \sinh 2a_5}{2 (2 \pi)^5 \ell_s^6 g_s c_5} \Rightarrow Q_5 = \frac{r_0^2 \sinh 2 a_5}{2 c_5}
	\]
	For the KK momentum, I assume it can be read off from the $dt d\gamma$ term \textbf{justify} (Maldacena writes this too, in his thesis below 2.34), which goes as $r_0^2 \sinh a_p \cosh a_p = \frac{r_0^2 \sinh 2a_p}{2} = c_p Q_p$, giving KK momentum
	\[
		Q_p = \frac{r_0^2 \sinh 2a_p}{2 c_p}
	\]
	
	\item 
	First, by analogy to 5D we expect an extremal metric of the form
	\[
		- \lambda^{-1/2} dt^2 + \lambda^{1/2} (dr^2 + r^2 d\Omega_2^2), \quad \lambda = \prod_{i=1}^4 (1 + \frac{r_i}{r})
	\]
	This will have a nonzero area $4 \pi \sqrt{r_1 r_2 r_3 r_4}$ only when all the $r_i \neq 0$. On the other hand the total mass is $M = \sum_{i=1}^4 M_i$ with $M_i = r_i/4G$. The question is what the charges correspond to at the level of a brane construction. 
	
	Towards this end, let's take IIA and compactify on $T^6$. We consider a D6 brane wrapping $x^1 \dots x^6$ together with a D2 wrapping $x^1, x^6$. $6-2=4$ is good, makes the state 1/4 BPS. We can also add KK momentum along the $1$ direction. 
	
	The crucial principle (Maldacena 2.5) is that if a scalar diverges at the horizon, the $d$-dimensional character of the solution is lost. For a single $p$-brane $p \neq 3$ the dilaton goes either to $\infty$ or $0$. In the case of the $D1$-$D5$ system, we needed branes symmetric about $p=3$ and differing by $4$ in order to give a BPS state with the dilaton tending to a constant $\frac14 \log H_1/H_5 \to \frac12 \log r_1/r_5$.
	
	We see now that this does not work with a D6 and D2. For a $p$-brane $e^{-2 \Phi} = H^{(p-3)/2}$ giving that D6-D2 gas a dilaton going as $e^{-2 \Phi} = H_6^{3/2} H_2^{-1/2}$. There's no (even dimensional) D-brane we could add in type $1$ that would save us, and adding fundamental strings would only give $e^{-2 \phi} = H_f$, which would not help. 
	
	But there is another extended object with the correct dilaton dependence as $e^{-2\Phi} = H^{-1}$. This is the NS5 brane! But will adding it break supersymmetry completely? On the contrary, the SUSY constraints from the D6 and D2 and KK momentum are:
	\[
		\epsilon_L = \Gamma^{0 12345 6} \epsilon_R, \quad \epsilon_L = \Gamma^{016} \epsilon_R, \quad \epsilon_L = \Gamma^{01} \epsilon_L, \quad  \epsilon_R = -\Gamma^{01} \epsilon_R
	\]
	The NS5 brane wrapping $1 \dots 5$ would give $\epsilon_L = \Gamma^{0 12345} \epsilon_L, \epsilon_R = - \Gamma^{012345} \epsilon_R$. This can be rewritten as
	\[
		\epsilon_{L,R} = \pm \Gamma^6 \Gamma^{0123456} \epsilon_{R,L} = \Gamma^6 \epsilon_{L,R}
	\]
	But $\epsilon_L = \pm \Gamma^6 \epsilon_L$ already follows from the prior supersymmetry constraints, so adding NS5 breaks nothing! 
	
	\item Directly applying the formula derived in problem \textbf{1} with $F=f^{-1/3}, C= f^{-1/3} h$ gives
	\[
		2\pi \frac{2}{\sqrt{F(r_0)} C'(r_0)} = 2 \pi r_0 \cosh(a_1) \cosh(a_5) \cosh(a_p) 
	\]
	
	\item This is direct by writing the differentials in terms of variables $r_0, a_1, a_5, a_p$:
	\begin{center}
		\includegraphics[scale=0.5]{"Figures/5D BH 1st Law"}
	\end{center}
	The variations do not involve arbitrary changes in the $N_{\pm i}$. This is not obvious from the form of the first law \textbf{as far as I can tell}, but $N_{\pm i}$ do need to be discrete in the brane interpretation.
	
	\item I have done this problem for Andy's class on quantum black holes. I will copy the full answer below: 
	\textbf{BTZ as a Quotient of AdS$_3$}
	The objective of this problem is to describe the precise way in which the BTZ black hole arises as a quotient of $AdS_3$. Take the embedding space to be $\RR^{2,2}$, with metric:
	\[
		\eta = \text{diag}(-1, -1, 1, 1)
	\]
	Denote the coordinates of the embedding space by $x^\mu =(x^0, x^1, x^2,x^3)$. $AdS_3$ is given by $x_\mu x^\mu = -\ell^2$. The Killing vectors generating isometries are given by $J_{\mu \nu} = x_{\nu} \d_\mu - x_\mu \d_\nu$. The most general Killing vector is then $\omega^{\mu \nu} J_{\mu \nu}$. 
	
	Define the identification subgroup of $AdS_3$ by 
	\[
		P \sim e^{t \xi} P, t \in 2 \pi n, \quad n \in \ZZ
	\]
	For this identification to make physical sense, it should not give rise to closed timelike curves. Unfortunately, in some regions, the $\xi$ used in this construction do give rise to CTC’s. Luckily, however, they are bounded by a region where $\xi \cdot \xi = 0$. The part of the spacetime where $\xi \cdot \xi = 0$ is then interpreted as a singularity in the causal structure, and the region where $\xi \cdot \xi < 0$ is cut out of the spacetime. Let
	\[
		\xi = \frac{r_+}{l} J_{12} - \frac{r_-}{l} J_{03}
	\]
	
	\begin{enumerate}
		\item \textbf{Write down $\omega_{\mu \nu}$ for this Killing vector, and the corresponding Casimir invariants for the $\SO(2,2)$ isometry group, which are given by}
		\[
			I_1= \omega_{\mu \nu} \omega^{\mu \nu}, \quad I_2 = \frac12 \epsilon^{\mu \nu \rho \sigma} \omega_{\mu \nu} \omega_{\rho \sigma}
		\]
		
		 We have 
		\[
			\omega_{12} = - \omega_{21} = \frac{r_+}{l}, \quad \omega_{03} = -\omega_{30} = - \frac{r_-}{l}
		\]
		Giving Casimirs:
		\[
			I_1 = -2 \frac{r_+^2 + r_-^2}{l^2}, \quad I_2 = 4 \frac{r_+ r_-}{l^2}
		\]
		\item \textbf{Find the allowed region $\xi \cdot \xi > 0$ in terms of $x^1, x^2$.}
		
		 We can write:
		\[
			\xi = \frac{r_+}{l} (x_2 \d_1 - x_1 \d_2) - \frac{r_-}{l} (x_3 \d_0 - x_0 \d_3) = \begin{pmatrix}
				-\frac{r_-}{l} x_3\\
				\frac{r_+}{l} x_2\\
				-\frac{r_+}{l} x_1\\
				\frac{r_-}{l} x_0
			\end{pmatrix} = \begin{pmatrix}
				-\frac{r_-}{l} x^3\\
				\frac{r_+}{l} x^2\\
				\frac{r_+}{l} x^1\\
				-\frac{r_-}{l} x^0
			\end{pmatrix}
		\]
		This vector has norm:
		\[
			\xi^2 = \frac{1}{l^2} [r_+^2 (x_1^2 - x_2^2) - r_-^2 ( x_3^2 - x_0^2)] = \frac{1}{\ell^2} (r_+^2 - r_-^2)(x_1^2 - x_2^2) + r_-^2
		\]
		This is $\geq 0$ when (assuming $r_+ > r_-$)
		\[
			x_1^2 - x_2^2 \geq - \frac{r_-^2 l^2}{r_+^2 - r_-^2}
		\]
		\item
		 \textbf{Find the regions for $\xi \cdot \xi \in \{(0, r_-^2), (r_-^2, r_+^2), (r_+^2, \infty) \}$ and identify whether the boundaries between them are timelike, spacelikeor null:}
		
		 Now let's look at 
		\[
			r_+^2 \leq \xi^2 \Rightarrow (r_+^2 - r_-^2)(x_1^2 - x_2^2) \geq (r_-^2 + r_+^2) l^2 \Rightarrow l^2 \leq x_1^2 - x_2^2 
		\]
		Next
		\[
			r_-^2 \leq \xi^2 \leq r_+^2 \Rightarrow 0 \leq x_1^2 - x_2^2 \leq l^2
		\]
		Finally 
		\[
			0 \leq \xi^2 \leq r_-^2 \Rightarrow - \frac{r_-^2 l^2}{r_+^2 - r_-^2} \leq x_1^2 - x_2^2 \leq 0
		\]
	
		All boundaries between these regions are null. The boundary $x_1^2 - x_2^2 = l^2$ implies $x_0^2 - x_3^2 =0$ and so is a null surface (cone). 
		Similarly, the boundary $x_1^2 - x_2^2 = 0$ implies $x_+ = \pm x_-$ which is again null surface. 
			%
		% The boundary between region 1 and 2 is given by $x_1^2 - x_2^2 = \frac{(r_+^2 + r_-^2) l^2}{r_+^2 - r_-^2}$
		% It is null
	
		\item \textbf{For region I use the coordinate transform given by $x^0 = \sqrt{B(r)} \sinh \tilde t, x^1 = \sqrt{A(r)} \cosh \tilde \phi, x^2 = \sqrt{A(r)} \sinh \tilde \phi, x^3 = \sqrt{A(r)} \cosh \tilde t$ where $A,B = l^2 \frac{r^2 - r^2_\mp}{r^2_+ - r^2_-}$ and $\tilde t, \tilde \phi = \frac{1}{l} (\pm \frac{t r_\pm}{l} \mp r_\mp \phi)$ to write the metric in a form}
		\[
			-N_\perp^2 dt^2 + N_\perp^{-2} dr^2 + r^2 (N_\phi dt + d\phi)^2
		\]
		
		Note we get $x_1^2 - x_2^2 = A(r)$, $x_3^2 - x_4^2 = B(r)$, and $B(r) = A(r) + l^2$ so that $B - A = l^2$ as required in AdS. At $r = r_+$, $A = - \frac{(r_+^2 - r_-^2) l^2}{r_+^2 - r_-^2}$ exactly saturating the boundary of region $1$. Simple differential manipulations
		\begin{center}
			\includegraphics[scale=0.5]{"Figures/BTZ metric"}
		\end{center}
		gives us
		\[
			N_{\perp} = \frac{\sqrt{(r^2 - r_-^2)(r^2 - r_+^2)}}{l r}, \quad N_{\phi} = \frac{r_+ r_-}{l r^2}
		\]
	
		For region II, taking $r_- < r < r_+$ makes $B$ negative, so we will keep $x_1, x_2$ as before and instead define 
		\[
			x^0 = -(-B(r))^{1/2} \cosh \tilde t, \quad x^3 = -(-B(r))^{1/2} \sinh \tilde t
		\]
		Here we have flipped the sign of $B$ together with exchanging $\sinh$ and $\cosh$ (so as to remain in coordinates satisfying the AdS constraint). 
		\textbf{This is exactly as in Kiritsis 13.7.8, referred to as the ``standard BTZ form''}
	
		We keep $\tilde t, \tilde \phi$ the same. This gives the same value for $N_{\perp}$ and $N_\phi$. Lastly for region III, $A$ also becomes negative, and we redefine $x^1, x^2$ while keeping $x^0, x^3$ from region II:
		\[
			x^1 = (-A(r))^{1/2} \sinh \tilde \phi, \quad x^3 = (-A(r))^{1/2} \cosh \tilde \phi
		\]
	
		The here $r$ ranges from $0$ to $\infty$ while $t, \phi$ are unrestricted and range from $-\infty$ to $\infty$
	
		\item
		\textbf{Compute the Killing vector $\xi$ in the $(t, r, \phi)$ coordinates and perform the identification. You should recognize the metric found in the previous part as the BTZ geometry. Identify $M,J$ in terms of $r_\pm$ and write the casimir invariants from part a) in terms of $M,J$}
		
		By computing the Jacobian $\frac{\d(x^{0},x^1,x^2,x^3)}{\d(t,r,\phi)}$ and judiciously guessing what vector I should push forward (way easier than trying to compute inverse Jacobians to pull back $\xi$), I see that $\xi = \d_\phi$ in our new basis: 
		\begin{center}
			\includegraphics[scale=0.5]{"Figures/BTZ dphi"}
		\end{center}
		Indeed, it is easy to see that $\d_\phi$ is killing from directly applying the Killing equation, and now we see it comes directly from a combination of the manifest $J_{\mu \nu}$ symmetries in the embedding space.  We can thus identify $\phi$ as a periodic variable and retain the space as a solution to Einstein's equations.
	
		Looking at the $N_{\perp}^2$ and $N_\phi^2$ contributions to $g_{00}$ we get:
		\[
			-g_{00} = - \frac{r_-^2 + r_+^2}{\l^2}   + \frac{r^2}{l^2} 
		\]
		This is a black hole in AdS with mass $M = \frac{r_-^2 + r_+^2}{l^2}$. Similarly, from the $dt\, d\phi$ component, we see that $2 r^2 N_\phi$ corresponds exactly to the angular momentum. We thus get
		\[
			M = \frac{r_+^2 + r_-^2}{l^2}, \quad J = \frac{2 r_+ r_-}{l}
		\]
		Note for global AdS when $r_-\to 0, r_+ \to -l^2$, we get mass $-l^2$. \textbf{Kiritsis adjusts the definition of mass by $+1$ making it $0$ in global AdS, so that it counts only the mass of the black hole.}
		
		This gives 
		\[
			I_1 = -2 M, \quad  I_2 = 2 J / l
		\]
	\end{enumerate}
	
	\item The KK reduction is not too bad:
	\[
	\begin{aligned}
		\int d^5 x \, \sqrt{\det g} \, \Big(e^{-2\phi} \Big[R &+ 4 \d_\mu \phi \d^\mu \phi + \frac14 \d_\mu G_{\alpha \beta} \d^\mu G^{\alpha \beta} - \frac14  G_{\alpha \beta} F_{\mu \nu}^\alpha F^{\mu \nu\, \beta} \Big] \\
		& - \frac14 \sqrt{G} G^{\alpha \beta} G^{\gamma \delta} H_{\mu \alpha \gamma} H^\mu_{\ \gamma \delta} - \frac14 \sqrt{G} G^{\alpha \beta} H_{\mu \nu \alpha} H^{\mu \nu}_\beta - \frac{1}{12} \sqrt{G} H_{\mu \nu \rho} H^{\mu \nu \rho}  \Big)
	\end{aligned}
	\]
	with $\phi = \Phi - \frac14 \log \det G_{\alpha \beta}$. 
	Here $H$ comes from the \emph{two form} not from the $B$ field.\textbf{I'm almost certain that the formula in Kiritsis is wrong }
	
	 We can rewrite this as
	\[
	\begin{aligned}
			\int d^5 x \, \sqrt{\det g} \, \Big(e^{-2\phi} \Big[R &+ 4 \d_\mu \phi \d^\mu \phi - \frac14 G^{\alpha \beta} G^{\gamma \delta} \big(  \d G_{\alpha \gamma} \d G_{\beta \delta} - e^{2\phi} \sqrt{G} \d C_{\alpha \gamma} \d C_{\beta \delta}  \big) -  \frac14  G_{\alpha \beta} F_{\mu \nu}^\alpha F^{\mu \nu\, \beta} \Big] \\
			& - \frac14 \sqrt{G} G^{\alpha \beta} H_{\mu \nu \alpha} H^{\mu \nu}_\beta - \frac{1}{12} \sqrt{G} H_{\mu \nu \rho} H^{\mu \nu \rho}  \Big)
		\end{aligned}
	\]
	Now we must take this to the Einstein frame. We perform a Weyl rescaling $g \to e^{4\phi/3} g$. This rescales fields (and changes the kinetic $\phi$ term) to give us the requisite action
	\[
	\begin{aligned}
		S_5 = \frac{1}{2 \kappa_5^2} \int \sqrt{-g} \Big[R &- \frac43 (\d \phi)^2 - \frac14 G^{\alpha \beta} G^{\gamma \delta} (\d G_{\alpha \gamma} \d G_{\beta \delta} + e^{2\phi} \sqrt{G} \d C_{\alpha \gamma} \d C_{\beta \delta})\\
		& - \frac{e^{-4 \phi/3}}{4} G_{\alpha \beta} F_{\mu \nu}^\alpha F^{\mu \nu\, \beta} - \frac{e^{2\phi /3}}{4} \sqrt{G} G^{\alpha \beta} H_{\mu \nu \alpha} H^{\mu \nu}_\beta - \frac{e^{-2 \phi/3}}{12} \sqrt{G} H_{\mu \nu \rho} H^{\mu \nu \rho}  \Big]
	\end{aligned}
	\]
	
	Each of the field strengths will obey:
	\[
		\begin{aligned}
			\dd\star [e^{-4 \phi/3} G_{\alpha \beta} \, F^{\alpha}_{\mu \nu} ]&= 0\\
			\dd\star [e^{2 \phi/3} \sqrt{G} G^{\alpha \beta}\,  H_{\alpha\, \mu \nu} ]&= 0\\
			\dd\star [e^{-2 \phi/3} \sqrt{G}\,  H_{\mu \nu \rho} ]&= 0\\
			\nabla^{\mu} [G^{\alpha \beta} G^{\gamma \delta} e^{2 \phi} \sqrt{G} \d_\mu C_{\beta \delta}] &= 0
		\end{aligned}
	\]
	The dilaton will obey:
	\[
		\frac83 \Box \, \phi - \frac12 e^{2\phi} \sqrt G \, G^{\alpha \beta} G^{\gamma \delta} \d C_{\alpha \gamma} \d C_{\beta \delta} + \frac{4}{3}  e^{-4 \phi}{3} G_{\alpha \beta} F^{\alpha}_{\mu \nu} F^{\beta\, \mu \nu} - \frac{1}{6} e^{2\phi/3} \sqrt{G} G^{\alpha \beta} H_{\mu \nu \alpha} H^{\mu \nu \beta} + \frac{1}{18} e^{-2\phi/3} \sqrt{G} H_{\mu \nu \rho} H^{\mu \nu \rho}
	\]
	Finally, the metric will obey:
	\[
		R_{\mu \nu} - \frac12 g_{\mu \nu} R - \frac43 \d_\mu \phi \d_\nu \phi - \frac14  G^{\alpha \beta} G^{\gamma \delta}( \d_\mu G_{\alpha \gamma} \d_\nu G_{\beta \delta} + e^{2 \phi} \sqrt{G} \d)
	\]
	
	The solution in question has nonzero (magnetic) $H_{\mu \nu \rho}$ and (electric) $H_{\mu \nu}$, which are functions of $r$ alone.
	\textbf{Finish}
	
	\textbf{show solution}
	
	\item Write $G_{\alpha \beta} = \sqrt{\frac{H_1}{H_2}} (\delta_{\alpha \beta} + h_{\alpha \beta})$
	
	\textbf{Finish}
	
	\item 
	Here we only take the two-form field strengths $H_{5 \mu \nu}$ $F^5_{\mu \nu}$ to be nontrivial. 
	
	By redefining $\tilde \phi = \phi+\frac12 \nu_5, \lambda = -\frac12 \phi + \frac34 \nu_5$, we get
	\[
		- (\d \tilde \phi)^2 - \frac43 (\d \lambda)^2 = - \frac43 (\d \phi)^2 - (\d \nu_5)^2
	\]
	The $\phi$ term matches, and the $\nu_5$ kinetic term together with the four $(\d \nu)^2$ terms exactly reproduces the expected $-\frac14 \d_\mu G_{\alpha \beta} \d^\mu G^{\alpha \beta }$. Let's look at the field strengths
	\[
	\begin{aligned}
		&e^{-2\phi/3} \sqrt G H_{\mu \nu \rho}^2 = e^{-2\phi/3 + \nu_5 + 4 \nu} H_{\mu \nu \rho}^2 = e^{\frac43 \lambda + 4 \nu}  H_{\mu \nu \rho}^2\\
		& e^{2\phi/3} \sqrt G  G^{55} H_{5 \nu \rho}^2 = e^{2\phi/3 - \nu_5 + 4 \nu} H_{5 \nu \rho}^2 = e^{-\frac43 \phi + 4 \nu} H_{5 \nu \rho}^2\\
		& e^{-4 \phi/3} G_{55} (F^5_{\mu \nu})^2 = e^{-4\phi/3 + 2 \nu_5}  (F^5_{\mu \nu})^2 = e^{\frac83 \lambda} (F^5_{\mu \nu})^2
	\end{aligned}
	\]
	These all exactly match. Note that these scalars \emph{have} a potential- they are not minimally coupled. Consequently, at the horizon, where the field strengths diverge, we expect that the values of these scalars will be fixed by the equations of motion. 
	
	\item This is direct:
	\begin{center}
		\includegraphics[scale=0.5]{"Figures/BH Scattering"}
	\end{center}
	Importantly, the derivative of the inner $R$ profile goes as $O(r_0^2)$, which makes it subleading in determining $B$. Thus, $B$ is $O(r_0^2)$. On the other hand, this derivative's $r_0^2$ dependence is important for determining the incoming flux: $\mathrm{Im} (h r^3 R^* \d_r R)$. For completeness I will add this part of the notebook as well: 
	\begin{center}
		\includegraphics[scale=0.5]{"Figures/BH Scattering 2"}
	\end{center}
	
	\item Redefining $K$ to be unitless, we have a relationship 
	\[
		e^{i x \cos \theta} = K \frac{e^{- i \omega x}}{x^{3/2}} Y_{000} + \text{higher moments}
	\]
	Let's integrate both over $S^3$, giving
	\[
		\int_0^\pi d\theta \int_0^\pi d\phi \int_0^{2\pi} d\psi \sin^2 \theta  \sin \phi e^{i x \cos \theta} = 2 \pi^2 \frac{J_1(x)}{x}  = K \frac{e^{i \omega x}}{x^{3/2}} \sqrt{2 \pi^2}
	\]
	The asymptotic form of $J_1$ is
	\[
		J_1(x) \sim \sqrt{\frac{2}{\pi x}} \cos(z - 3 \pi /4)
	\]
	So then, up to a phase
	\[
		2 \pi \sqrt{2 \pi} = K \sqrt{2 \pi^2} \Rightarrow K = \sqrt{4 \pi}
	\]
	as required. 
	
	\item 
	In spacetime dimension $d+1$, one can write a black hole metric as:
	\[
		ds^2 = - f^{-1+1/(d-1)} h dt^2 + f^{1/(d-1)} \left[\frac{dr^2}{h} + r^2 d \Omega_{d-1}^2 \right] \Rightarrow \sqrt{-g} = f^{1/(d-1)} r^{d-1}, \quad h = 1 - \frac{r_0^{d-1}}{r^{d-1}}
	\]
	I will ignore the details of $f$ for now, since it depends a lot on the dimension and charges. $h$ is simply what reproduces the Schwarzschild solution for a totally uncharged black hole. What does matter is that the horizon is at $r = r_0$, giving a horizon area $A = \Omega_{d-1} r_0^{(d-1)} f(r_0)^{1/2}$, meaning that to leading order as $r \to r_0$ we have
	\[
		f(r) \approx \frac{R_H^{2(d-1)}}{r_0^{2(d-1)}}, \quad R_H^{d-1} = \frac{A}{\Omega_{d-1}}
	\]
	This is all we need for the near-horizon data.
	
	\begin{center}
		\includegraphics[scale=0.2]{"Drawings/BH Scatter"}
	\end{center}
	
	 What matters is that the throat size that is determined by $f$ is much larger than the extremality parameter $r_0$. For a minimal scalar $\Box \phi = 0$ in $d+1$ dimensions we get
	\[
		\left[\frac{h}{r^{d-1}} \d_r \left(h r^{d-1} \d_r \right) + \omega^2 f \right] R_\omega (r) = 0
	\]
	For $r$ small and close to $r_0$ define the coordinate $\sigma$ by
	\[
		d\sigma = \frac{dr}{h(r) r^{d-1}} \Rightarrow \d_\sigma = h(r) r^{d-1} \d_r
	\]
	This give us that the wave equation becomes
	\[
		[\d_\sigma^2 + \omega^2 r^{d-1} f(r)] R_\omega (r) = 0
	\]
	Now, in the $r \to r_0$ limit this simplifies to
	\[
		[\d_\sigma^2 + \omega^2 R_H^{2(d-1)}] R_\omega(r) = 0 \Rightarrow R = \tilde A e^{-i \omega \sigma R_H^{d-1}} \approx \tilde A ( 1 - i \omega \sigma R_H^{d-1})
	\]
	where we have picked the sign in the exponent so that the wave in the near region is purely incoming. In the final approximation, we're looking at the extreme near-horizon limit. Further, keeping $r \omega$ small but looking at large $r$ compared to $r_0$ we see that the behavior of $\sigma$ is given by $\sigma \approx -\frac{r^{-d+2}}{d-2}$, yielding 
	\begin{equation}\label{eq:near_zone}
		 R(r) \approx \tilde A (1 - i \omega \frac{R_H^{d-1}}{(d-2) r^{d-2}})
	\end{equation}
	
	Now for the far region:
	
	Redefining $\psi = r^{\frac{d-1}{2}} R$ and introducing the tortoise coordinate $dr_* = dr/h$ so that $\d_{r_*} = h \d_r$ we get
	\[
		\Big[- \frac{d^2}{dr_*^2} + \underbrace{\frac{(d-1)(d-3)}{4 r^2} \left(1 - \frac{r_0^{d-2}}{r^{d-2}} \right) \left(1+ \frac{d-1}{d-3} \frac{r_0^{d-2}}{r^{d-2}}\right) - \omega^2 f(r)}_{V(r_*)} \Big] \psi(r) = 0
	\]
	This reproduces \textbf{13.8.4} when $d=4$.
	We don't expect this to be solvable, and so we will work at it by matching. Again, $r_0, r_p \ll r_m \ll r_1, r_5$.

	For $r$ large, $r = r_*$ and we can divide through by $\omega$ giving $\rho = r \omega$. The equation then reduces to
	\[
		\left(\frac{d^2}{d\rho^2} + 1 - \frac{(d-1)(d-3)}{4 \rho^2} \right) \psi
	\]
	This has a solution in terms of Bessel functions:
	\[
		\psi = \sqrt{\frac{\pi \rho}{2}} [A J_{-1+d/2} (\rho) + B Y_{-1+d/2} (\rho)]
	\]
	This implies that asymptotically:
	\[
	  R \approx \frac{1}{r^{(d-1)/2}} [e^{i \omega r} e^{-i \pi/4} e^{-i (d-2)/4} (A - B e^{i (d-2)  \frac \pi2}) + e^{-i \omega r} e^{i \pi/4} e^{i (d-2)/4} (A - B e^{-i (d-2)  \frac \pi2})]
	  \]
  	The absorption probability is then
  	\[
  		\Gamma = 1 - \Bigg|\frac{1 + \frac{B}{A} e^{i \frac12 (d-2)}}{1 + \frac{B}{A} e^{-i\frac12 (d-2)}} \Bigg|^2.
  	\]
	We need an odd number of spatial dimensions for this to work, reflecting the fact that the Bessel functions degenerate in these cases. \textbf{There is probably a cleaner way here.}
	  
	Taking $\rho = r \omega \ll 1$ at $r = r_m$ for matching, the Bessel functions will become at leading order:
	\begin{equation}
		\begin{aligned}\label{eq:far_zone}
				R &\approx \frac{1}{(\omega r)^{(d-1)/2}}\sqrt{\frac{\pi \omega r}{2}} \left(A \frac{2^{1-d/2}}{\Gamma(d/2)} (r\omega)^{-1+d/2} + B \frac{2^{(d-2)/2}}{\Gamma(2-d/2) (r \omega)^{d/2-1}} \right)\\
				&=  A \frac{\sqrt{\pi} 2^{(1-d)/2}}{\Gamma(d/2)}  + B \frac{\sqrt\pi 2^{(d-3)/2}}{\Gamma(2-d/2) (\omega r)^{d-1}} 
		\end{aligned}
	\end{equation}

	Now let us match \eqref{eq:far_zone} onto \eqref{eq:near_zone}. This is direct, and gives:
	\[
		\frac{A}{\tilde A} = \frac{\Gamma(d/2)}{2^{(1-d)/2} \sqrt\pi}, \qquad  \frac{B}{\tilde A} = i \frac{\Gamma(2-d/2) (\omega R_H)^{d-1} 2^{(3-d)/2}}{\sqrt\pi (2-d) } \Rightarrow \frac{B}{A} = i \frac{2^{2-d} \Gamma(2-d/2) (\omega R_H)^{d-1} }{(2-d) \Gamma(d/2)}
	\]
	We will then get in the $\omega \to 0$ limit:
	\[
		\Gamma = \Big|\frac{2^{3-d} \Gamma(2-d/2) (\omega R_H)^{d-1}}{(d-2) \Gamma(d/2)} \Big|
	\]
	
	Now, following the discussion of \textbf{13.18}, we must account for  the conversion factor $K$ from partial waves to plane waves. In $d$ spatial dimensions we get this to be
	\[
		K = \sqrt{\frac{(2\pi)^{d-1}}{\omega^{d-1} \Omega_{d-1}}} 
	\]
	This gives
	\[
	\begin{aligned}
		\sigma_{abs} &= \Gamma |K|^2 = \Big|\frac{(2\pi)^{d-1} \Gamma(\frac d2)}{\omega^{d-1} 2 \pi^{d/2}} \frac{2^{3-d} \Gamma(2-d/2) (\omega R_H)^{d-1}}{(d-2) \Gamma(d/2)} \Big|\\
		 &= \Big|\frac{2 \pi^{d/2-1} R_H^{d-1} \Gamma(2-d/2)}{(d-2)} \Big| = \Big|\frac{2 \pi^{d/2-1} R_H^{d-1} \pi}{(d-2) \Gamma(d/2 - 1) \sin(d/2-1)} \Big|\\
		 &= \frac{\mathbf{2} \pi^{d/2} R_H^{d-1}}{\Gamma(d/2)} = A_H.
	\end{aligned}
	\]
	as required. \textbf{Literally after all that I'm off only by a factor of $2$}.
	

	% For $r$ near the horizon, at \emph{zeroth order} keeping only the incoming (positive frequency wave), I see that the equations become:
	% \[
	% 	\Big[ \frac{d^2}{dr_*^2} +  \frac{\omega^2 R_H^{2 (d-1)}}{r_0^{2(d-1)}}  \Big] \psi(r) \Rightarrow \psi = \frac{1}{r^{(d-1)/2}} e^{-i \omega r_* \frac{R_H^{d-1}}{r_0^{d-1}}}  \approx \frac{1}{r^{(d-1)/2}} \Big(1 - i \omega \frac{R_H^{d-1}}{(d-2) r^{(d-2)}} \Big)
	% 	% \frac{h}{r^{d-1}} \d_r [h r^{d-1} \d_r] + \frac{\omega^2 R_H^{2 (d-1)}}{r_0^{2(d-1)}}
	% \]
	% It is not obvious that zeroth order is good enough to do the matching between the near-field and far-field at the intermediate $r = r_m$. One can, however argue, that any corrections of this zeroth order equation for $\psi$ will receive corrections that are subleading in $r_0$, which will translate over to the matching conditions. We will see that this will give us the horizon area for $\sigma_{abs}(\omega = 0)$ as a nontrivial check. In other words - the deep near-horizon geometry really is \emph{all} that matters. We will therefore continue with this zeroth order solution.
		%
	% Solving this gives
	%
	% \textbf{Put in mathematica}
	%
	% % Keeping just the incoming (positive frequency) wave near the horizon, we get:
% 	\[
% 		\tilde A \left(1 - \frac{r_0^{d-2}}{r^{d-2}} \right)^{-i r_0 \omega \frac{R_H^{d-1}}{(d-2) r_0^{d-1}}}, \quad
% 	\]
	% The same analysis will give $B = 0$ for the reflected wave. We get


	

	% Let's repeat the argument, following Mathur's paper using $D=p+2$ spacetime dimensions. For a minimal scalar $\Box \phi = 0$ and we can decompose this as $R_\omega(r) e^{i \omega t}$ for appropriate coordinates $r, t$. The metric is now different from 5D. I have some freedom in my choice of representation for the metric. I will pick a particularly simple form for the metric. Namely, any spherically-symmetric solution can be brought to isotropic coordinates:
	% 	\[
	% 		ds^2 = - f(r) dt^2 + g(r) [dr^2 + r^2 d\Omega_{p}^2]
	% 	\]
	%
	% 	A minimally-coupled scalar satisfies the equation:
	% 	\[
	% 		\left(\frac{1}{r^p g^{\frac{p+1}{2}} f^{1/2}}\d_r[ r^{p} g^{\frac{p-1}{2}} f^{1/2} \d_r] + \frac{1}{f} \omega^2 \right) R_\omega = 0 \Rightarrow ((r^{p} g^{\frac{p-1}{2}} f^{1/2} \d_r)^2 + [r^2 g]^p \omega^2 ) R_\omega = 0
	% 	\]
	% 	Redefining a tortoise-like coordinate $\rho$ so that $\d_\rho = \frac{dr}{r^{p} g^{\frac{p-1}{2}} f^{1/2} }$ the differential equation considerably simplifies to
	% 	\[
	% 		(\d_\rho^2 + [r^2 g]^p \omega^2) R_\omega = 0
	% 	\]
	% 	The horizon position $r_H$ gives area $A = r_H^{p} g(r_H)^{p/2} \Omega_p$. The true ``radius'' of the black hole is then $R_H^{p} := A/\Omega_p = r_H^{p} g(r_H)^{p/2} $ Near the horizon, this equation becomes to leading order:
	% 	\[
	% 		(\d_\rho^2 + R_H^{2p} \omega^2) R_\omega = 0 \Rightarrow R_\omega = e^{-i R_H^{p} \omega \sigma}
	% 	\]
	%
	
	
	\item We have seen that the Ricci scalar for a D$p$ brane solution takes the exact form:
	\[
		R = \frac{L^{2(7-p)}}{4 r^{\frac{p-3}{2}}} \frac{(p+1) (p-3) (p-7)^2}{(r^{7-p} + L^{7-p})^{5/2}}
	\]
	For $p = 3$, this vanishes identically. On the other hand, the dilaton EOM yields
	\[
		R = 4 (\nabla \Phi)^2 - 4 \Box\, \Phi
	\]
	since the solution $\Phi$ is a constant, at leading order about a classical solution, this equation reads
	\[
		\Box \Phi = 0.
	\]
	Thus, the dilaton is indeed minimal. We do not expect such nice simplification for other D$p$ branes. 
	
	Now, let us calculate cross section per unit D3 brane volume.
	We consider $s$-wave scattering. This wave equation in the D3 background $g_{rr} = \sqrt{H(r)}, \sqrt g = r^5 \sqrt{H(r)}$ translates to
	\[
		0 = \Big(\frac{1}{\sqrt g} \d_r g^{rr} \sqrt{g} \d_r  + \omega^2 g^{tt} \Big) R(r)  = \left( \frac{1}{r^5 \sqrt{H}} \d_r r^5 \d_r + \omega^2 \sqrt{H}  \right) R \Rightarrow \frac{1}{r^5} \d_r r^5 \d_r R + \omega^2 \left(1 + \frac{L^4}{r^4} \right) 
	\]
	As before, lets redefine $\psi = r^{5/2} R$. We get the equation
	\[
		0 = \psi''(r) + \left[\omega^2 \left(1 + \frac{L^4}{r^4}\right)  - \frac{15}{4 r^2} \right] \psi(r) \Rightarrow V_{eff} = \frac{15}{4 r^2} - \omega^2 \left(1 + \frac{L^4}{r^4}\right)
	\]
	Again, this does not look like it has an analytic solution (actually apparently it does and its a Matthieu function, but we don't really know that). Taking $\rho = \omega r$ and looking at $\rho \gg 1$, we drop the $L^4/\rho^4$ term and obtain solutions
	\[
		\psi = \sqrt{\frac{\pi \rho}{2}} \left[A J_2 (\rho) + B Y_2(\rho)\right]
	\]
	For large $\rho$ these asymptote to
	\[
		R \approx \frac{1}{2 r^{5/2}} [e^{i r \omega} (A - i B ) e^{3 i \pi/4} + e^{-i r \omega} (A  + i B) e^{-3i \pi/4}]
	\]
	For the small $r$ limit, on the other hand, we get
	\[
		\psi = \sqrt{\frac{\pi \rho}{2  (L \omega)^2}} \left[\tilde A J_2 (\frac{(L \omega)^2}{\rho}) + \tilde B Y_2(\frac{(L \omega)^2}{\rho}) \right]
	\]
	For this to be an incoming wave in this region, we require the combination 
	\begin{equation} \label{eq:D3inc}
		R = \tilde A \frac{(L \omega)^4}{\rho^2} \left[ J_2 (\frac{(L \omega)^2}{\rho}) + i Y_2(\frac{(L \omega)^2}{\rho}) \right]
	\end{equation}
	We want to match this at an intermediate value of $r$. Take $\rho \ll 1$, and look at low frequencies. This will allow us to write the first solution as
	\[
		R = \frac{1}{r^{5/2}} \sqrt{\frac{\pi \rho}{2}} (A J_2(\rho) + B J_2(\rho)) \approx \frac18 \sqrt{\frac{\pi \omega^5}{2}} A - \frac{2}{r^4 \sqrt{\omega^3 \pi/2}} B
	\]
	We see that for small $r$, the second term blows up, and we expect that for matching to hold, we must take $B=0$. 
	
	Meanwhile, equation \eqref{eq:D3inc} for small $r$ gives in the $\omega \to 0$ limit an expansion: 
	\[
		R \approx - \frac{4 i \tilde A}{\pi}
	\]
	For these to match we must have:
	\[
		\frac18 \sqrt{\frac{\pi \omega^5}{2}} A = - \frac{4 i \tilde A}{\pi} \Rightarrow \frac{\tilde A}{A} = \frac{1}{32} \sqrt{\frac{\omega^5 \pi^3}{2}} i
	\]
	The conserved flux is 
	\[
		\mathcal F = \frac{1}{2 i} \left[r^5 R^* \d_r R - c.c. \right] 
	\]
	For the incoming wave this is $\mathcal F_{in} = -\omega |A/2|^2$. For the absorbed one, a quick Mathematica computation gives
	 $\mathcal F_{abs} = -\frac{2 L^8 \omega^4}{\pi} |\tilde A|^2$
\begin{center}
	\includegraphics[scale=0.5]{"Figures/Rabs"}
\end{center}
	This gives
	\[
		R_{abs} = \frac{\mathcal F_{abs}}{\mathcal F_{in}} = \frac{|\tilde A|^2}{|A|^2} \frac{8 L^8 \omega^3}{\pi} = \frac{\pi^2 (L \omega)^8}{(16)^2}
	\]
	Now, following the discussion of \textbf{13.18}, we must account for  the conversion factor $K$ from partial waves to plane waves. In $D$ spatial dimensions we get this to be
	\[
		K = \sqrt{\frac{(2\pi)^{D-1}}{\omega^{D-1} \Omega_{D-1}}} \to 32 \pi^2
	\]
	in our case of $D = 6$. Altogether we get
	\[
		\sigma_{abs} = K^2 R_{abs} = \frac{\pi^4 L^8 \omega^3}{8}
	\]
	as required.
	
	The fact that the $B$ coefficient did not come into play confirms once again that the near-horizon regime is all that matters in the calculation of $\sigma_{abs}$
	
	For higher partial waves, the Bessel functions involved take the form $J_{\ell+2}$. The outer region will continue to have $B = 0$ enforced, and look like
	\[
		\frac{A}{\rho^2} J_{2 + \ell}(\rho)
	\]
	while the inner region will look like
	\[
		\tilde A \frac{(L \omega)^2}{\rho^2} \Big[ J_{2+\ell}(\frac{(L \omega)^2}{\rho})+ i Y_{2+\ell}(\frac{(L \omega)^2}{\rho}) \Big]
	\]
	This will give a match like $\tilde A \sim (\omega L)^{-2 \ell} A$. This makes it so that their ratio squared goes as $(\omega L)^{4 \ell}$. The flux calculations remain the same. Altogether we expect $\sigma_{abs}$ to scale as $L^8 \omega^3 (L \omega)^{2\ell}$ for higher partial waves.
	
	Finally, the Hawking emission rate remains zero, since it involves a factor of $e^{-\beta \omega}$ and $\beta = 1/T = \infty$ for an extremal $p$-brane. There are likely corrections to this \emph{beyond} the semiclassical level of analysis.
	
	\item Let's expand:
	\[
		\prod_{n=1}^\infty \frac{(1+q^n)^8}{(1-q^n)^8} = 1 + 16 q + 144 q^2 + 960 q^3 + 5264 q^4 + \dots
	\]
	
	For $N=0$ $(T^4)^N/S_N$ is just a point which has trivial cohomology ring with dimension $1$. 
	
	For $N=1$ we recover $T^4$ which has $4 \times 4$ cocycles generated as an alternating algebra by the elements $dx^i, i = 1 \dots 4$, giving dimension $16$. Note that we should view $dx^i$ as \emph{fermionc elements} corresponding to the odd cohomology, and even elements such as $1, dx^i \wedge dx^j, dx^1 \wedge dx^2 \wedge dx^3 \wedge dx^3$ as bosonic.
	
	For $N=2$ we get $T^{8}/S_2$ identifying points of two separate $T_4$s. Each individual $T_4$ has all of its cycles remaining intact, giving $2 \times 2^4 = 32$ cycles untouched. The remaining $2^8 - 2^4$ cycles are half-killed, giving $2 \times 2^4 + (2^8 - 2 \times 2^4)/2 = 144$. Although this gives the right answer, I see that its not the most generalizable way to look at things. There will always be an untwisted sector of this orbifold, as well as twisted sectors in 1-1 correspondence with conjugacy classes of $S_N$. The untwisted sector simply considers $N$ particle states on $T^4$. There are $8$ fermionic elements and $8$ bosonic elements in the cohomology. The types of 2-particle states are thus:
	\[
		\underbrace{\frac{8 \times 9}{2}}_{\text{bose-bose}}  + \underbrace{\frac{8 \times 7}{2}}_{\text{fermi-fermi}} + 	\underbrace{8 \times 8}_{\text{bose-fermi}} = 128
	\]
	The twisted sector here is a single copy of $T^4$, and any cycle is allowed. We thus get an additional $16$ terms, giving $144$ as desired. 
	
	Now let's look at $N=3$, the first case where $S_N$ becomes nonabelian. We expect an untwisted sector, corresponding to the system of $3$ point particles on $T^4$. This gives 
	\[
		\frac{8 \times  9 \times 10}{3!} + \frac{8 \times  7 \times 6}{3!} + \frac{8 \times 9}{2!} \times 8 + \frac{8 \times 7}{2!} \times 8 = 688
	\]
	 as well as two twisted sectors, in 1-1 correspondence with the conjugacy classes $(123)$ and $(12)(3)$ of $S_3$. The former gives a single $T^4$, whose cohomology is $16$. The other gives two (independent!) $T^4$s, whose cohomology then is $16 \times 16$. Altogether we get:
	 \[
	 	688 + 16^2 + 16 = 960
	 \]
	 as required.
	 
	 Let's finally do $N=4$. 
	 \begin{center}
	 	\includegraphics[scale=0.5]{"Figures/T4 cohomo"}
	 \end{center}
	
	The generating function, for a manifold $M$ with $f$ odd cycles and $b$ even cycles, consists of taking the generators of $H^*(M)$ to be $\alpha_{-1}^a, a = 1 \dots \dim H^* (M)$. For each $S_n$ twisted sector of $n$ copies of $M$, we introduce ``twisted modes'' $\alpha_{-n}^a$.
	
	Then, the generating function consists of taking products over all $n$ so that for a given $S_N$, the full orbifold is built up from taking all the ways one can partition $N$ in terms of subsectors twisted by $n_i, \sum n_i = N$. Thus, we look for the $q^N$ coefficient in:
	\[
		\prod_{i=1}^\infty \frac{(1+q^n)^f}{(1-q^n)^b}.
	\]
	
	\item A derivation of Cardy's Formula: 
	
	Take the CFT to have continuous spectrum, which we can write in terms of a $\delta$-function based $\rho(\Delta)$ as
	\[
		Z(\tau) = \int_0^\infty d\Delta \rho(\Delta) e^{2 \pi i \tau \Delta}
	\]
	We can invert this using a Bromwich integral:
	\[
		\rho(\Delta) = \int_C d\tau Z(\tau) e^{-2\pi i \tau \Delta}
	\]
	where $C$ is the contour running parallel but slightly above the real axis, enclosing the upper half plane. In the $q$-disk this would run close to the boundary of the disk. 
	
	Now, we would like an expression for $Z(\tau)$ as $\Im \, \tau \to 0$, namely the high-temperature limit. We know that $Z(\tau \to \infty) = \dim \mathcal H_0$, the space of ground states, which we take to consist of only a unique $\ket 0$, so we take this to be $1$. Further, we know that 
	\[
		q^{-c/24} Z(\tau) = e^{-2\pi i \tau c/24} Z(\tau) = Z(-1/\tau) e^{\frac{2\pi i c}{24 \tau}}
	\]
	is modular invariant. This implies that 
	\[
		Z(\tau \to 0) \approx Z(\infty) e^{2 \pi i \frac{c}{24 \tau} } =e^{2 \pi i \frac{c}{24 \tau} } 
	\]
	We now can approximate the integral:
	\[
		\rho(\Delta) \approx \int_{-\infty}^\infty d\tau e^{-2 \pi i (\tau \Delta - \frac{c}{24 \tau})}
	\]
	This gives a stationary value at
	\[
		\Delta + \frac{c}{24 \tau^2} = 0 \Rightarrow \tau = i \sqrt{\frac{c}{24 \Delta}}
	\]
	Plugging this back and interpreting $\rho$ as just an expected number of states at a given level $\Omega(N)$ gives
	\[
		\Omega(N) = \rho(\Delta) \approx e^{2 \pi \sqrt{\frac{c \Delta}{6}}} \Rightarrow S = \log \Omega(N) \approx 2 \pi \sqrt{\frac{c}{6}} \sqrt{N}
	\]
	I did this just for the left-movers, but taking left and right movers together gives the desired result:
	\[
		S = \log \Omega(N_L, N_R) \approx  2 \pi \sqrt{\frac{c}{6}} (\sqrt{N_L} +  \sqrt{N_R})
	\]
	
	\item
	As we've seen before, a single free boson the partition function is $\Tr(e^{-\beta L_0}) = \eta^{-1}$ while for a two free fermions, bosonization give the identical result. For a single free fermion then, at leading order (which means just retaining the same central charge) this gives $\eta^{-1/2}$. For $n_f$ copies of this system we simply exponentiate to obtain the leading piece:
	\begin{equation}\label{eq:susypart}
		\left(\prod_{n=1}^\infty \frac{1}{1-e^{-\beta n/R}} \right)^{\frac32 n_f}
	\end{equation}
	This can be directly written using $q = e^{-\beta/ R}$ ie $\tau = i \beta /2 \pi R$
	\[
		\left(\frac{q^{1/{24}}}{\eta(\tau)}\right)^{\frac32 n_f}
	\]
	We need the high temperature limit, for which $q^{1/24}$ is subleading and can be ignored. 
	Now the $\eta$ function satisfies $\eta(\tau) = \sqrt{i/\tau} \, \eta(-1/\tau)$ 
	\[
		\eta(\beta n/R \to 0) = \sqrt{\frac{2 \pi R}{\beta n}}\, e^{-\frac{(2 \pi)^2 R}{24 \beta}}
	\]
	The square-root term is also subleading and we obtain to leading order
	\[
		\log Z = \frac32 n_f \frac{(2\pi)^2 R}{24 \beta}
	\]
	which is exactly what we want, once we remove units from $R$, $R \to \ell_s R$.
	
	In general, we also have fermionic contributions modifying the numerator in equation \eqref{eq:susypart}. Here, again $\tau = i \beta / 2 \pi R$. For a single periodic \emph{or} antiperiodic fermion we will have traces that gives partition functions of the form:
	\[
	\begin{aligned}
		\Tr_{P} [e^{-\beta L_0/2\pi R}]  = \prod_{n = 0}^\infty (1 + q^{\frac{\beta}{2 \pi R} n})  = \sqrt{\frac{q^{1/24} \theta_{2}(\tau)}{\eta}}\\
		\Tr_{A} [e^{-\beta L_0/2\pi R}]  = \prod_{n = 0}^\infty (1 + q^{\frac{\beta}{2 \pi R} (n+1/2)})  = \sqrt{\frac{q^{1/24} \theta_{3}(\tau)}{\eta}}
	\end{aligned}
	\]
	Taking the infinite temperature limit sets $\tau \to 0, q \to \infty$ giving respectively
	\[
		\begin{aligned}
			\sqrt{ \frac{\theta_4 (-1/\tau)}{\eta(-1/\tau)}} \approx e^{\frac{2 \pi i}{48 \tau}} = e^{\frac{(2 \pi)^2 R}{48 \beta}}\\
			\sqrt{ \frac{\theta_3 (-1/\tau)}{\eta(-1/\tau)}} \approx e^{\frac{2 \pi i}{48 \tau}} = e^{\frac{(2 \pi)^2 R}{48 \beta}}
		\end{aligned}
	\]
	So in both cases we retain the same contribution as the $\eta^{-1/2}$ divergence, with sub-leading terms being different. 
	
	
	
\end{enumerate}
% section chapter_13_black_holes_and_entropy_in_string_theory (end)

\end{document}
	